\lecture{11}{18.08.2020}{Nonlinear Control: Sliding Mode and Synergetic Control}

We will cover Sliding Mode Control (SMC) and Synergetic Control approaches to nonlinear control. Both are based on the idea of establishing a \emph{linear relation} between the state variables to simplify the solution. This way, the (state of the) system is forced to move to a certain locus/manifold $\psi(x)=0$. That is, through some constraints, we map the original system to a simpler one.

Note we are not talking about linearizing the system through some approximation, but we are adding to the system so the result is linear.

In traditional feedback control, the structure of the feedback is fixed, that is, our input $u(t)$ which must be defined based on the state of the system $x(t)$ is set to be $u(t)=f\left( x(t) \right) =k^{T}x(t)$. In the \textbf{Variable Structure System (VSS)}, the system is allowed to change its structure at any instant. So the design problem becomes the selection of parameters of each structure and the definition of the switching logic.

So in a double-integrator system (second-order) in a closed-loop, given by \[
    \ddot{x}(t)=-\Psi x(t)
\] in which $\Psi$ can alternate between $\Psi=a_1^{2},\Psi=a_2^{2}$, where $0<a_2<1<a_1$, we know that the solution in both cases is not asymptotically stable. Even more, both are of the form of a complex conjugate with null real part (see Note 1 below).

\begin{note}
    A solution for \[
    \ddot{x}(t) = -K x(t)
    \] , or any similar system, can be easily found by assuming that the solution may be in the form of $e^{\lambda t}$. Then we have \[
    \lambda^{2}e^{\lambda t} = -K e^{\lambda t}
    \] which implies \[
    \lambda^{2}+K = 0 \implies \lambda = \pm i\sqrt{K}
    \].
\end{note}

If we understand the state variables of the system as $x_1=x$ and $x_2=\dot{x}$, then our system becomes \[
\begin{cases}
    \dot{x}_1(t)=x_2 \\
    \dot{x}_2=-\Psi x_1
\end{cases}
\] and if we define a switching of \[
\Psi = \begin{cases}
    a_1^{2}, & \dot{x}x>0 \\
    a_2^{2}, & \dot{x}x < 0
\end{cases}
\] 

% STOPPED @ 14:21
