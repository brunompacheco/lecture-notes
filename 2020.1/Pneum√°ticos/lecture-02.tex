\lecture{2}{22.08.2020}{Componentes do Sistema Hidráulico}

\section*{Válvulas}

\subsection*{Servoválvula e Válvulas Proporcionais}

Vimos já válvulas direcionais, que possuem estados discretos, normalmente: fechado, paralelo e cruzado.

\begin{definition}
    \emph{Válvulas de Controle Contínuo} (VCC) são válvulas que controlam o fluxo de energia de um sistema de modo contínuo, em resposta a um sinal de entrada também contínuo.
\end{definition}

Elas possuem os mesmos estados das válvulas direcionais, mas \emph{trocam de estado de forma contínua}.
 
\subsubsection*{Válvulas Proporcionais}

É uma VCC, mas possui uma construção com dois solenóides proporcionais.

A sua vazão é derivada a partir da vazão através de um orifício \[
q_V = cd\cdot A_0\cdot \sqrt{\frac{2\Delta p}{\rho}} 
\], onde $A_0$ é a área do orifício, $\Delta p$ é a diferença de pressão entre as extremidades, $\rho$ é o coeficiente de algo relacionado ao fluido, e $cd$ é um coeficiente relacionado ao formato do orifício. Assim, a vazão da válvula carretel de 4 vias pode ser calculada como \[
q_{VC} = Kv\cdot \frac{x}{x_n}\cdot \sqrt{\Delta p_t} 
\], onde $Kv$ é o coeficiente de vazão, $x$ é o deslocamento do carretel, e $x_n$ é o deslocamento nominal do carretel. Temos que $\Delta p_t = 2\cdot \Delta p_{via}$ para pressão uniforme ao longo do circuito. Além disso, podemos utilizar também $\frac{i}{i_n}$ ou $\frac{U}{U_n}$, dependendo do coeficiente de vazão.


