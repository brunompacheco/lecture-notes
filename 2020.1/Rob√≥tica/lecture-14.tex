\lecture{14}{15.10.2020}{Cinemática Diferencial}

Cinemática diferencial é o próximo passo: estudo da propagação das velocidades em um robô.

Definimos a velocidade vetorial de um ponto através de vetores unitários. De forma genérica como \[
\bm{v} = \begin{bmatrix} \dot{x} \\ \dot{y} \\ \dot{z} \end{bmatrix} = \|\bm{v}\|\bm{\dot{u}}
,\] em que $\bm{\dot{u}}$ é um vetor unitário.

Assim, dado um vetor $\bm{q}$ dos parâmetros de D-H, o problema da cinemática direta diferencial é o mapeamento $\bm{\dot{q}} \to \bm{v}$, onde $\bm{v}$ é a velocidade no efetuador, ou seja, \[
\bm{\dot{q}} = \begin{cases}
    \dot{\theta} \\ \dot{d}
\end{cases}  \to \bm{v} = \begin{cases}
    \dot{p} \\ \omega
\end{cases} 
,\] sendo $p$ as posições do efetuador e $\omega$ a sua velocidade angular. Esse mapeamento é feito através do Jacobiano.

\begin{eg}
    Em um robô RPRR em um espaço bidimensional, temos \[
    \bm{q} = \begin{bmatrix} \theta_1 \\ d_2 \\ \theta_3 \\ \theta_4 \end{bmatrix} \implies \dot{\bm{q}} = \begin{bmatrix} \dot{\theta}_1 \\ \dot{d}_2 \\ \dot{\theta}_3 \\ \dot{\theta}_4 \end{bmatrix} 
    .\]

    Ao mesmo tempo, o efetuador possui velocidade descrita por \[
    \bm{v} = \begin{bmatrix} \dot{p}_x \\ \dot{p}_y \\ \omega \end{bmatrix} 
    .\] 

    Fazemos o mapeamento através da matriz jacobiana $J$ de forma que \[
    \bm{v} = J \bm{\dot{q}}
    .\] 
\end{eg}

\subsection*{Singularidade}

Singularidade acontece quando em uma posição, existe a redução das possibilidades de movimento do efetuador. Ou seja, independente da atuação feita nas juntas, não é possível se deslocar nas direções e sentidos antes possíveis. Podemos analisar isso através da Jacobiana, que indica uma redução das dimensões da imagem da transformação pelo seu \emph{rank}.

\subsection*{Redundância}

Quando o robô possui juntas que permitem o mesmo movimento do atuador, possibilitando mais posturas para uma mesma posição. Isso possibilita, \emph{e.g.}, evitar colisões nos trajetos.

