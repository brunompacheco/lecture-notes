\lecture{25}{26.11.2020}{Introdução à Geração de Trajetórias}

O principal objetivo da geração de trajetórias é facilitar o controle de posição, velocidade, aceleração. Uma movimentação para uma posição desejada, através da cinemática inversa, resulta em uma mudança brusca dos parâmetros dos atuadores, ou seja, um esforço muito grande.

Uma trajetória lida com as variações da posição, velocidade e aceleração ao longo do tempo, no espaço de juntas. A partir de uma posição inicial, deseja-se atingir uma posição final através de posições intermediárias. Calcula-se a posição e velocidade para essas posições para que sejam suaves.

Se esse trabalho de definição e cálculo de cinemática inversa é feita no momento de execução, chama-se online. Se é feito previamente, chama-se offline.

Trajetórias ponto-a-ponto são geradas quando não importa o caminho, mas o ponto inicial e final. Assim, a trajetória se resume a deslocamentos "diretos" de um ponto para o seguinte.

Trajetórias contínuas são geradas através de interpolação numérica das posições e orientações desejadas, gerando curvas parametrizadas. Naturalmente, são caracterizadas pela ordem da interpolação.

Normalmente, utilizamos funções de terceira ordem para a geração de trajetórias, resultando em uma velocidade continua e uma aceleração descontínua mas bem controlada. Dessa forma, a trajetória torna-se \[
q(t) = a_3t^3 + a_2t^2 + a_1t + a_0
.\] Pode-se encontrar esses coeficientes através das condições iniciais e finais desejadas para a posição e velocidade.

\section*{Perfil de velocidade trapezoidal}

Muito comum é impor um perfil de velocidade trapezoidal, ou seja, sendo $t_i =0$, a posição varia de forma constante, salvo por um período $t_c$ de aceleração e desaceleração (constantes). Veja que nessa situação, a aceleração do sistema é constante (em módulo) durante os períodos de aceleração e desaceleração e tem valor \[
\ddot{q}_{max} = \frac{\dot{q}_{max}}{t_c}
.\]

Normalmente definimos o período de aceleração a partir da aceleração máxima suportada pelo robô, integrando a curva da velocidade. Chegamos em \[
t_c = \frac{t_f}{2} -\sqrt{\frac{\ddot{q}_{max}t_f^2 - 4\Delta q}{4 \ddot{q}_{max}}} 
.\] 

