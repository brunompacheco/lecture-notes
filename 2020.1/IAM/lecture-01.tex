\lecture{1}{31.09.2020}{Introdução à Automação da Manufatura}

\section*{Materiais}

Classificação básica:
\begin{description}
\item[Metais] alta condutividade térmica e elétrica; podem receber polimento. \emph{Muitos elétrons livres}.
    \item[Polímeros] baixa densidade; isolantes térmicos e elétrico; refletem pouca luz. \emph{Poucos elétrons livres}.
    \item[Cerâmicos] relativamente duros e frágeis. \emph{Mistura de metais e polímeros}.
\end{description}

Ligações interatômicas:
\begin{description}
    \item[Iônicas] metal com não metal; transporte de elétrons.
    \item[Covalentes] compartilhamento de elétrons; típica de polímeros.
    \item[Metálicas] elétrons livres.
\end{description}

Entendemos os átomos como esferas rígidas. Então vemos as estruturas cristalinas como o arranjo dessas esferas em células, que são o menor grupo que representa a estrutura do resto da estrutura cristalina.

As faces das estruturas cristalinas definem as propriedades do material naquele plano.o

Fator de empacotamento = fração do volume ocupado por átomos.

A solidificação do material começa a formação do cristal através de núcleos, que vão se expandindo. Essas regiões com formação unidirecional cristalina é chamada de \textbf{grão}. O tamanho dos grãos influencia nas propriedades do corpo.

Corpos não puros:
\begin{description}
    \item[Soluções Substitucionais] quando átomos de um material não metálico substituem aleatoriamente ou ordenadamente os átomos metálicos;
    \item[Soluções intersticiais] quando os átomos ocupam os espaços vazios das células da estrutura cristalina;
\end{description}

Defeitos como falta de átomos ou células mal-arranjadas (discordantes das demais) são responsáveis por dar características aos materiais. Podem ser pontuais, de arestas, helicoidais ou até mesmo superficiais.

