\lecture{2}{02.09.2020}{Revisão e Diagrama de Equilíbrio, TTT, CCT e Tratamentos}

\section*{Processos de Fabricação}

Processos de fabricação:
\begin{itemize}
    \item Fundição
    \item Usinagem
    \item Soldagem
    \item Metalurgia do Pó
    \item Conformação Mecânica
\end{itemize}

Os processos afetam a estrutura do aço de diferentes formas, alterando o arranjo e formato dos grãos.

\subsubsection*{Fundição}

Solidificação se da por núcleos e a partir da superfícies, então altera a estrutura.

\subsubsection*{Usinagem}

Não altera a estrutura da peça, ótimo acabamento de superfície, mas \textbf{custoso}.

\subsubsection*{Soldagem}

Gera uma região de fusão, que acarreta os mesmos problemas da fundição, além das proximidades, que são aquecidas e geram alterações na estrutura.

\subsubsection*{Metalurgia do Pó}

Pó metálico é compactado na forma desejada e aquecido para que aconteça difusão entre as partículas. Acaba formando zonas ocas distribuídas pela peça.

Resultado é uma peça com certa porosidade $\implies$ baixa resistência e baixa ductibilidade.

Aplicações:
\begin{itemize}
    \item Principal aplicação para metais com ponto de fusão baixo, tornando o processo barato;
\item 	Também utilizado para filtros;
   \item  Também pode ser utilizado para peças auto-lubrificantes
\end{itemize}

\subsubsection*{Conformação}

Excelentes características e propriedades para o material resultante, pois pode alterar as propriedades para o material resultante, controlando a temperatura também. Só factível para lotes muito grandes pelos altos custos.

Processos classificados pela ação mecânica:
\begin{itemize}
    \item Compressão direta $\implies$ forjamento, laminação
\item Compressão indireta $\implies$	perfilamento e extrusão
	\item Tração $\implies$ estiramento, dobramento e cisalhamento ou corte
\end{itemize}

Conformação a frio se dá entre 0 e 0,3 da temperatura de fusão do material (escala Kelvin) e ocorre o encruamento, morno entre 0,3 e 0,5 e acontece também a recuperação, e a quente entre 0,5 até a temperatura de fusão e acontece também a recuperação e recristalização.

Também classificáveis pela temperatura do processo. A frio, gera deformação dos grãos nos sentidos da ação mecânica, gera um material mais resistente, mais duro, mas pouco dúctil. A quente, há movimentação atômica e os grãos voltam à situação anterior, aumentando ductibilidade mas piorando o acabamento superficial (alta rugosidade).

Solução: deformação a frio e recozimento.

