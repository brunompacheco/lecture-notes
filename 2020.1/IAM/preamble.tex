% Some basic packages
\usepackage[utf8]{inputenc}
\usepackage[T1]{fontenc}
\usepackage{textcomp}
\usepackage[portuguese]{babel}
\usepackage{url}
\usepackage{graphicx}
\usepackage{float}
\usepackage{booktabs}
\usepackage{enumitem}

\pdfminorversion=7

% reduces left margin
\usepackage[a4paper,lmargin=0.5in, rmargin=1in]{geometry}

% slide settings
\usepackage{xparse}
\setlength\columnsep{20pt} % This is the default columnsep for all pages
\newcommand*{\slidesfile}{filename-not-set}
\def\isfirstslide{1}
\newcommand{\insertslide}[1]{
\begin{wrapfigure}{l}{0.6\linewidth}
    \centering
    \fbox{\includegraphics[page=#1,width=\linewidth]{\slidesfile}}
\end{wrapfigure}
}
\newcount\currslidenum
\newcommand{\newslide}[2][]{
    \ifthenelse{\equal{\isfirstslide}{1}}{}{\clearpage}
    \def\isfirstslide{0}
    \currslidenum=#2
    \section*{\ifthenelse{\equal{#1}{}}{Slide #2}{#1}}
    \hrule
    \insertslide{#2}
}
\newcommand{\nextslide}[1][]{
    \advance \currslidenum by 1
    \ifthenelse{\equal{#1}{}}{%
    \newslide{\the\currslidenum}}{%
    \newslide[#1]{\the\currslidenum}}
}
\newcommand{\insertminislides}[2][0]{
    \begin{wrapfigure}{l}{0.6\linewidth}
	\fbox{\includegraphics[page=#2,width=\linewidth]{\slidesfile}}
	\ifnum#1>0 %
	    \begin{subfigure}{\linewidth}
		% TODO: fix figs alignment
		\ifnum#1<3 %
		    \def\subfigwidth{.48\linewidth}
		\else
		    \def\subfigwidth{.3\linewidth}
		\fi

		\newcount\minislides
		\minislides=#2
		\advance \minislides by 1
		\newcount\looplimit
		\looplimit=#2
		\advance \looplimit by #1
		\advance \looplimit by 1

		\loop
		\fbox{\includegraphics[page=\the\minislides,width=\subfigwidth]{\slidesfile}}
		\advance \minislides by 1
		\ifnum \minislides<\looplimit
		\repeat
	    \end{subfigure}
	\fi
    \end{wrapfigure}
}
\newcommand{\insertmultipleslides}[3][]{
    \begin{wrapfigure}{l}{.6\linewidth}
	\ifthenelse{\equal{#2}{#3}}{
	    \fbox{\includegraphics[page=#3,width=\linewidth]{\slidesfile}}
	}{
	    % sets highlighted slide
	    \newcount\hlslide
	    \ifthenelse{\equal{#1}{}}{%
		\hlslide=#2
	    }{%
		\hlslide=#1
	    }
	    \fbox{\includegraphics[page=\the\hlslide,width=\linewidth]{\slidesfile}}

	    \begin{subfigure}{\linewidth}
		\newcount\currminislide
		\currminislide=#2
		\newcount\lastminislide
		\lastminislide=#3
		\newcount\nminislides
		\nminislides=\lastminislide
		\advance \nminislides -\currminislide
		\advance \nminislides by 1

		\ifnum \nminislides < 3 %
		    \def\subfigwidth{.48\linewidth}
		\else
		    \def\subfigwidth{.3\linewidth}
		\fi
		
		\advance \lastminislide by 1
		\loop
		\fbox{\includegraphics[page=\the\currminislide,width=\subfigwidth]{\slidesfile}}
		\advance \currminislide by 1
		\ifnum \currminislide < \lastminislide
		\repeat
	        
	    \end{subfigure}
	}
    \end{wrapfigure}
}

\usepackage{keycommand}

\newkeycommand*\nextslides[title,slide,until,highlight]{
    \newcount\startslide
    \ifthenelse{\equal{\commandkey{slide}}{}}{
	\advance \currslidenum by 1
	\startslide=\the\currslidenum
    }{
	\startslide=\commandkey{slide}
    }
    \ifthenelse{\equal{\commandkey{until}}{}}{
	\currslidenum=\the\startslide
    }{
	\currslidenum=\commandkey{until}
    }

    \ifthenelse{\equal{\isfirstslide}{1}}{}{\clearpage}
    \def\isfirstslide{0}
    
    \section*{\ifthenelse{\equal{\commandkey{title}}{}}{%
	Slide \ifthenelse{\equal{\commandkey{highlight}}{}}{%
	    \the\startslide}{\commandkey{highlight}
	}
    }{%
	\commandkey{title}
    }}
    \hrule

    \ifthenelse{\equal{\commandkey{until}}{}}{
	\insertmultipleslides[\commandkey{highlight}]{\the\startslide}{\the\startslide}
    }{
	\insertmultipleslides[\commandkey{highlight}]{\the\startslide}{\commandkey{until}}
    }
}


% Don't indent paragraphs, leave some space between them
\usepackage{parskip}

% Hide page number when page is empty
\usepackage{emptypage}
\usepackage{subcaption}
\usepackage{multicol}
\usepackage{xcolor}

% wrap text around figures
\usepackage{wrapfig}

% Other font I sometimes use.
% \usepackage{cmbright}

% Math stuff
\usepackage{amsmath, amsfonts, mathtools, amsthm, amssymb}
% Fancy script capitals
\usepackage{mathrsfs}
\usepackage{cancel}
% Bold math
\usepackage{bm}
% Some shortcuts
\newcommand\N{\ensuremath{\mathbb{N}}}
\newcommand\R{\ensuremath{\mathbb{R}}}
\newcommand\Z{\ensuremath{\mathbb{Z}}}
\renewcommand\O{\ensuremath{\emptyset}}
\newcommand\Q{\ensuremath{\mathbb{Q}}}
\newcommand\C{\ensuremath{\mathbb{C}}}
\renewcommand\L{\ensuremath{\mathcal{L}}}

% Easily typeset systems of equations (French package)
\usepackage{systeme}

% Put x \to \infty below \lim
\let\svlim\lim\def\lim{\svlim\limits}

%Make implies and impliedby shorter
\let\implies\Rightarrow
\let\impliedby\Leftarrow
\let\iff\Leftrightarrow
\let\epsilon\varepsilon

% Add \contra symbol to denote contradiction
\usepackage{stmaryrd} % for \lightning
\newcommand\contra{\scalebox{1.5}{$\lightning$}}

% \let\phi\varphi

% Command for short corrections
% Usage: 1+1=\correct{3}{2}

\definecolor{correct}{HTML}{009900}
\newcommand\correct[2]{\ensuremath{\:}{\color{red}{#1}}\ensuremath{\to }{\color{correct}{#2}}\ensuremath{\:}}
\newcommand\green[1]{{\color{correct}{#1}}}

% horizontal rule
\newcommand\hr{
    \noindent\rule[0.5ex]{\linewidth}{0.5pt}
}

% hide parts
\newcommand\hide[1]{}

% si unitx
\usepackage{siunitx}
\sisetup{locale = FR}

% Environments
\makeatother
% For box around Definition, Theorem, \ldots
\usepackage{mdframed}
\mdfsetup{skipabove=1em,skipbelow=0em}
\theoremstyle{definition}
\newmdtheoremenv[nobreak=true]{definitie}{Definitie}
\newmdtheoremenv[nobreak=true]{eigenschap}{Eigenschap}
\newmdtheoremenv[nobreak=true]{gevolg}{Gevolg}
\newmdtheoremenv[nobreak=true]{lemma}{Lemma}
\newmdtheoremenv[nobreak=true]{propositie}{Propositie}
\newmdtheoremenv[nobreak=true]{stelling}{Stelling}
\newmdtheoremenv[nobreak=true]{wet}{Wet}
\newmdtheoremenv[nobreak=true]{postulaat}{Postulaat}
\newmdtheoremenv{conclusie}{Conclusie}
\newmdtheoremenv{toemaatje}{Toemaatje}
\newmdtheoremenv{vermoeden}{Vermoeden}
\newtheorem*{herhaling}{Herhaling}
\newtheorem*{intermezzo}{Intermezzo}
\newtheorem*{notatie}{Notatie}
\newtheorem*{observatie}{Observatie}
\newtheorem*{exe}{Exercise}
\newtheorem*{opmerking}{Opmerking}
\newtheorem*{praktisch}{Praktisch}
\newtheorem*{probleem}{Probleem}
\newtheorem*{terminologie}{Terminologie}
\newtheorem*{toepassing}{Toepassing}
\newtheorem*{uovt}{UOVT}
\newtheorem*{vb}{Voorbeeld}
\newtheorem*{vraag}{Vraag}

\newmdtheoremenv[nobreak=true]{definition}{Definition}
\newtheorem*{eg}{Example}
\newtheorem*{notation}{Notation}
\newtheorem*{previouslyseen}{As previously seen}
\newtheorem*{remark}{Remark}
\newtheorem*{note}{Note}
\newtheorem*{problem}{Problem}
\newtheorem*{observe}{Observe}
\newtheorem*{property}{Property}
\newtheorem*{intuition}{Intuition}
\newmdtheoremenv[nobreak=true]{prop}{Proposition}
\newmdtheoremenv[nobreak=true]{theorem}{Theorem}
\newmdtheoremenv[nobreak=true]{corollary}{Corollary}

% End example and intermezzo environments with a small diamond (just like proof
% environments end with a small square)
\usepackage{etoolbox}
\AtEndEnvironment{vb}{\null\hfill$\diamond$}%
\AtEndEnvironment{intermezzo}{\null\hfill$\diamond$}%
% \AtEndEnvironment{opmerking}{\null\hfill$\diamond$}%

% Fix some spacing
% http://tex.stackexchange.com/questions/22119/how-can-i-change-the-spacing-before-theorems-with-amsthm
\makeatletter
\def\thm@space@setup{%
  \thm@preskip=\parskip \thm@postskip=0pt
}


% Exercise 
% Usage:
% \exercise{5}
% \subexercise{1}
% \subexercise{2}
% \subexercise{3}
% gives
% Exercise 5
%   Exercise 5.1
%   Exercise 5.2
%   Exercise 5.3
\newcommand{\exercise}[1]{%
    \def\@exercise{#1}%
    \subsection*{Exercise #1}
}

\newcommand{\subexercise}[1]{%
    \subsubsection*{Exercise \@exercise.#1}
}


% \lecture starts a new lecture (les in dutch)
%
% Usage:
% \lecture{1}{di 12 feb 2019 16:00}{Inleiding}
%
% This adds a section heading with the number / title of the lecture and a
% margin paragraph with the date.

% I use \dateparts here to hide the year (2019). This way, I can easily parse
% the date of each lecture unambiguously while still having a human-friendly
% short format printed to the pdf.

\usepackage{xifthen}
\def\testdateparts#1{\dateparts#1\relax}
\def\dateparts#1 #2 #3 #4 #5\relax{
    \marginpar{\small\textsf{\mbox{#1 #2 #3 #5}}}
}

\def\@lecture{}%
\newcommand{\lecture}[4][]{
    \ifthenelse{\isempty{#1}}{}{
	\clearpage
	\renewcommand\slidesfile{#1}
	\def\isfirstslide{1}
	\currslidenum=0
    }
    \ifthenelse{\isempty{#4}}{%
        \def\@lecture{Aula #2}%
    }{%
        \def\@lecture{Aula #2: #4}%
    }%
    \subsection{\@lecture}
    \marginpar{\small\textsf{\mbox{#3}}}
}



% These are the fancy headers
\usepackage{fancyhdr}
\pagestyle{fancy}

% LE: left even
% RO: right odd
% CE, CO: center even, center odd
% My name for when I print my lecture notes to use for an open book exam.
% \fancyhead[LE,RO]{Gilles Castel}

\fancyhead[RO,LE]{\@lecture} % Right odd,  Left even
\fancyhead[RE,LO]{}          % Right even, Left odd

\fancyfoot[RO,LE]{\thepage}  % Right odd,  Left even
\fancyfoot[RE,LO]{}          % Right even, Left odd
\fancyfoot[C]{\leftmark}     % Center

\makeatother




% Todonotes and inline notes in fancy boxes
\usepackage{todonotes}
\usepackage{tcolorbox}

% Make boxes breakable
\tcbuselibrary{breakable}

% Verbetering is correction in Dutch
% Usage: 
% \begin{verbetering}
%     Lorem ipsum dolor sit amet, consetetur sadipscing elitr, sed diam nonumy eirmod
%     tempor invidunt ut labore et dolore magna aliquyam erat, sed diam voluptua. At
%     vero eos et accusam et justo duo dolores et ea rebum. Stet clita kasd gubergren,
%     no sea takimata sanctus est Lorem ipsum dolor sit amet.
% \end{verbetering}
\newenvironment{verbetering}{\begin{tcolorbox}[
    arc=0mm,
    colback=white,
    colframe=green!60!black,
    title=Opmerking,
    fonttitle=\sffamily,
    breakable
]}{\end{tcolorbox}}

% Noot is note in Dutch. Same as 'verbetering' but color of box is different
\newenvironment{noot}[1]{\begin{tcolorbox}[
    arc=0mm,
    colback=white,
    colframe=white!60!black,
    title=#1,
    fonttitle=\sffamily,
    breakable
]}{\end{tcolorbox}}




% Figure support as explained in my blog post.
\usepackage{import}
\usepackage{xifthen}
\usepackage{pdfpages}
\usepackage{transparent}
\newcommand{\incfig}[1]{%
    \def\svgwidth{\columnwidth}
    \import{./figures/}{#1.pdf_tex}
}

% Fix some stuff
% %http://tex.stackexchange.com/questions/76273/multiple-pdfs-with-page-group-included-in-a-single-page-warning
\pdfsuppresswarningpagegroup=1


% My name
\author{Bruno M. Pacheco}

% Fixes sections starting at 0
\setcounter{chapter}{1}% Not using chapters, but they're used in the counters

% Hides the numbers of sections but keeps them at the toc
\setcounter{secnumdepth}{0}
