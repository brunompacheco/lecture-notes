\lecture[presentations/lecture_14_uncertainty.pdf]{14}{19.06.2020}{Uncertainty}

\nextslides[slide=4] 

\nextslides

\nextslides

\nextslides

\nextslides

\nextslides

Indeterminate event is recorded as normal, but we are not sure whether it happened or didn't happen.

\nextslides[until=11]

$N(t,2)$ is a normal distribution centered at timestamp $t$ with variance 2.

\nextslides

Highlight to Bayesian approach, which distinguishes this from the frequentist approach!

\nextslides

Remember that the weight we give to the uncertain is a degree of belief!

\nextslides[slide=15] 

\nextslides[until=18, highlight=16] 

Recall alignments: alignments are a method to measure the conformance of a given trace with a given model.

\nextslides

Note that $\left< A,B,C,D,E \right>$ is valid because of the uncertainty of the timestamp!

\nextslides

The "optimized" procedures are for bounds, the real value is through conformance checking/alignment calculation.

\nextslides

\nextslides

\nextslides

\nextslides[slide=27] 

\nextslides

\nextslides

\nextslides

\nextslides[until=34, highlight=33] 

These are simulated settings to validate the hypothesis, previously formulated as questions.

Note: p is a probability reference.

\nextslides[until=36, highlight=35] 

\nextslides[slide=38]

This will be a deeper explanation of the graph synthesis from the previous slides.

\nextslides

\nextslides

\nextslides[slide=42,until=44,highlight=44]

\nextslides

That is, the very first slides that we add to the graph.

\nextslides[until=49, highlight=49]

We create two entries for each timestamp boundary, marking with subscript L and R.

\nextslides

\nextslides[until=70,highlight=51]

\nextslides[until=75] 

Formal arguments for the correctness of the algorithm.

From my understanding, the search stops as soon as it finds a certain successor (either the end of an uncertain or a certain) because it wouldn't be possible to connect to anything after that, as it is certain that a following event already happened.

\nextslides[until=80] 

Upper and lower complexity bounds. Both quadratic, aka $\Theta(n^{2})$.


\nextslides[until=82] 

\nextslides[slide=84] 
