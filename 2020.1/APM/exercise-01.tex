\documentclass[a4paper]{report}
\input{./preamble.tex}

\title{Exercises 1}

\begin{document}
\maketitle

\section*{Functions}

\exercise{0}

\[
	f_2 = \left\{ \left( p_1,t_1 \right), \left( p_4,t_1 \right), \left( p_2,t_2 \right),\left( p_3,t_2 \right), \left( p_4,t_3 \right), \left( p_6,t_3 \right) ,\left( p_5,t_4 \right) , \left( p_3,t_4 \right)   \right\} 
.\] 

\exercise{1}

\subexercise{a)} $f_1$ is \textbf{Injective}
\subexercise{b)} $f_3$ is just a function once $f_3(2) = f_3(-2)$ and $f_3(x) \ge 0 \forall x\in \R$
\subexercise{a)} $f_4$ is \textbf{Surjective} once $\forall y\in \R^{+}, f_4^{-1}(y) = |\sqrt{y}| $
\subexercise{a)}

$f_5$ is \textbf{Bijective}

See that $f_5$ maps odds and evens in non-overlapping ranges and, within these cases, it is invertible.

\section*{Basic Petri Nets}

\exercise{2}

\subexercise{1}

\subexercise{1.a)}

Markings after each transition:
\begin{enumerate}
	\item $\left[ p_1,p_2 \right] $
	\item $\left[ p_2,p_3 \right] $
\end{enumerate}
As $\cdot e = \left[ p_1,p_2 \right]$, this is not possible. There is no combination of places to be removed that would make this firing sequence execultable as it is impossible by the execution of the transition $b$, but it would be possible if the transition  $a$ was executed twice.

\subexercise{1.b)}

Markings after each transition:
\begin{enumerate}
	\item $\left[ p_1,p_2 \right] $
	\item $\left[ p_1^{2}, p_2^{2} \right] $ 
	\item $\left[ p_1,p_2,p_3,p_4 \right] $
	\item $\left[ p_1,p_3,p_4^{2} \right] $
	\item $\left[ p_1,p_4,p_5 \right] $
\end{enumerate}

\subexercise{1.c)}

Markings after each transition:
\begin{enumerate}
	\item $\left[  p_1,p_2\right] $
	\item $\left[  p_3,p_4\right] $
	\item $\left[ p_5 \right] $
\end{enumerate}
The last $d$ transition is not possible. No places to be removed, as neither of its $\cdot d$ has tokens by its second activation.

\subexercise{2}
\subexercise{2.a)}

Marking after each transition:
\begin{enumerate}
	\item $\left[ p_1,p_2\right] $
	\item $\left[ p_1,p_4\right] $
	\item $\left[ p_1\right] $
\end{enumerate}
Transition $c$ is not possible, it would be solved through the removal of $f$.

\subexercise{2.b)}

Marking after each transition:
\begin{enumerate}
	\item $\left[ p_1,p_2\right] $
	\item $\left[ p_1,p_4\right] $
	\item $\left[ p_1, p_5\right] $
	\item $\left[ p_1,p_2\right] $
\end{enumerate}

\subexercise{3}
\subexercise{3.a)}
Marking after each transition:
\begin{enumerate}
	\item $\left[ p_1\right] $
	\item $\left[ p_2\right] $
	\item $\left[ p_3\right] $
\end{enumerate}

\subexercise{3.b)}

Marking after each transition:
\begin{enumerate}
	\item $\left[ p_1\right] $
\end{enumerate}
An extra token at $p_0$ would make it executable.

\subexercise{4}
\subexercise{4.a)}
Marking after each transition:
\begin{enumerate}
	\item $\left[ p_1\right] $
	\item $\left[ p_2\right] $
	\item $\left[ \O\right] $
\end{enumerate}
\subexercise{4.b)}
Marking after each transition:
\begin{enumerate}
	\item $\left[ p_1\right] $
	\item $\left[ p_1^{2}\right] $
	\item $\left[ p_1,p_2\right] $
	\item $\left[ p_1\right] $
\end{enumerate}
Transition $e$ is not possible, but it would be if transition $c$ was fired a second time before it.

\subexercise{4.c)}
Marking after each transition:
\begin{enumerate}
	\item $\left[ p_1\right] $
	\item $\left[ p_1^{2}\right] $
	\item $\left[ p_1^{3}\right] $
	\item $\left[ p_1^{4}\right] $
	\item $\left[ p_1^{5}\right] $
	\item $\left[ p_1^{6}\right] $
\end{enumerate}

\section*{Petri Net Construction}

\exercise{3}

\begin{figure}[H]
    \centering
    \includegraphics[width=1\textwidth]{figures/ex1_3.png}
    \caption{3) Attending APM Petri Net}
    \label{fig:ex1_3}
\end{figure}

\section*{Advanced Petri Nets}

\exercise{4}

\subexercise{1}

\begin{figure}[H]
    \centering
    \includegraphics[width=0.8\textwidth]{figures/ex1_4_1_a.pdf}
    \caption{4.1a)}
    \label{fig:figures-ex1_4_a-pdf}
\end{figure}

\begin{figure}[H]
    \centering
    \includegraphics[width=0.8\textwidth]{figures/ex1_4_1_b.pdf}
    \caption{4.1b) Traffic Light Free-choice PN}
    \label{fig:ex1_4_1_b}
\end{figure}

\subexercise{2}

\begin{figure}[H]
    \centering
    \includegraphics[width=0.8\textwidth]{figures/ex1_4_2_a.pdf}
    \caption{4.2a)}
    \label{fig:figures-ex1_4_2_a-pdf}
\end{figure}

\begin{figure}[H]
    \centering
    \includegraphics[width=0.8\textwidth]{figures/ex1_4_2_b.pdf}
    \caption{4.2b)}
    \label{fig:ex1_4_2_b-pdf}
\end{figure}

\subexercise{3}

\subexercise{3.a)}

We have only two transitions in the net with a positive outcome of tokens (that
generate more tokens than it consumes). Given that these tokens are always generated in different
places and that this transitions are always followed by negative outcome transitions with the same
absolute outcome (+1/-1), we have that $k=1$.

\subexercise{3.b)}

Yes, giving that there is only one sequence of transitions possible and that it is a cyclic PN, it
is live.  $\left< t_4,t_7, t_6, t_5, t_0, t_3, t_2, t_1, t_4 \right>$

\subexercise{3.c)}

Yes, given the answer above.

\end{document}
