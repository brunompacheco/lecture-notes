\documentclass[a4paper]{report}
\input{./preamble.tex}

\title{Exercises 1}

\begin{document}
\maketitle

\section*{Functions}

\exercise{0}

\[
	f_2 = \left\{ \left( p_1,t_1 \right), \left( p_4,t_1 \right), \left( p_2,t_2 \right),\left( p_3,t_2 \right), \left( p_4,t_3 \right), \left( p_6,t_3 \right) ,\left( p_5,t_4 \right) , \left( p_3,t_4 \right)   \right\} 
.\] 

\exercise{1}

\subexercise{a)} $f_1$ is \textbf{Injective}
\subexercise{b)} $f_3$ is just a function once $f_3(2) = f_3(-2)$ and $f_3(x) \ge 0 \forall x\in \R$
\subexercise{a)} $f_4$ is \textbf{Surjective} once $\forall y\in \R^{+}, f_4^{-1}(y) = |\sqrt{y}| $
\subexercise{a)}

$f_5$ is \textbf{Bijective}

See that $f_5$ maps odds and evens in non-overlapping ranges and, within these cases, it is invertible.

\section*{Basic Petri Nets}

\exercise{2}

\subexercise{1}

\subexercise{1.a)}

Markings after each transition:
\begin{enumerate}
	\item $\left[ p_1,p_2 \right] $
	\item $\left[ p_2,p_3 \right] $
\end{enumerate}
As $\cdot e = \left[ p_1,p_2 \right]$, this is not possible. There is no combination of places to be removed that would make this firing sequence execultable as it is impossible by the execution of the transition $b$, but it would be possible if the transition  $a$ was executed twice.

\subexercise{1.b)}

Markings after each transition:
\begin{enumerate}
	\item $\left[ p_1,p_2 \right] $
	\item $\left[ p_1^{2}, p_2^{2} \right] $ 
	\item $\left[ p_1,p_2,p_3,p_4 \right] $
	\item $\left[ p_1,p_3,p_4^{2} \right] $
	\item $\left[ p_1,p_4,p_5 \right] $
\end{enumerate}

\subexercise{1.c)}

Markings after each transition:
\begin{enumerate}
	\item $\left[  p_1,p_2\right] $
	\item $\left[  p_3,p_4\right] $
	\item $\left[ p_5 \right] $
\end{enumerate}
The last $d$ transition is not possible. No places to be removed, as neither of its $\cdot d$ has tokens by its second activation.

\subexercise{2}
\subexercise{2.a)}

Marking after each transition:
\begin{enumerate}
	\item $\left[ p_1,p_2\right] $
	\item $\left[ p_1,p_4\right] $
	\item $\left[ p_1\right] $
\end{enumerate}
Transition $c$ is not possible, it would be solved through the removal of $f$.

\subexercise{2.b)}

Marking after each transition:
\begin{enumerate}
	\item $\left[ p_1,p_2\right] $
	\item $\left[ p_1,p_4\right] $
	\item $\left[ p_1, p_5\right] $
	\item $\left[ p_1,p_2\right] $
\end{enumerate}

\subexercise{3}
\subexercise{3.a)}
Marking after each transition:
\begin{enumerate}
	\item $\left[ p_1\right] $
	\item $\left[ p_2\right] $
	\item $\left[ p_3\right] $
\end{enumerate}

\subexercise{3.b)}

Marking after each transition:
\begin{enumerate}
	\item $\left[ p_1\right] $
\end{enumerate}
An extra token at $p_0$ would make it executable.

\subexercise{4}
\subexercise{4.a)}
Marking after each transition:
\begin{enumerate}
	\item $\left[ p_1\right] $
	\item $\left[ p_2\right] $
	\item $\left[ \O\right] $
\end{enumerate}
\subexercise{4.b)}
Marking after each transition:
\begin{enumerate}
	\item $\left[ p_1\right] $
	\item $\left[ p_1^{2}\right] $
	\item $\left[ p_1,p_2\right] $
	\item $\left[ p_1\right] $
\end{enumerate}
Transition $e$ is not possible, but it would be if transition $c$ was fired a second time before it.

\subexercise{4.c)}
Marking after each transition:
\begin{enumerate}
	\item $\left[ p_1\right] $
	\item $\left[ p_1^{2}\right] $
	\item $\left[ p_1^{3}\right] $
	\item $\left[ p_1^{4}\right] $
	\item $\left[ p_1^{5}\right] $
	\item $\left[ p_1^{6}\right] $
\end{enumerate}

\section*{Petri Net Construction}

\ctikzfig{figures/ex1_3}

\begin{tikzpicture}[node distance=1.3cm,>=stealth',bend angle=45]
	\tikzstyle{place}=[fill=white, draw=black, shape=circle, minimum size=6mm, node distance=1.3cm]
	\tikzstyle{transition}=[fill=white, draw=black, shape=rectangle, minimum size=4mm, node distance=1.3cm]

	\node [style=place] (0) at (2, 0) {};
	\node [style=place] (2) at (0, 0) {};
	\node [style=place] (3) at (5, 0) {};
	\node [style=transition, label=registering] (4) at (-1, 0) {};
	\node [style=transition, label=intro] (5) at (1, 0) {};
	\node [style=transition, label=postpone] (6) at (3, 0) {};
	\node [style=transition, label=left:{stay tuned}] (7) at (2, -1) {};
	\node [style=transition, label=enjoy] (8) at (6, 1) {};
	\node [style=transition, label=above:{focus on other lec.}] (9) at (4, 1) {};
	\node [style=place] (11) at (7, -2) {};
	\node [style=transition, label=hurry] (12) at (6, 0) {};
	\node [style=place] (13) at (9.5, -2) {};
	\node [style=transition, label=above:panic] (14) at (8.25, -2) {};
	\node [style=transition, label=right:{exam}] (15) at (7, -3) {};
	\node [style=place] (16) at (6, -3) {};
	\node [style=place] (17) at (2, -3) {};
	\node [style=transition, label=recent lecture] (18) at (1, -2) {};
	\node [style=place] (19) at (0, -2) {};
	\node [style=transition, label=left:{recent instruction}] (20) at (-1, -2.5) {};
	\node [style=place] (21) at (0, -3) {};
	\node [style=transition, label=excited] (22) at (1, -3) {};
	\node [style=transition, label={below:prepare relaxed}] (23) at (3, -3) {};
	\node [style=place] (24) at (4, -3) {};
	\node [style=transition, label=below:{exam}] (25) at (5, -3) {};
	\node [style=place] (26) at (6, -1) {};
	\node [style=place] (27) at (6, -2) {};
	\node [style=transition, label={below:A2}] (28) at (5, -2) {};
	\node [style=transition, label={below:A1}] (29) at (5, -1) {};
	\node [style=place] (30) at (4, -0.5) {};
	\node [style=place] (31) at (3, -1) {};
	\node [style=place] (32) at (7, 0) {};
	\node [style=transition, label=above:{check instruction}] (33) at (7, -1) {};
	\node [style=transition, label=above:{watch lecture}] (34) at (9.5, -1) {};

	\draw [style=edge] (4) to (2);
	\draw [style=edge] (2) to (5);
	\draw [style=edge] (5) to (0);
	\draw [style=edge] (0) to (6);
	\draw [style=edge] (0) to (7);
	\draw [style=edge] (6) to (3);
	\draw [style=edge] (3) to (8);
	\draw [style=edge] (9) to (3);
	\draw [style=edge] (11) to (15);
	\draw [style=edge] (13) to (15);
	\draw [style=edge] (15) to (16);
	\draw [style=edge] (7) to (17);
	\draw [style=edge] (17) to (18);
	\draw [style=edge] (18) to (19);
	\draw [style=edge] (19) to (20);
	\draw [style=edge] (20) to (21);
	\draw [style=edge] (21) to (22);
	\draw [style=edge] (22) to (17);
	\draw [style=edge] (17) to (23);
	\draw [style=edge] (23) to (24);
	\draw [style=edge] (24) to (25);
	\draw [style=edge] (25) to (16);
	\draw [style=edge] (3) to (9);
	\draw [style=edge] (8) to (3);
	\draw [style=edge] (3) to (12);
	\draw [style=edge] (29) to (26);
	\draw [style=edge] (26) to (28);
	\draw [style=edge] (28) to (27);
	\draw [style=edge] (27) to (15);
	\draw [style=edge] (3) to (29);
	\draw [style=edge] (30) to (29);
	\draw [style=edge] (29) to (3);
	\draw [style=edge] (6) to (30);
	\draw [style=edge] (17) to (29);
	\draw [style=edge] (29) to (17);
	\draw [style=edge] (7) to (31);
	\draw [style=edge] (31) to (29);
	\draw [style=edge] (27) to (25);
	\draw [style=edge] (32) to (33);
	\draw [style=edge] (12) to (32);
	\draw [style=edge] (32) to (34);
	\draw [style=edge] (34) to (13);
	\draw [style=edge] (33) to (11);
	\draw [style=edge] (11) to (14);
	\draw [style=edge] (13) to (14);
	\draw [style=edge] (14) to (32);
\end{tikzpicture}

\end{document}
