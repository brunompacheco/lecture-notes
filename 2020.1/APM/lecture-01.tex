\lecture{1}{13.04.2020}{Introduction to Process Mining}


\section*{Overview}

\subsection*{Definitions}

About Process Mining.

\begin{definition}
	A Process is a discrete collection of activities executed to achieve a goal.
\end{definition}

\begin{definition}
	Mining is ``Gaining knowledge of, and, insights in (business) processes by analysing the event data stored during execution of the process''.
\end{definition}

Basic process data example:

\begin{table}[htpb]
	\centering
	\caption{Basic example of event data}
	\label{tab:event-data}
	\begin{tabular}{|c|c|c|}
		\hline
		package & task & timestamp \\
		\hline
		1eg081hr  & load      & 12:36 \\
		1eg081hr  & dispatch  & 12:38 \\
		41yp39he  & load      & 12:37 \\
		\hline
	\end{tabular}
\end{table}

The `package' column is the \emph{Case identifier}, it identifies the instance of the process. `task' is the \emph{Activity} column, it explains what is being performed at a data record (row). `timestamp' is the \emph{Timestamp} column, a time reference to the data record. The last one is not necessary and can come in several different ways, even sometimes having record for beginning and end.

We can also have different data in a log, but these are the bare basics.

\begin{definition}
	An \emph{Event} captures the (partial) execution of an activity within a process instance.
\end{definition}

Events are the atoms of our data, the rows in the dataset.

\begin{definition}
	A \emph{Trace} is a collection of events related to a process instance.
\end{definition}

We can make traces by the \emph{Case identifier} reference.

\subsection*{Process Mining - Analysis}

There are three branches of Process Mining:

\begin{definition}
	\emph{Process Discovery} is deriving process models from process logs.
\end{definition}

\begin{definition}
	\emph{Conformance Checking} is assessing if reality in the data is conforming to specification (from a process model).
\end{definition}

\begin{definition}
	\emph{Process Enhancement} is providing performance information to a process model (as an overlay).
\end{definition}

Process discovery acts from event logs, in the model. Conformance checking acts between the model and the event logs, acting on both. Enhancement acts from logs on the models, with a feedback from the model.
