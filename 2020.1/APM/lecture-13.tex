\lecture[presentations/lecture_13_privacy_preservation.pdf]{13}{18.06.2020}{Privacy Preservation}

\nextslides[slide=2] 

Care for people's right

\nextslides

This presentation will focus in confidentiality.

\nextslides

First more related to cryptography, not the focus here.

\nextslides[slide=6] 

The worst case lies when purely out of the identifiers it is possible to identify the individuals, which is usually not the case as we often just use incremental identifiers.

\nextslides[slide=9] 

PPDP: you have sensitive data, but you can't remove it because it is required for the analysis. So the task is too protect this information from the publishing.

\nextslides

We are narrowing down our scope of work for the lecture.

\nextslides

\nextslides[slide=13] 

\nextslides

\nextslides

\nextslides

\emph{k-Anonimity} does not refer to the operation, but to the goal specification. For each value combination of the QIDs, there must be at least $k$ records in the table.

\nextslides

It refers to the diversity of the sensitive attributes in regard to the QIDs combinations.

\nextslides

\nextslides

For the trace condition, PPDP in PM is considered a multi-dimensional problem.

\nextslides

\nextslides

High variability: parallelism, looping, low "representativeness" (?).

\nextslides[until=26] 

These are the types of \emph{a priori} information that the attacker can have.

\nextslides[slide=27] 

BK = Background Knowledge.

\nextslides

\nextslides

\nextslides[until=31] 

\nextslides

\nextslides

\nextslides

Advantage of MVT is that they are much less numerous than the violating traces.

\nextslides

\nextslides

These are measures to prioritize events to be removed (less loss, more privacy gain).

\nextslides[until=44] 

It looks like a tree search, but no idea.

You do not only remove the event/trace from the tree, but the whole branch!

\nextslides

So it is better than k-anonimity because it preserves more information in the event log.

\nextslides


