\lecture{4}{20.04.2020}{Basic Discovery and Inductive Miner}

\section*{Basic Process Discovery}

Reminder: Process Discovery is translating event logs into process models.

\subsection*{Transition relationships}

Alpha algorithm is based on a few relationships that can be extracted from an event log.

\begin{description}
    \item[Direct succession] $x > y \iff$ for some case, $x$ is directly followed by $y$.
\end{description}

\emph{E.g.},

Given an event log like
\[
L_1=\left[ \left<a,b,c,d \right>^3, \left<a,c,b,d \right>^2, \left<a,e,d \right> \right] 
\] 
We can obtain the following set of relations
\[
\left\{ a>b,a>c,a>e,b>c,b>d,c>b,c>d,e>d  \right\}
\] 

\begin{description}
    \item[Causality] $x\to y \iff x>y \land \neg y>x$
\end{description}

In our example, then, the causality relations are
\[
\left\{ a\to b, a\to c, a\to e, b\to d, c\to d, e\to d \right\} 
\] 

\begin{description}
    \item[Parallel] $x||y \iff x>y \land y>x$
\end{description}

In our example, then, the parallel relations are
\[
\left\{ b||c, c||b \right\} 
\] 

\begin{description}
    \item[Choice] $x\#y \iff \neg x>y \land \neg y>x$
\end{description}

These relations are all that do not showed up before.

\begin{remark}
    The group $Choice(A)$ of choice relationships can be defined as  $Choice(A) = DS(A) \ A\times
    A$, of the $DS(A)$ set of direct succession relationships.
\end{remark}

\subsection*{Patterns}

Patterns that can occur in the net. Given transitions $a, b$ and $c$,
\begin{description}
    \item[Sequence] $a\to b$, when they are directly followed, there is no option for a
    \item[XOR-split] $a\to b, a\to c, b\#c$, when the token has two options after $a$: $b$ XOR $c$ 
    \item[XOR-join] $b\to c, a\to c, a\#c$
    \item[AND-split] $a\to b, a\to c, b||c$
\end{description}

\subsection*{Alpha Miner}

So in our example, we can use the relations and the patterns to build the net seen in slide 49.

We start by building a matrix with the relations described before, having the transitions as
indexes, with the addition of the $\gets = \to^{-1} $ relation. Notice this is actually an
upper-triangular matrix, once it is symmetric along the diagonal.

\begin{remark}
    we connect a set of transitions $X$ with a set of transitions $Y$ if every  $x,x' \in X$ has
    $x\#x'$, every $y,y'\in Y$ has $y\#y'$ and every $x\in X$ has $x\to y$.
\end{remark}

Notice that $x\neq x'$ is not a requirement, so it applies for \textbf{singleton sets}. So in our
example, $X=\left\{ a \right\} , Y=\left\{ b \right\} $ applies and you can connect the transitions.

The \emph{Alpha Miner} algorithm is based on this premise, but has a few caveats:

\begin{enumerate}
    \item Performs initial trial on the event log and gets the transitions
    \item Gets the \emph{largest} sets $A$ and  $B$ that contain transitions with choice relation within
	the sets, but sequence relations from $A$ to $B$
    \item Generates places for this pair of sets and for the initial and final transitions (acquired
	at step 1)
    \item Connects the place to the transitions in  $A$ and $B$, and to the final and initial places
	to the respective transitions
    \item Repeats steps 3 and 4 for all the $A$ and $B$ possible (keeping the \emph{largest}
	characteristic)
\end{enumerate}

Alpha Miner does not handle self loops! It creates a disconnection. See Alpha+ for a solution (I
guess it filters it at the beginning).

\subsection*{Heuristic Miner}

The \emph{Heuristic Miner} takes on account the frequency of occurrence to create a causal graph.

\[
    x\to y = \frac{|x>y| - |y>x|}{|x>y| + |y>x|+1}
\] 

Upon this, we set a threshold and draw the graph for the relations that are above that value (e.g.,
0.75).

\section*{Inductive Miner Basic Idea}

The original problem is that the Alpha Miner generates very complex and hard to read PN, and the
Heuristic Miner only shows what is on the data, does not generalize enough (?).

The \textbf{Inductive Miner} generates very nice graphs. It actually generates a process tree. It
discovers a process trees from a directly follows graph, which is a graph from the direct succession
relations.

Given an event log $L:A*\to \N$, 
\begin{enumerate}
    \item Gets the direct succession relations and builds a directed graph
    \item Assuming that this graph is generated from a process tree
    \item Recursively splits the graph and finds structures (operations)
\end{enumerate}
