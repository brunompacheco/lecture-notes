\documentclass[a4paper]{report}
% Some basic packages
\usepackage[utf8]{inputenc}
\usepackage[T1]{fontenc}
\usepackage{textcomp}
\usepackage[english]{babel}
\usepackage{url}
\usepackage{graphicx}
\usepackage{float}
\usepackage{booktabs}
\usepackage{enumitem}

\pdfminorversion=7

% Don't indent paragraphs, leave some space between them
\usepackage{parskip}

% Hide page number when page is empty
\usepackage{emptypage}
\usepackage{subcaption}
\usepackage{multicol}
\usepackage{xcolor}

% Other font I sometimes use.
% \usepackage{cmbright}

% Math stuff
\usepackage{amsmath, amsfonts, mathtools, amsthm, amssymb}
% Fancy script capitals
\usepackage{mathrsfs}
\usepackage{cancel}
% Bold math
\usepackage{bm}
% Some shortcuts
\newcommand\N{\ensuremath{\mathbb{N}}}
\newcommand\R{\ensuremath{\mathbb{R}}}
\newcommand\Z{\ensuremath{\mathbb{Z}}}
\renewcommand\O{\ensuremath{\emptyset}}
\newcommand\Q{\ensuremath{\mathbb{Q}}}
\newcommand\C{\ensuremath{\mathbb{C}}}
\renewcommand\L{\ensuremath{\mathcal{L}}}

% Package for Petri Net drawing
\usepackage[version=0.96]{pgf}
\usepackage{tikz}
\usetikzlibrary{arrows,shapes,automata,petri}
\usepackage{tikzit}
\input{petri_nets_style.tikzstyles}

% Easily typeset systems of equations (French package)
\usepackage{systeme}

% Put x \to \infty below \lim
\let\svlim\lim\def\lim{\svlim\limits}

%Make implies and impliedby shorter
\let\implies\Rightarrow
\let\impliedby\Leftarrow
\let\iff\Leftrightarrow
\let\epsilon\varepsilon

% Add \contra symbol to denote contradiction
\usepackage{stmaryrd} % for \lightning
\newcommand\contra{\scalebox{1.5}{$\lightning$}}

% \let\phi\varphi

% Command for short corrections
% Usage: 1+1=\correct{3}{2}

\definecolor{correct}{HTML}{009900}
\newcommand\correct[2]{\ensuremath{\:}{\color{red}{#1}}\ensuremath{\to }{\color{correct}{#2}}\ensuremath{\:}}
\newcommand\green[1]{{\color{correct}{#1}}}

% horizontal rule
\newcommand\hr{
    \noindent\rule[0.5ex]{\linewidth}{0.5pt}
}

% hide parts
\newcommand\hide[1]{}

% si unitx
\usepackage{siunitx}
\sisetup{locale = FR}

% Environments
\makeatother
% For box around Definition, Theorem, \ldots
\usepackage{mdframed}
\mdfsetup{skipabove=1em,skipbelow=0em}
\theoremstyle{definition}
\newmdtheoremenv[nobreak=true]{definitie}{Definitie}
\newmdtheoremenv[nobreak=true]{eigenschap}{Eigenschap}
\newmdtheoremenv[nobreak=true]{gevolg}{Gevolg}
\newmdtheoremenv[nobreak=true]{lemma}{Lemma}
\newmdtheoremenv[nobreak=true]{propositie}{Propositie}
\newmdtheoremenv[nobreak=true]{stelling}{Stelling}
\newmdtheoremenv[nobreak=true]{wet}{Wet}
\newmdtheoremenv[nobreak=true]{postulaat}{Postulaat}
\newmdtheoremenv{conclusie}{Conclusie}
\newmdtheoremenv{toemaatje}{Toemaatje}
\newmdtheoremenv{vermoeden}{Vermoeden}
\newtheorem*{herhaling}{Herhaling}
\newtheorem*{intermezzo}{Intermezzo}
\newtheorem*{notatie}{Notatie}
\newtheorem*{observatie}{Observatie}
\newtheorem*{exe}{Exercise}
\newtheorem*{opmerking}{Opmerking}
\newtheorem*{praktisch}{Praktisch}
\newtheorem*{probleem}{Probleem}
\newtheorem*{terminologie}{Terminologie}
\newtheorem*{toepassing}{Toepassing}
\newtheorem*{uovt}{UOVT}
\newtheorem*{vb}{Voorbeeld}
\newtheorem*{vraag}{Vraag}

\newmdtheoremenv[nobreak=true]{definition}{Definition}
\newtheorem*{eg}{Example}
\newtheorem*{notation}{Notation}
\newtheorem*{previouslyseen}{As previously seen}
\newtheorem*{remark}{Remark}
\newtheorem*{note}{Note}
\newtheorem*{problem}{Problem}
\newtheorem*{observe}{Observe}
\newtheorem*{property}{Property}
\newtheorem*{intuition}{Intuition}
\newmdtheoremenv[nobreak=true]{prop}{Proposition}
\newmdtheoremenv[nobreak=true]{theorem}{Theorem}
\newmdtheoremenv[nobreak=true]{corollary}{Corollary}

% End example and intermezzo environments with a small diamond (just like proof
% environments end with a small square)
\usepackage{etoolbox}
\AtEndEnvironment{vb}{\null\hfill$\diamond$}%
\AtEndEnvironment{intermezzo}{\null\hfill$\diamond$}%
% \AtEndEnvironment{opmerking}{\null\hfill$\diamond$}%

% Fix some spacing
% http://tex.stackexchange.com/questions/22119/how-can-i-change-the-spacing-before-theorems-with-amsthm
\makeatletter
\def\thm@space@setup{%
  \thm@preskip=\parskip \thm@postskip=0pt
}


% Exercise 
% Usage:
% \exercise{5}
% \subexercise{1}
% \subexercise{2}
% \subexercise{3}
% gives
% Exercise 5
%   Exercise 5.1
%   Exercise 5.2
%   Exercise 5.3
\newcommand{\exercise}[1]{%
    \def\@exercise{#1}%
    \subsection*{Exercise #1}
}

\newcommand{\subexercise}[1]{%
    \subsubsection*{Exercise \@exercise.#1}
}


% \lecture starts a new lecture (les in dutch)
%
% Usage:
% \lecture{1}{di 12 feb 2019 16:00}{Inleiding}
%
% This adds a section heading with the number / title of the lecture and a
% margin paragraph with the date.

% I use \dateparts here to hide the year (2019). This way, I can easily parse
% the date of each lecture unambiguously while still having a human-friendly
% short format printed to the pdf.

\usepackage{xifthen}
\def\testdateparts#1{\dateparts#1\relax}
\def\dateparts#1 #2 #3 #4 #5\relax{
    \marginpar{\small\textsf{\mbox{#1 #2 #3 #5}}}
}

\def\@lecture{}%
\newcommand{\lecture}[3]{
    \ifthenelse{\isempty{#3}}{%
        \def\@lecture{Lecture #1}%
    }{%
        \def\@lecture{Lecture #1: #3}%
    }%
    \subsection*{\@lecture}
    \marginpar{\small\textsf{\mbox{#2}}}
}



% These are the fancy headers
\usepackage{fancyhdr}
\pagestyle{fancy}

% LE: left even
% RO: right odd
% CE, CO: center even, center odd
% My name for when I print my lecture notes to use for an open book exam.
% \fancyhead[LE,RO]{Gilles Castel}

\fancyhead[RO,LE]{\@lecture} % Right odd,  Left even
\fancyhead[RE,LO]{}          % Right even, Left odd

\fancyfoot[RO,LE]{\thepage}  % Right odd,  Left even
\fancyfoot[RE,LO]{}          % Right even, Left odd
\fancyfoot[C]{\leftmark}     % Center

\makeatother




% Todonotes and inline notes in fancy boxes
\usepackage{todonotes}
\usepackage{tcolorbox}

% Make boxes breakable
\tcbuselibrary{breakable}

% Verbetering is correction in Dutch
% Usage: 
% \begin{verbetering}
%     Lorem ipsum dolor sit amet, consetetur sadipscing elitr, sed diam nonumy eirmod
%     tempor invidunt ut labore et dolore magna aliquyam erat, sed diam voluptua. At
%     vero eos et accusam et justo duo dolores et ea rebum. Stet clita kasd gubergren,
%     no sea takimata sanctus est Lorem ipsum dolor sit amet.
% \end{verbetering}
\newenvironment{verbetering}{\begin{tcolorbox}[
    arc=0mm,
    colback=white,
    colframe=green!60!black,
    title=Opmerking,
    fonttitle=\sffamily,
    breakable
]}{\end{tcolorbox}}

% Noot is note in Dutch. Same as 'verbetering' but color of box is different
\newenvironment{noot}[1]{\begin{tcolorbox}[
    arc=0mm,
    colback=white,
    colframe=white!60!black,
    title=#1,
    fonttitle=\sffamily,
    breakable
]}{\end{tcolorbox}}




% Figure support as explained in my blog post.
\usepackage{import}
\usepackage{xifthen}
\usepackage{pdfpages}
\usepackage{transparent}
\newcommand{\incfig}[1]{%
    \def\svgwidth{\columnwidth}
    \import{./figures/}{#1.pdf_tex}
}

% Fix some stuff
% %http://tex.stackexchange.com/questions/76273/multiple-pdfs-with-page-group-included-in-a-single-page-warning
\pdfsuppresswarningpagegroup=1


% My name
\author{Bruno M. Pacheco}

 
\begin{document}
 
\title{Lista 2}
\author{Bruno M. Pacheco (16100865)\\
H-Cálculo IV}
 
\maketitle
 
\exercise{1}

\begin{quote}
Sejam $f,g: A\to \R$ integráveis tais que $f\le g$. Defina o conjunto \[
\hat{C} = \left\{ ( x,y ) \in  \R^{n+1}: x\in  A, f( x ) \le  y \le  g( x )  \right\} 
\] Verifique que Ĉ é J-mensurável (em $\R^{n+1}$).
\end{quote}

Primeiro, vemos que $\hat{C}$ é limitado uma vez que $A$ é limitado e $f,g$ são integráveis, logo, sua imagem é limitada. Ou seja, escolha $L\in \R$ tal que 
\begin{align*}
    -L &< \inf\left\{ f\left( x \right) : x\in A \right\} \\
    L&> \sup \left\{ g\left( x \right) : x\in A \right\} 
,\end{align*}
então \[
\hat{C}\subseteq \hat{A} := A\times \left[ -L,L \right] 
.\] 

Agora, veja que, por construção, \[
\partial \hat{C} \subseteq \hat{A} = \left(  \partial A \times \left[ -L,L \right]  \right) \cup \left( int A \times \left[ -L,L \right]  \right) 
.\] 

Tome $z_0=\left( x_0,y_0 \right) \in \partial \hat{C}$. Seja $x_0 \in int A \setminus \left(  D_f \cup D_g\right) $ e suponha $y_0 \in \left( f\left( x_0 \right) , g\left( x_0 \right)  \right) $. Pela continuidade de $f$ e $g$ em $x_0$ podemos escolher $U_{x_0} \subseteq A$ aberto com $x_0\in U_{x_0}$ tal que \[
U_{x_0} \cap \left( D_f \cup D_g \right) = \O
.\] Defina $l = \min\left\{ \left| y_0-f\left( x_0 \right)  \right| , \left| y_0-g\left( x_0 \right)  \right|  \right\} $. Assim $\left( y_0-l,y_0+l \right) \subseteq \left( f\left( x_0 \right) , f\left( x_0 \right)  \right) $. Ou seja, \[
U = U_{x_0}\times \left( y_0-l,y_0+l \right)
\] é um aberto, $z_0\in U$ e $U\subseteq \hat{C}$, o que contradiz $z_0\in \partial \hat{C}$. Logo, $y_0\in \left\{ f\left( x_0 \right) , g\left( x_0 \right)  \right\} $ \[
\implies \left\{ \left( x_0,y_0 \right) \in \partial \hat{C} : x_0\in int A \setminus \left( D_f\cup D_g \right) \right\} \subseteq graph\left( f \right) \cup graph\left( g \right) 
.\] Portanto, concluímos que \[
\partial \hat{C} \subseteq \left(  \partial A \times \left[ -L,L \right]  \right) \cup \left(  \left( D_f \cup D_g \right) \times \left[ -L,L \right]  \right) \cup graph\left( f \right) \cup graph\left( g \right)  \tag{$*$}
.\] Ou seja, nos basta mostrar que cada um desses conjuntos possui medida nula.

Seja $\epsilon>0$. Como $f$ é integrável, escolha $P$ partição de $A$ tal que \[
S\left( f,P \right) - s\left( f,P \right) < \epsilon
.\] Assim, 
\begin{align*}
    \sum_{B\in \mathcal{B}\left( P \right) } \left( M_B - m_B \right) vol B & <\epsilon
.\end{align*}
Defina, então, \[
\mathcal{B}' := \left\{ B \times \left[ m_{B}, M_{B} \right] : B\in \mathcal{B}\left( P \right)  \right\} 
.\] Da forma como definimos, fica claro que $graph\left( f \right) \subseteq \cup \mathcal{B}'$. Agora, veja que \[
\sum_{B'\in \mathcal{B}'} vol B' = \sum_{B\in \mathcal{B}\left( P \right) } vol \left[ m_B,M_B \right] . vol B = \sum_{B\in \mathcal{B}\left( P \right) } \left( M_B-m_B \right)  vol B < \epsilon
.\] Portanto, $graph\left( f \right) $ possui medida nula. Da mesma forma, $graph\left( g\right) $ possui medida nula.

Agora, veja que para $C\subseteq\R^{n}$ com medida nula, podemos escolher $\left\{ B_i \right\} _{i\in \N}$ de forma que \[
C \subseteq \bigcup_{i\in \N} B_i
\] e \[
\sum_{i\in \N} vol_{\R^{n}} B_i < \frac{\epsilon}{vol_{\R^{m}} X}
,\] $\forall \epsilon> 0$ e $X\subseteq\R^{m}$. Dessa forma, é fácil ver que $\left\{ B'_i := B_i \times X \right\} $ é uma cobertura de $C\times X$ e \[
\sum_{i\in \N} vol_{\R^{n+m}} B_i' = \sum_{i\in \N} vol_{\R^{n}}B_i . vol_{\R^{m}} X < \epsilon
.\] 

Isso nos mostra que, uma vez que $\partial A, D_f$ e $D_g$ tem medida nula, seu produto cartesiano com $\left[ -L, L \right] $ também tem medida nula. Assim, ($*$) nos mostra que $\partial \hat{C}$ possui medida nula e, portanto, $\hat{C}$ é J-mensurável.

\exercise{2}

\begin{quote}
    Mostre que a definição de volume de um conjunto $C$ J-mensurável não depende do n-bloco $A$.
\end{quote}

Tome $A_1, A_2$ n-blocos tais que $\overline{C}\subseteq int A_1$ e $\overline{C}\subseteq int A_2$. Tome $P_1$ partição de $A_1$ e $P_2$ partição de $A_2$. Defina
\begin{align*}
    Q_1 &= P_1 \cup \left( C\cap P_2 \right) \\
    Q_2&= P_2 \cup \left( C\cap P_1 \right)
\end{align*}
e veja que refinam $P_1$ e $P_2$ respectivamente. Assim,
\begin{align*}
    S\left( \chi_C^{A_1}, P_1 \right) \ge S\left( \chi_C^{A_1}, Q_1 \right) &= \sum_{B\in \mathcal{B}\left( Q_1 \right) } M_B^{\chi_C^{A_1}} vol B \\
    &= \sum_{B\in \mathcal{B}\left( Q_1 \right) : B\cap C \neq \O} vol B
.\end{align*}
Mas, por construção, \[
\left\{ B\in \mathcal{B}\left( Q_1 \right) : B\cap C\neq \O \right\} = \left\{ B\in \mathcal{B}\left( Q_2 \right) : B\cap C \neq \O \right\} 
.\] Então
\begin{align*}
    \sum_{B\in \mathcal{B}\left( Q_1 \right) : B\cap C \neq \O} vol B &= \sum_{B\in \mathcal{B}\left( Q_2 \right) : B\cap C \neq \O} vol B \\
    &= \sum_{B\in \mathcal{B}\left( Q_2 \right) } M_B^{\chi_C^{A_2}} vol B \\
    &= S\left( \chi_C^{A_2}, Q_2 \right) 
,\end{align*}
ou seja, como nossas partições foram arbitrária, \[
\inf S\left( \chi_C^{A_2} \right) \le \inf S\left( \chi_C^{A_1} \right)  \tag{$*$}
.\] 

De forma análoga,
\begin{align*}
    s\left( \chi_C^{A_1}, P_1 \right) \le  s\left( \chi_C^{A_1}, Q_1 \right) &= \sum_{B\in \mathcal{B}\left( Q_1 \right) } m_B^{\chi_C^{A_1}} vol B \\
    &= \sum_{B\in \mathcal{B}\left( Q_1 \right) : B \subseteq C} vol B \\
    &= \sum_{B\in \mathcal{B}\left( Q_2 \right) : B \subseteq C} vol B \\
    &= \sum_{B\in \mathcal{B}\left( Q_2 \right) } m_B^{\chi_C^{A_2}} vol B \\
    &= s\left( \chi_C^{A_2}, Q_2 \right) 
\end{align*}
\[
\implies \sup s\left( \chi_C^{A_2} \right) \ge \sup s\left( \chi_C^{A_1} \right) 
.\] Junto com ($*$), temos \[
\underline{\int}_{A_1}\chi_C^{A_1} \le \underline{\int}_{A_2}\chi_C^{A_2} \le \overline{\int}_{A_2}\chi_C^{A_2}\le \overline{\int}_{A_1}\chi_C^{A_1}
.\] Pela integrabilidade dessas funções, concluímos que \[
vol_{A_1}\left( C \right) = vol_{A_2}\left( C \right) 
.\] 



\end{document}
