\lecture{7}{22.02.2021}{Mudança de variável}

Retomamos o assunto da aula anterior com um exemplo concreto.

\begin{eg}
    Sejam
    \begin{align*}
        f: X\subseteq\R^{2} &\longrightarrow \R \\
        \left( x,y \right)  &\longmapsto f\left( x,y \right)  = x^2+y^2 \\
        g: X\subseteq\R^2 &\longrightarrow \R \\
        \left( x,y \right)  &\longmapsto g\left( x,y \right)  = 1
    ,\end{align*}
    onde $X = \left\{ \left( x,y \right) \in \R^{2} : x^2+y^2\le 1 \right\} $. Veja que $X$ é J-mensurável pois sua fronteira é uma imersão de uma subvariedade de ordem menor ($\R$), portanto tem medida nula. Assim, $\hat{C} = \left\{ \left( x,y,z \right) \in \R^3 : \left( x,y \right) \in X, f\left( x,y \right) \le z\le g\left( x,y \right)  \right\}$ também é J-mensurável.

    Para calcular o volume
    \begin{align*}
	vol \hat{C} &= \int_X 1 - \int_X f \\
	&= vol_{\R^2}\left( X \right) - \int_X f
    .\end{align*}
    Portanto, reduzimos o problema ao cálculo da integral de $f$.

    Para tal, tome um n-bloco $A$ de forma que $\overline{X}\subseteq int A$. Escolhemos $A = \left[ -L,L \right]^2 , L>0$. Podemos agora estender $f$ para $A$ na forma de $\hat{f}$, de forma que \[
    \int_X f = \int_A \hat{f} \chi_X
    ,\] ao invés de utilizar a extensão por zero. De forma mais explícita, com vistas de aplicar Fubini,
    \begin{align*}
	\int_X f = \int_A \hat{f} \chi_X &= \int_{\left[ -L,L \right] \times \left[ -L,L \right] } \hat{f}\left( x,y \right) \chi_X \left( x,y \right) dx dy \\
	&= \int_{\left[ -L,L \right] } \left[ \overline{\int}_{\left[ -L,L \right] }\hat{f}\left( x,y \right) \chi_X\left( x,y \right) dx \right] dy \\
    .\end{align*}
    Explicitamente, temos \[
	\hat{f}\left( x,y \right) \chi_X\left( x,y \right) = \left( x^2 + y^2 \right) \chi_{\left[ -\sqrt{1-y^2} , \sqrt{1-y^2}  \right] }\left( x \right) \chi_{[-1,1]}\left( y \right) \tag{$*$}
    ,\] onde a função $\chi_X$ foi reescrita uma vez que
    \begin{align*}
	\left( x,y \right) \in X &\iff x^2+y^2 \le 1 \\
				 &\iff x^2 \le 1-y^2 \\
				 &\iff -\sqrt{1-y^2} \le x\le \sqrt{1-y^2} 
    .\end{align*}
    Assim, reescrevemos a integral como
    \begin{align*}
	\int_A \hat{f} \chi_X &= \int_{\left[ -L,L \right] } \left[ \overline{\int}_{\left[ -L,L \right] } \left( x^2 + y^2 \right) \chi_{\left[ -\sqrt{1-y^2} , \sqrt{1-y^2}  \right] }\left( x \right) dx \right] \chi_{[-1,1]}\left( y \right) dy \\
	&= \int_{\left[ -1,1 \right] } \left[ \overline{\int}_{\left[ -\sqrt{1-y^2} , \sqrt{1-y^2}  \right] } \left( x^2 + y^2 \right) dx \right]  dy
    .\end{align*}
    Agora veja que $x^2 + y^2$ é integrável, uma vez que é contínua em $\left[ -\sqrt{1-y^2} , \sqrt{1-y^2}  \right] $, um intervalo J-mensurável. Assim, podemos substituir a integral superior pela integral, resultando na equação \[
	\int_A \hat{f} \chi_X= \int_{\left[ -1,1 \right] } \left[ \int_{\left[ -\sqrt{1-y^2} , \sqrt{1-y^2}  \right] } \left( x^2 + y^2 \right) dx \right]  dy
    ,\] reduzindo o problema efetivamente à integrais de funções em uma dimensão. Desenvolvemos, então, essas integrais:
    \begin{align*}
	\int_A \hat{f} \chi_X&= \int_{\left[ -1,1 \right] } \left( \frac{x^3}{3} + xy^2 \right)\Big|_{-\sqrt{1-y^2} }^{\sqrt{1-y^2} } dy \\
	&= \int_{\left[ -1,1 \right] }\left( \frac{2}{3}\left( 1-y^2 \right)^{\frac{3}{2}} + 2y^2\sqrt{1-y^2} \right) dy
    .\end{align*}
    Podemos fazer a substituição trigonométrica $y=\sin \theta, -\frac{\pi}{2}\le \theta \le \frac{\pi}{2}, \implies dy = \cos \theta d\theta$ 
    \begin{align*}
	\int_A \hat{f} \chi_X &= \int_{-\frac{\pi}{2}}^{\frac{\pi}{2}} \left[ \frac{2}{3}\cos^3\theta + 2\sin^2\theta \cos\theta \right] \cos\theta d\theta \\
		 &= \int_{-\frac{\pi}{2}}^{\frac{\pi}{2}} \left[ \frac{2}{3}\cos^4\theta + 2\sin^2\theta \cos^2\theta \right] d\theta
    .\end{align*}

    Enquanto isso, o volume
    \begin{align*}
	\int_X 1 =\int_A \chi_X &= \int_{\left[ -1,1 \right] } \left[ \int_{\left[ -\sqrt{1-y^2} , \sqrt{1-y^2}  \right] } 1 dx \right]  dy \\
				&= \int_{-1 }^1  2\sqrt{1-y^2} dy
    .\end{align*}
    substituição trigonométrica novamente
    \begin{align*}
	\int_X 1 &= 2\int_{-\frac{\pi}{2}}^{\frac{\pi}{2}} \cos^2\theta d\theta \\
	&= \left( \theta + \sin\theta\cos\theta \right)\Big|_{-\frac{\pi}{2}}^{\frac{\pi}{2}} = \pi
    .\end{align*}
\end{eg}

Vemos que mesmo aplicando Fubini, temos um problema mais complicado que deveria. Isso se origina, em particular, pois lidamos com volumes/funções com simetria circular em um sistema de coordenadas retangular.

\section*{Mudança de Variável}

\begin{note}
    Demonstração completa: Cap. 09 "Análise Real", v. 2.
\end{note}

A motivação (principal) por trás deste teorema é simplificar o cálculo de integrais.

\begin{intuition}
    Em 1 dimensão. Para $I,J$ intervalos, 
    \begin{align*}
        f: J &\longrightarrow \R \\
        g: I &\longrightarrow J \\
    \end{align*}
    de classe $C^{1}$, $g$ um difeomorfismo crescente (portanto uma bijeção com inversa diferenciável, ou seja, sua derivada não zera em nenhum ponto). Definimos, naturalmente $f\circ g : I\to \R$. Assim, veja que, para $\left[ a,b \right] \subseteq I$\[
    g\left( \left[ a,b \right] \right) = \left[ c,d \right] = \left[ g\left( a \right) ,g\left( b \right)  \right] 
    .\] Seja $F: J\to \R$ primitiva de $f$. Pelo teorema fundamental do cálculo, \[
    \int_c^{d}f\left( t \right) dt = F\left( d \right) -F\left( c \right) 
    .\] Agora veja que \[
    \left( F \circ  g \right) ' = \left( F' \circ  g \right) . g' = \left( f \circ  g \right) . g'
    ,\] portanto \[
    \int_a^{b} f \circ  g \left( t \right) g'\left( t \right) dt = F \circ  g\left( b \right) - F \circ  g \left( a \right) = F\left( d \right) - F\left( c \right) = \int_c^{d}f\left( t \right) dt
    \] \[
    \implies \int_{\left[ c,d \right] } f = \int_{g\left( \left[ a,b \right]  \right) } f = \int_{\left[ a,b \right] } \left( f \circ  g \right) g' 
    \] que é basicamente o teorema de mudança de variáveis para $\R$.

    Para integrais múltiplas ($\R^{n}$), a fórmula se torna \[
    \int_{\varphi\left( K \right) } f = \int_K \left( f\circ \varphi \right) \left| det J_{\varphi} \right| 
    ,\] sendo $\varphi$ um difeomorfismo de classe $C^{1}$ e $f$ somente integrável.
\end{intuition}

\begin{theorem}
    (Mudança de Variáveis) Sejam $\varphi : U \to  V$ um difeomorfismo de classe $C^{1}$ entre abertos $U,V \subseteq \R^{m}$, $K\subset U$ um compacto J-mensurável e $f: \varphi\left( K \right)  \longrightarrow \R$ uma função integrável. Então $f\circ \varphi: K \longrightarrow \R$ é integrável e \[
    \int_{\varphi\left( K \right) } f = \int_K \left( f\circ \varphi \right) \left| det.  \varphi' \right| 
    .\]

    De forma explícita, \[
    \int_{\varphi\left( K \right) }f\left( y \right) dy = \int_K f\left( \varphi\left( x \right)  \right) \cdot \left| det. \varphi\left( x \right)' \right| dx
    .\] 
\end{theorem}

\begin{proof}
    Sejam $D_f$ os pontos de descontinuidade de $f$. Veja que \[
    D_{f\circ \varphi} = \varphi^{-1}\left( D_f \right) 
    .\] Como $f$ tem medida nula e a inversa de $\varphi$ também é localmente Lipschitz, $D_{f\circ \varphi}$ tem medida nula, ou seja, garantimos a integrabilidade de $f\circ \varphi$.
\end{proof}

\begin{note}
    Seja $T$ uma transformação linear, então \[
    vol T\left( X \right) = \left| det. T \right| \cdot vol X
    .\] 
\end{note}

\begin{note}
    $\forall a,b \in I$\[
    \|f\left( b \right) - f\left( a \right)  \| \le \left( b-a \right) \cdot \sup \left\{ \|f'\left( x \right) \| : x\in I \right\} 
    .\] Ainda mais, se $f$ é diferenciável em $I$, então $\exists c\in I$ tal que \[
    \|f\left( b \right) - f\left( a \right)  \| \le \left( b-a \right) \cdot f'\left( x \right) 
    .\] 
\end{note}

\begin{note}
    \[
    det\left( \left( f\circ g \right) '\left( x \right)  \right) = det f'\left( g\left( x \right)  \right) \cdot det g'\left( x \right) 
    .\] 
\end{note}

