As máquinas de corrente alternada estão presente fortemente na indústria pois grande parte delas é conectada direto na central de rede elétrica. Portanto, geradores, bombas, compressores, motores elétricos são exemplos de máquinas de corrente alternada, e atuam nos mais diversos processos  nas  fábricas. Como por exemplo os compressores que são empregados nas geladeiras e freezers domésticos, máquinas de lavar, ar condicionados. 

As aplicações dos motores síncronos na indústria, na maioria das vezes, resultam em vantagens econômicas e operacionais consideráveis devido as suas características de funcionamento. As principais vantagens são:

\begin{itemize}
\item Correção do fator de Potência: O motor síncrono pode ajudar a reduzir os custos de energia elétrica e melhorar o rendimento do sistema de energia, corrigindo o fator de potência na rede elétrica onde está instalado.

\item Alta capacidade de Torque:  A máquina síncrona é projetado com alta capacidade de sobrecarga, mantendo a velocidade constante mesmo em aplicações com grandes variações de carga.

\item Velocidade Constante: máquinas síncronas mantêm a velocidade constante tanto nas situações de sobrecarga como também durante momentos de oscilações de tensão.

\item Alto Rendimento: São mais eficiente na conversão de energia elétrica em mecânica, gerando maior economia de energia. O motor síncrono é projetado para operar com alto rendimento e fornecer um melhor aproveitamento de energia para uma
grande variedade de carga.

\item Maior Estabilidade na Utilização com Inversores de Frequência: Pode atuar em uma ampla faixa de velocidade, mantendo a estabilidade independente da variação de carga.

\end{itemize}

