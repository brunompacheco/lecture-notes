O motor de corrente contínua permite o controle da velocidade, pois é mais fácil de acionar em relação à maquinas de corrente alternada. O uso de conversores estáticos a tiristor ou transistor é a melhor solução para a variação da velocidade.

Portanto, devido a variedade de estruturas disponíveis, é necessário o uso de critérios para estabelecer qual a melhor solução para cada caso. Os critérios mais importantes são:

\begin{enumerate}
    \item Característica torque-velocidade: A presença de um conversor modifica a característica torque-velocidade natural do motor, é preciso então que as novas características sejam adequadas para a carga que se deseja acionar.
    
    \item Fator de potência: Estruturas e modos diferentes de controlá-las podem provocar consumos de reativos diferentes para uma mesma carga e uma mesma máquina. O fator de potência é dado por:
    \begin{equation*}
        FP = \frac{Potencia de entrada}{Volt.Amperes de entrada} = \frac{V.I_{1}.\cos{\phi_{1}}}{V.I}
    \end{equation*}
    
    onde:
    
    $I_{1} \rightarrow$ Valor eficaz da corrente fundamental.
    
    $I \rightarrow$ Valor eficaz da corrente total.
    
    $\phi_{1}  \rightarrow$  Ângulo entre a tensão e a corrente fundamental.
    
    \item Conteúdo Harmônico: É definido pela expressão a seguir:
    \[I_{h} = \frac{\sqrt{I^{2} - I_{1}^{2}}}{I_{1}}\]
    
    Quanto maior o conteúdo harmônico introduzido na rede, mais indesejável é o sistema conversor-motor CC.
    
    \item Fator de forma da corrente de armadura: O fator de forma é definido por:
    \begin{equation*}
        FF = \frac{I_{a_{eficaz}}}{I_{a_{medio}}}
    \end{equation*}
    
    $I_{a}$ é a corrente de armadura do motor.
    
    Quanto maior o fator de forma, maior será as perdas do motor. A corrente máxima do motor sempre deve ser respeitada.
    
    \item Corrente de pico: Quanto maior a corrente de pico, para uma corrente média, pior será o funcionamento do comutador do motor. A corrente máxima do motor sempre deve ser respeitada.
    
    \item Rendimento: Diferentes estruturas de comando podem propiciar rendimentos diferentes, para um dado motor e para uma dada carga.
\end{enumerate}

