\section{Considerações Finais}

Este trabalho apresenta uma abordagem geral das principais máquinas elétricas de uso industrial e seu acionamento, atentando aos seus princípios de funcionamento, seu equacionamento básico e seus aspectos construtivos, além das principais aplicações para elas.

De suma importância é a compreensão dessas máquinas, uma vez que são primordiais na indústria moderna. As máquinas de corrente contínua, por exemplo, são de grande importância na área de controle e automação, apesar de seu alto custo tanto de aquisição quanto de manutenção. Também as máquinas de corrente alternada e seu acionamento apresentam baixa manutenção e alta robustez, sendo de grande aplicabilidade em ambientes dos mais adversos.

Por isso, o presente trabalho se mostra essencial como uma base para o estudo das máquinas e das suas aplicações principais, uma vez que apresenta um conhecimento fundamental principalmente para o domínio da indústria de controle e automação.

