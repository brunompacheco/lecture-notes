\section{Introdução}

Partindo do conhecimento desenvolvido no trabalho anterior, este aborda duas grandes áreas do estudo de máquinas elétricas: máquinas de corrente contínua e máquinas de corrente alternada. A abordagem tomada é similar: parte-se dos princípios e ruma-se às aplicações através primeiro dos modelos e, após, dos aspectos construtivos.

Inicia-se com as máquinas de corrente contínua, em um estudo introdutório sobre seu funcionamento e acionamento. Então, estudam-se as máquinas de corrente alternada síncronas e assíncronas, focando nas máquinas de indução. Por fim, são estudados os motores de passo, fundamentais na indústria moderna.

Dessa forma, este trabalho visa desenvolver um conhecimento introdutório e formar uma base sólida para o domínio das máquinas elétricas no ambiente, principalmente, industrial, grande área de atuação da engenharia. 

