\lecture{2}{08.02.2021}{Amplificadores Operacionais (AMPOP)}

\section*{Circuitos realimentados}

Como o ganho de um amplificador operacional é geralmente (e idealmente) muito grande, geralmente se faz um arranjo com ele de forma realimentada.

\subsection*{Realimentação inversora}

Se realimentamos a saída do amplificador à entrada inversora. Intermediamos a conexão da tensão de entrada $V_{in}$ à entrada inversora com um resistor $R_1$, assim como a tensão de saída $V_o$, utilizando um resistor $R_2$. Devido ao ganho do amplificador, que tende ao infinito, a tensão diferencial tende a zero pela relação $V_o = A\cdot V_{diferencial}$ (aceita). Assim, podemos observar que a relação entre as tensões de entrada do circuito e saída é \[
\frac{V_o}{V_{in}} = - \frac{R_2}{R_1}
.\] 


