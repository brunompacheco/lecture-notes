\lecture{1}{01.02.2021}{Introdução à amplificadores}

\section*{Conceitos básicos de amplificadores}

Um amplificador é, essencialmente, um componente que mapeia uma tensão de entrada $V_{in}$ em uma tensão de saída $V_{out}$ de forma linear, ou seja, $V_{out}=A V_{in}$, onde $A$ é um ganho estático. Entretanto, todo amplificador é limitado pela sua alimentação, ou seja, ele é limitado superiormente e inferiormente pelas suas alimentações $V+$ e $V-$. Assim, podemos modelar ele de forma mais realista como \[
    V_{out} = \min\left\{ V+, \max\left\{ V-, A V_{in} \right\}  \right\} 
.\] 

