\lecture{1}{01.09.2020}{Conceitos Básicos}

Principais conceitos:
\begin{description}
    \item[Cinemática] ramo da física que trata da geometria do movimento, \emph{sem análise das causas}.
	\begin{description}
	    \item[Cinemática aplicada] ramo da engenharia que utiliza conhecimento de cinemática para projetar mecanismos e máquinas
	\end{description}
    \item[Rigidez] é a característica de um corpo que mantém a mesma distância entre suas partículas, que não se deforma. A partir deste conceito se derivam os conceitos de corpo rígido e corpo deformável
    \item[Sistema dinâmico] conjunto de partículas, corpos rígidos e suas interrelações
    \item[Cadeia cinemática] conjunto de corpos rígidos sujeitos a restrições mecânicas
	\begin{description}
	    \item[Elos] corpos rígidos de uma cadeia cinemática
	    \item[Acoplamento] é uma restrição entre elos de uma cadeia. Direto é por contato, indireto é sem contato.
	    \item[Aberta] uma cadeia em que existe somente um caminho entre os pares cinemáticos
	    \item[Fechada] todo elo possui, no mínimo, duas conexões com outros elos
	\end{description}
    \item[Deslocamento] mudança de posição de um elo. Quando medido ao longo do tempo é chamado \emph{Caminho}. Se um caminho possui velocidades prescritas, é chamado \emph{Trajetória}
    \item[Graus de liberdade] número de variáveis independentes necessário para determinar de forma unívoca o comportamento de um sistema
    \item[Espaço de trabalho] ambiente onde os mecanismos vão operar; todo espaço de trabalho possui seu grau de liberdade expresso por $\lambda$ 
    \item[Pares cinemáticos] par de elementos unidos por um acoplamento direto. A realização física de um par cinemático é chamada \emph{Junta}.
\end{description}

