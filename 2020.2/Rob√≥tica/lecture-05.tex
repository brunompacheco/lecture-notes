\lecture{5}{15.09.2020}{Operador de Rotação}

Um operador de rotação $R_i^{j}$ de um corpo definido em uma referência $i$ para uma referência $j$ é definido a partir das projeções dos vetores unitários (paralelos aos eixos) da referência destino na referência original. Então se a referência original $i$ possui vetores unitários $x_i$, $y_i$ e $z_i$ (considerando um espaço tridimensional), o operador de rotação pode ser definido como \[
    R = \begin{bmatrix} | & | & | \\ x^{i}_j & y^{i}_j & z_j^{i} \\ | & | & | \end{bmatrix} 
\] em que, e.g., $x_j^{i}$ é a projeção do vetor $x_j$ na referência $i$. Assim, a rotação de um ponto qualquer $P$ é definida pelo produto matricial $R'P$.

\begin{eg}
    Uma rotação de $+\frac{\pi}{2}$ em torno do eixo z pode ser definida pelo operador de rotação \[
	R' = \begin{bmatrix} 0 & -1 & 0 \\ 1 & 0 & 0 \\ 0 & 0 & 1 \end{bmatrix} 
    \]. Dado um ponto \[
    P = \begin{bmatrix} 2 \\ 2 \\ 0 \end{bmatrix} 
    \], sua rotação em torno do eixo z seria \[
    P' = R'P = \begin{bmatrix} -2 \\ 2 \\ 0 \end{bmatrix} 
    \].
\end{eg}

Para rotacionar um corpo, basta definir o operador de rotação das referências e aplicar a operação a cada ponto do corpo.

\section*{Rotações elementares}

Podemos decompor uma rotação qualquer nos seus componentes em torno dos 3 eixos do espaço. Ou seja, para aplicarmos uma rotação qualquer a um ponto, aplicamos, por convenção, as rotações \emph{elementares} em torno dos 3 eixos, e sequência.

Assim, definimos, por exemplo, a rotação de um ângulo $\theta_x$ em torno do eixo x (utilizando a regra da mão direita para definir a orientação), como \[
    R_x\left( \theta_x \right) = \begin{bmatrix} 1 & 0 & 0 \\ 0 & \cos\theta_x & - \sin\theta_x \\ 0 & \sin\theta_x & \cos\theta_x \end{bmatrix} 
\].

As demais rotações são definidas como \[
R_y\left( \theta_y \right)  = \begin{bmatrix} 
    \cos\theta_y & 0 & \sin\theta_y \\
    0 & 1 & 0 \\
    - \sin\theta_y & 0 & \cos\theta_y \\
\end{bmatrix} 
\] e \[
R_z\left( \theta_z \right) = \begin{bmatrix} 
    \cos\theta_z & -\sin\theta_z & 0 \\
    \sin\theta_z & \cos\theta_z & 0 \\
    0 & 0 & 1
\end{bmatrix} 
\].

Note que toda matriz de rotação elementar possui determinante 1 e linhas/colunas com norma 1.

