\lecture{21}{12.11.2020}{Cinemática Inversa (continuação)}

Para um robô constituído de duas juntas rotativas parametrizadas por $\theta_1$ e $\theta_2$, temos \[
\begin{cases}
    P_x = l_1\cos\theta_1 + l_2 \cos\theta_1+\theta_2 \\
    P_y = l_1\sin\theta_1 + l_2 \sin\theta_1+\theta_2 \\
    \phi_z = \theta_1 + \theta_2
\end{cases}
,\] queremos encontrar
\begin{align*}
\theta_1 = f_1\left( P_x, P_y \right) \\
\theta_2 = f_2\left( P_x, P_y \right) 
.\end{align*}
Sendo $r$ a distância do efetuador à base, temos \[
    r^2 = P_x^2 + P_y^2 \\
\] e, pela lei dos cossenos, \[
r^2 = l_1^2+l_2^2 -2l_1l_2\cos\beta 
,\] onde $\beta = \pi-\theta_2$. Agora, podemos computar \[
\sin\beta = \pm\sqrt{1-\cos^2\beta} 
,\] que nos permite calcular \[
\beta = atan2\left( \sin\beta, \cos\beta \right) 
.\] Note que o sinal de $\beta$ equivale a escolha de cotovelo para cima ou para baixo.

Para $\theta_1$, podemos aplicar a lei dos cossenos para o ângulo $\varphi$ adjacente à $\theta_1$ \[
    l_2^2 = l_1^2 + r^2 - 2l_1r\cos\varphi
,\] que, de forma similar ao $\theta_2$, nos permite calcular \[
\varphi = atan2\left( \sin\varphi, \cos\varphi \right) 
\] e, portanto, \[
\theta_1 = atan2\left( P_y, P_x \right) -\varphi
.\] 

Agora, adicionando a orientação $\phi_z$ e trabalhando com um robô 3R, podemos abordar o problema como uma redução para o problema do robô 2R acima. Veja que \[
\begin{cases}
    \bm{P} = \begin{bmatrix} P_x \\ P_y \end{bmatrix} = \begin{bmatrix} 
    l_1\cos\theta_1 + l_2\cos\left( \theta_1 +\theta_2 \right) +l_3\cos\left( \theta_1 + \theta_2 + \theta_3 \right) \\
    l_1\sin\theta_1 + l_2\sin\left( \theta_1 +\theta_2 \right) +l_3\sin\left( \theta_1 + \theta_2 + \theta_3 \right)
\end{bmatrix} \\
\phi_z = \theta_1 + \theta_2 + \theta_3
\end{cases}
.\] Definindo um ponto de trabalho $\bm{P_w}$ no atuador 3, temos \[
\bm{P_w} + l_3\begin{bmatrix} \cos\phi \\ \sin\phi \end{bmatrix} = \bm{P}
,\] ou seja, \[
\begin{cases}
    P_{w_x} = P_x - l_3\cos\phi = l_1 \cos\theta_1 + l_2\cos\left( \theta_1 + \theta_2 \right) \\
    P_{w_y} = P_x - l_3\sin\phi = l_1 \sin\theta_1 + l_2\sin\left( \theta_1 + \theta_2 \right) \\
\end{cases}
,\] problema idêntico ao resolvido acima.

A abordagem para o caso plano é a de tentar reduzir o problema para um caso mais simples.

