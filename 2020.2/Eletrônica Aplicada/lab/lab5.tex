\documentclass[a4paper]{report}
\input{./preamble.tex}
 
\begin{document}
 
\title{Laboratório 5}
\author{Bruno M. Pacheco (16100865)\\
Pedro Y. F. Ceripes (18100681) \\
EEL 7550 - Eletrônica Aplicada}
 
\maketitle
\section*{Objetivo}
 
Nesta experiência visamos adquirir familiaridade com duas variantes muito utilizadas do diodo. A primeira, o LED, em um arranjo simples, e a segunda, o diodo Zener, em um circuito regulador e em outro ceifador.
 
\section*{Simulações}

\subsection*{Circuito 1}
\subsection*{a}

Seguindo as instruções do roteiro, foi montado o circuito representado pela figura \ref{fig:figures-lab5-1-a-png} abaixo no software LTspice XII.

\begin{figure}[H]
    \centering
    \includegraphics[width=0.8\textwidth]{figures/lab5-1-a.PNG}
    \caption{Primeiro circuito configurado no LTSpice.}
    \label{fig:figures-lab5-1-a-png}
\end{figure}

\subsubsection*{b}

Considerando uma queda de tensão no LED de 2V, podemos calcular R como:

\begin{equation*}
    R = \frac{V_{fonte}-V_{D0}}{I_D} = \frac{5-2}{0.02} = 150 \Omega
\end{equation*}

\subsubsection*{c}

Definimos $R_x = 150 \Omega$, como esse é um valor comercial, podemos dizer que $R_1 = 150 \Omega$. Após isso simulamos o circuito e alcançamos os resultados abaixo.

\begin{table}[H]
    \centering
    \begin{tabular}{c | c}
	Medida & Resultado \\
	\hline
	$R_{x}$ & $150 \Omega$ \\
	$V_{D}$ & 1,94 V \\
	$I_{D}$ & 20,42 mA
\end{tabular}
\end{table}

\subsubsection*{d}
Foi dobrada a resistência utilizada na simulação e refeita a simulação. Dessa forma, foi obtido os resultados abaixo.

\begin{table}[H]
    \centering
    \begin{tabular}{c | c}
	Medida & Resultado \\
	\hline
	$R_{x}$ & $300 \Omega$ \\
	$V_{D}$ & 1,85 V \\
	$I_{D}$ & 10,49 mA
\end{tabular}
\end{table}

Analisando os resultado é possível perceber que ocorre uma queda pequena na tensão diodo e outra grande, cerca de 50\%, na tensão. Sendo assim, o aumento da resistência faz com que a potência do LED diminua, diminuindo também o brilho gerado por ele.

\subsection*{Circuito 2}

O circuito foi montado conforme a figura \ref{fig:figures-lab5_1-png}.

\begin{figure}[H]
    \centering
    \includegraphics[width=0.8\textwidth]{figures/lab5-2.png}
    \caption{Circuito regulador de tensão com diodo Zener no LTSpice.}
    \label{fig:figures-lab5_1-png}
\end{figure}

\subsubsection*{a}

De acordo com o datasheet fornecido, o componente 1N750 possui tensão Zener de 4,7 volts, portanto, precisamos de uma queda de tensão mínima desse valor sobre a resistência da carga $R_L$, ou seja, \[
    V_{out} = I_L \cdot R_L = \left( I_R-I_Z \right) R_L
.\] Assumindo que o diodo Zener é ideal, ou seja, $V_{out} < 4,7 \implies I_Z = 0$, temos $I_R = I_L$, ou seja, \[
I_R = \frac{12}{R_1 + R_L}
.\] Assim,
\begin{align*}
    V_{out} = \frac{12}{R_1 + R_L}R_L < 4,7 \\
    \implies \left( 12 - 4,7 \right) R_L < 4,7 R_1 \\
    \implies R_L < 361 \Omega
,\end{align*}
i.e., $R_{L,min} = 361 \Omega$.

\subsubsection*{b}

Veja que $R_L \to \infty \implies I_Z = I_R$, i.e., \[
    I_R = \frac{12-V_Z}{R_1} \approx 13 \text{ mA}
.\] Portanto, encontra-se dentro da corrente máxima de 85 mA estabelecida pelo fabricante. Além disso, também nesse caso, temos uma dissipação de potência de \[
P_Z = 0,013 \cdot V_Z \approx 61,3 mW
.\] 

\subsubsection*{c}

\begin{table}[H]
    \centering
    \caption{Resultados do experimento com o circuito regulador de tensão com diodo Zener.}
    \label{tab:1-zener}
    \begin{tabular}{c | c | c | c | c}
    $R_L\,\left[ \Omega \right] $ & $I_L\,\left[ \text{mA} \right]$ & $I_R\,\left[ \text{mA} \right]$ & $I_Z\,\left[ \text{mA} \right]$ & $V_Z\,\left[ \text{V} \right]$ \\
    \hline
    Sem carga & 0 & 13,1 & 13,1 & 4,68 \\
    1,2 k$\Omega$ & 3,9 & 13,1 & 9,2 & 4,66 \\
    560 $\Omega$ & 8,3 & 13,2 & 4,9 & 4,62 \\
    330 $\Omega$ & 13,0 & 13,7 & 0,7 & 4,3 \\
    \end{tabular}
\end{table}

Vemos que o comportamento condiz com o esperado pelos resultados teóricos, ou seja, com a resistência abaixo do valor calculado o diodo Zener já não mais opera como um regulador de tensão.

\subsection*{Circuito 3}
\subsubsection*{a}
Conforme instruído no roteiro, foi montado o circuito no software LTspice como é mostrado na figura \ref{fig:figures-lab5-3-a-png}.

\begin{figure}[H]
    \centering
    \includegraphics[width=0.8\textwidth]{figures/lab5-3-a.PNG}
    \caption{Terceiro circuito configurado no LTSpice.}
    \label{fig:figures-lab5-3-a-png}
\end{figure}

\subsubsection*{b}

Após isso, o circuito acima foi simulado e a figura \ref{fig:figures-lab5-3-b-png} mostra o sinal de entrada e de saída nas cores azul e verde, respectivamente.

\begin{figure}[H]
    \centering
    \includegraphics[width=0.8\textwidth]{figures/lab5-3-b.PNG}
    \caption{Simulação do terceiro circuito.}
    \label{fig:figures-lab5-3-b-png}
\end{figure}

\subsubsection*{c}

Utilizando as ferramentas do LTspice e a simulação apresentada no item anterior, foi possível preencher a tabela abaixo.

\begin{table}[H]
    \centering
    \begin{tabular}{c | c}
	Medida & Resultado \\
	\hline
	$V_{i,pp}$ & 20V \\
	$V_{o,pp}$ & 10,84 V \\
	$V_{o,min}$ & -5,42 V \\
	$V_{0,max}$ & 5,42
\end{tabular}
\end{table}

\section{Conclusão}

Vimos os usos mais comuns de dois componentes de vasta utilização: o diodo emissor de luz e o diodo Zener. Quanto ao uso do LED no componente TQLP690C, apesar dos resultados não ideais, isso não se mostra um problema uma vez que esse componente é visado normalmente como indicador de algum estado de operação. Por outro lado, o diodo Zener é fundamental no seu arranjo como regulador de tensão, por exemplo, como referência de tensão de fontes lineares de tensão, sendo o seu desempenho e precisão cruciais. O mesmo pode-se dizer sobre esse componente quando utilizado como ceifador, uma vez que impacta diretamente o sinal produzido.

\end{document}
