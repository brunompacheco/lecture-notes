\documentclass[a4paper]{report}
\input{./preamble.tex}
 
\begin{document}
 
\title{Laboratório 10}
\author{Bruno M. Pacheco (16100865)\\
Pedro Y. F. Ceripes (18100681) \\
EEL 7550 - Eletrônica Aplicada}

\maketitle
\section*{Objetivo}

As simulações descritas neste roteiro visam experimentar o uso de MOSFETs em um arranjos de amplificador, em particular, amplificador fonte comum.

\section*{Simulações}

\subsection*{Circuito 1}
\subsection*{a}

Como requisitado no roteiro, foi montado o circuito completo, porém note que as fontes de tensão que representam Vs estão com amplitude igual a zero. Logo, os únicos componentes que determinam o comportamento do sistema no equilíbrio são os resistores R1, R2, RS, RD e Vcc. O circuito pode ser observado na figura abaixo:

\begin{figure}[H]
    \centering
    \includegraphics[width=0.6\textwidth]{figures/lab10-1.PNG}
    \caption{Primeiro circuito simulado no LTspice.}
\end{figure}

Com isso, foi possível completar a tabela abaixo:

\begin{table}[H]
    \centering
    \begin{tabular}{c | c | c | c | c}
	$I_G$ [mA]  & $I_D$ [mA] & $I_{S}$ [mA]  & $V_{GS}$ [V]  & $V_{DS}$ [V]\\
    \hline
    0 & 1.3 & 1.3 & 1.7 & 1.5 
    \end{tabular}
\end{table}

Como $V_{GS} > V_{th}$, considerando $V_{th} = 1.6 V$, e também $V_{DS} >  | V_{GS} - V_{th} |$, logo o transistor está operando em saturação, atuando como amplificador.

\subsection*{b}

Para esse exercício foi modificada a tensão Vs, para duas fontes senoidais com amplitudes e frequências diferentes, como mostrado na figura abaixo:

\begin{figure}[H]
    \centering
    \includegraphics[width=0.6\textwidth]{figures/lab10-2.PNG}
    \caption{Segundo circuito simulado no LTspice.}
\end{figure}

Para analisar o ganho, é necessário checar o sinal de entrada e de saída. O gráfico desses sinais pode ser observado na figura abaixo:

\begin{figure}[H]
    \centering
    \includegraphics[width=0.8\textwidth]{figures/lab10-3.PNG}
    \caption{Análise da tensão de entrada pela de saída.}
\end{figure}

Para definir o ganho, devemos utilizar a relação: $G = \cfrac{Vout}{Vin}$. Considerando o ponto observado, temos que:
$$
G = \cfrac{Vout}{Vin} \approx \cfrac{210mV}{-70mV} = -3  
$$

Analisando a curva, também percebemos que o ganho não é alterado por estas faixas de frequência, ou seja o MOSFET está se comportando conforme o esperado para chaveamentos ou freqências altas.

\section{Conclusão}

A experiência demonstra o uso de um MOSFET em um arranjo amplificador de fonte comum. No circuito, uma fonte CC é utilizada tanto para o sinal amplificado quanto para polarizar o \textit{gate} do transistor, enquanto uma fonte AC fornece o sinal a ser amplificado. Analisaram-se o impacto da grandeza dos componentes utilizados no ganho, i.e., na quantidade de amplificação do sinal de entrada como visto pela carga simulada.

Em geral, os resultados, apesar de impactados pela não idealidade simulada do componente principal, seguiram o esperado pelo estudo teórico desse tipo de amplificador.

\end{document}
