\documentclass[a4paper]{report}
\input{./preamble.tex}
 
\begin{document}
 
\title{Lista 6}
\author{Bruno M. Pacheco (16100865)\\
EPS 5211 - Programação Econômica e Financeira}
 
\maketitle
 
\exercise{1}

De acordo com as informações fornecidas, podemos montar o fluxo de caixa relaciona ao investimento no equipamento novo:
\begin{table}[H]
    \centering
    \label{tab:label}
    \begin{tabular}{c | c}
	ano & $a_j$ \\
	\hline
	0 & -80.000,00 \\
	1 & 7.500,00 \\
	2 & 8.300,00 \\
	3 & 9.100,00 \\
	4 & 8.000,00 \\
	5 & 8.500,00 \\
	6 & 9.000,00 \\
	7 & 8.800,00 \\
	8 & 8.900,00 \\
	9 & 8.500,00
    \end{tabular}
\end{table}
E adicionar o fluxo de caixa terminal de 20.000,00. Assim, a TIR desse investimento resulta em 3,49 \%, ou seja, inferior ao custo de capital da empresa, não justificando o investimento.

\exercise{2}

B) mutuamente exclusivos.

\exercise{3}

A) somente o enfoque de aceitação-rejeição é suficiente.

\exercise{5}

O VPL de cada projeto é:
\begin{description}
    \item[Projeto A] R\$ 4.862,09
    \item[Projeto B] R\$ 4.424,48
    \item[Projeto C] R\$ 5.299,70
\end{description}

Portanto, provavelmente o projeto C é mais interessante para a empresa.

\exercise{6}

Analisamos a VPL anualizada de cada projeto:
\begin{description}
    \item[Projeto A] R\$ 4.076,00
    \item[Projeto B] R\$ 4.199,15
    \item[Projeto C] R\$ 4.041,12
\end{description}

Podemos também calcular a TIR para cada projeto:
\begin{description}
    \item[Projeto A] 15,40 \%
    \item[Projeto B] 15,54 \%
    \item[Projeto C] 14,82 \%
\end{description}

Assim, nos resta a letra D)

\exercise{7}

O fluxo de caixa referente ao projeto pode ser descrito como:
\begin{table}[H]
    \centering
    \begin{tabular}{c | c}
	período	& $a_j$ \\
	\hline
	0	& 45000 \\
	1	& 10300 \\
	2	& 9800  \\
	3	& 10600 \\
	4	& 10100 \\
	5	& 11100 \\
	6	& 11100 \\
	7	& 11100 \\
	8	& 14100 \\
    \end{tabular}
\end{table}
ou seja, o investimento tem um VPL de R\$ 6.708,93. C)

\exercise{8}

Passos para calcular a TIR:
\begin{table}[H]
    \centering
    \begin{tabular}{l}
5500000 CHS g CFo \\
150000 g CFj \\
200000 g CFj \\
230000 g CFj \\
f IRR $\implies$ 2,52
    \end{tabular}
\end{table}

B)

\exercise{11}

Fluxo de caixa:
\begin{table}[H]
    \centering
    \begin{tabular}{c | c}
	período	& $a_j$ \\
	\hline
	0	& R\$ 800.000 \\
	1	& R\$ 215.000 \\
	2	& R\$ 185.000 \\
	3	& R\$ 184.800 \\
	4	& R\$ 154.300 \\
	5	& R\$ 114.800 \\
	6	& R\$ 1.114.800\\
    \end{tabular}
\end{table}
VPL = R\$ 220.896,44

\exercise{12}

E) somente I e IV

\exercise{13}

Somente o projeto C, uma vez que tem maior TIR (e é superior ao custo de capital). O projeto E, ainda que caiba no budget, tem TIR inferior ao custo de capital.

\exercise{14}

C)

\exercise{15}

Passos para calcular a TIR:
\begin{table}[H]
    \centering
    \begin{tabular}{l}
30000 CHS g CFo \\
7000 g CFj \\
7500 g CFj \\
8200 g CFj \\
3 g Nj \\
f IRR $\implies$ 9,26
    \end{tabular}
\end{table}

C)



\end{document}
