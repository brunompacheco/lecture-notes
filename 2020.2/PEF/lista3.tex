\documentclass[a4paper]{report}
\input{./preamble.tex}
 
\begin{document}
 
\title{Lista 3}
\author{Bruno M. Pacheco (16100865)\\
EPS 5211 - Programação Econômica e Financeira}
 
\maketitle
 
\section*{Taxas Proporcionais - Juros Simples}

\exercise{1}

\[
i_m . 6 = i_s \implies i_s = 0,18
.\] 

\[
i_m . 12 = i_a \implies i_a = 0,36
.\] 

\exercise{2}

\subexercise{a}

\[
i_m . 5 = i_s = 0,9 \implies i_m = 0,15
.\]
\subexercise{b}

\[
i_m . 12 = i_a = 2,208 \implies i_m = 0,184
.\] 

\subexercise{c}

\[
i_m . 24 = i_b = 0,96 \implies i_m = 0,04
.\] 

\exercise{3}

\subexercise{a}

\[
i_a = 12 . i_m = 12 . 0,025 = 0,30
.\] 

\subexercise{b}

\[
i_a = 3 . i_q = 3 . 0,56 = 1,68
.\] 
\subexercise{c}

\[
i_a = \frac{12}{5} i_{5m} =\frac{12}{5} 0,325 = 0,78
.\] 

\section*{Taxas Equivalentes - Juros Compostos}

\exercise{1}

\[
\left( 1+i_a \right) ^{1} = \left( 1+i_t \right) ^{4} = 1,08^{4} \implies i_a = 0,3605
.\] 

\exercise{2}

\[
\left( 1+i_a \right) ^{1} = \left( 1+i_m \right) ^{12} =  1,01 ^{12} \implies i_a = 0,1268
,\] \[
\left( 1+i_s \right) ^{1} = \left( 1+i_m \right) ^{6} =  1,01 ^{6} \implies i_a = 0,0615
.\] 

\exercise{3}

\[
\left( 1+i_a \right) ^{1} = \left( 1+i_t \right) ^{4} =  1,03 ^{4} \implies i_a = 0,1255
,\] \[
\left( 1+i_s \right) ^{1} = \left( 1+i_t \right) ^{2} =  1,03 ^{2} \implies i_a = 0,0609
.\] 

\exercise{4}

\[
VF = 100\left( 1+0,126825 \right) ^{4} = 161,22
,\] \[
VF = 100\left( 1+0,061520 \right) ^{8} = 161,22
\] e \[
VF = 100\left( 1+0,01 \right) ^{48} = 161,22
.\] 

\exercise{5}

\[
\left( 1 + i_m \right) ^{12} = 1 + i_a = 1,10 \implies i_m = 0,0080
.\] 

\exercise{6}

\[
\left( 1+i_d \right) ^{30} = 1 + i_m = 1,015 \implies i_d = 0,0004964
.\] 

\exercise{7}

\[
\left( 1+i_m \right) ^{6} = 1 + i_s = 1,58 \implies i_m = 0,07922
.\] 

\exercise{8}

\[
\left( 1 + i_{sem} \right) ^{52} = 1 + i_a = 1,78 \implies i_{sem} = 0,01115
.\]

\end{document}
