\documentclass[a4paper]{report}
\input{./preamble.tex}
 
\begin{document}
 
\title{Exam}
\author{Bruno M. Pacheco (16100865)\\
DAS410079 - Modelagem para Otimização}
 
\maketitle
 
\exercise{1}

Let's start  by defining \[
\bm{x} = \begin{bmatrix} x_{Personal} \\ x_{Car} \\ x_{Home} \\ x_{Farm} \\ x_{Commercial} \end{bmatrix} 
\] our working variable that indicates the amount of money destined to each type of loan. It is clear that \[
0\le \|\bm{x}\|_{l_1} \le 12
,\] that is, the boundaries of the investment.

Besides that, we must enforce \[
x_{Farm} + x_{Commercial} = 0,4 * 12
.\] Also, as stated, \[
x_{Home} \ge 0,5 \left( x_{Personal} + x_{Car} + x_{Home} \right) 
\] \[
\implies x_{Home} \ge  x_{Personal} + x_{Car}
.\] 

On the bad debts, let's define \[
    \bm{d} = \begin{bmatrix} d_{Personal} \\ d_{Car} \\ d_{Home} \\ d_{Farm} \\ d_{Commercial} \end{bmatrix} 
\] the probability of bad debt for each type of loan. Now, the policy of the bank specifies that \[
    \frac{1}{\|\bm{x}\|_{l_1}} \bm{d}^{T}\bm{x} \le 0,04
\] \[
    \implies \bm{d}^{T}\bm{x} \le \begin{bmatrix} 0,04 \\ 0,04 \\ 0,04 \\ 0,04 \end{bmatrix} ^{T} \bm{x}
\] \[
\implies \left( \bm{d} - 0,04 \right)^{T} \bm{x} \le  0
,\] where the subtraction is considered element-wise.

Finally, we define \[
    \bm{i} = \begin{bmatrix} i_{Personal} \\ i_{Car} \\ i_{Home} \\ i_{Farm} \\ i_{Commercial} \end{bmatrix} 
\] the interest rate for each investment. So the expected profit, which is the goal of our optimisation, can be determined as \[
\sum_{t \in T} i_t x_t \left( 1-d_t \right)
,\] where $T=\left\{ Personal, Car, Home, Farm, Commercial \right\} $.

Thus, the programming can be written
\begin{align*}
    \max_{\bm{x}} \quad & \sum_{t \in T} i_t x_t \left( 1-d_t \right) \\
    \textrm{s.t.} \quad & \left( \bm{d} - 0,04 \right)^{T} \bm{x} \le 0 \\
      & x_{Home} \ge  x_{Personal} + x_{Car} \\
      & x_{Farm} + x_{Commercial} = 0,4 * 12 \\
      & 0\le \|\bm{x}\|_{l_1} \le 12 \\
	  & x_t \ge 0,\, t \in  T
.\end{align*}

\exercise{3}

We define $x_i \in \Z^{+}$ as the number of computers produced at plant $i$. We know that the production must be limited by the demand, that is, \[
\sum_{i=1}^{4} x_i \le 20 000
.\]

Now, we must also model the applicability of the fixed cost. So by using $y_i \in \left\{ 0,1 \right\} $, we can enforce
\begin{align*}
    x_i &\ge 1 - M \left( 1-y_i \right) \\
    x_i &\le 0 + M y_i 
,\end{align*}
where $M$ is a big enough number. This way, $y_i$ will indicate whether plant $i$ is in operation. This allows us to model the profit, which we want to maximize, as \[
-\bm{y}^{T} \begin{bmatrix} 9000000 \\ 5000000 \\ 3000000 \\ 1000000 \end{bmatrix} + \bm{x}^{T}\begin{bmatrix} 3500 - 1000 \\ 3500 - 1700 \\ 3500-2300 \\ 3500-2900 \end{bmatrix} 
.\] 

So our optimization problem can be formulated as
\begin{align*}
    \max_{\bm{x}} \quad & -\bm{y}^{T} \begin{bmatrix} 9000000 \\ 5000000 \\ 3000000 \\ 1000000 \end{bmatrix} + \bm{x}^{T}\begin{bmatrix} 2500 \\ 1800 \\ 1200 \\ 600 \end{bmatrix} \\
    \textrm{s.t.} \quad & \sum_{i=1}^{4} x_i \le 20 000 \\
			&x_i \ge 1 - M \left( 1-y_i \right),&i=1,\ldots,4 \\
			&x_i \le 0 + M y_i,& i=1,\ldots,4 \\
			& \bm{x} \le \begin{bmatrix} 10000 \\ 8000 \\ 9000 \\ 6000 \end{bmatrix} \\
			& x_i \in \Z^{+},\, y_i \in \left\{ 0,1 \right\} ,&i=1,\ldots,4
.\end{align*}

The optimal solution was found using Gurobipy at a profit of \$ 25,600,000 by producing 10,000 computers at plant 1, 8,000 computers at plant 2 and 2,000 computers at plant 4, none being produced at plant 3.

\end{document}
