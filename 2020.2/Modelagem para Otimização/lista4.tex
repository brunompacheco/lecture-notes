\documentclass[a4paper]{report}
\input{./preamble.tex}
 
\begin{document}
 
\title{Lista 4}
\author{Bruno M. Pacheco (16100865)\\
DAS410079 - Modelagem para Otimização}
 
\maketitle
 
\exercise{1}

Dada a formulação do problema de Hillier e Lieberman, podemos montá-lo na forma tabular conforme 

\begin{table}[H]
    \centering
    \begin{tabular}{c | c | c | c | c | c | c | c}
    Variável básica  & $Z$ & $x_1$ & $x_2$ & $x_3$ & $x_4$  & $x_5$ & Lado direito  \\
    \hline
    $Z$ & 1 & -3 & -5 & 0 & 0 & 0 & 0 \\
    $x_3$ & 0 & 1 & 0 & 1 & 0 & 0 & 4 \\
    $x_4$ & 0 & 0 & 2 & 0 & 1 & 0 & 12 \\
    $x_5$ & 0 & 3 & 2 & 0 & 0 & 1 & 18
    \end{tabular}
\end{table}

Assim, seleciona-se a variável $x_2$ como entrante e $x_4$ como sainte.

\begin{table}[H]
    \centering
    \begin{tabular}{c | c | c | c | c | c | c | c}
    Variável básica  & $Z$ & $x_1$ & $x_2$ & $x_3$ & $x_4$  & $x_5$ & Lado direito  \\
    \hline
    $Z$ & 1 & -3 & 0 & 0 & $\frac{5}{2}$ & 0 & 30 \\
    $x_3$ & 0 & 1 & 0 & 1 & 0 & 0 & 4 \\
    $x_2$ & 0 & 0 & 1 & 0 & $\frac{1}{2}$ & 0 & 6 \\
    $x_5$ & 0 & 3 & 0 & 0 & -1 & 1 & 6
    \end{tabular}
\end{table}

Então, $x_1$ se torna a entrante e $x_5$ a sainte.

\begin{table}[H]
    \centering
    \begin{tabular}{c | c | c | c | c | c | c | c}
    Variável básica  & $Z$ & $x_1$ & $x_2$ & $x_3$ & $x_4$  & $x_5$ & Lado direito  \\
    \hline
    $Z$ & 1 & 0 & 0 & 0 & $\frac{3}{2}$ & 1 & 36 \\
    $x_3$ & 0 & 0 & 0 & 1 & $\frac{1}{3}$ & $-\frac{1}{3}$ & 2 \\
    $x_2$ & 0 & 0 & 1 & 0 & $\frac{1}{2}$ & 0 & 6 \\
    $x_1$ & 0 & 1 & 0 & 0 & $-\frac{1}{3}$ & $\frac{1}{3}$ & 2
    \end{tabular}
\end{table}

Vê-se que chegamos em uma solução, com \[
\begin{bmatrix} x_1 \\ x_2 \end{bmatrix} = \begin{bmatrix} 2 \\ 6 \end{bmatrix} 
\] e custo 36.

\exercise{2}

Introduzindo 3 variáveis básicas,

\begin{table}[H]
    \centering
    \begin{tabular}{c | c | c | c | c | c | c | c}
	Variável básica  & $Z$ & $x_1$ & $x_2$ & $x_3$ & $x_4$ & $x_5$ & Lado direito  \\
    \hline
	$Z$ & 1 & 4 & -1 & 0 & 0 & 0 & 0 \\
	$x_3$ & 0 & 2 & 1 & 1 & 0 & 0 & 8 \\
	$x_4$ & 0 & 0 & 1 & 0 & 1 & 0 & 5 \\
	$x_5$ & 0 & 1 & -1 & 0 & 0 & 1 & 4
    \end{tabular}
\end{table}

Escolhemos $x_2$ como a variável saindo e $x_3$ como a entrante.

\begin{table}[H]
    \centering
    \begin{tabular}{c | c | c | c | c | c | c | c}
	Variável básica  & $Z$ & $x_1$ & $x_2$ & $x_3$ & $x_4$ & $x_5$ & Lado direito  \\
    \hline
	$Z$ & 1 & 4 & 0 & 1 & 0 & 0 & 8 \\
	$x_2$ & 0 & 2 & 1 & 1 & 0 & 0 & 8 \\
	$x_4$ & 0 & 0 & 0 & -1 & 1 & 0 & 5 \\
	$x_5$ & 0 & 1 & 0 & 1 & 0 & 1 & 4
    \end{tabular}
\end{table}

Então $\bm{x} = \left( 0, 8 \right) $ é uma solução ótima para o problema.

\exercise{3}

Podemos dividir o dia em intervalos de 8 horas (turnos dos ônibus) com sobreposição, de forma que o primeiro turno cubra os dois primeiros intervalos do gráfico, o segundo os intervalos 2 e 3 e assim por diante até que o último turno cubra o último intervalo e o primeiro. Formulamos, então, a alocação de ônibus por cada turno como \[
    \bm{x} = \begin{bmatrix} x_1 \\ \vdots \\ x_6 \end{bmatrix} \in \Z^{6}
.\] 

Definamos, então, \[
\bm{f} = \begin{bmatrix} 4 \\ 8 \\ 10 \\ 7 \\ 12 \\ 4 \end{bmatrix} 
\] a frota necessária por cada intervalos do gráfico. Assim, nota-se que \[
f_i \le x_i + x_{i-1},\, 1 < i \le 6
.\] Dessa forma, podemos escrever a restrição de forma matricial através de \[
A = \begin{bmatrix}
    1 & 0 & 0 & 0 & 0 & 1 \\
    1 & 1 & 0 & 0 & 0 & 0 \\
    0 & 1 & 1 & 0 & 0 & 0 \\
    0 & 0 & 1 & 1 & 0 & 0 \\
    0 & 0 & 0 & 1 & 1 & 0 \\
    0 & 0 & 0 & 0 & 1 & 1
\end{bmatrix} 
,\] i.e., \[
A \bm{x} \ge \bm{f}
.\] 

Dessa forma, podemos formular nosso problema como
\begin{align*}
    \min_{\bm{x}} \quad & \|\bm{x} \|_{s} \\
    \textrm{s.t.} \quad & A \bm{x} \ge \bm{f} \\
      & \bm{x} \ge \bm{0}
,\end{align*}
onde $\|\cdot \|_s$ é a norma da soma.

Utilizando o Gurobi, encontrou-se a solução ótima com uma frota de 26 ônibus, distribuídos nos turnos conforme a tabela abaixo.

\begin{table}[H]
    \centering
    \begin{tabular}{c | c}
    Turno & \# ônibus \\
    \hline
    00:00 - 08:00 & 0 \\
    04:00 - 12:00 & 8 \\
    08:00 - 16:00 & 2 \\
    12:00 - 20:00 & 5 \\
    16:00 - 00:00 & 7 \\
    20:00 - 04:00 & 4
    \end{tabular}
\end{table}

\end{document}
