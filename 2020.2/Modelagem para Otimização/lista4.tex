\documentclass[a4paper]{report}
% Some basic packages
\usepackage[utf8]{inputenc}
\usepackage[T1]{fontenc}
\usepackage{textcomp}
\usepackage[english]{babel}
\usepackage{url}
\usepackage{graphicx}
\usepackage{float}
\usepackage{booktabs}
\usepackage{enumitem}

\pdfminorversion=7

% Don't indent paragraphs, leave some space between them
\usepackage{parskip}

% Hide page number when page is empty
\usepackage{emptypage}
\usepackage{subcaption}
\usepackage{multicol}
\usepackage{xcolor}

% Other font I sometimes use.
% \usepackage{cmbright}

% Math stuff
\usepackage{amsmath, amsfonts, mathtools, amsthm, amssymb}
% Fancy script capitals
\usepackage{mathrsfs}
\usepackage{cancel}
% Bold math
\usepackage{bm}
% Some shortcuts
\newcommand\N{\ensuremath{\mathbb{N}}}
\newcommand\R{\ensuremath{\mathbb{R}}}
\newcommand\Z{\ensuremath{\mathbb{Z}}}
\renewcommand\O{\ensuremath{\emptyset}}
\newcommand\Q{\ensuremath{\mathbb{Q}}}
\newcommand\C{\ensuremath{\mathbb{C}}}
\renewcommand\L{\ensuremath{\mathcal{L}}}

% Package for Petri Net drawing
\usepackage[version=0.96]{pgf}
\usepackage{tikz}
\usetikzlibrary{arrows,shapes,automata,petri}
\usepackage{tikzit}
\input{petri_nets_style.tikzstyles}

% Easily typeset systems of equations (French package)
\usepackage{systeme}

% Put x \to \infty below \lim
\let\svlim\lim\def\lim{\svlim\limits}

%Make implies and impliedby shorter
\let\implies\Rightarrow
\let\impliedby\Leftarrow
\let\iff\Leftrightarrow
\let\epsilon\varepsilon

% Add \contra symbol to denote contradiction
\usepackage{stmaryrd} % for \lightning
\newcommand\contra{\scalebox{1.5}{$\lightning$}}

% \let\phi\varphi

% Command for short corrections
% Usage: 1+1=\correct{3}{2}

\definecolor{correct}{HTML}{009900}
\newcommand\correct[2]{\ensuremath{\:}{\color{red}{#1}}\ensuremath{\to }{\color{correct}{#2}}\ensuremath{\:}}
\newcommand\green[1]{{\color{correct}{#1}}}

% horizontal rule
\newcommand\hr{
    \noindent\rule[0.5ex]{\linewidth}{0.5pt}
}

% hide parts
\newcommand\hide[1]{}

% si unitx
\usepackage{siunitx}
\sisetup{locale = FR}

% Environments
\makeatother
% For box around Definition, Theorem, \ldots
\usepackage{mdframed}
\mdfsetup{skipabove=1em,skipbelow=0em}
\theoremstyle{definition}
\newmdtheoremenv[nobreak=true]{definitie}{Definitie}
\newmdtheoremenv[nobreak=true]{eigenschap}{Eigenschap}
\newmdtheoremenv[nobreak=true]{gevolg}{Gevolg}
\newmdtheoremenv[nobreak=true]{lemma}{Lemma}
\newmdtheoremenv[nobreak=true]{propositie}{Propositie}
\newmdtheoremenv[nobreak=true]{stelling}{Stelling}
\newmdtheoremenv[nobreak=true]{wet}{Wet}
\newmdtheoremenv[nobreak=true]{postulaat}{Postulaat}
\newmdtheoremenv{conclusie}{Conclusie}
\newmdtheoremenv{toemaatje}{Toemaatje}
\newmdtheoremenv{vermoeden}{Vermoeden}
\newtheorem*{herhaling}{Herhaling}
\newtheorem*{intermezzo}{Intermezzo}
\newtheorem*{notatie}{Notatie}
\newtheorem*{observatie}{Observatie}
\newtheorem*{exe}{Exercise}
\newtheorem*{opmerking}{Opmerking}
\newtheorem*{praktisch}{Praktisch}
\newtheorem*{probleem}{Probleem}
\newtheorem*{terminologie}{Terminologie}
\newtheorem*{toepassing}{Toepassing}
\newtheorem*{uovt}{UOVT}
\newtheorem*{vb}{Voorbeeld}
\newtheorem*{vraag}{Vraag}

\newmdtheoremenv[nobreak=true]{definition}{Definition}
\newtheorem*{eg}{Example}
\newtheorem*{notation}{Notation}
\newtheorem*{previouslyseen}{As previously seen}
\newtheorem*{remark}{Remark}
\newtheorem*{note}{Note}
\newtheorem*{problem}{Problem}
\newtheorem*{observe}{Observe}
\newtheorem*{property}{Property}
\newtheorem*{intuition}{Intuition}
\newmdtheoremenv[nobreak=true]{prop}{Proposition}
\newmdtheoremenv[nobreak=true]{theorem}{Theorem}
\newmdtheoremenv[nobreak=true]{corollary}{Corollary}

% End example and intermezzo environments with a small diamond (just like proof
% environments end with a small square)
\usepackage{etoolbox}
\AtEndEnvironment{vb}{\null\hfill$\diamond$}%
\AtEndEnvironment{intermezzo}{\null\hfill$\diamond$}%
% \AtEndEnvironment{opmerking}{\null\hfill$\diamond$}%

% Fix some spacing
% http://tex.stackexchange.com/questions/22119/how-can-i-change-the-spacing-before-theorems-with-amsthm
\makeatletter
\def\thm@space@setup{%
  \thm@preskip=\parskip \thm@postskip=0pt
}


% Exercise 
% Usage:
% \exercise{5}
% \subexercise{1}
% \subexercise{2}
% \subexercise{3}
% gives
% Exercise 5
%   Exercise 5.1
%   Exercise 5.2
%   Exercise 5.3
\newcommand{\exercise}[1]{%
    \def\@exercise{#1}%
    \subsection*{Exercise #1}
}

\newcommand{\subexercise}[1]{%
    \subsubsection*{Exercise \@exercise.#1}
}


% \lecture starts a new lecture (les in dutch)
%
% Usage:
% \lecture{1}{di 12 feb 2019 16:00}{Inleiding}
%
% This adds a section heading with the number / title of the lecture and a
% margin paragraph with the date.

% I use \dateparts here to hide the year (2019). This way, I can easily parse
% the date of each lecture unambiguously while still having a human-friendly
% short format printed to the pdf.

\usepackage{xifthen}
\def\testdateparts#1{\dateparts#1\relax}
\def\dateparts#1 #2 #3 #4 #5\relax{
    \marginpar{\small\textsf{\mbox{#1 #2 #3 #5}}}
}

\def\@lecture{}%
\newcommand{\lecture}[3]{
    \ifthenelse{\isempty{#3}}{%
        \def\@lecture{Lecture #1}%
    }{%
        \def\@lecture{Lecture #1: #3}%
    }%
    \subsection*{\@lecture}
    \marginpar{\small\textsf{\mbox{#2}}}
}



% These are the fancy headers
\usepackage{fancyhdr}
\pagestyle{fancy}

% LE: left even
% RO: right odd
% CE, CO: center even, center odd
% My name for when I print my lecture notes to use for an open book exam.
% \fancyhead[LE,RO]{Gilles Castel}

\fancyhead[RO,LE]{\@lecture} % Right odd,  Left even
\fancyhead[RE,LO]{}          % Right even, Left odd

\fancyfoot[RO,LE]{\thepage}  % Right odd,  Left even
\fancyfoot[RE,LO]{}          % Right even, Left odd
\fancyfoot[C]{\leftmark}     % Center

\makeatother




% Todonotes and inline notes in fancy boxes
\usepackage{todonotes}
\usepackage{tcolorbox}

% Make boxes breakable
\tcbuselibrary{breakable}

% Verbetering is correction in Dutch
% Usage: 
% \begin{verbetering}
%     Lorem ipsum dolor sit amet, consetetur sadipscing elitr, sed diam nonumy eirmod
%     tempor invidunt ut labore et dolore magna aliquyam erat, sed diam voluptua. At
%     vero eos et accusam et justo duo dolores et ea rebum. Stet clita kasd gubergren,
%     no sea takimata sanctus est Lorem ipsum dolor sit amet.
% \end{verbetering}
\newenvironment{verbetering}{\begin{tcolorbox}[
    arc=0mm,
    colback=white,
    colframe=green!60!black,
    title=Opmerking,
    fonttitle=\sffamily,
    breakable
]}{\end{tcolorbox}}

% Noot is note in Dutch. Same as 'verbetering' but color of box is different
\newenvironment{noot}[1]{\begin{tcolorbox}[
    arc=0mm,
    colback=white,
    colframe=white!60!black,
    title=#1,
    fonttitle=\sffamily,
    breakable
]}{\end{tcolorbox}}




% Figure support as explained in my blog post.
\usepackage{import}
\usepackage{xifthen}
\usepackage{pdfpages}
\usepackage{transparent}
\newcommand{\incfig}[1]{%
    \def\svgwidth{\columnwidth}
    \import{./figures/}{#1.pdf_tex}
}

% Fix some stuff
% %http://tex.stackexchange.com/questions/76273/multiple-pdfs-with-page-group-included-in-a-single-page-warning
\pdfsuppresswarningpagegroup=1


% My name
\author{Bruno M. Pacheco}

 
\begin{document}
 
\title{Lista 4}
\author{Bruno M. Pacheco (16100865)\\
DAS410079 - Modelagem para Otimização}
 
\maketitle
 
\exercise{1}

Dada a formulação do problema de Hillier e Lieberman, podemos montá-lo na forma tabular conforme 

\begin{table}[H]
    \centering
    \begin{tabular}{c | c | c | c | c | c | c | c}
    Variável básica  & $Z$ & $x_1$ & $x_2$ & $x_3$ & $x_4$  & $x_5$ & Lado direito  \\
    \hline
    $Z$ & 1 & -3 & -5 & 0 & 0 & 0 & 0 \\
    $x_3$ & 0 & 1 & 0 & 1 & 0 & 0 & 4 \\
    $x_4$ & 0 & 0 & 2 & 0 & 1 & 0 & 12 \\
    $x_5$ & 0 & 3 & 2 & 0 & 0 & 1 & 18
    \end{tabular}
\end{table}

Assim, seleciona-se a variável $x_2$ como entrante e $x_4$ como sainte.

\begin{table}[H]
    \centering
    \begin{tabular}{c | c | c | c | c | c | c | c}
    Variável básica  & $Z$ & $x_1$ & $x_2$ & $x_3$ & $x_4$  & $x_5$ & Lado direito  \\
    \hline
    $Z$ & 1 & -3 & 0 & 0 & $\frac{5}{2}$ & 0 & 30 \\
    $x_3$ & 0 & 1 & 0 & 1 & 0 & 0 & 4 \\
    $x_2$ & 0 & 0 & 1 & 0 & $\frac{1}{2}$ & 0 & 6 \\
    $x_5$ & 0 & 3 & 0 & 0 & -1 & 1 & 6
    \end{tabular}
\end{table}

Então, $x_1$ se torna a entrante e $x_5$ a sainte.

\begin{table}[H]
    \centering
    \begin{tabular}{c | c | c | c | c | c | c | c}
    Variável básica  & $Z$ & $x_1$ & $x_2$ & $x_3$ & $x_4$  & $x_5$ & Lado direito  \\
    \hline
    $Z$ & 1 & 0 & 0 & 0 & $\frac{3}{2}$ & 1 & 36 \\
    $x_3$ & 0 & 0 & 0 & 1 & $\frac{1}{3}$ & $-\frac{1}{3}$ & 2 \\
    $x_2$ & 0 & 0 & 1 & 0 & $\frac{1}{2}$ & 0 & 6 \\
    $x_1$ & 0 & 1 & 0 & 0 & $-\frac{1}{3}$ & $\frac{1}{3}$ & 2
    \end{tabular}
\end{table}

Vê-se que chegamos em uma solução, com \[
\begin{bmatrix} x_1 \\ x_2 \end{bmatrix} = \begin{bmatrix} 2 \\ 6 \end{bmatrix} 
\] e custo 36.

\exercise{2}

Introduzindo 3 variáveis básicas,

\begin{table}[H]
    \centering
    \begin{tabular}{c | c | c | c | c | c | c | c}
	Variável básica  & $Z$ & $x_1$ & $x_2$ & $x_3$ & $x_4$ & $x_5$ & Lado direito  \\
    \hline
	$Z$ & 1 & 4 & -1 & 0 & 0 & 0 & 0 \\
	$x_3$ & 0 & 2 & 1 & 1 & 0 & 0 & 8 \\
	$x_4$ & 0 & 0 & 1 & 0 & 1 & 0 & 5 \\
	$x_5$ & 0 & 1 & -1 & 0 & 0 & 1 & 4
    \end{tabular}
\end{table}

Escolhemos $x_2$ como a variável saindo e $x_3$ como a entrante.

\begin{table}[H]
    \centering
    \begin{tabular}{c | c | c | c | c | c | c | c}
	Variável básica  & $Z$ & $x_1$ & $x_2$ & $x_3$ & $x_4$ & $x_5$ & Lado direito  \\
    \hline
	$Z$ & 1 & 4 & 0 & 1 & 0 & 0 & 8 \\
	$x_2$ & 0 & 2 & 1 & 1 & 0 & 0 & 8 \\
	$x_4$ & 0 & 0 & 0 & -1 & 1 & 0 & 5 \\
	$x_5$ & 0 & 1 & 0 & 1 & 0 & 1 & 4
    \end{tabular}
\end{table}

Então $\bm{x} = \left( 0, 8 \right) $ é uma solução ótima para o problema.

\exercise{3}

Podemos dividir o dia em intervalos de 8 horas (turnos dos ônibus) com sobreposição, de forma que o primeiro turno cubra os dois primeiros intervalos do gráfico, o segundo os intervalos 2 e 3 e assim por diante até que o último turno cubra o último intervalo e o primeiro. Formulamos, então, a alocação de ônibus por cada turno como \[
    \bm{x} = \begin{bmatrix} x_1 \\ \vdots \\ x_6 \end{bmatrix} \in \Z^{6}
.\] 

Definamos, então, \[
\bm{f} = \begin{bmatrix} 4 \\ 8 \\ 10 \\ 7 \\ 12 \\ 4 \end{bmatrix} 
\] a frota necessária por cada intervalos do gráfico. Assim, nota-se que \[
f_i \le x_i + x_{i-1},\, 1 < i \le 6
.\] Dessa forma, podemos escrever a restrição de forma matricial através de \[
A = \begin{bmatrix}
    1 & 0 & 0 & 0 & 0 & 1 \\
    1 & 1 & 0 & 0 & 0 & 0 \\
    0 & 1 & 1 & 0 & 0 & 0 \\
    0 & 0 & 1 & 1 & 0 & 0 \\
    0 & 0 & 0 & 1 & 1 & 0 \\
    0 & 0 & 0 & 0 & 1 & 1
\end{bmatrix} 
,\] i.e., \[
A \bm{x} \ge \bm{f}
.\] 

Dessa forma, podemos formular nosso problema como
\begin{align*}
    \min_{\bm{x}} \quad & \|\bm{x} \|_{s} \\
    \textrm{s.t.} \quad & A \bm{x} \ge \bm{f} \\
      & \bm{x} \ge \bm{0}
,\end{align*}
onde $\|\cdot \|_s$ é a norma da soma.

Utilizando o Gurobi, encontrou-se a solução ótima com uma frota de 26 ônibus, distribuídos nos turnos conforme a tabela abaixo.

\begin{table}[H]
    \centering
    \begin{tabular}{c | c}
    Turno & \# ônibus \\
    \hline
    00:00 - 08:00 & 0 \\
    04:00 - 12:00 & 8 \\
    08:00 - 16:00 & 2 \\
    12:00 - 20:00 & 5 \\
    16:00 - 00:00 & 7 \\
    20:00 - 04:00 & 4
    \end{tabular}
\end{table}

\end{document}
