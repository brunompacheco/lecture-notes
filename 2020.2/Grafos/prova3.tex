\documentclass[a4paper]{report}
\input{./preamble.tex}
 
\begin{document}
 
\title{Prova 3}
\author{Bruno M. Pacheco (16100865)\\
Grafos}
 
\maketitle
 
\exercise{1}

Considerando que o comportamento dos motoristas é organizado no sentido de maximizar o fluxo total de $s$ para $t$, podemos utilizar o algoritmo de Ford-Fulkerson para obter o fluxo máximo em $G$. Dessa forma, podemos obter o fluxo através de uma intersecção $u\in V$ através do fluxo através dos arcos saintes de $u$ (ou entrantes em $u$, seria o mesmo resultado). Um exemplo dessa abordagem pode ser visto no algoritmo abaixo.

\begin{algorithm}
    \KwIn{um grafo $G=\left( V, A, c \right) $, vértices $s,t \in V$}
    $f \gets \,$FordFulkerson$\left( G,s,t \right) $ \\
    $F_u \gets \sum_{\left( u,v \right) \in N^{+}\left( u \right)  } f\left( \left( u,v \right)  \right) ,\, \forall u \in V$ \\
    \Return{$\left\{ u\in V : F_u = \max F \right\} $}
\end{algorithm}

\exercise{2}

O grafo CPM obtido a partir do conjunto de atividades fornecido pode ser visto na figura abaixo.

\begin{figure}[h]
    \centering
    \includegraphics[width=0.8\textwidth]{cpm.png}
\end{figure}

Dessa forma, performamos o \emph{forward-pass} e o \emph{backward-pass} para encontrar a seguinte relação entre os eventos e os seus tempos mínimos/máximos de ocorrência:
\begin{table}[h]
    \centering
    \begin{tabular}{c | c | c}
     & $E_j$ & $T_j$ \\
     \hline
	1 & 0 & 0 \\
	2 & 3 & 3 \\
	3 & 7 & 7 \\
	4 & 9 & 9 \\
	5 & 11 & 11
    \end{tabular}
\end{table}

As atividades críticas são $B, C, D, F $.

\exercise{3}

Obter o emparelhamento máximo de $G'$ com apenas um caminho aumentante alternante não é possível para todo grafo $G$ como estipulado no enunciado. Um exemplo seria um grafo $G$ com vértices $u\in X$, $v\in Y$ não emparelhados por $M$ de forma que não exista um caminho de $u$ para $v$. Ainda mais, suponha que $Z_x = \left\{ \left( x,v \right)  \right\} $ e $Z_y = \left\{ \left( y,u \right)  \right\} $. Veja que não existe sequer um caminho que passe por $x$ e $y$ e, ainda assim, o emparelhamento máximo do grafo $G'$ inclui os emparelhamentos $x-v$ e $u-y$, ou seja, seriam necessários dois caminhos aumentantes alternantes para obter o emparelhamento máximo de $G'$.

\exercise{4}

Veja que, como os vértices de $T$ não possuem arestas entre si, os vizinhos de cada vértice de $T$ terão no máximo $k$ cores distintas. Dessa forma, podemos, no pior caso, adicionar uma nova cor e utilizá-la para todos os vértices de $T$ uma vez que eles não são vizinhos entre si. Ou seja, $G'$ tem no máximo $k+1$ cores.

\end{document}
