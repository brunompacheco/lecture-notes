\section{Introdução}

É bem sabido que as máquinas elétricas possibilitaram a gigantesca maioria das grandes revoluções que aconteceram nos últimos anos. Desde indústrias altamente automatizadas até o conforto disponível em grandes lares. Hoje é raro encontrar ambientes sem a presença de máquinas elétricas.

Por isso, se torna essencial o estudo de seu acionamento, principalmente quando se fala no seu controle. Duas famílias de componentes são de essencial domínio para que o engenheiro possa realizar o acionamento de uma máquina elétrica com sucesso e de forma versátil: retificadores e conversores.

Este trabalho, direcionado à disciplina de Máquina e Acionamentos Elétricos para Automação, visa apresentar o princípio de funcionamento, equacionamento básico e principais aplicações de retificadores e conversores em suas principais formas construtivas para o acionamento de máquinas elétricas. Ou seja, serão cobertos retificadores à diodo e à tiristor, enquanto conversores CC-CC e CC-CA serão o foco da segunda parte.

