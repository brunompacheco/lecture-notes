Quando o enrolamento do estator for conectado a uma fonte de alimentação trifásica e o circuito do rotor é fechado, as correntes induzidas no rotor irão produzir um campo magnético que irá interagir com o campo magnético girante no entreferro, dando origem a um torque. O rotor, se livre, irá iniciar o movimento de rotação. De acordo com a lei de Lenz, o rotor gira na direção do campo magnético girante de tal maneira que a velocidade relativa entre o campo magnético girante e o enrolamento do rotor diminua. O rotor irá atingir uma velocidade de rotação n que é inferior à velocidade de rotação síncrona. A diferença entre a velocidade síncrona do campo magnético girante e a velocidade do rotor é denominada escorregamento e é definida através da equação:

$$S = \frac{n_s - n}{n_s}$$

A frequência da corrente induzida no circuito do rotor é dada pela expressão:

$$f_2 = s  f_1$$

