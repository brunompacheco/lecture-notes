\lecture{23}{19.11.2020}{Cinemática Inversa Numérica}

A soluções analíticas funcionam bem para robôs de cadeia cinemática simples. Quando temos estruturas complexas, punhos não-esféricos, manipuladores redundantes ou uma alta relação não-linear entre variáveis do espaço de juntas e cartesiano, \emph{utiliza-se como solução da cinemática inversa a integração no tempo da cinemática diferencial}.

Na cinemática diferencial direta, temos \[
    \bm{v} = J \dot{\bm{q}}
,\] sendo $\bm{q}$ os parâmetros das juntas e $\bm{v}$ as velocidades do atuador. Assim, podemos obter \[
\bm{q}(t) = \bm{q}_0 + \int_{t_0}^{t} J^{-1} \bm{v}
.\] Dessa forma, basta aplicar um método numérico de integração.

Utilizando o método de Euler, aproximando a operação de diferença de forma linear, isto é, \[
\bm{v} = \frac{\bm{s}(t) - \bm{s}(t-\Delta t)}{\Delta t}
.\] De forma discreta, \[
\bm{s}(t) = \bm{s}_i
.\] Assim, definimos \[
\bm{q}_{i+1} = \bm{q}_i + J^{-1}\left( \bm{q}_i \right) \left( \bm{s}_d -\bm{s}_i \right) 
,\] onde $\bm{s}_d$ é a posição desejada. Essa equação é repetida enquanto $\bm{s}_d - \bm{s}_i$ for significativo.

