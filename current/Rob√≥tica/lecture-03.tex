\lecture{3}{08.09.2020}{Partições, Variações e Inversões}

Além de especificar mobilidade, dimensão do espaço e número de circuitos, outras informações são necessárias para definir um mecanismo unicamente.

\begin{description}
    \item[Partições] conjunto de elos diferentes que entram na formação da cadeia cinemática, i.e., quais elos estão disponíveis para construção;
    \item[Variações] formas disponíveis de ligar os elos das partições;
    \item[Inversões] Diferentes formas de fixação dos elos. As inversões de uma variação são definidas a partir de um elo como base.
\end{description}

\begin{note}
    Cadeias degeneradas (com mobilidade 0, e.g., um triângulo, um circuito de dois elos binários) não compõem variações.
\end{note}

\begin{note}
    Os conceitos acima são "hierárquicos", a partir de uma partição podemos formar as variações, e uma dada variação tem suas inversões.
\end{note}

\begin{definition}
    (Degeneração) Uma cadeia é dita \emph{degenerada} se:
    \begin{itemize}
        \item Possui uma sub-cadeia de Baranov;
	\item Possui um fracionamento.
    \end{itemize}
\end{definition}

