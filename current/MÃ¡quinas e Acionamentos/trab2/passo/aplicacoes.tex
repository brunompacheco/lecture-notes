A inércia do rotor resultante de altas velocidades de rotação pode ocasionar perdas de passo, desestabilizando a operação e, portanto, dificultando o uso de motores de passo em aplicações que requerem alta velocidade. O mesmo se aplica para situações em que um alto torque é aplicado ao rotor, uma vez que a resposta do mesmo é resultante da atração entre o rotor e a bobina energizada, ou seja, um limitante ao torque possível.

Seu uso é recomendado para aplicações que requerem o posicionamento preciso principalmente por não possuir um erro cumulativo. Além disso, também é um motor muito vantajoso pela sua rápida aceleração e desaceleração. Algumas aplicações bem conhecidas são:
\begin{itemize}
\item Bombas peristálticas
\item Atuadores lineares
\item Equipamentos médicos
\item Rotuladoras
\item Etiquetadoras 
\item Scanners
\item Impressoras 
\item Máquinas com controle numérico (CNC)
\end{itemize}

