Motores de passo são máquinas que convertem pulsos elétricos em movimentos mecânicos, rotacionando com variações angulares discretas e bem definidas: os passos. Normalmente, esses são incrementos angulares pequenos do rotor, rotacionando aos pulsos nos seus terminais. Dessa forma, eles permitem um controle não-realimentado natural do ângulo rotacionado através da quantidade de pulsos aplicados. Sua velocidade rotacional, também, é determinada pela frequência dos pulsos.

