\section{Considerações Finais}

Neste trabalho foi redigida uma visão concisa e geral de retificadores e conversores, discorrendo sobre princípios de funcionamento, equacionamento básico das variáveis de interesse e principais aplicações das variantes.

Mais especificamente, retificadores controlados ou não (utilizando tiristores ou diodos, resp.) foram estudados quanto a suas aplicações para uma ou três fases, além das variações nas implementações para meia onda e onda completa. Quanto aos conversores, distinguiu-se entre conversores CC-CC e CC-CA (ou inversores). Analisou-se os arranjos mais comuns dos circuitos para essas finalidades, além das consequências desses quanto ao resultado dos sinais para a carga e para a alimentação.

Por fim, pode-se dizer que este estudo resulta, principalmente, em um aprendizado em bom nível de profundidade dos componentes tratados, servindo como uma base sólida para estudos futuros.

