\documentclass[a4paper]{report}
\input{./preamble.tex}
 
\begin{document}
 
\title{Lista 1}
\author{Bruno M. Pacheco (16100865)\\
EPS 5211 - Programação Econômica e Financeira}
 
\maketitle
 
\exercise{1}

Sabemos que o valor a prazo para uma taxa de juros de 6,5 \% pode ser calculado como \[
VF = 450,00 \cdot 1,065 ^{n}
,\] onde $n$ é o número de meses. Assim, sabendo que $VF=600,00$, podemos calcular \[
n = \log_{1,065} \frac{600}{450} = 4,57 \text{ meses}
.\] 

\exercise{2}

Para um regime de juros compostos, utilizando meses como referência, temos \[
64.800,00 = 40.000,00 \cdot \left( 1+i \right)  ^{4} \implies 1+i = 1,1282 \implies i = 12,82 \text{ \% a.m.}
\] 

Para um regime de juros simples, \[
64.800,00 = 40.000,00 \cdot \left( 1 + i \cdot 4 \right)  \implies  i = 15,5\text{ \% a.m.}
\] 

\exercise{3}

Primeiro, queremos que $VF = VP \cdot 5$. Então, para um regime de juros compostos, \[
5\cdot VP = VP \cdot \left( 1+0,28 \right) ^{n}
,\] portanto, \[
n = \log_{1,28}5 = 6,52\text{ anos}
.\] 

Para um regime  de juros simples, \[
    5 \cdot  VP = VP \cdot \left( 1 + 0,28\cdot n \right) \implies n = \frac{4}{0,28} = 14,29\text{ anos}
.\] 

\exercise{4}

Para os valores fornecidos, temos \[
10.000 = VP \cdot 1,12 ^{2,25}
,\] onde $2,25$ representa 2 anos e 3 meses. Assim, podemos calcular que deve ser aplicado um montante de R\$ 7.749,25.

\exercise{5}

\[
VF = 10,000 \cdot 1,03387 ^{7} = \text{R\$ }12.625,88
.\] 

\exercise{6}

Pelo enunciado, sabemos que \[
VF = 10.000 = VP \cdot 1,015^{\frac{21}{30}}
,\] o que implica em um montante aplicado de R\$ 9.896,32.

\exercise{7}

Considerando o ano comercial como 360 dias, temos \[
VF = 10.000 \cdot 1,15^{\frac{18}{360}} = \text{R\$ }10.070,13
.\] 

\exercise{8}

Sabemos que o rendimento pode ser calculado como \[
    J = VP\cdot (1+i)^{n} - VP
.\] Queremos que $J = VP$, portanto  \[
VP = VP\cdot (1+i)^{n} - VP \implies 2 = \left( 1+i \right) ^{n}
,\] portanto, com a taxa enunciada, \[
n = \log_{1,05}2 = 14,21\text{ meses}
.\] 

\exercise{9}

\[
J = 27.473 = 83.000 \cdot 0,10 \cdot n \implies n = 3,31 \text{ anos}
.\] 

\exercise{10}

Entre 3 de abril e 6 de junho percorreram-se 2 meses e 3 dias, ou seja, 2,1 meses. Assim, podemos calcular \[
J = 7.000\cdot 0,01^{2,1} = \text{R\$ }147,81
.\] 

\exercise{11}

\subexercise{a}

\begin{align*}
    & 100.000 = VP\cdot \left(1+0,13  \right) ^{4\cdot 4} \\
    & \implies VP = \text{R\$ }14.149,62
.\end{align*}

\subexercise{b}

\begin{align*}
    & 100.000 = VP\cdot \left(1+0,18  \right) ^{4} \\
    & \implies VP = \text{R\$ }51.589,89
.\end{align*}

\subexercise{c}

\begin{align*}
    & 100.000 = VP\cdot \left(1+0,14  \right) ^{2 \cdot 4} \\
    & \implies VP = \text{R\$ }35.055,91
.\end{align*}

\subexercise{d}

\begin{align*}
    & 100.000 = VP\cdot \left(1+0,12  \right) ^{12 \cdot 4} \\
    & \implies VP = \text{R\$ }434,05
.\end{align*}

\exercise{12}

Queremos que \[
2\cdot VP = VP\cdot \left( 1 + i \right) ^{\frac{42}{12}}
,\] o que nos permite chegar em \[
i = 2^{\frac{12}{42}} - 1 = 21,90 \text{ \%}
\] 

\exercise{13}

\[
6.161,85 = 22.000 \cdot \left( 1,025 ^{n} - 1 \right)  \implies n = 10\text{ meses}
.\] 

\exercise{14}

\[
6.947,21 = VP \cdot \left( 1,095 \right) ^{2 + \frac{4}{12}} \implies VP = \text{R\$ }5.621,40
.\] 

\end{document}
