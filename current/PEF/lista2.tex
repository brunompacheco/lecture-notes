\documentclass[a4paper]{report}
\input{./preamble.tex}
 
\begin{document}
 
\title{Lista 2}
\author{Bruno M. Pacheco (16100865)\\
EPS 5211 - Programação Econômica e Financeira}
 
\maketitle
 
\exercise{1}

\[
VF = 3.000\cdot \left( 1 + 0.06 \cdot  5 \right) \implies VF = \text{R\$ }3.900,00
.\] 

\exercise{2}

\[
    VF = 750 = VP\cdot \left( 1 + 0,1 \cdot 5 \right) \implies VP = \text{R\$ }500,00
.\] 

\exercise{3}

\[
VF = 400 = 200 \cdot \left( 1 + i * 5 \right) \implies i = 20\text{ \%}
.\] 

\exercise{4}

\[
VF = 10.000\cdot \left( 1 + 0,12 \right) = \text{R\$ }11.200,00
.\] 

\exercise{5}

\[
VF = 10.000 = VP \left( 1 + 0,015 \cdot 12 \right) \implies VP = \text{R\$ }8.474,58
\] 

\exercise{6}

\[
VF = 2\cdot VP = VP \left( 1 + 0,02 \cdot n \right) \implies n = 50\text{ meses}
\] 

\exercise{7}

\[
VF = 134,00 = 68,00 \left( 1 + 0,02 \cdot n \right) \implies n = 48,5\text{ meses}
\] 

\exercise{8}

\[
    VF = 1.250,00 = 1.000,00 \left( 1 + i\cdot 20 \right) \implies i = 1,3\text{ \%}
\] 

\exercise{9}

\[
J = 10.000,00 \cdot 0,012 \cdot \frac{18}{30} = \text{R\$ }72,00
\] 

\exercise{10}

\[
J = 100.000,00 \cdot 0,03 \cdot \left( 5 \cdot 3 \right) = \text{R\$ }45.000,00
\] 

\exercise{11}

\begin{align*}
    J &=20.000 \cdot 0,06 \cdot \frac{69}{30} = \text{R\$ }2.760,00 \\
    VF &= VP + J = \text{R\$ }22.760,00
.\end{align*}

\exercise{12}

\[
J = 400,00 = 1.000,00 \cdot 0,02 \cdot n \implies n = 20\text{ meses}
\] 

\exercise{13}

\[
J = 18.000,00 = 200.000,00 \cdot i \cdot \frac{72}{360} \implies i = 45\text{ \%}
\] 

\exercise{14}

\[
    VF = 60.000,00 = VP \left( 1 + 0,03\cdot (3\cdot 2) \right) \implies VP = \text{R\$ }50.847,045
\] 

\end{document}
