\documentclass[a4paper]{report}
\input{./preamble.tex}
 
\begin{document}
 
\title{Lista 4}
\author{Bruno M. Pacheco (16100865)\\
EPS 5211 - Programação Econômica e Financeira}
 
\maketitle
 
\exercise{1}

\subexercise{a}

\[
\left( 1 + i_a \right) = \left( 1 + \frac{i_N}{12} \right)^{12} = 1,0075^{12} \implies i_a = 0,0938
,\] onde $i_N$ é a taxa nominal anual.

\subexercise{b}

\[
\left( 1 + i_a \right) = \left( 1 + \frac{i_N}{4} \right)^{4} = 1,0225^{4} \implies i_a = 0,0930
.\] 

\subexercise{c}

\[
\left( 1 + i_a \right) = \left( 1 + \frac{i_N}{2} \right)^{2} = 1,0450^{2} \implies i_a = 0,0920
.\] 

\exercise{2}

\[
\left( 1 + i_t \right) = \left( 1 + \frac{i_N}{12} \right)^{3} = 1,0125^{3} \implies i_t = 0,0380
.\] 

\exercise{3}

\[
\left( 1 + i_m \right)^{3} = \left( 1 + \frac{i_N}{4} \right) = 1,025 \implies i_m = 0,008265
.\] 

\exercise{4}

\[
VF = 1.000 \left( 1 + \frac{i_N}{12}  \right) ^{24} = 1000 \left( 1,0075 \right) ^{24} = 1.196,41
.\] 

\exercise{5}

\[
\left( 1 + i_m \right)^{3} = \left( 1 + \frac{i_N}{4} \right) = 1,02125 \implies i_m = 0,007034
.\] 

\exercise{6}

\[
\left( 1 + i_t \right) = \left( 1 + \frac{i_N}{12} \right)^{3} = 1,0095^{3} \implies i_t = 0,0288
\] e \[
\left( 1 + i_a \right) = \left( 1 + \frac{i_N}{12} \right)^{12} = 1,0095^{12} \implies i_t = 0,1201
.\] 

\exercise{7}

\[
VF = 1.000 \left( 1 + \frac{i_N}{12}  \right) ^{24} = 1000 \left( 1,0085 \right) ^{24} = 1.225,24
.\] 

\exercise{8}

Para a segunda aplicação, \[
67.895,08 = VF_1 \left( 1 + \frac{0,60}{4} \right)^{4} \implies VF_1 = 38.819,23
.\] Assim, a primeira aplicação \[
VF_1 = 38.819,23 = VP \left( 1 + \frac{0,48}{12} \right) ^{12} \implies VP = 24.240,76
.\] 

\end{document}
