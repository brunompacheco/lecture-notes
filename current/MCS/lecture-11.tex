\lecture{11}{18.08.2020}{Nonlinear Control: Sliding Mode and Synergetic Control}

We will cover Sliding Mode Control (SMC) and Synergetic Control approaches to nonlinear control. Both are based on the idea of enforcing a \emph{linear relation} between the state variables to simplify the solution. This way, the (state of the) system is forced to move to a certain locus/manifold $\psi(x)=0$. That is, through applying some constraints, we map the original system to a simpler one.

Note we are not talking about linearizing the system through some approximation, but we are adding to the system so the result is linear.

\section*{Variable Structure System}

In traditional feedback control, the structure of the feedback is fixed, that is, our input $u(t)$ which must be defined based on the state of the system $x(t)$ is set to be $u(t)=f\left( x(t) \right) =k^{T}x(t)$. In the \textbf{Variable Structure System (VSS)}, the system is allowed to change its structure at any instant. So the design problem becomes the selection of parameters of each structure and the definition of the switching logic.

So in a double-integrator system (second-order) in a closed-loop, given by \[
    \ddot{x}(t)=-\Psi x(t)
\] in which $\Psi$ can alternate between $\Psi=a_1^{2},\Psi=a_2^{2}$, where $0<a_2<1<a_1$, we know that the solution in both cases is not asymptotically stable. Even more, both are of the form of a complex conjugate with null real part (see Note 1 below).

\begin{note}
    A solution for \[
    \ddot{x}(t) = -K x(t)
    \] , or any similar system, can be easily found by assuming that the solution may be in the form of $e^{\lambda t}$. Then we have \[
    \lambda^{2}e^{\lambda t} = -K e^{\lambda t}
    \] which implies \[
    \lambda^{2}+K = 0 \implies \lambda = \pm i\sqrt{K}
    \].
\end{note}

If we understand the state variables of the system as $x_1=x$ and $x_2=\dot{x}$, then our system becomes \[
\begin{cases}
    \dot{x}_1(t)=x_2(t) \\
    \dot{x}_2(t)=-\Psi x_1(t)
\end{cases}
\] and if we define a switching of \[
\Psi = \begin{cases}
    a_1^{2}, & \dot{x}x>0 \\
    a_2^{2}, & \dot{x}x < 0
\end{cases}
\] we can check the stability through Lyapunov. So defining the Lyapunov function as \[
V = x_1^{2} + x_2^{2}
\] we have 
\begin{align*}
    \dot{V} &= 2 \dot{x}_1 x_1 + 2 \dot{x}_2 x_2 \\
	    &= 2 x_2 x_1 - 2 \Psi x_1 x_2 \\
	    &= 2x_1x_2(1-\Psi) \\
    \implies \dot{V} &= \begin{cases}
	2x_1x_2(1-a_1^{2}), & \dot{x}x > 0 \\
	2x_1x_2(1-a_2^{2}), & \dot{x}x < 0 \\
    \end{cases}
\end{align*}
or, using the state variables, \[
    \dot{V} = \begin{cases}
	2x_1x_2(1-a_1^{2}), & x_2x_1 > 0 \\
	2x_1x_2(1-a_2^{2}), & x_2x_1 < 0 \\
    \end{cases}
\] which is always negative, therefore, the trajectory moves towards the equilibrium point (the origin).

\section*{Sliding Mode Control}

The SMC approach aims to keep the states close to a region where the switching happens.

So keeping with the previous example with \[
\ddot{x}(t) = u(t)
\] , we can choose \[
\begin{cases}
    \dot{x}_1 = x_2 \\
    \dot{x}_2 = u(t)
\end{cases}
\] where \[
u(t) = -sign(x) = \begin{cases}
    -1, & kx_1 + x_2 > 0 \\
    1, & kx_1+x_2 < 0
\end{cases}
\]. Note that the switching function represents a line with slope $-k$ in the phase plot. Therefore, if we consider a Lyapunov function  \[
V(x_1,x_2) = \frac{1}{2} \left( kx_1 + x_2 \right) ^{2}
\] (which is a classic quadratic function but on the plane that we defined for the state variables), we have
\begin{align*}
    \dot{V} &= \left( kx_1 + x_2 \right) * \left( k\dot{x}_1 + \dot{x}_2 \right) \\
	    &=  \left( kx_1 + x_2 \right) * \left( kx_2 + u \right) \\
	    &=  \left( kx_1 + x_2 \right) * kx_2 + \left( kx_1 + x_2 \right) * u \\
\end{align*}
and also, we have that $\left( kx_1 + x_2 \right) * u = - \|\left( kx_1 + x_2 \right)\|$ by the definition of $u$, thus  \[
    \dot{V} \le \|\left( kx_1 + x_2 \right)\|(k|x_2| - 1)
\] which is smaller than zero whenever $|x_2| < \frac{1}{k}$. So now we know that, given the aforementioned condition, the system converges to the plane defined, so we \textbf{guarantee a linear relation between the two state variables}.

\subsection*{Problem Formulation?}

Given a surface $s$ defined over the state variables, so $s(x) = 0$, one condition is that the surface must be \textbf{locally attractive}, that is
\begin{align*}
    V &= \frac{1}{2}s(x)^{2} \\
    \implies \dot{V} &=\nabla V \cdot \dot{s}(x) \\
		     &= s\dot{s} < 0
\end{align*}
and $s\dot{s}<0$ (reachability condition)  means that, "above" the surface, or when $s > 0$, by the condition, $\dot{s}<0$, which results in the attraction to the surface. The same happens for $s<0$. 

\subsection*{Summary}

We saw a use of the VSS with a switching structure that ensures the state variables are always in a neighborhood of the defined plane, which guarantees a linear relation between two state variables.

\section*{Synergetic Control}

We assume that our system can be formulated as \[
    \dot{x} = f(x) + g(x)u
\] and try to achieve a simpler system by bounding it to a manifold $\Psi$ such that \[
T \dot{\Psi}(x) + \Psi(x) = 0
\]. So we look for a function $u(x)$ that satisfies this condition. Also, we define the value of $T$, so the speed in which the system approaches the surface.

The biggest difference from SMC is that we do not define a discrete set of input approaches, so the more the switching speed requirement increase, the more we approach synergetic control.

So the challenges are:
\begin{itemize}
    \item Is the manifold reachable?
    \item Is the manifold stable?
    \item If there is a desired specific point in the manifold, is it reachable through the manifold?
\end{itemize}

