\lecture{9}{01.03.2021}{Subvariedades de $\R^{n}$}

Nesta segunda parte da matéria, trataremos de
\begin{enumerate}
    \item Subvariedades do $\R^{n}$ (com e sem bordo)
    \item Formas alternadas
\end{enumerate}

E, depois, falaremos de integração de formas diferenciais em subvariedades + Teorema de Stokes.

Utilizaremos o Spivak e o Análise Real vol. 3, do Elon, além das notas de aula do Martin Weilandt, que são uma tradução das notas da professora Helga Baum.

\section*{Subvariedades de $\R^{n}$}
ou "superfícies", via Elon.

O exemplo mais clássico de subvariedade é o gráfico de uma função, ou seja, dada uma função $f: U\subseteq\R^{n} \longrightarrow \R^{m}$, seu gráfico é o conjunto \[
Graph\left( f \right) = \left\{ \left( x, f\left( x \right)  \right)  \in \R^{m+n} : x\in U\right\} 
\] de medida nula em $\R^{n+m}$.

\begin{definition}
    Seja $f: U\subseteq\R^{n} \longrightarrow \R^{m}$, $c\in \R^{m}$ é um \emph{valor regular} sse $x\in f^{-1}\left( c \right) $, \[
    Df_x: \R^{n} \longrightarrow \R^{m}
    \] é sobrejetora.
\end{definition}

Nossa definição de subvariedade aqui será diferente da definição vista em Cálculo III, se aterá somente a subvariedades de $\R^{n}$ (não inclui espaços de Banach), e incluirá o caso com bordo.

\subsection*{Digressão: semi-espaço superior $\mathbb{H}^{n}$}

Fixe $n\in \N^*$. Por definição, \[
\mathbb{H}^{n} := \left\{ \left( x_1,\ldots,x_n \right) \in \R^{n}: x_n \ge 0 \right\} 
.\] Além disso, \[
\partial \mathbb{H}^{n}:= \left\{ \left( x_1,\ldots,x_n \right) \in \R^{n}: x_n = 0 \right\} 
.\] Veja que essa definição coincide com a definição de fronteira topológica de $\mathbb{H}^{n}$. Também, \[
int \mathbb{H}^{n} = \left\{ \left( x_1,\ldots,x_n \right) \in \mathbb{H}^{n}: x_n > 0 \right\} 
.\] Dessa forma, também é válido que $\mathbb{H}^{n}= int \mathbb{H}^{n} \cup \partial \mathbb{H}^{n}$.

Veja que um aberto em $\mathbb{H}^{n}$ é $U\subseteq\mathbb{H}^{n}$ tal que \[
U = \widetilde{U} \cap \mathbb{H}^{n}
,\] onde $\widetilde{U}$ é um aberto de $\R^{n}$. Ou seja, isso é $\mathbb{H}^{n}$ munido da topologia induzida por $\R^{n}$. Dessa forma, temos dois tipos de abertos de $\mathbb{H}^{n}$:
\begin{itemize}
    \item $U\subseteq int \mathbb{H}^{n}$, logo, $U$ é um aberto de $\R^{n}$;
    \item $U\cap \partial \mathbb{H}^{n} \neq  \O$, logo, não é um aberto de $\R^{n}$.
\end{itemize}

\begin{definition}
    $f: U\subseteq \mathbb{H}^{n} \longrightarrow \R^{m}$, $U$ aberto em $\mathbb{H}^{n}$, é de classe $C^{k}$ se $\forall x_0\in U$, $\exists \widetilde{U}\subseteq\R^{n}$ aberto de $\R^{n}$ com $x_0\in \widetilde{U}$ e $\widetilde{g}: \widetilde{U} \longrightarrow \R^{m}$ de classe $C^{k}$ (no sentido usual) tal que \[
    \widetilde{U} \cap \mathbb{H}^{n} \subseteq U\text{ e }\widetilde{g}\Big|_{\widetilde{U}\cap \mathbb{H}^{n}} = f\Big|_{\widetilde{U}\cap \mathbb{H}^{n}}
    .\] 
\end{definition}

\begin{remark}
    Seja $g: V\subseteq\R^{n} \longrightarrow \R^{m}$, $V$ aberto de $\R^{n}$, de classe $C^{k}$ no sentido usual; se $U:=V\cap \mathbb{H}^{n}\neq \O$, então \[
    f:= g\Big|_U: U\subseteq \mathbb{H}^{n} \longrightarrow \R^{n}
    \] é de classe $C^{k}$ no sentido de $\mathbb{H}^{n}$.
\end{remark}

\begin{prop}
    Seja $f: U\subseteq\mathbb{H}^{n} \longrightarrow \R^{m}$ de classe $C^{k}$. Então
    \begin{enumerate}
        \item $f$ é contínua;
	\item se $x_0\in U$, $\widetilde{g}: \widetilde{U}\subseteq\R^{n} \longrightarrow \R^{m}$, $\hat{g}: \hat{U} \subseteq\R^{n} \longrightarrow \R^{m} $ são de classe $C^{k}$ tal que $x_0\in \widetilde{U}\cap \hat{U}$, $\widetilde{U}\cap \mathbb{H}^{n}, \hat{U}\cap \mathbb{H}^{n} \subseteq U$ e \[
	f\Big|_{\hat{U}\cap \widetilde{U}\cap \mathbb{H}^{n}} =\widetilde{g}\Big|_{\hat{U}\cap \widetilde{U}\cap \mathbb{H}^{n}} = \hat{g}\Big|_{\hat{U}\cap \widetilde{U}\cap \mathbb{H}^{n}}
	,\] então \[
	D\widetilde{g}_{x_0} = D\hat{g}_{x_0}
	.\] 
    \end{enumerate}
\end{prop}

\begin{demo}
    (1) Fixe $x_0\in U$. Seja $\widetilde{g}: \widetilde{U} \subseteq\R^{n} \longrightarrow \R^{m} $ de classe $C^{k}$ com $\widetilde{U}\cap \mathbb{H}^{n}\subseteq U$ e \[
    \widetilde{g}\Big|_{\widetilde{U}\cap \mathbb{H}^{n}} = f\Big|_{\widetilde{U}\cap \mathbb{H}^{n}}
.\] $\widetilde{g}$ é contínua (em $x_0$). Portanto $f$ é contínua em $x_0$.

    (2) Escolha $\hat{g}: \hat{U}\subseteq\R^{n} \longrightarrow \R^{m}$ de classe $C^{k}$, $\hat{U}\cap \mathbb{H}^{n}\subseteq U$, \[
    \hat{g}\Big|_{\hat{U}\cap \mathbb{H}^{n}} = f\Big|_{hat{U}\cap \mathbb{H}^{n}}
    .\] Desejo provar que \[
    \frac{\partial \hat{g}_i}{\partial x_j} \left( x_0 \right) = \frac{\partial \widetilde{g}_i}{\partial x_j} \left( x_0 \right) , i=1,\ldots,m ; j=1,\ldots,n
    .\] Veja que \[
    \frac{\partial \widetilde{g}_i}{\partial x_j} \left( x_0 \right) := \lim_{t \to 0} \frac{\widetilde{g}_i\left( x_0 + t.e_j\right) -\widetilde{g}\left( x_0 \right)  }{t}
    .\] Por definição, já temos que \[
    \widetilde{g}\left( x_0 \right) = \hat{g}\left( x_0 \right) =f\left( x_0 \right) 
    ,\] portanto, nos basta provar o outro componente, ou seja, a equivalência das funções na vizinhança de $x_0$. Agora, se $x_0\in int \mathbb{H}^{n}$ , então $\exists \epsilon>0$ tal que \[
    B_\epsilon^{\R^{n}}\left( x_0 \right) \subseteq U\cap int \mathbb{H}^{n}
    .\]  Como $\widetilde{U}\cap hat{U}$ é aberto de $\R^{n}$, diminuindo $\epsilon$, se necessário, podemos assumir que \[
    B_\epsilon^{\R^{n}}\left( x_0 \right) \subseteq\widetilde{U}\cap hat{U}\cap \mathbb{H}^{n}\subseteq U
    .\] Se $0<\left| t \right| <\epsilon $, então, $x_0+t.e_j \in \hat{U}\cap \widetilde{U}\cap \mathbb{H}^{n}$ e, portanto, \[
    \widetilde{g}_i\left( x_0+t.e_j \right) = f\left( x_0+t.e_j \right) = \hat{g}\left( x_0 + t.e_j \right) 
    ,\] portanto, não há nada a provar. Assuma, então, $x_0\in U\cap \partial \mathbb{H}^{n}$. Então, $x_0$ é da forma $\left( x_0^{1},\ldots,x_0^{n-1},0 \right) $. Se $j\neq n$, então \[
    x_0+t.e_j \in \partial \mathbb{H}^{n}
    .\] Como ambos $\hat{U}, \widetilde{U}$ são abertos, é possível escolher $\delta > 0$ tal que \[
    \left| t \right| < \delta \implies x_0+ t.e_j\in \hat{U}\cap \widetilde{U}\cap \partial \mathbb{H}^{n} \subseteq U\cap \widetilde{U}\cap \hat{U}
    ,\] logo, \[
    f\left( x_0+t.e_j \right) = \widetilde{g}\left( x_0+t.e_j \right) = \hat{g}\left( x_0+t.e_j \right) 
    \] e, portanto, as derivadas parciais coincidem. Finalmente, considere $j=n$. Por um lado \[
    \frac{\partial \widetilde{g}_i}{\partial x_n} \left( x_0 \right) = \lim_{t \to 0} \frac{\widetilde{g}_i\left( x_0+t.e_n \right) -\widetilde{g}_i\left(  x_0\right) }{t}
\] é um limite que existe, uma vez que as funções são de classe $C^{k}$. Além disso, é o mesmo que \[
= \lim_{t \to 0^{+}} \frac{\widetilde{g}_i\left( x_0+t.e_n \right) -\hat{g}_i\left(  x_0\right) }{t}
.\] Podemos, então, fazer $t$ pequeno o suficiente de forma que $x_0+t.e_n\in \widetilde{U}\cap \widetilde{V}\cap \mathbb{H}^{n}$. Assim, \[
\widetilde{g}_i\left( x_0+t.e_n \right) = \hat{g}_i\left( x_0+t.e_n \right)
.\] Como $\hat{g}$ também é contínua, seu limite existe e o limite à esquerda e à direita coincidem. Portanto, mesmo para $j=n$, as derivadas coincidem.
\end{demo}

\begin{definition}
    Seja $f: U\subseteq\mathbb{H}^{n} \longrightarrow \R^{m}$ de classe $C^{k}$. Dado $x_0\in U$ e $\widetilde{g}: \widetilde{U}\subseteq\R^{n} \longrightarrow \R^{m}$ $C^{k}$ com $x_0\in \widetilde{U}$, $\widetilde{U}\cap \mathbb{H}^{n} \subseteq U$ e $\widetilde{g}\Big|_{\widetilde{U}\cap \mathbb{H}^{n}} = f\Big|_{\widetilde{U}\cap \mathbb{H}^{n}}$, então \[
    Df_{x_0} := D\widetilde{g}_{x_0} : \R^{n} \longrightarrow \R^{m}
    .\] 
\end{definition}

Veja que pela proposição anterior, essa definição se mostra independente da função $\widetilde{g}$.

\begin{problem}
    Sejam $f, g: U\subseteq\mathbb{H}^{n} \longrightarrow \R^{m}$ e $h: U\subseteq \mathbb{H}^{n} \longrightarrow \R$, todas de classe $C^{k}$. Prove que:
    \begin{enumerate}
        \item $f+g$ e $f . h$, operações ponto-a-ponto, são de classe $C^{k}$;
	\item Suponha $\varphi : W\subseteq\mathbb{H}^{m} \longrightarrow \R^{p}$ de classe $C^{k}$ e $f\left( U \right) \subseteq W$. Então $\varphi \circ f: U\subseteq\mathbb{H}^{n} \longrightarrow \R^{p}$ é de classe $C^{k}$.
    \end{enumerate}
\end{problem}

\begin{definition}
    Sejam $U,V$ abertos de $\mathbb{H}^{n}$ e $\varphi : U \longrightarrow V$ bijeção. $\varphi $ é um \emph{difeomorfismo de classe $C^{k}$} se $\varphi : U \longrightarrow \R^{n}$ e $\varphi ^{-1}: V  \longrightarrow \R^{n}$ são de classe $C^{k}$.
\end{definition}

\begin{theorem}
    Sejam $U,V \subseteq \mathbb{H}^{n}$ abertos e $\varphi : U \longrightarrow V$ um difeomorfismo de classe $C^{k}$. Então \[
    \varphi \left( U\cap \partial \mathbb{H}^{n} \right) = V\cap \partial \mathbb{H}^{n}
    .\] 
\end{theorem}

\begin{proof}
    Seja $x\in U\cap int \mathbb{H}^{n}$. Então, podemos tomar um aberto $\widetilde{U} \ni x$ de forma que $\widetilde{U}\subseteq U$ e $\varphi ( \widetilde{U} ) \subseteq V \cap int \mathbb{H}^{n}$, uma vez que $\varphi' $ é um isomorfismo. Ou seja, \[
    x\in U\cap int \mathbb{H}^{n} \implies \varphi \left( x \right) \not\in V\cap \partial \mathbb{H}^{n}
    .\] Veja que o mesmo pode ser feito para $\varphi ^{-1}$, ou seja, \[
    \varphi^{-1} \left( V\cap int \mathbb{H}^{n} \right) \subseteq U\cap int \mathbb{H}^{n}
    .\] Dessa forma, é fácil ver que 
    \begin{align*}
	\varphi \left( U\setminus int \mathbb{H}^{n} \right) \subseteq V\setminus int \mathbb{H}^{n} &\implies \varphi \left( U\cap \partial \mathbb{H}^{n} \right) \subseteq V\cap \partial  \mathbb{H}^{n} \\
	\varphi^{-1} \left( V\setminus int \mathbb{H}^{n} \right) \subseteq U\setminus int \mathbb{H}^{n} &\implies \varphi^{-1} \left( V\cap \partial \mathbb{H}^{n} \right) \subseteq U\cap \partial  \mathbb{H}^{n} \\
	 \therefore \varphi \left( U\cap \partial \mathbb{H}^{n} \right) &= V\cap \partial \mathbb{H}^{n}
    .\end{align*}
\end{proof}

\subsection*{Subvariedades com bordo}

Utilizamos $\mathbb{H}^{n}$ como espaço de parametrização de subvariedades com bordo.

\begin{definition}
    (Parametrizações locais) Seja $M\subseteq\R^{n}$. Uma \emph{parametrização local de classe $C^{k}$ e dimensão $m$} de $M$ é uma aplicação $\varphi : U_0\subseteq\mathbb{H}^{m} \longrightarrow \R^{n}$ de classe $C^{k}$ tal que 
    \begin{enumerate}
	\item $\forall x\in U_0$ um aberto de $\mathbb{H}^{m}$, $D\varphi _{x}: \R^{m} \longrightarrow \R^{n}$ é injetora (\emph{imersão})
	\item $\varphi \left( U_0 \right) \subseteq M$ é um \emph{aberto em $M$}, i.e., $\exists \widetilde{U}\subseteq\R^{n}$ aberto tal que $\varphi \left( U_0 \right) =M\cap \widetilde{U}$
	\item $\varphi : U_0 \longrightarrow \varphi \left( U_0 \right) \subseteq M$ é um homeomorfismo, i.e., uma bijeção contínua com inversa contínua.
    \end{enumerate}
\end{definition}

\begin{definition}
    (Subvariedade de $\R^{n}$) Um conjunto $M\subseteq\R^{n}$ é \emph{subvariedade de dimensão $m$ e classe $C^{k}$} (de $\R^{n}$) se $\forall p\in M, \exists \varphi : U_0\subseteq\mathbb{H}^{m} \longrightarrow M$ uma parametrização de classe $C^{k}$ e dimensão $m$ com $p\in \varphi \left( U_0 \right) $.
\end{definition}

Veja que para subvariedade na definição usual, com parametrizações a partir de abertos de $\R^{m}$, podem ser parametrizadas em $\mathbb{H}^{m}$, se utilizando de uma translação para jogar os abertos em $int \mathbb{H}^{m}$.

\begin{definition}
    Seja $M\subseteq\R^{n}$ uma subvariedade de classe $C^{k}$ e dimensão $m$. Uma parametrização $\varphi : U_0\subseteq\mathbb{H}^{m} \longrightarrow M$ em um ponto $p \in M$ é dita:
    \begin{itemize}
        \item \emph{interior} se $\varphi ^{-1}\left( p \right) \in  int \, \mathbb{H}^{m}$;
	\item \emph{de bordo} se $\varphi ^{-1}\left( p \right) \in \partial \mathbb{H}^{m}$.
    \end{itemize}
\end{definition}

\begin{definition}
    (Borde e interior) Seja $M\subseteq\R^{n}$ uma subvariedade de dimensão $m$ e classe $C^{k}$. O \emph{bordo de $M$} \[
    \partial M := \left\{ p\in M: \exists \varphi : U_0\subseteq\mathbb{H}^{m} \longrightarrow M \text{ de bordo com }\varphi \left( U_0 \right) \ni p \right\} 
    .\] O \emph{interior de $M$} \[
    int M := \left\{ p \in M : \exists \varphi : U_0\subseteq\mathbb{H}^{m} \longrightarrow M \text{ interior com } \varphi \left( U_0 \right) \ni p\right\} 
    .\] 
\end{definition}

