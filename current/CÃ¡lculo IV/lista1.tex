\documentclass[a4paper]{report}
% Some basic packages
\usepackage[utf8]{inputenc}
\usepackage[T1]{fontenc}
\usepackage{textcomp}
\usepackage[english]{babel}
\usepackage{url}
\usepackage{graphicx}
\usepackage{float}
\usepackage{booktabs}
\usepackage{enumitem}

\pdfminorversion=7

% Don't indent paragraphs, leave some space between them
\usepackage{parskip}

% Hide page number when page is empty
\usepackage{emptypage}
\usepackage{subcaption}
\usepackage{multicol}
\usepackage{xcolor}

% Other font I sometimes use.
% \usepackage{cmbright}

% Math stuff
\usepackage{amsmath, amsfonts, mathtools, amsthm, amssymb}
% Fancy script capitals
\usepackage{mathrsfs}
\usepackage{cancel}
% Bold math
\usepackage{bm}
% Some shortcuts
\newcommand\N{\ensuremath{\mathbb{N}}}
\newcommand\R{\ensuremath{\mathbb{R}}}
\newcommand\Z{\ensuremath{\mathbb{Z}}}
\renewcommand\O{\ensuremath{\emptyset}}
\newcommand\Q{\ensuremath{\mathbb{Q}}}
\newcommand\C{\ensuremath{\mathbb{C}}}
\renewcommand\L{\ensuremath{\mathcal{L}}}

% Package for Petri Net drawing
\usepackage[version=0.96]{pgf}
\usepackage{tikz}
\usetikzlibrary{arrows,shapes,automata,petri}
\usepackage{tikzit}
\input{petri_nets_style.tikzstyles}

% Easily typeset systems of equations (French package)
\usepackage{systeme}

% Put x \to \infty below \lim
\let\svlim\lim\def\lim{\svlim\limits}

%Make implies and impliedby shorter
\let\implies\Rightarrow
\let\impliedby\Leftarrow
\let\iff\Leftrightarrow
\let\epsilon\varepsilon

% Add \contra symbol to denote contradiction
\usepackage{stmaryrd} % for \lightning
\newcommand\contra{\scalebox{1.5}{$\lightning$}}

% \let\phi\varphi

% Command for short corrections
% Usage: 1+1=\correct{3}{2}

\definecolor{correct}{HTML}{009900}
\newcommand\correct[2]{\ensuremath{\:}{\color{red}{#1}}\ensuremath{\to }{\color{correct}{#2}}\ensuremath{\:}}
\newcommand\green[1]{{\color{correct}{#1}}}

% horizontal rule
\newcommand\hr{
    \noindent\rule[0.5ex]{\linewidth}{0.5pt}
}

% hide parts
\newcommand\hide[1]{}

% si unitx
\usepackage{siunitx}
\sisetup{locale = FR}

% Environments
\makeatother
% For box around Definition, Theorem, \ldots
\usepackage{mdframed}
\mdfsetup{skipabove=1em,skipbelow=0em}
\theoremstyle{definition}
\newmdtheoremenv[nobreak=true]{definitie}{Definitie}
\newmdtheoremenv[nobreak=true]{eigenschap}{Eigenschap}
\newmdtheoremenv[nobreak=true]{gevolg}{Gevolg}
\newmdtheoremenv[nobreak=true]{lemma}{Lemma}
\newmdtheoremenv[nobreak=true]{propositie}{Propositie}
\newmdtheoremenv[nobreak=true]{stelling}{Stelling}
\newmdtheoremenv[nobreak=true]{wet}{Wet}
\newmdtheoremenv[nobreak=true]{postulaat}{Postulaat}
\newmdtheoremenv{conclusie}{Conclusie}
\newmdtheoremenv{toemaatje}{Toemaatje}
\newmdtheoremenv{vermoeden}{Vermoeden}
\newtheorem*{herhaling}{Herhaling}
\newtheorem*{intermezzo}{Intermezzo}
\newtheorem*{notatie}{Notatie}
\newtheorem*{observatie}{Observatie}
\newtheorem*{exe}{Exercise}
\newtheorem*{opmerking}{Opmerking}
\newtheorem*{praktisch}{Praktisch}
\newtheorem*{probleem}{Probleem}
\newtheorem*{terminologie}{Terminologie}
\newtheorem*{toepassing}{Toepassing}
\newtheorem*{uovt}{UOVT}
\newtheorem*{vb}{Voorbeeld}
\newtheorem*{vraag}{Vraag}

\newmdtheoremenv[nobreak=true]{definition}{Definition}
\newtheorem*{eg}{Example}
\newtheorem*{notation}{Notation}
\newtheorem*{previouslyseen}{As previously seen}
\newtheorem*{remark}{Remark}
\newtheorem*{note}{Note}
\newtheorem*{problem}{Problem}
\newtheorem*{observe}{Observe}
\newtheorem*{property}{Property}
\newtheorem*{intuition}{Intuition}
\newmdtheoremenv[nobreak=true]{prop}{Proposition}
\newmdtheoremenv[nobreak=true]{theorem}{Theorem}
\newmdtheoremenv[nobreak=true]{corollary}{Corollary}

% End example and intermezzo environments with a small diamond (just like proof
% environments end with a small square)
\usepackage{etoolbox}
\AtEndEnvironment{vb}{\null\hfill$\diamond$}%
\AtEndEnvironment{intermezzo}{\null\hfill$\diamond$}%
% \AtEndEnvironment{opmerking}{\null\hfill$\diamond$}%

% Fix some spacing
% http://tex.stackexchange.com/questions/22119/how-can-i-change-the-spacing-before-theorems-with-amsthm
\makeatletter
\def\thm@space@setup{%
  \thm@preskip=\parskip \thm@postskip=0pt
}


% Exercise 
% Usage:
% \exercise{5}
% \subexercise{1}
% \subexercise{2}
% \subexercise{3}
% gives
% Exercise 5
%   Exercise 5.1
%   Exercise 5.2
%   Exercise 5.3
\newcommand{\exercise}[1]{%
    \def\@exercise{#1}%
    \subsection*{Exercise #1}
}

\newcommand{\subexercise}[1]{%
    \subsubsection*{Exercise \@exercise.#1}
}


% \lecture starts a new lecture (les in dutch)
%
% Usage:
% \lecture{1}{di 12 feb 2019 16:00}{Inleiding}
%
% This adds a section heading with the number / title of the lecture and a
% margin paragraph with the date.

% I use \dateparts here to hide the year (2019). This way, I can easily parse
% the date of each lecture unambiguously while still having a human-friendly
% short format printed to the pdf.

\usepackage{xifthen}
\def\testdateparts#1{\dateparts#1\relax}
\def\dateparts#1 #2 #3 #4 #5\relax{
    \marginpar{\small\textsf{\mbox{#1 #2 #3 #5}}}
}

\def\@lecture{}%
\newcommand{\lecture}[3]{
    \ifthenelse{\isempty{#3}}{%
        \def\@lecture{Lecture #1}%
    }{%
        \def\@lecture{Lecture #1: #3}%
    }%
    \subsection*{\@lecture}
    \marginpar{\small\textsf{\mbox{#2}}}
}



% These are the fancy headers
\usepackage{fancyhdr}
\pagestyle{fancy}

% LE: left even
% RO: right odd
% CE, CO: center even, center odd
% My name for when I print my lecture notes to use for an open book exam.
% \fancyhead[LE,RO]{Gilles Castel}

\fancyhead[RO,LE]{\@lecture} % Right odd,  Left even
\fancyhead[RE,LO]{}          % Right even, Left odd

\fancyfoot[RO,LE]{\thepage}  % Right odd,  Left even
\fancyfoot[RE,LO]{}          % Right even, Left odd
\fancyfoot[C]{\leftmark}     % Center

\makeatother




% Todonotes and inline notes in fancy boxes
\usepackage{todonotes}
\usepackage{tcolorbox}

% Make boxes breakable
\tcbuselibrary{breakable}

% Verbetering is correction in Dutch
% Usage: 
% \begin{verbetering}
%     Lorem ipsum dolor sit amet, consetetur sadipscing elitr, sed diam nonumy eirmod
%     tempor invidunt ut labore et dolore magna aliquyam erat, sed diam voluptua. At
%     vero eos et accusam et justo duo dolores et ea rebum. Stet clita kasd gubergren,
%     no sea takimata sanctus est Lorem ipsum dolor sit amet.
% \end{verbetering}
\newenvironment{verbetering}{\begin{tcolorbox}[
    arc=0mm,
    colback=white,
    colframe=green!60!black,
    title=Opmerking,
    fonttitle=\sffamily,
    breakable
]}{\end{tcolorbox}}

% Noot is note in Dutch. Same as 'verbetering' but color of box is different
\newenvironment{noot}[1]{\begin{tcolorbox}[
    arc=0mm,
    colback=white,
    colframe=white!60!black,
    title=#1,
    fonttitle=\sffamily,
    breakable
]}{\end{tcolorbox}}




% Figure support as explained in my blog post.
\usepackage{import}
\usepackage{xifthen}
\usepackage{pdfpages}
\usepackage{transparent}
\newcommand{\incfig}[1]{%
    \def\svgwidth{\columnwidth}
    \import{./figures/}{#1.pdf_tex}
}

% Fix some stuff
% %http://tex.stackexchange.com/questions/76273/multiple-pdfs-with-page-group-included-in-a-single-page-warning
\pdfsuppresswarningpagegroup=1


% My name
\author{Bruno M. Pacheco}

 
\begin{document}
 
\title{Lista 1}
\author{Bruno M. Pacheco (16100865)\\
H-Cálculo IV}
 
\maketitle
 
\exercise{3-1}

Seja $\epsilon>0$, tome $P = P_1\times P_2$ uma partição de $X=\left[ 0,1 \right] \times \left[ 0,1 \right] $ com $P_1=\{0,\frac{1}{2}-\epsilon, \frac{1}{2},1\} $. Então, \[
\mathcal{B}\left( P \right) = B^{(1)}\cup B^{(\epsilon)}\cup B^{(2)}
,\] onde $B^{(1)}=\{B^{(1)}_i=\left[ 0,\frac{1}{2}-\epsilon \right] \times \left[ t_{2_i},t_{2_{i+1}} \right]\}$, $B^{(2)}=\{B^{(2)}_i=\left[ \frac{1}{2}, 1 \right] \times \left[ t_{2_i},t_{2_{i+1}} \right]\}$ e $B^{(\epsilon)}=\{B^{(\epsilon)}_i=\left[ \frac{1}{2}-\epsilon, \frac{1}{2} \right] \times \left[ t_{2_i},t_{2_{i+1}} \right]\}$, sendo $ \left[ t_{2_i},t_{2_{i+1}} \right] \text{ subintervalo de }P_2 $.

Assim, \[
M_{B^{(1)}_i}=m_{B^{(1)}_i} = 0, \forall B^{(1)}_i\in B^{(1)}
\] e \[
M_{B^{(2)}_i}=m_{B^{(2)}_i} = 1, \forall B^{(2)}_i\in B^{(2)}
,\] mas \[
M_{B^{(\epsilon)}_i}= 1 \text{ e }m_{B^{(\epsilon)}_i} = 0, \forall B^{(\epsilon)}_i\in B^{(\epsilon)}
.\] Também, é fácil ver que
\begin{align*}
    \sum_{B^{(1)}_i\in B^{(1)}} vol\left( B^{(1)}_i \right) &= \frac{1}{2} - \epsilon \\
    \sum_{B^{(2)}_i\in B^{(2)}} vol\left( B^{(2)}_i \right) &= \frac{1}{2} \\
    \sum_{B^{(\epsilon)}_i\in B^{(\epsilon)}} vol\left( B^{(\epsilon)}_i \right) &= \epsilon
.\end{align*}

Portanto,
\begin{align*}
    S\left( f,P \right) = &\sum_{B^{(1)}_i\in B^{(1)}} M_{B^{(1)}_i} vol\left( B^{(1)}_i \right) + \\
			  &\sum_{B^{(\epsilon)}_i\in B^{(\epsilon)}} M_{B^{(\epsilon)}_i} vol\left( B^{(\epsilon)}_i \right) + \\
&\sum_{B^{(2)}_i\in B^{(2)}} M_{B^{(2)}_i} vol\left( B^{(2)}_i \right)\\
    =& \frac{1}{2} + \epsilon
,\end{align*}
enquanto, de forma similar, \[
s\left( f,P \right) = \frac{1}{2}
.\] 

Agora, como $\epsilon$ pode ser arbitrariamente pequeno, \[
    \overline{\int}_A f = \underline{\int}_A f = \frac{1}{2}
,\] logo, $f$ é integrável como se desejava demonstrar.

\exercise{3-2}

Seja $X=\left\{ x_1,\ldots,x_N \right\} \subset A$ o conjunto dos pontos em que $g$ difere de $f$, i.e., \[
x_i\in X \iff g\left( x_i \right) \neq f\left( x_i \right) 
.\] 

Seja $\epsilon>0$, escolha $P_\epsilon$ partição de $A$ tal que \[
    \forall x_i \in X, \exists! B_i\in \mathcal{B}\left( P_\epsilon \right) \text{ tal que } x_i\in  B_i \subset B_{\left( \frac{\epsilon}{N} \right)^{\frac{1}{n}}}\left( x_i \right) 
.\] Note que $B_{\left( \frac{\epsilon}{N} \right)^{\frac{1}{n}}}\left( x_i \right) $ é uma bola de raio $\left( \frac{\epsilon}{N} \right)^{\frac{1}{n}}$ em torno de $x_i$, não um n-bloco como $B_i$. Definamos, então, $\mathcal{B}'\subset \mathcal{B}\left( P_\epsilon \right) $ o conjunto do n-blocos $B_i$ que satisfazem a condição acima.

Assim,
\begin{align*}
    S\left( g,P_\epsilon \right) &=\sum_{B\in \mathcal{B}\left( P_\epsilon \right) } M_B^{g} vol\left( B \right) \\
			&= S\left( f, P_\epsilon \right) - \sum_{B_i\in \mathcal{B}' } M_{B_i}^{f} vol\left( B_i \right) + \sum_{B_i\in \mathcal{B}' } M_{B_i}^{g} vol\left( B_i \right) \\
			&< S\left( f, P_\epsilon \right) + \epsilon\sum_{B_i\in \mathcal{B}' } \left| M_{B_i}^{g} - M_{B_i}^{f} \right|
,\end{align*}
uma vez que $B_i \subset B_{\left( \frac{\epsilon}{N} \right)^{\frac{1}{n}}}\left( \cdot  \right)  \implies vol B_i < \frac{\epsilon}{N} $. Note que $M_B^{f}$ e $M_B^{f}$ denotam o supremo das respectivas funções no n-bloco $B$. De forma análoga,
\begin{align*}
    s\left( g,P_\epsilon \right) > s\left( f, P_\epsilon \right) - \epsilon\sum_{B_i\in \mathcal{B}' } \left| m_{B_i}^{g} - m_{B_i}^{f} \right|
.\end{align*}
Como $\epsilon$ pode ser arbitrariamente pequeno, \[
\inf S\left( g \right) = \inf S\left( f \right) 
\] e, da mesma forma \[
\sup s\left( g \right) = \sup s\left( f \right) 
,\] ou seja, \[
\overline{\int}_A g = \underline{\int}_A g = \int_A f
.\] 

\exercise{3-4}

($\implies$)
Tome $\epsilon>0$. Pela integrabilidade de $f$, sabemos que $\exists Q$ partição de $A$ que refina $P$ tal que  \[
S\left( f, Q \right) -s\left( f, Q\right) < \epsilon
.\] 
Agora tome $Q' = \{t\in Q : t \in S\} $, ou seja, uma "restrição" da partição $Q$ ao n-bloco $S$. Então, é fácil ver que $\mathcal{B}\left( Q' \right) \subset \mathcal{B}\left( Q \right) $ e \[
B'\in \mathcal{Q'} \implies B'\subset S
.\] Então,
\begin{align*}
    S\left( f,Q \right) &= \sum_{B\in \mathcal{B}\left( Q \right) } M_B vol B \\
			&= \sum_{B\in \mathcal{B}\left( Q \right)\setminus\mathcal{B}\left( Q' \right)  } M_B vol B + \sum_{B'\in \mathcal{B}\left( Q' \right) } M_{B'} vol B' \\
			&= S\left( f|_{S} , Q'\right) + \sum_{B\in \mathcal{B}\left( Q \right)\setminus\mathcal{B}\left( Q' \right)  } M_B vol B
\end{align*}
e, de forma análoga, \[
			s\left( f,Q \right) = s\left( f|_{S} , Q'\right) + \sum_{B\in \mathcal{B}\left( Q \right)\setminus\mathcal{B}\left( Q' \right)  } m_B vol B
.\] Portanto, \[
S\left( f, Q \right) -s\left( f, Q\right) = S\left( f|_{S}, Q' \right) - s\left( f|_{S}, Q' \right) +\sum_{B\in \mathcal{B}\left( Q \right)\setminus\mathcal{B}\left( Q' \right)  } \left( M_B - m_B \right) vol B  < \epsilon
\] \[
\implies S\left( f|_{S}, Q' \right) - s\left( f|_{S}, Q' \right) < \epsilon
,\] ou seja, $f|_{S}$ é integrável.

($\impliedby$)
Seja $\{S_i\} _{i=1,\ldots,N}$ uma enumeração dos n-blocos de $\mathcal{B}\left( P \right) $, ou seja, \[
\bigcup_{i=1}^{N}S_i = \mathcal{B}\left( P \right) 
.\] Seja $\epsilon>0$. Então, pela integrabilidade de $f|_{S_i}, i=1,\ldots,N$, é possível escolher $\left\{ Q_i \right\} _{i=1,\ldots,N}$ partições de forma que \[
S\left( f|_{S_i},Q_i \right) - s\left( f|_{S_i},Q_i \right) < \frac{\epsilon}{N}  , i=1,\ldots,N
.\] Note que $Q=\bigcup_{i=1}^{N}Q_i$ é uma partição de $A$ e, ainda mais, $Q\subset P$.

Agora, veja que \[
\mathcal{B}\left( Q \right) = \bigcup_{i=1}^{N}\mathcal{B}\left( Q_i \right) 
,\] portanto
\begin{align*}
    S\left( f,Q \right) &= \sum_{B\in \mathcal{B}\left( Q \right) } M_B^{f} vol B \\
			&= \sum_{i=1}^{N} \sum_{B\in \mathcal{B}\left( Q_i \right) } M_B^{f|_{S_i}} vol B \\
			&= \sum_{i=1}^{N} S\left( f|_{S_i}, Q_i \right)
\end{align*}
e, de forma análoga, \[
s\left( f, Q \right) = \sum_{i=1}^{N} s\left( f|_{S_i}, Q_i \right)
.\] 
Com efeito, \[
S\left( f, Q \right) - s\left( f,Q \right) = \sum_{i=1}^{N} \left( S\left( f|_{S_i}, Q_i \right) - s\left( f|_{S_i}, Q_i \right)\right) < \epsilon
.\] 

\exercise{3-8}

Defina $B' = \left[ a_1,b_1 \right] \times \ldots\times \left[ a_n,b_n \right] $. Se $a_i<b_i, i=1,\ldots,n$, então $\exists \epsilon_0>0$ de forma que \[
b_i - a_i > \epsilon_0^{\frac{1}{n}}, i=1,\ldots,n
.\] Sendo assim, \[
vol B' > \epsilon_0
.\] Agora veja que se $B_1,\ldots,B_k$ são tais que \[
B' \subset B_1\cup \ldots\cup B_k
,\] então \[
\sum_{i=1}^{k} vol B_i > vol B' > \epsilon_0 > 0
.\] 

\exercise{3-9}

\subexercise{a}

Seja $X\subset \R^{n}$ um conjunto com conteúdo nulo. Então, $\exists B_1,\ldots,B_k$ n-blocos fechados de forma que \[
X\subset \bigcup_{i=1} ^{k}B_i
.\] Claramente, n-blocos são limitados. Isso pode ser verificado utilizando a norma do máximo, sendo a distância entre quaisquer dois elementos do  n-bloco majorada pelo tamanho de seu maior intervalo. Também temos que a união de conjuntos limitados é também um conjunto limitado. Assim sendo, $X\subset B_1\cup \ldots\cup B_k$ é um conjunto limitado. Portanto, como $med X = 0 \implies X$ é limitado, então $X$ não limitado $\implies med X \neq 0$.

\subexercise{b}

Primeiro, claramente $\N\subset \R$ é fechado. Agora, $\forall i\in \N$, defina \[
B_i = \left[  i-\frac{\epsilon}{2^{i+1}}, i+\frac{\epsilon}{2^{i+1}}\right] 
.\] Assim, \[
\N\subset \bigcup_{i\in \N} B_i
\] e \[
\sum_{i\in \N} vol B_i = \sum_{i\in \N} \frac{\epsilon}{2^{i}} < \epsilon
.\]

Entretanto, $\N$ não é limitado e, portanto, não tem conteúdo zero.

\exercise{3-10}

\subexercise{a}

Seja $C$ um conjunto de conteúdo nulo. Então $\forall \epsilon>0, \exists B_1,\ldots,B_k$ n-blocos fechados tais que \[
C \subseteq B_1\cup \ldots\cup B_k
\] e \[
\sum_{i=1}^{k} vol B_i < \epsilon
\]  Como a união de n-blocos fechados é também um conjunto fechado, necessariamente \[
\overline{C} \subseteq B_1\cup \ldots\cup B_k
,\] uma vez que o fecho de um conjunto é o menor fechado que contém o conjunto. Assim, vemos que \[
\partial C \subseteq B_1\cup \ldots\cup B_k
\] e, portanto, $\partial C$ tem conteúdo nulo.

\subexercise{b}

Tome $X=\Q\cap \left[ 0,1 \right] \subset\R$. Claramente $X$ é limitado, uma vez que $X\subset \left[ 0,1 \right] $. Ainda mais, $X$ tem medida nula pois $\Q$ tem medida nula e $X\subset \Q$. Agora, sabemos que \[
int X = \O \text{ e } \overline{X} = \left[ 0,1 \right] 
.\] Assim, \[
\partial X = \overline{X} \setminus int X = \left[ 0,1 \right] 
,\] que não possui medida nula.

\exercise{3-11}

Pelo enunciado, temos o conjunto $A\subset \left[ 0,1 \right] $ formado pela união dos intervalos abertos $\left( a_i, b_i \right) $ de forma que cada número racional no intervalo $\left( 0,1 \right)$ está contido em algum intervalo $\left( a_i, b_i \right) $. Do problema 1-18, sabemos que $\partial A = \left[ 0,1 \right] - A$.

Suponha que $\partial A$ tem medida nula e $\sum_{i=1}^{\infty} \left( b_i - a_i \right) < 1$. Tome \[
0<\epsilon < 1 - \sum_{i=1}^{\infty} \left( b_i - a_i \right)
.\] Escolha, então, $\{B_i\} _{i\in \N}$ de forma que \[
\partial A \subseteq \bigcup_{i\in \N} B_i
\] e \[
\sum_{i\in \N} vol B_i < \epsilon
.\] Então, como $\left[ 0,1 \right] = \partial A \cup A$, \[
\left[ 0,1 \right] \subseteq A \cup \bigcup_{i\in \N} B_i 
.\] Portanto, \[
vol \left[ 0,1 \right] \le \sum_{i=1}^{\infty} \left( b_i - a_i \right) + \sum_{i\in \N} vol B_i < 1
,\] uma contradição.

\exercise{3-12}

(Problema 1-30)
Sejam $x_1,\ldots,x_{n} \in \left[ a,b \right] $ distintos. Suponha, sem perda de generalidade, que $a=x_0<x_1<\ldots<x_n<x_{n+1}=b$. Veja que, como $f$ é crescente, \[
f\left( x_{i-1} \right) \le f\left( x_i \right) \le f\left( x_{i+1} \right) , i=1,\ldots,n
.\] Agora, veja que $\forall \delta>0$ tal que $\delta <\min \left\{ x_{i+1}-x_i, x_i - x_{i-1} \right\}$ \[
x,y\in B_\delta\left( x_i \right) \implies \left| f(x)-f(y) \right| \le  f\left( x_{i+1} \right) - f\left( x_{i-1} \right)
,\] $i=1,\ldots,n$. Ou seja, \[
o\left( f,x_i \right) \le f\left( x_{i+1} \right) -f\left( x_{i-1} \right) 
.\] Então 
\begin{align*}
    \sum_{i=1}^{n} o\left( f, x_i \right) &\le  \sum_{i=1}^{n} \left( f\left( x_{i+1} \right) - f\left( x_{i-1} \right)  \right) \\
					  &= f\left( x_{n+1} \right) - f\left( x_n \right) + f\left( x_1 \right)  - f\left( x_0 \right) \\
					  &\le  f\left( b \right) - f\left( a \right) \tag{*}
.\end{align*}

Veja que \[
\sum_{i=1}^{n} o\left( f,x_i \right) < f\left( b \right) -f\left( a \right) 
\] não acontece para toda função $f$ crescente, vide a função \[
f\left( x \right) = \begin{cases}
    f\left( b \right) &, x>x_j \\
    f\left( a \right) &, x\le x_j
\end{cases}
\] que, para $x_1,\ldots,x_j,\ldots,x_n \in \left[ a,b \right] $, possui \[
\sum_{i=1}^{n} o\left( f,x_i \right) = o\left( f, x_j \right) = f\left( b \right) - f\left( a \right) 
.\]

Queremos, então, mostrar que o conjunto $F=\left\{ x: f\text{ é descontínua em }x \right\}$ tem medida nula. Primeiro, defina o conjunto \[
O_n = \left\{ x : o\left( f,x \right) > \frac{f\left( b \right) -f\left( a \right) }{n} \right\} 
.\] Veja que o conjunto $O_n$ não deverá ter mais do que $n-1$ elementos, uma vez que, do contrário, \[
\sum_{x\in O_n} o\left( f,x \right) > f\left( b \right) -f\left( a \right) 
,\] impossível por (*). Agora, sabemos que $x\in \left[ a,b \right] $ tal que $f$ é descontinua em $x \iff \exists \epsilon>0$ tal que $o\left( f,x \right) > \epsilon$. Assim, vemos que $\forall x\in F, \exists N\in \N$ tal que $x\in O_N$, portanto \[
F \subseteq \bigcup_{n\in \N} O_n
\] . Como todo $O_n$ é finito, possui medida nula e, portanto, a união possui também medida nula, logo, $F$ possui medida nula.

\exercise{3-14}

Tome $P_0$ partição de $A$ de forma que \[
\{x:f\cdot g\left( x \right) = 0\} \subseteq P_0
.\] Pelo exercício 3-4, provar a integrabilidade de $f\cdot g$ é o mesmo que provar a integrabilidade de $f\cdot g$ em cada n-bloco $S\subseteq \mathcal{B}\left( P_0 \right) $. Para tanto, segmentamos a análise em dois casos que são exaustivos e mutualmente exclusivos.

($f\cdot g\left( x \right) \ge 0, \forall x\in S$)
Nesse caso, veja que, $\forall B\in \mathcal{B}\left( P \right) $, onde $P$ é uma partição de $S$, \[
    M_B^{f\cdot g} \le  M_B^{f} \cdot M_B^{g} \\
\] e \[
    m_B^{f\cdot g} \ge  m_B^{f} \cdot m_B^{g} \\
.\] Agora seja $\epsilon>0$ arbitrário. Como ambas as funções são integráveis, podemos tomar $\epsilon_0 < \sqrt{\epsilon} $ de forma que
\begin{align*}
    & S\left( f,P \right) - s\left( f,P \right) < \epsilon_0 \\
    & S\left( g,P \right) - s\left( g,P \right) < \epsilon_0
.\end{align*}
Agora, veja que
\begin{align*}
    \left( S\left( f,P \right) - s\left( f,P \right) \right) \left( S\left( g,P \right) - s\left( g,P \right) \right) &\ge \sum_{B\in \mathcal{B}\left( P \right) } \left( M_B^{f}M_B^{g} + m_B^{f}m_B^{g} - M_B^{f}m_B^{g} - m_B^{f}M_B^{g} \right) vol B \tag{*} \\
															 &\ge \sum_{B\in \mathcal{B}\left( P \right) } \left( M_B^{f}M_B^{g} - m_B^{f}M_B^{g} \right) vol B \\
															 &\ge \sum_{B\in \mathcal{B}\left( P \right) } \left( M_B^{f}M_B^{g} - m_B^{f}m_B^{g} \right) vol B \\
															 &\ge  \sum_{B\in \mathcal{B}\left( P \right) } \left( M_B^{f\cdot g} - m_B^{f\cdot g} \right) vol B \\
															 &= S\left( f\cdot g, P \right) - s\left( f\cdot g, P \right) 
.\end{align*}
Portanto, \[
S\left( f\cdot g, P \right) - s\left( f\cdot g, P \right) < \epsilon_0^2 < \epsilon
,\] ou seja, $f\cdot g\left( x \right) \ge 0, \forall x\in S \implies f\cdot g$ é integrável em $S$.

($f\cdot g\left( x \right) \le 0, \forall x\in S$)
Sem perda de generalidade, suponha $f\left( x \right) \ge 0, g\left( x \right) \le 0, \forall x\in S$. De forma análoga, $\forall B\in \mathcal{B}\left( P \right) $, onde $P$ é uma partição de $S$,
\begin{align*}
    M_B^{f\cdot g} \le  m_B^{f} \cdot M_B^{g} \\
    m_B^{f\cdot g} \ge   M_B^{f} \cdot m_B^{g}
.\end{align*}
Veja que
\begin{align*}
    M_B^{f}M_B^{g} + m_B^{f}m_B^{g} &> m_B^{f}M_B^{g} \\
    M_B^{f}m_B^{g} + m_B^{f}M_B^{g} &> M_B^{f}m_B^{g}
\end{align*}
e (*) ainda é válida, portanto
\begin{align*}
    \left( S\left( f,P \right) - s\left( f,P \right) \right) \left( S\left( g,P \right) - s\left( g,P \right) \right) &\ge   \sum_{B\in \mathcal{B}\left( P \right) } \left( M_B^{f\cdot g} - m_B^{f\cdot g} \right) vol B \\
															 &= S\left( f\cdot g, P \right) - s\left( f\cdot g, P \right) 
.\end{align*}
Ou seja, $\forall \epsilon>0$, escolhemos $\epsilon_0$ da mesma forma que no caso anterior e chegamos na mesma conclusão: $f\cdot g$ é integrável em S.

\exercise{3-15}

Seja $\epsilon > 0$. Tome $B_1,\ldots,B_k$ n-blocos fechados de forma que \[
C\subseteq B_1\cup \ldots\cup B_k
\] e \[
\sum_{i=1}^{k} vol B_i < \epsilon
.\] Escreva \[
B_i = \prod_{j=1}^{n} \left[ a^{(i)}_j, b^{(i)}_j \right] , i=1,\ldots,k
.\] Então, defina o n-bloco \[
A = \prod_{j=1}^{n} \left[ \min\left\{ a^{(i)}_j :i=1,\ldots,k\right\}, \max\left\{ b^{(i)}_j :i=1,\ldots,k\right\}  \right]  
.\] Veja que \[
B_i \subseteq A, i=1,\ldots,k \implies B_1\cup \ldots\cup B_k \subseteq A  
,\] ou seja, $C \subseteq A$.

Como $C\subset A$ e $A$ é claramente limitado, então $C$ é limitado. Como visto no exercício 3-10.a, $\partial C$ tem conteúdo nulo, então também possui medida nula. Portanto, $C$ é Jordan-mensurável.

Agora, tome uma partição $P$ de $A$ de forma que $B_i \in \mathcal{B}\left( P \right) , i=1,\ldots,k$. Veja que \[
\forall B \in \mathcal{B}\left( P \right) \setminus \{B_i\} _{i=1,\ldots,k}, M_B = m_B = 0
,\] enquanto \[
M_{B_i} = 1, m_{B_i} \in \left\{ 0,1 \right\} 
.\] Assim, temos
\begin{align*}
    S\left( \chi_C, P \right) &= \sum_{B \in \mathcal{B}\left( P \right) } M_B vol B \\
    &= \sum_{i=1}^{k} vol B_i < \epsilon
\end{align*}
e \[
    0 \le s\left( \chi_C, P \right) \le \sum_{i=1}^{k} vol B_i < \epsilon
.\] Como $\epsilon$ é arbitrário \[
\overline{\int}_A \chi_C = \underline{\int}_A \chi_C = 0
.\] 

\exercise{3-16}

Similar ao exercício 3-10.b, tome $C = \Q\cap \left[ 0,1 \right] $. Como já visto, $C$ possui medida nula e é limitado. Agora seja $A$ um n-bloco (neste caso, um intervalo fechado) tal que $C \subseteq A$ e $P$ uma partição de $A$. Então, $\forall B\in \mathcal{B}\left( P \right), B\cap C \neq \O $ da mesma forma que $B \cap A\setminus C \neq \O$. Portanto, \[
M_B^{\chi_C} = 1\text{ e } m_B^{\chi_C}=0, \forall B\in \mathcal{B}\left( P \right) 
,\] ou seja, 
\begin{align*}
    S\left( f, P \right) &= \sum_{B\in \mathcal{B}\left( P \right) } vol B = vol A \\
    s\left( f,P \right) &= 0 \\
.\end{align*}
Como $P$ foi escolhida de forma arbitrária, \[
\overline{\int}_A \chi_C = vol A \neq \underline{\int}_A \chi_C = 0
.\]

\exercise{3-17}

Seja $P$ uma partição de $A$. Seja $\epsilon > 0$, então escolha $\left\{ B_i \right\} _{i\in \N}$ de forma que \[
C \subseteq \bigcup_{i\in \N} B_i
\] e \[
\sum_{i\in \N} vol B_i < \epsilon
.\] Então, veja que, $\forall B\in \mathcal{B}\left( P \right) $, \[
m_B^{\chi_C} = 1 \implies \exists i_0\in \N \text{ tal que } B \subseteq B_{i_0}
,\] mas como \[
vol B_{i_0} < \sum_{i\in \N} vol B_i < \epsilon
,\] isso implicaria em $B$ ser um conjunto de conteúdo nulo, o que contradiz os resultados do exercício 3-8. Assim, concluímos que \[
m_B^{\chi_C} = 0, B \in \mathcal{B}\left( P \right) \implies s\left( \chi_C, P \right) = 0
\] \[
\implies \underline{\int}_A \chi_C = 0 \implies \int_A \chi_C = 0
.\]

\exercise{3-18}

Tome $\epsilon > 0$ e $n\in \N^*$. Então, $\exists P$ partição de $A$ de forma que \[
S\left( f, P  \right) - s\left( f, P \right) < \frac{\epsilon}{n}
.\] 
Como $\int_A f=0$ e $f$ é não-negativa, \[
s\left( f, P \right) = 0
,\] portanto \[
S\left( f, P \right) = \sum_{B\in \mathcal{B}\left( P \right) } M_B vol B  < \frac{\epsilon}{n}
.\] Agora defina $O_n = \{x : f\left( x \right) > \frac{1}{n}\} $. Tome $B_1,\ldots,B_{k} \in \mathcal{B}\left(P  \right) $ de forma que $B_i \cap O_n \neq \O, i=1,\ldots,k$. Claramente, \[
O_n \subseteq \bigcup_{i=1} ^{k} B_i
.\] Veja então que $M_{B_i} > \frac{1}{n}, i=1,\ldots,k$ e, assim,
\begin{align*}
    \frac{1}{n} \sum_{i=1}^{k} vol B_i &< \sum_{i=1}^{k} M_{B_i} vol B_i \\
				       &< \sum_{B\in \mathcal{B}\left( P \right) } M_B vol B \\
				       &= S\left( f, P \right) < \frac{\epsilon}{n}
,\end{align*}
ou seja, \[
\sum_{i=1}^{k} vol B_i < \epsilon
\] e, portanto, $O_n$ tem conteúdo nulo.

Defina $F = \{x : f\left( x \right) \neq 0\}$. Veja que, $\forall x \in F \exists N\in \N^{*}$ tal que $x\in O_N$. Assim, \[
F \subseteq \bigcup_{n\in \N^{*}} O_n
\] e, como $O_n$ tem conteúdo nulo e, portanto, medida nula, sua união também possui medida nula, logo $F$ possui medida nula.

\exercise{3-19}

Pelo enunciado, $U\subset \left[ 0,1 \right] $\[
U = \bigcup_{i\in \N} \left( a_i, b_i \right) 
\] de forma que $\Q \subset U$, $med U \neq 0$ e \[
\sum_{i\in \N} \left( b_i - a_i \right) < 1
.\] 

Sabemos que \[
\partial U = \left[ 0,1 \right] - U
\] e \[
med \partial U \neq 0
.\] 

Suponha $f$ integrável. Tome $0<\epsilon < 1-\sum_{i\in \N} \left( b_i - a_i \right) $. Escolha uma partição $P$ de $\left[ 0,1 \right] $ de forma que \[
S\left( f,P \right) - s\left( f,P \right) < \epsilon
\] . Defina então \[
B' = \{B : B\in \mathcal{B}\left( P \right) \text{ e } B \subseteq U\} 
.\] Por definição, \[
\bigcup_{B\in B'} \subseteq U
.\] Assim, como $vol\left( B_i\cap B_j \right) = 0, B_i, B_j \in B'$, \[
\sum_{B\in B'} vol B \le \sum_{i\in \N} vol\left( a_i, b_i \right) < 1- \epsilon
.\] 

Agora, veja que, $\forall B\in B'$, $M_B = m_B = 1$, enquanto $\forall B\in \mathcal{B}\left( P \right) \setminus B'$, $M_B = 1$, uma vez que $\Q\cap B \neq \O$, e $m_B=0$, visto que $\exists x\in B$ tal que $x\in \left[ 0,1 \right] \setminus U=\partial U$. Assim, podemos ver que \[
S\left( f,P \right) = \sum_{B\in \mathcal{B}\left( P \right) } M_B vol B = \sum_{B \in \mathcal{B}\left( P \right) } vol B = vol \left[ 0,1 \right] = 1
\] e \[
s\left( f,P \right) = \sum_{B \in B'} 1 vol B + \sum_{B\in \mathcal{B}\left( P \right) \setminus B'} 0 vol B = \sum_{B\in B'} vol B
,\] ou seja, 
\begin{align*}
& S\left( f,P \right)  - s\left( f,P \right) < \epsilon  \\
& \implies 1 - \sum_{B\in B'} vol B < \epsilon \\
& \implies \sum_{B\in B'} vol B > 1 - \epsilon
,\end{align*}
uma contradição.

\exercise{3-20}

Defina $O_n = \{x : o\left( f, x \right) > \frac{f\left( b \right) -f\left( a \right) }{n}\} $. Veja que, $\forall \epsilon_0 > 0, \exists N_0\gg\frac{f\left( b \right) -f\left( a \right) }{\epsilon_0}$ de forma que \[
o\left( f, x \right) \ll \epsilon_0, \forall x \in \left[ a,b \right] \setminus O_{N_0}
.\] Portanto, todo n-bloco $B \subseteq \left[ a,b \right] \setminus O_{N_0}$ admite uma partição $Q_B$ de forma que \[
S\left( f|_B,Q_B \right) - s\left( f|_B, Q_B \right) < \epsilon_0
,\] ou seja, \[
\sum_{B_B \in \mathcal{B}\left( Q_B \right) } \left( M_{B_B}-m_{B_B} \right) vol B_B < \epsilon_0
.\] 

Fixe, então, $\epsilon > 0$ e escolha $N\gg \frac{2\left(  f\left( b \right) -f\left( a \right) \right) }{\epsilon}$ análogo ao anterior. Como já vimos que todo conjunto $O_n$ é finito, podemos escolher n-blocos $B_1,\ldots,B_k$ de forma que $O_N\subseteq B_1,\ldots,B_k$ e \[
\sum_{i=1}^{k} vol B_i < \frac{\epsilon}{2\left(  f\left( b \right) - f\left( a \right) \right) }
.\] Escreva $B^{'} = \{ B_1, \ldots, B_k\}$. Tome $P$ uma partição de $\left[ a,b \right] $ de forma que $B^{'} \subseteq \mathcal{B}\left( P \right) $. Assim, \[
\sum_{B\in B^{'}} \left( M_B - m_B \right) vol B < \sum_{B \in B^{'}} \left( f\left( b \right) -f\left( a \right)  \right) vol B < \frac{\epsilon}{2} \tag{$*$}
.\] Agora, tome uma partição $Q \supseteq P$ de forma que, definindo $B^{''} := \{B : B \subseteq \left[ a,b \right] \setminus O_N\} \subseteq \mathcal{B}\left( Q \right)$, \[
\sum_{B\in B^{''}} \left( M_B - m_B \right) vol B < \frac{\epsilon}{2} \tag{$**$}
.\] Portanto, como
\begin{align*}
    S\left( f,Q \right) -s\left( f,Q \right) &= \sum_{B\in \mathcal{B}\left( Q \right) } \left( M_B - m_B \right) vol B \\
    &= \sum_{B\in B^{'}} \left( M_B - m_B \right) vol B + \sum_{B\in B^{''}} \left( M_B - m_B \right) vol B
,\end{align*}
($*$) e ($**$) nos dão que \[
S\left( f,Q \right) -s\left( f,Q \right) < \epsilon
.\] 

\exercise{3-21}


($\implies$)
Seja $C$ Jordan-mensurável. Veja que $\mathcal{S}_2 \subseteq \mathcal{S}_1$, ou seja, queremos provar que \[
\sum_{S\in \mathcal{S}} vol S < \epsilon
\] onde \[
\mathcal{S} = \{S : S\cap C \neq \O, S\cap A\setminus C \neq \O\} \subseteq \mathcal{B}\left( P \right) 
.\] Veja que $\partial C$ é limitado e fechado, portanto, por Heine-Borel, é compacto. Além disso, como $C$ é Jordan-mensurável, $\partial C$ tem medida nula. Logo, $C$ tem conteúdo nulo. Então podemos escolher $B_1,\ldots,B_k $ n-blocos de forma que \[
\partial C \subseteq B_1\cup \ldots\cup B_k
\] e \[
\sum_{i=1}^{k} vol B_i < \epsilon
.\] Agora, escolha uma partição $P$ de $A$ de forma que $B_1,\ldots,B_k \in \mathcal{B}\left( P \right) $. Assim, é fácil ver que $\mathcal{S} = \{B_1,\ldots,B_k\} $, logo \[
\sum_{S\in \mathcal{S}} vol S = \sum_{i=1}^{k} vol B_i < \epsilon
.\] 

($\impliedby$)
Se assumimos que $\forall \epsilon>0, \exists P$ partição de $A$ de forma que \[
\sum_{S\in \mathcal{S}} vol S < \epsilon
,\] então é fácil ver que $\mathcal{S}$ é uma cobertura finita de n-blocos para $\partial C$, logo $\partial C$ possui conteúdo nulo e, por consequência, medida nula. Além disso, $A$ é limitado $\implies C\subseteq A$ é limitado. Portanto, $C$ é Jordan-mensurável.


\end{document}
