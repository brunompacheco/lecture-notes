\lecture{3}{08.02.2021}{Conjuntos de Medida Nula}

Temos a notação \[
\mathcal{R}\left( A \right) = \left\{ f:A\to \R / f \text{ integrável} \right\} 
.\] Agora, podemos dizer que $\mathcal{C}\left( A \right) \subset \mathcal{R}\left( A \right) $, uma vez que $\mathcal{R}\left( A \right) $ é um espaço vetorial com operações ponto a ponto.

Também podemos definir a operação
\begin{align*}
    \mathcal{I}: \mathcal{R}\left( A \right)  &\longrightarrow \R \\
    f &\longmapsto \mathcal{I}(f) = \int_A f
\end{align*}
que é linear.

Agora, utilizando a norma \[
    \|f\|_{\infty} = sup\left\{ |f\left( x \right) | : x\in A \right\} 
,\] temos que \[
\|\mathcal{I}\left( f \right) \| = \left| \int_A f \right| \le \int_A \left| f \right| 
.\] 

\section*{Conjuntos de medida nula}

\begin{definition}
    Um conjunto $X\subset \R^{n}$ tem \emph{medida (de Lebesgue) nula} se $\forall \epsilon>0$, é possível encontrar $\{B_n\}_{n\in \N}$ coleção de n-blocos (ou vazios) tal que
    \begin{itemize}
        \item $X \subset \cup_{n\in \N}B_n$; e
	\item $\sum_{n\in \N} vol\left( B_n \right) < \epsilon$.
    \end{itemize}
    Nesse caso, escrevemos $med X = 0$.
\end{definition}

Alguns exemplos:
\begin{eg}
    1) $X=\O$ tem medida nula
\end{eg}
\begin{eg}
    2) $X = \{x_0\}$, com $x_0\in \R^{n}$, tem medida nula. Basta escolher um n-bloco que tenha intervalos constituintes centrados nos componentes de $x_0$ e com tamanho que totalize um volume menor que $\epsilon$. Ou seja, dado $\epsilon>0$ e $x_0 = \left( x_0^{1},\ldots,x_0^{n} \right) $, ponha 
    \begin{align*}
    & B_1 = \left[ x_0^{1}-\frac{\left(  \frac{\epsilon}{2}\right) ^{\frac{1}{n}}}{2}, x_0^{1}+\frac{\left(  \frac{\epsilon}{2}\right) ^{\frac{1}{n}}}{2}  \right] \times \ldots \\
    .\end{align*}
    Então, \[
    vol \left( B_1 \right) = \left( \left(  \frac{\epsilon}{2}\right)^{\frac{1}{n}} \right)^{n} = \frac{\epsilon}{2} < \epsilon
    ,\] portanto, $med X = 0$.
\end{eg}

\begin{prop}
    Seja $\left\{ X_k \right\}_{k\in \N}$ conjunto de subconjuntos de $\R^{n}$ com \[
    med_{\R^{n}}X_k = 0 , \forall k\in \N
    .\] Então \[
    med\left( \bigcup_{k\in \N} X_k \right) = 0
    .\] 
\end{prop}

\begin{demo}
    Seja $\epsilon>0$. $\forall k\in \N$, escolha $\left\{ B_l^{(k)} \right\}_{l\in \N}$ de n-blocos, com \[
    X_k \subset  \bigcup_{l\in \N}B_l^{(k)}
    \]  e \[
    \sum_{l\in \N} vol\left( B_l^{(k)} \right) < \frac{\epsilon}{2^{k}}
    ,\] ou seja, dado que todo $X_k$ tem medida nula, pegamos os n-blocos que contém seus elementos e indexamos esses conjuntos de n-blocos.

    Veja que $\left\{ B_l^{(k)} \right\}_{k,l \in \N}$ é uma coleção enumerável de n-blocos tal que \[
    X:= \bigcup_{k\in \N}X_k \subset \bigcup _{l,k\in \N}B_l^{(k)}
    .\] Assim, só nos basta provar que eles possuem volume somado menor que $\epsilon$.

    Veja que \[
    \sum_{k\in \N} \sum_{l\in \N} vol\left( B_l^{(k)} \right) < \sum_{k\in N} \frac{\epsilon}{2^{k}} = \epsilon\sum_{k\in N} \frac{1}{2^{k}}
.\] Entretanto, a série $\sum_{n=1}^{\infty} \frac{1}{2^{n}} \to 1$, então podemos afirmar que \[
\epsilon\sum_{k\in N} \frac{1}{2^{k}} < \epsilon
,\] ou seja, \[
\sum_{k,l \in  \N} vol\left( B_l^{(k)} \right) < \epsilon
.\] 
\end{demo}

\begin{eg}
    3) $\Q\subset \R$ tem medida nula: $\Q = \cup_{k\in \N}\left\{ r_k \right\} $, onde fixamos uma enumeração $\Q=\left\{ r_1,r_2,\ldots,r_k,\ldots \right\}$. Como $med \left\{ r_k \right\} = 0, \forall k\in \N$, então, pela proposição acima, \[
    med \Q = 0.
    \] 

    Mais geralmente, qualquer conjunto enumerável $X \subset \R^{n}$ tem medida nula.
\end{eg}

\begin{eg}
    4) Se $X \subset  \R^{n}$ tem medida nula, então $int\left( X \right) = \O$ (interior vazio).

    Se $X$ fosse tal que $int\left( X \right) \neq \O$, então $\exists A\subset X$, sendo $A$ um n-bloco. Temos que qualquer cobertura enumerável $A\subset \cup B_i$ de n-blocos é tal que \[
    \sum_{i\in \N} vol B_i \ge vol A
    .\] Logo, $med A \neq 0 \implies med X \neq 0$.

    Para uma demonstração mais minuciosa, ver E. Lima, Curso de Análise vol. 2, Cap. VI, seção 2, proposição C.
\end{eg}

\begin{note}
    Veja que a recíproca é claramente falsa:
    $\R \setminus \Q\subset \R$ tem $int\left( \R\setminus \Q \right) =\O$ mas se $med\left( \R \setminus \Q \right) = 0$, então teríamos $med \R = med\left( \left( \R \setminus \Q \right) \cup \Q \right) = 0$ (uma vez que ambos possuem medida nula), uma contradição.

    Isso também nos mostra que todo aberto não-vazio que contém um conjunto de medida nula não pode possuir medida nula [TODO: ???].
\end{note}

\begin{eg}
    5) $X =  \R^{n} \times \left\{ 0_{\R^{m}} \right\} \subset \R^{n+m}$ possui medida nula.

    Claramente não conseguimos cobrir com uma quantidade finita de blocos.

    \begin{remark}
	Considere $P = (p_1,\ldots,p_n) \in \Z^{n}$ e n-blocos da forma \[
        B_P^{(n)} = \left[ p_1, p_1+1 \right] \times \ldots\times \left[ p_n, p_n+1 \right] 
        ,\] então \[
        \bigcup_{P\in \Z^{n}}B_P^{(n)} = \R^{n}
        .\] Estamos cobrindo $\R^{n}$ com n-blocos delimitados pelos inteiros.
    \end{remark}

    Fixe $\epsilon>0$. Fixe uma enumeração $P^{1},\ldots,P^{l},\ldots$ de $\Z^{n}$. Defina \[
    B_l = B^{(n)}_{P_l} \times \left[ 0, \left( \frac{\epsilon}{2^{l}} \right)^{\frac{1}{n}} \right]\times \ldots\times \left[ 0, \left( \frac{\epsilon}{2^{l}} \right)^{\frac{1}{n}} \right]
    ,\] ou seja, uma extensão do bloco como na observação com intervalos de medida controlada por $\epsilon$. Então, claramente, \[
    X \subset  \bigcup _l B_l
    .\] Agora, veja que 
    \begin{align*}
	& vol_{\R^{n+m}}B_l = \left(  vol_{\R^{n}} B_{P_l}^{(n)}\right)  \left( \frac{\epsilon}{2^{l}} \right) = \frac{\epsilon}{2^{l}} \\
	& \implies \sum_{l} vol_{\R^{n+m}}B_l = \sum_{l} \frac{\epsilon}{2^{l}} < \epsilon
    .\end{align*}
\end{eg}

\begin{prop}
    (ver Teorema 3, sec. 2, p. 151, "Análise Real", vol. 2, E. Lima)

    Para $X\subset \R^{n}$, são equivalentes:
    \begin{itemize}
        \item $med X = 0$;
	\item $\forall \epsilon>0$, $\exists $ n-blocos abertos $B^{k} := \left( a^{k}_1,b_1^{k} \right)\times \ldots\times \left( a^{k}_n,b_n^{k} \right) $ com $X\subset \cup _k B^{k}$, \[
	\sum_{k\in \N} vol B^{k} < \epsilon
	;\] 
	\item $\forall \epsilon>0$, $\exists $ n-cubos fechados, da forma \[
	B^{k}= \left[ a_1^{k},b_1^{k} \right] \times \ldots\times \left[ a_n^{k},b_n^{k} \right]
	\] com $b_i^{k} - a_i^{k}=L \forall i \in \left\{ 1,\ldots,k \right\} $, tal que $X\subset \cup _{k\in \N}B^{k}$ e \[
	\sum_{k\in \N} \left( L^{k} \right) ^{n} < \epsilon
	;\] 
	\item Mesmo do anterior, mas com n-cubos abertos.
    \end{itemize}

    \begin{remark}
	Para que consigamos mostrar a equivalência entre o uso de blocos/cubos abertos e fechados, basta que, partindo de um elemento fechado, definamos um elemento aberto estendendo o conjunto com um valor $\delta > 0$ arbitrário, então, como o volume é contínuo, então podemos fazer com que $\delta$ seja tão pequeno quanto queiramos, fazendo com que o volume do aberto vá para o volume do fechado conforme $\delta\to 0$. Ou seja, restringimos o volume do fechado para que seja menor que $\frac{\epsilon}{2}$. por exemplo, e estendendo o volume com um $\delta$ tal que o volume do aberto seja $\frac{3\epsilon}{4}$. A volta é mais fácil, uma vez que podemos tomar o fecho dos abertos, que preserva o volume.

	Para a equivalência entre blocos e cubos, primeiro vemos que blocos com dimensões racionais podem ser partidos em cubos. Então, utilizamos a mesma ideia acima, estendendo os blocos de forma arbitrariamente pequena para que suas dimensões sejam racionais.
    \end{remark}
\end{prop}

\begin{prop}
    Seja $f : X\subset \R^{n}\to \R^{n}$ Lipschitz. Se $med X = 0$, então $med\left(  f\left( X \right)\right)   = 0$.
\end{prop}
\begin{demo}
    Fixe $\|.\|$ como a norma do máximo em $\R^{n}$. Fixe $\epsilon>0$. Tome $c>0$ a constante de Lipschitz de $f$, de forma que $\|f\left( x \right) -f\left( y \right) \| \le  c \|x-y\| \forall x, y \in X$. Seja $\hat{C}_k$, $k\in \N$ n-cubos (os intervalos constituintes possuem mesmo tamanho, neste caso, $L_k$), tal que \[
    \hat{C}_k \subset X, \forall k\in \N
    \] e \[
    vol \hat{C}_k = L_k^{n} < \epsilon'
    \] com $\epsilon'$ a ser escolhido, mas arbitrário uma vez que $med X = 0$.

    Veja que $x, y \in \hat{C}_k\subset X \implies \|x-y\|\le L_k$. Então \[
    j=1,\ldots,n\text{, } \left| f_j\left( x \right) - f_j\left( y \right)  \right| \le cL_k
    .\] Fixe, para cada \[
    \hat{C}_k = \left[ a_1^{(k)}, b_1^{(k)} \right] \times \ldots\times \left[ a_n^{(k)}, b_n^{(k)} \right]
    ,\] \[
    a^{(k)} := \left( a_1^{(k)}, \ldots, a_n^{(k)} \right) \in \hat{C}_k\subset X
    \] \[
    x_j\in \left[ a_j^{(k)}-L_k, a_j^{(k)}+L_k \right] , \forall x\in \hat{C}_k
    \]  \[
    \implies f_j\left( x \right) \in \left[ f_j\left( a^{(k)} \right) -cL_k, f_j\left( a^{(k)} \right) +cL_k \right] \forall x \in \hat{C}_k
    ,\] mas isso implica em \[
    f\left( x \right) \in \prod_{j=1}^{n}  \left[ f_j\left( a^{(k)} \right) -cL_k, f_j\left( a^{(k)} \right) +cL_k \right] =: B_k
    ,\] ou seja, \[
    f\left( \hat{C}_k \right) \subset B_k
    .\] \[
    f\left( X \right) \subset \bigcup_{k\in \N} B_k
    .\] Agora veja que 
    \begin{align*}
    & vol B_k = \left( 2cL_k \right) ^{n} = 2^{n}c^{n} vol \hat{C}_k < 2^{n}c^{n}\epsilon' \\ 
    & \implies \sum_{k=1}^{\infty} vol B_k = 2^{n}c^{n}\sum_{k=1}^{\infty} vol \hat{C}_k = 2^{n}c^{n}\sum_{k=1}^{\infty} \epsilon'
    .\end{align*}
    Como $\epsilon'$ é um valor arbitrário pela medida nula de $X$, podemos escolhê-lo como $\epsilon' < \frac{\epsilon}{2^{k+n}c^{n}}$, então \[
    \sum_{k=1}^{\infty} vol B_k < 2^{n}c^{n}\sum_{k=1}^{\infty}\frac{\epsilon}{2^{k+n}c^{n}} = \epsilon
    ,\] ou seja, \[
    med f\left( X \right) = 0
.\] 
\end{demo}

\begin{corollary}
    Seja $f: U\subset \R^{n}\to \R^{m}$ Lipschitz. Se $m>n$, então \[
    med_\R^{m} f\left( U \right) = 0
    .\] 
\end{corollary}

\begin{proof}
    Defina 
    \begin{align*}
        F: U\times \R^{m-n} &\longrightarrow \R^{m} \\
        \left( x,y \right)  &\longmapsto F\left( x,y \right) = f\left( x \right) 
    .\end{align*}

    Tome $c> 0$ tal que \[
	\|f(x) - f(x')\|_{\R^{m}}\le c \|x-x'\|_{\R^{n}} \text{, }\forall x,x' \in U
    .\] Tome em $\R^{n},\R^{m}$ a norma da soma. Nesse caso, \[
    \|F\left( x,y \right) - F\left( x',y' \right) \|_{\R^{m}} = \|f\left( x \right) - f\left( x' \right) \|_{\R^{m}} \le c\|x-x'\|_{\R^{n}}
    .\] mas \[
    c\|x-x'\|_{\R^{n}} \le c\left( \|x-x'\|_{\R^{n}}+\|y-y'\|_{\R^{m-n}} \right) = c\|\left( x,y \right) -\left( x',y' \right) \|_{\R^{m}}
    ,\] ou seja, \[
    \|F\left( x,y \right) - F\left( x',y' \right) \|_{\R^{m}} \le c\|\left( x,y \right) -\left( x',y' \right) \|_{\R^{m}}
    ,\] i.e., $F$ é Lipschitz.

    Além disso, \[
    F\left( U\times \left\{ 0_{\R^{m-n}} \right\}  \right) = f\left( U \right) 
    .\] Entretanto, como \[
    U\times \left\{ 0_{\R^{m-n}} \right\} \subset \R^{n}\times \left\{ 0_{\R^{m-n}} \right\}
    \] que tem medida nula em $\R^{m}$, $U\times \left\{ 0_{\R^{m-n}} \right\}$ tem medida nula em $\R^{m}$ e, portanto, \[
     med_{\R^{m}}f\left( U \right)  = med_{\R^{m}}F\left( U\times \left\{ 0_{\R^{m-n}} \right\} \right) = 0
    .\] 
\end{proof}

