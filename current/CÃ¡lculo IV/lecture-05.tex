\lecture{5}{15.02.2021}{Teorema de Riemann-Lebesgue}

\section*{Teorema de Riemann-Lebesgue}

\begin{definition}
    (Número de Lebesgue) Seja $\left( M,d \right) $ um espaço métrico. Seja $C=\{U_\lambda\} _{\lambda \in \Lambda}$ uma cobertura aberta de $M$ tal que \[
    M = \bigcup_{\lambda \in  \Lambda} U_\lambda
    .\] Um número $\delta >0$ é dito \emph{número de Lebesgue} de $C$ se $\forall X\subseteq M$, $X\neq \O$ \[
    diam X = \sup \left\{ d\left( x,y \right) : x.y\in X \right\} < \delta \implies \exists \lambda \in \Lambda \text{ t.q. } X\subseteq U_\lambda
    .\] 
\end{definition}
\begin{lemma}\label{lemma:lebesgue-1}
    Se $\left( M,d \right) $ é um espaço métrico compacto, qualquer cobertura aberta tem número de Lebesgue.
\end{lemma}
\begin{proof}
    Suponha que $\{U_\lambda\} _{\lambda\in \Lambda}$ é uma cobertura aberta de $\left( M,d \right) $ sem número de Lebesgue: $\forall n\in \N$, $\frac{1}{n}$ não é um número de Lebesgue. Então $\exists X_n\subseteq M$, $X_n \neq \O$ de forma que \[
    diam X_n < \frac{1}{n}
    \] mas com \[
    X_n \cap \left( M\setminus U_\lambda \right) \neq \O, \forall \lambda \in \Lambda
    .\] Escolha para cada $n\in \N, x_{n}\in X_n$. Dessa forma \[
    \left( x_n \right) _{n\in \N}\subseteq M
    .\] A menos de passar para uma subsequência, podemos assumir que a sequência converge \[
    x_n \to x_0 \in  U_{\lambda_0}
    \] para algum $\lambda_0 \in \Lambda$.
    Seja $\epsilon>0$ tal que \[
    B_{2\epsilon}\left( x_0 \right) \subseteq U_{\lambda_0}
    .\] Tome $n_0\in \N$ tal que \[
    X_{n_0}\subseteq B_{\epsilon}\left( x_0 \right) 
    \] e $ n_0 > \frac{1}{\epsilon} \implies \frac{\epsilon}{n_0} < \epsilon$. Portanto, $\forall x\in X_{n_0}$, \[
    d\left( x,x_0 \right) \le d\left( x, x_{n_0} \right) + d\left( x_{n_0}, x_0 \right) 
    \] mas \[
    d\left( x, x_{n_0} \right) \le diam X_{n_0} < \epsilon
    \] e \[
    d\left( x_{n_0}, x_0 \right) < \epsilon
    \] logo \[
    d\left( x,x_0 \right) < 2\epsilon \implies x\in B_{2\epsilon}\left( x_0 \right) \subseteq U_{\lambda_0}
    \] uma contradição.
\end{proof}

\begin{lemma}\label{lemma:lebesgue-2}
    Seja $f:A\subseteq\R^{n}\to \R$ limitada, com $A$ n-bloco. Suponha que $\exists \epsilon>0$ tal que \[
    o\left( f, x \right) < \epsilon, \forall x\in A
    .\] Nesse caso, $\exists P$ partição de $A$ tal que \[
    S\left( f,P \right) -s\left( f, P \right) < \epsilon . vol A
    .\] 
\end{lemma}
\begin{proof}
    (Vamos trabalhar com a norma do máximo em $\R^{n}$)
    Para cada $x\in A$, $\exists \delta_x > 0$ tal que \[
    o\left( f, x, \delta_x \right) = M\left( f,x,\delta_x \right) -m\left( f,x,\delta_x \right) < \epsilon
.\] Veja que \[
\{B_{\delta_x}\left( x \right) \} _{x\in A}
\] é uma cobertura aberta de $A$, um espaço métrico compacto com métrica induzida pela norma.
Pelo lema \ref{lemma:lebesgue-1}, sabemos que pode-se escolher $\delta > 0$ o número de Lebesgue para essa cobertura. Escolha uma partição $P_\delta=P_1\times \ldots\times P_n$ de $A$ com $\left| P_i \right| < \frac{\delta}{2}$ (essa norma é a maior diferença entre os intervalos induzidos pela partição) \[
\implies diam B < \delta, \forall B\in \mathcal{B}\left( P_\delta \right) 
\] \[
\implies B\subseteq B_{\delta_x}\left( x \right) \cap A
\] para algum $x\in A$.
Então
\begin{align*}
    S\left( f,P_\delta \right) - s\left( f, P_\delta \right) &= \sum_{B\in \mathcal{B}\left( P_\delta \right) } \left( M_B - m_B \right) vol B \\
							     &<  \sum_{B\in \mathcal{B}\left( P_\delta \right) } \left( M\left( f,x,\delta_x \right) -m\left( f,x,\delta_x \right)  \right)  vol B \\
							     &<  \sum_{B\in \mathcal{B}\left( P_\delta \right) } \epsilon vol B = \epsilon vol A
,\end{align*}
onde $\delta_x$ é relativo ao aberto que contém o n-bloco $B$, para cada $B$.

\end{proof}

\begin{theorem}
    (Riemann-Lebesgue) Se $f:A\to \R$ ($A$ um n-bloco) é uma função limitada, $f$ é integrável \emph{sse} \[
    D_f := \{x\in A: f\text{ não é contínua em }x\} 
    \] tem medida nula.
\end{theorem}

\begin{proof}
    ($\impliedby$) Suponha $D_f$ com medida nula. Fixe $\epsilon>0$. Considere \[
	D_\epsilon := \{x\in A : o\left( f,x \right) \ge \epsilon\} \subseteq D_f
    ,\] logo, $D_\epsilon$ tem medida nula. Além disso, $D_\epsilon$ é fechado em $A$, um compacto, portanto $D_\epsilon$ é compacto. Assim, concluímos que $D_\epsilon$ tem conteúdo nulo. Portanto, $\exists B_1,\ldots,B_k$ n-blocos abertos de forma que \[
	D_\epsilon \subseteq B_1\cup \ldots\cup B_k
    \] e \[
	\sum_{i=1}^{k} vol B_i < \epsilon
	.\] Pelo lema \ref{lemma:lebesgue-1}, sabendo que $\delta$ é o número de lebesgue da cobertura $B_1,\ldots,B_k$, podemos escolher partição $P$ de $A$ de forma que $\forall B\in \mathcal{B}\left( P \right) $ tal que $B\cap D_\epsilon \neq \O \implies diam B < \delta$. Assim, para cada n-blocos $B$, temos \[
    B\cap D_\epsilon = \O \text{, formando o conjunto $S_1\left( P \right) $}
    \] ou \[
    B \subseteq B_i\text{, formando o conjunto $S_2\left( P \right) $}
    \] para algum $i\in \{1,\ldots,k\}$. Agora veja que \[
    S\left( f, P \right) - s\left( f,P \right) \le \sum_{B\in S_1\left( P \right)	 } \left( M_B - m_B \right) vol B + \sum_{B\in S_2\left( P \right)	 } \left( M_B - m_B \right) vol B
    .\] Escreva \[
    M = \sup \{\left| f\left( x \right)  \right| : x\in A\} \ge 0
    ,\] possível uma vez que $f$ é limitada. Além disso, $\forall x\in A$, $-M \le f\left( x \right) \le M$. Assim, \[
    S\left( f, P \right) - s\left( f,P \right) \le \sum_{B\in S_1\left( P \right)	 } \left( M_B - m_B \right) vol B + 2M \sum_{B\in S_2\left( P \right)	 } vol B
    .\] Como \[
    \bigcup_{B\in S_2\left( P \right) } B \subseteq B_1\cup \ldots\cup B_k
    ,\] então \[
    \sum_{B\in S_2\left( P \right) } vol B < \epsilon
    .\] Mas para cada n-bloco $B\in S_1\left( P \right) $, temos \[
    o\left( f, x \right) < \epsilon, \forall x\in B
    \] uma vez que $B\cap D_\epsilon = \O$. Assim, podemos aplicar o lema \ref{lemma:lebesgue-2} a cada n-bloco $B\in S_2\left( P \right) $ e encontrar uma partição $Q_B$ de $B$ de forma que \[
    \sum_{B'\in \mathcal{B}\left( Q_B \right) } \left( M_{B'} - m_{B'} \right) vol B' < \epsilon vol B
    .\] Refinamos a partição $P$ para uma nova partição $Q$ de modo que os subblocos de $Q$ coincidem com os de $P$ para os que intersectam os $B_i$s ou são os $B\in \mathcal{B}\left( Q_B \right) $. Portanto
    \begin{align*}
	S\left( f, P \right) - s\left( f,P \right) &\le \sum_{B\in S_1\left( P \right)	 } \left( M_B - m_B \right) vol B + 2M \sum_{B\in S_2\left( P \right)	 } vol B \\
						   &< \left( 2M + \epsilon \right) \sum_{B\in \mathcal{B}\left( Q \right) } vol B
    .\end{align*}
    Agora basta tomar uma partição $Q' \supseteq Q $ de forma que \[
    \sum_{B\in \mathcal{B}\left( Q' \right) } vol B < \frac{\epsilon}{2M + \epsilon}
    .\] 

    ($\implies$)
    Suponha $f$ integrável. $\forall k\in \N$, \[
    D_{\frac{1}{k}} = \left\{  x\in A : o\left( f,x \right) \ge \frac{1}{k}\right\}  \implies D_f = \bigcup_{k\in \N} D_{\frac{1}{k}}
    .\] Então, basta provar que \[
    med D_{\frac{1}{k}} = 0, k\in \N
    .\] Fixe $k_0\in \N$. Seja $\epsilon>0$. Sabemos também que $D_{\frac{1}{k}}$ é fechado em $A$ e, portanto, compacto. Seja $P$ uma partição de $A$ de modo que \[
    S\left( f,P \right)  - s\left( f,P \right)  < \frac{\epsilon}{k_0}
    .\] Seja \[
    S_{k_0} = \left\{ B\in \mathcal{B}\left( P \right) : B\cap D_{\frac{1}{k_0}}\neq \O \right\} 
    .\] Agora, vemos que \[
    \sum_{B\in S_{k_0}} \left( M_B - m_B \right) vol B \le S\left( f,P \right) -s\left( f,P \right) < \frac{\epsilon}{k_0}
    \] ao mesmo tempo que \[
    M_B - m_B \ge \frac{1}{k_0}
    .\] Portanto, \[
    \frac{1}{k_0}\sum_{B\in S_{k_0}} vol B \le \sum_{B\in S_{k_0}} \left( M_B-m_b \right) vol B < \frac{\epsilon}{k_0}
    \] ou seja, $S_{k_0}$ é uma cobertura de $D_{\frac{1}{k_0}}$, i.e., \[
    D_{\frac{1}{k_0}}\subseteq \bigcup_{B\in S_{k_0}} B
    \] por construção e \[
    \sum_{B\in S_{k_0}} vol B < \epsilon
    .\] Logo $D_{\frac{1}{k_0}}$ tem conteúdo nulo, portanto $D_f$ tem medida nula.
\end{proof}

\begin{eg}
    $f: \left[ 0,1 \right] \to \R$ definida como \[
    f\left( x \right) = \begin{cases}
	0 &, \text{ se }x\in \left( \R\setminus\Q \right) \cap \left( 0,1 \right) \\
	\frac{1}{q} &, \text{ se } x = \frac{p}{q} \in \left[ 0,1 \right] , p,q \in \Z^{+}, q>0, p,q \text{ primos entre si}
    \end{cases}
.\] (Spivak) $f$ é contínua nos irracionais e descontínua em $\Q \cap \left[ 0,1 \right] $. $f$ é integrável e \[
\int_{\left[ 0,1 \right] }f = \sup \left\{ s\left( f,P \right)  : P \text{ partição}\right\} = 0
.\] 
\end{eg}

\section*{Integração em domínios irregulares}

\begin{definition}
    Seja $X\subseteq\R^{n}$ conjunto qualquer. A \emph{função indicadora} de $X$ ou \emph{função característica} de $X$
    \begin{align*}
        \chi_X: \R^{n} &\longrightarrow \R \\
        x &\longmapsto \chi_X(x) = \begin{cases}
	    1 &, \text{ se }x\in X \\
	    0 &, \text{ se }x\notin X
        \end{cases}
    .\end{align*}
\end{definition}

\begin{intuition}
    Nossa situação desejada é de um n-bloco $A$ que contém o domínio $X$ da nossa função, de forma que podemos olha para a função característica de $X$ restrita a $A$, ou seja
    \begin{align*}
        \chi_X: A &\longrightarrow \R \\
        x &\longmapsto \chi_X(x) = \begin{cases}
	    1 &, \text{ se }x\in X \\
	    0 &, \text{ se }x\notin X
        \end{cases}
    .\end{align*}
    Veja que essa função é limitada.
\end{intuition}

\begin{prop}
    Seja $A$ um n-bloco, $\overline{C}\subseteq int A$ e $\chi_C:A\to \R$. Então, são equivalentes:
    \begin{enumerate}
        \item $\chi_C$ é integrável
	\item $\partial C$ tem medida nula
	\item $\partial C$ tem conteúdo nulo
    \end{enumerate}
\end{prop}

\begin{demo}
    A equivalência entre (2) e (3) é imediata uma vez que $\partial C$ é um conjunto fechado por definição (interseção de dois fechados) e $\partial C \subseteq A$, sendo $A$ um compacto, então $\partial C$ é um compacto. Portanto, basta provar (1)$\iff$(2).

    Vamos provar que \[
    D_{\chi_C} = \partial C
    .\] Seja $x\not\in \partial C$, $x\in A$. Sabemos que \[
    \overline{C} = \partial C \cup \left( int C \right) 
,\] portanto, (a) $x\not\in \overline{C}$ ou (b) $x\in int C$.

(a)$\implies \exists \delta>0$ tal que $B_{\delta}\left( x \right) \cap C = \O$ \[
\implies \forall y\in B_{\delta}\left( x \right) \cap A, \chi_C\left( y \right) = 0
,\] portanto $\left| \chi_C\left( y \right) - \chi_C\left( x \right)  \right| =0$, logo $\chi_C$ é contínua em $x \implies x\not\in D_{\chi_C}$ .

(b)$\implies \exists \delta>0$ tal que $B_\delta\left( x \right) \subseteq C$ \[
\forall y\in B_\delta\left( x \right) \cap A, \chi_C\left( y \right) = 0
,\] portanto $\left| \chi_C\left( y \right) -\chi_C\left( x \right)  \right| =0$, logo $x\not\in D_{\chi_C}$.

Assim, vemos que $D_{\chi_C}\subseteq\partial C$.

Por outro lado, seja $x_0\in \partial C$. Escolha $\delta_0>0$ tal que \[
B_{\delta_0}\left( x_0 \right) \subseteq A
.\] Tome $\epsilon = \frac{1}{2}$. Veja que $\forall 0<\delta < \delta_0$\[
B_{\delta}\left( x_0 \right) \cap C \neq  \O \text{ e }B_{\delta}\left( x_0 \right) \cap \left( A\setminus C \right) \neq \O
.\] Portanto, tome $y\in B_{\delta}\left( x_0 \right) \cap C$ e $z\in B_{\delta}\left( x_0 \right) \cap \left( A\setminus C \right)$. Então \[
\left| \chi_C\left( y \right) -\chi_C\left( x_0 \right)  \right| = 1
\] ou \[
\left| \chi_C\left( z \right) -\chi_C\left( x_0 \right)  \right| = 1
,\] ou seja, ambos são maiores que $\epsilon$, portanto $\chi_C$ não é contínua em $x_0$. Logo \[
\partial C \subseteq D_{\chi_C}
\] \[
\implies \partial C = D_{\chi_C}
.\] Resultado segue de Riemann-Lebesgue.
\end{demo}

