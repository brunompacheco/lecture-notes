\documentclass[a4paper]{report}
% Some basic packages
\usepackage[utf8]{inputenc}
\usepackage[T1]{fontenc}
\usepackage{textcomp}
\usepackage[english]{babel}
\usepackage{url}
\usepackage{graphicx}
\usepackage{float}
\usepackage{booktabs}
\usepackage{enumitem}

\pdfminorversion=7

% Don't indent paragraphs, leave some space between them
\usepackage{parskip}

% Hide page number when page is empty
\usepackage{emptypage}
\usepackage{subcaption}
\usepackage{multicol}
\usepackage{xcolor}

% Other font I sometimes use.
% \usepackage{cmbright}

% Math stuff
\usepackage{amsmath, amsfonts, mathtools, amsthm, amssymb}
% Fancy script capitals
\usepackage{mathrsfs}
\usepackage{cancel}
% Bold math
\usepackage{bm}
% Some shortcuts
\newcommand\N{\ensuremath{\mathbb{N}}}
\newcommand\R{\ensuremath{\mathbb{R}}}
\newcommand\Z{\ensuremath{\mathbb{Z}}}
\renewcommand\O{\ensuremath{\emptyset}}
\newcommand\Q{\ensuremath{\mathbb{Q}}}
\newcommand\C{\ensuremath{\mathbb{C}}}
\renewcommand\L{\ensuremath{\mathcal{L}}}

% Package for Petri Net drawing
\usepackage[version=0.96]{pgf}
\usepackage{tikz}
\usetikzlibrary{arrows,shapes,automata,petri}
\usepackage{tikzit}
\input{petri_nets_style.tikzstyles}

% Easily typeset systems of equations (French package)
\usepackage{systeme}

% Put x \to \infty below \lim
\let\svlim\lim\def\lim{\svlim\limits}

%Make implies and impliedby shorter
\let\implies\Rightarrow
\let\impliedby\Leftarrow
\let\iff\Leftrightarrow
\let\epsilon\varepsilon

% Add \contra symbol to denote contradiction
\usepackage{stmaryrd} % for \lightning
\newcommand\contra{\scalebox{1.5}{$\lightning$}}

% \let\phi\varphi

% Command for short corrections
% Usage: 1+1=\correct{3}{2}

\definecolor{correct}{HTML}{009900}
\newcommand\correct[2]{\ensuremath{\:}{\color{red}{#1}}\ensuremath{\to }{\color{correct}{#2}}\ensuremath{\:}}
\newcommand\green[1]{{\color{correct}{#1}}}

% horizontal rule
\newcommand\hr{
    \noindent\rule[0.5ex]{\linewidth}{0.5pt}
}

% hide parts
\newcommand\hide[1]{}

% si unitx
\usepackage{siunitx}
\sisetup{locale = FR}

% Environments
\makeatother
% For box around Definition, Theorem, \ldots
\usepackage{mdframed}
\mdfsetup{skipabove=1em,skipbelow=0em}
\theoremstyle{definition}
\newmdtheoremenv[nobreak=true]{definitie}{Definitie}
\newmdtheoremenv[nobreak=true]{eigenschap}{Eigenschap}
\newmdtheoremenv[nobreak=true]{gevolg}{Gevolg}
\newmdtheoremenv[nobreak=true]{lemma}{Lemma}
\newmdtheoremenv[nobreak=true]{propositie}{Propositie}
\newmdtheoremenv[nobreak=true]{stelling}{Stelling}
\newmdtheoremenv[nobreak=true]{wet}{Wet}
\newmdtheoremenv[nobreak=true]{postulaat}{Postulaat}
\newmdtheoremenv{conclusie}{Conclusie}
\newmdtheoremenv{toemaatje}{Toemaatje}
\newmdtheoremenv{vermoeden}{Vermoeden}
\newtheorem*{herhaling}{Herhaling}
\newtheorem*{intermezzo}{Intermezzo}
\newtheorem*{notatie}{Notatie}
\newtheorem*{observatie}{Observatie}
\newtheorem*{exe}{Exercise}
\newtheorem*{opmerking}{Opmerking}
\newtheorem*{praktisch}{Praktisch}
\newtheorem*{probleem}{Probleem}
\newtheorem*{terminologie}{Terminologie}
\newtheorem*{toepassing}{Toepassing}
\newtheorem*{uovt}{UOVT}
\newtheorem*{vb}{Voorbeeld}
\newtheorem*{vraag}{Vraag}

\newmdtheoremenv[nobreak=true]{definition}{Definition}
\newtheorem*{eg}{Example}
\newtheorem*{notation}{Notation}
\newtheorem*{previouslyseen}{As previously seen}
\newtheorem*{remark}{Remark}
\newtheorem*{note}{Note}
\newtheorem*{problem}{Problem}
\newtheorem*{observe}{Observe}
\newtheorem*{property}{Property}
\newtheorem*{intuition}{Intuition}
\newmdtheoremenv[nobreak=true]{prop}{Proposition}
\newmdtheoremenv[nobreak=true]{theorem}{Theorem}
\newmdtheoremenv[nobreak=true]{corollary}{Corollary}

% End example and intermezzo environments with a small diamond (just like proof
% environments end with a small square)
\usepackage{etoolbox}
\AtEndEnvironment{vb}{\null\hfill$\diamond$}%
\AtEndEnvironment{intermezzo}{\null\hfill$\diamond$}%
% \AtEndEnvironment{opmerking}{\null\hfill$\diamond$}%

% Fix some spacing
% http://tex.stackexchange.com/questions/22119/how-can-i-change-the-spacing-before-theorems-with-amsthm
\makeatletter
\def\thm@space@setup{%
  \thm@preskip=\parskip \thm@postskip=0pt
}


% Exercise 
% Usage:
% \exercise{5}
% \subexercise{1}
% \subexercise{2}
% \subexercise{3}
% gives
% Exercise 5
%   Exercise 5.1
%   Exercise 5.2
%   Exercise 5.3
\newcommand{\exercise}[1]{%
    \def\@exercise{#1}%
    \subsection*{Exercise #1}
}

\newcommand{\subexercise}[1]{%
    \subsubsection*{Exercise \@exercise.#1}
}


% \lecture starts a new lecture (les in dutch)
%
% Usage:
% \lecture{1}{di 12 feb 2019 16:00}{Inleiding}
%
% This adds a section heading with the number / title of the lecture and a
% margin paragraph with the date.

% I use \dateparts here to hide the year (2019). This way, I can easily parse
% the date of each lecture unambiguously while still having a human-friendly
% short format printed to the pdf.

\usepackage{xifthen}
\def\testdateparts#1{\dateparts#1\relax}
\def\dateparts#1 #2 #3 #4 #5\relax{
    \marginpar{\small\textsf{\mbox{#1 #2 #3 #5}}}
}

\def\@lecture{}%
\newcommand{\lecture}[3]{
    \ifthenelse{\isempty{#3}}{%
        \def\@lecture{Lecture #1}%
    }{%
        \def\@lecture{Lecture #1: #3}%
    }%
    \subsection*{\@lecture}
    \marginpar{\small\textsf{\mbox{#2}}}
}



% These are the fancy headers
\usepackage{fancyhdr}
\pagestyle{fancy}

% LE: left even
% RO: right odd
% CE, CO: center even, center odd
% My name for when I print my lecture notes to use for an open book exam.
% \fancyhead[LE,RO]{Gilles Castel}

\fancyhead[RO,LE]{\@lecture} % Right odd,  Left even
\fancyhead[RE,LO]{}          % Right even, Left odd

\fancyfoot[RO,LE]{\thepage}  % Right odd,  Left even
\fancyfoot[RE,LO]{}          % Right even, Left odd
\fancyfoot[C]{\leftmark}     % Center

\makeatother




% Todonotes and inline notes in fancy boxes
\usepackage{todonotes}
\usepackage{tcolorbox}

% Make boxes breakable
\tcbuselibrary{breakable}

% Verbetering is correction in Dutch
% Usage: 
% \begin{verbetering}
%     Lorem ipsum dolor sit amet, consetetur sadipscing elitr, sed diam nonumy eirmod
%     tempor invidunt ut labore et dolore magna aliquyam erat, sed diam voluptua. At
%     vero eos et accusam et justo duo dolores et ea rebum. Stet clita kasd gubergren,
%     no sea takimata sanctus est Lorem ipsum dolor sit amet.
% \end{verbetering}
\newenvironment{verbetering}{\begin{tcolorbox}[
    arc=0mm,
    colback=white,
    colframe=green!60!black,
    title=Opmerking,
    fonttitle=\sffamily,
    breakable
]}{\end{tcolorbox}}

% Noot is note in Dutch. Same as 'verbetering' but color of box is different
\newenvironment{noot}[1]{\begin{tcolorbox}[
    arc=0mm,
    colback=white,
    colframe=white!60!black,
    title=#1,
    fonttitle=\sffamily,
    breakable
]}{\end{tcolorbox}}




% Figure support as explained in my blog post.
\usepackage{import}
\usepackage{xifthen}
\usepackage{pdfpages}
\usepackage{transparent}
\newcommand{\incfig}[1]{%
    \def\svgwidth{\columnwidth}
    \import{./figures/}{#1.pdf_tex}
}

% Fix some stuff
% %http://tex.stackexchange.com/questions/76273/multiple-pdfs-with-page-group-included-in-a-single-page-warning
\pdfsuppresswarningpagegroup=1


% My name
\author{Bruno M. Pacheco}

 
\begin{document}
 
\title{Prova 1}
\author{Bruno M. Pacheco (16100865)\\
H-Cálculo IV}
 
\maketitle
 
\exercise{1}

Escreva $\left\{ q_k \right\} _{k\in \N}$ uma enumeração de $\Q$. Defina, então \[
E = \left\{ B_k:= \left[ q_k -\frac{1}{2^{k+1}}, q_k+\frac{1}{2^{k+1}}\right]  \right\} 
.\] Veja que \[
\R\subseteq \bigcup_{k\in \N} B_k
\] e \[
\sum_{k\in \N} vol. B_k = \sum_{k\in \N} \frac{1}{2^{k}} < 1
.\] Agora, tome $\left\{ B_{k'} \right\} _{k'\in \N}$ uma enumeração de $E^{n}$. Temos, então \[
\R^{n} \subseteq \bigcup_{k'\in \N} B_{k'}
\] e \[
\sum_{k'\in \N} vol. B_{k'} < n
.\]

Fixe $\epsilon  > 0$. Como $med._{\R^{m}} X = 0$, podemos escolher $\left\{ B_i \right\} _{i\in \N}$ n-blocos em $\R^{m}$ de forma que \[
X \subseteq \bigcup_{i\in \N} B_i
\] e \[
\sum_{i\in \N} vol. B_i < \frac{\epsilon}{n}
.\]

Agora veja que $\left\{ B_i\times B_{k'} \right\} _{i, k'\in \N}$ é um conjunto enumerável de n-blocos de $\R^{n+m}$ tal que \[
X\times Y \subseteq \bigcup_{i, k'\in \N} B_i\times B_{k'}
\] e \[
\sum_{i, k'\in \N} vol. \left( B_i\times B_{k'} \right) = \sum_{i,k'\in \N} vol. B_i \cdot vol. B_{k'} \le \left(  \sum_{i\in \N} vol. B_i \right)  \left(  \sum_{k'\in \N} vol. B_{k'}\right) < \epsilon
.\] Ou seja, \[
med._{\R^{n+m}} X\times Y = 0
.\] 

\exercise{2}

\subexercise{a)}

Com $n=1$, seja $X=\left[ 0,1 \right] $ e 
\begin{align*}
    f: X &\longrightarrow \R \\
    x &\longmapsto f(x) = \begin{cases}
	1 &, x\in \Q \\
	0 &, x\not\in \Q
    \end{cases}
.\end{align*}

Veja que \[
S_f = \Q\cap \left[ 0,1 \right] 
,\] que tem medida nula. Entretanto, $f$ não é integrável, uma vez que é descontínua em todo os pontos de $X$.

\subexercise{b)}

Como $f$ é integrável, é limitada. Assim, tome $L\in \R$ tal que $-L\le f\left( x \right) \le L, \forall x\in X$. Agora, como $med.S_f = 0$, $\exists E=\left\{ B_i \right\} _{i\in \N}$ tal que \[
S_f \subseteq\bigcup_{i\in \N} B_i
\] e \[
\sum_{i\in \N} vol. B_i < \frac{\epsilon}{L}
.\] 

Tome $P$ partição de $X$ tal que $E\subseteq\mathcal{B}\left( P \right) $. Veja que $B\in \left(  \mathcal{B}\left( P \right) \setminus E\right) \implies M_B = m_B = 0 $. Dessa forma,
\begin{align*}
    S\left( f, P \right) &= \sum_{B\in \mathcal{B}\left( P \right) } M_B vol. B \\
    &= \sum_{B\in \left( \mathcal{B}\left( P \right) \setminus E \right) } M_B vol. B + \sum_{i\in \N} M_{B_i} vol.B_i \\
    &\le L \sum_{i\in \N} vol. B_i < \epsilon
.\end{align*}
De forma análoga, vemos que \[
s\left( f,P \right) \ge  -L \sum_{i\in \N} vol. B_i > -\epsilon
.\] Portanto, \[
-\epsilon < \int_X f < \epsilon \implies \int_X f = 0
.\] 

\exercise{3}

Nos restringindo primeiro ao espaço tridimensional, vemos que o princípio de Cavalieri segue a intuição de que, dados dois objetos bem comportados, se quais cortes que eu faça em ambos os objetos implica em vistas de áreas iguais, então esses objetos devem ter o mesmo volume.

Para provar isso, partimos do volume de $X$, ou seja, dado um n-bloco $A = A_n\times A_1 \subseteq\R^{n+1}$ tal que $A_n\in \R^{n}$ e $\overline{X},\overline{Y} \subseteq int A$, \[
vol. X = \int_{A} \chi_X
.\] Por Fubini, temos que \[
\int_{\R^{n+1}} \chi_X = \int_{A_1} \overline{\int_{A_n}} \chi_X\left( x, t \right) dx dt
.\] Veja que, por definição, para $t\in \R$, $x\in \R^{n}$ \[
\left( x,t \right) \in X \iff x \in X_t
.\] Assim, \[
\int_{A_1} \overline{\int_{A_n}} \chi_X\left( x, t \right) dx dt = \int_{A_1}\overline{\int_{A_n}} \chi_{X_t} \left( x \right) dx dt
.\] Agora, veja que \[
vol. X_t = \int_{A_n} \chi_{X_t} = \overline{\int_{A_n}} \chi_{X_t}
,\] portanto \[
\int_{A_1}\overline{\int_{A_n}} \chi_{X_t} \left( x \right) dx dt = \int_{A_1} vol. X_t dt
.\]

Se é verdade que $vol. X_t = vol. Y_t$ e aplicando os passos anteriores de forma análoga para $vol. Y$, temos
\begin{align*}
    vol. X &= \int_{A_1} vol. X_t \\
    &= \int_{A_1} vol. Y_t = vol. Y
.\end{align*}

\exercise{4}

\subexercise{a)}

Utilizando a distância euclidiana, temos que $B$ pode ser escrito como \[
B = \overline{B_2}\left( 0,0,0 \right) \cap \left( \R^{2}\times \R^{+} \right) \cap \overline{C_1^{z}\left( 0,0 \right) }
,\] onde $\overline{B_2}$ é uma bola fechada de raio 2 e $\overline{C_1^{z}}$ é um cilindro fechado de raio 1 em torno do eixo $z$. Dessa forma, \[
\partial B \subseteq S_2\left( 0,0,0 \right) \cup \left\{ \left( x,y,0 \right) , x,y\in \R \right\} \cup \partial C_1^{z}\left( 0,0 \right) 
.\] Como esses 3 conjuntos são subvariedades de dimensão 2, vimos que possuem medida nula e, portanto sua união também possui medida nula. Assim, como $B$ é claramente limitado, temos que é J-mensurável.

Um esboço de uma seção do conjunto, que pode ser descrito por uma revolução dessa seção, pode ser visto na figura abaixo.

\begin{figure}[H]
    \centering
    \includegraphics[width=0.8\textwidth]{figures/CalcIV_Q4b.pdf}
    \caption*{Esboço de uma seção do conjunto.}
    \label{fig:figures-CalcIV_Q4b-pdf}
\end{figure}

\subexercise{b)}

Tome $A :=A_x \times A_y\times A_z\subseteq\R^{3} $ um n-bloco com $B\subseteq int. A$. Assim, podemos calcular seu volume como \[
vol. B = \int_A \chi_B
.\] 

Por Fubini,  
\begin{align*}
    \int_A \chi_B &= \int_{A_x \times A_y} \overline{\int_{A_z}} \chi_B
.\end{align*}
Mas veja que \[
\chi_B\left( x,y,z \right) = \chi_{B_1\left( 0,0 \right)  }\left( x, y \right) \cdot \chi_{\left[ 0, 4-x^2 - y^2 \right] }\left( z \right) 
,\] logo, também pela continuidade da função constante,
\begin{align*}
    \int_{A_x \times A_y} \overline{\int_{A_z}} \chi_B &=  \int_{B_1\left( 0,0 \right) } \overline{\int}_{\left[ 0, 4-x^2-y^2 \right] } 1 \\
    &=  \int_{B_1\left( 0,0 \right) } \int_{\left[ 0, 4-x^2-y^2 \right] } 1 \\
    &= \int_{B_1\left( 0,0 \right) } \left( 4-x^2-y^2 \right)
.\end{align*}

Agora, visando aplicar a mudança de variáveis, defina
\begin{align*}
    \varphi : \left( 0,1 \right) \times \left( -\frac{\pi}{2},\frac{\pi}{2} \right)  &\longrightarrow \R^{2} \\
    \left( r,\theta \right)  &\longmapsto \varphi \left( r,\theta \right)  = \left( r\cos\theta, r\sin\theta \right) 
.\end{align*}
Veja que, assim definida, $\varphi $ é um difeomorfismo e \[
\int_{B_1\left( 0,0 \right) } \left( 4-x^2-y^2 \right) = \int_{\varphi \left( \left( 0,1 \right) \times \left( -\frac{\pi}{2}, \frac{\pi}{2} \right)  \right) } \left( 4-x^2-y^2 \right)
\] \[
    J_{\varphi } = \begin{bmatrix} \cos\theta & -r\sin\theta \\ \sin\theta & r\cos\theta \end{bmatrix} \implies \left| J_{\varphi } \right| = \left| r \right| = r
.\] Portanto,
\begin{align*}
    \int_{B_1\left( 0,0 \right) } \left( 4-x^2-y^2 \right) &= \int_0^{1} \int_{\frac{-\pi}{2}}^{\frac{\pi}{2}}  \left( 4-\left( r^2 \cos^2\theta + r^2\sin^2\theta \right)  \right) r d\theta dr \\
    &= \int_0^{1} \int_{\frac{-\pi}{2}}^{\frac{\pi}{2}} \left( 4 - r^2 \right) r d\theta dr \\
    &= \pi \int_0^{1}\left( 4r - r^3 \right) dr  \\
    &= \pi \left( 2r^2 - \frac{r^{4}}{4} \right)\Big|_0^{1} \\
    &=  \pi\left( 2 - \frac{1}{4} \right) \\
    &= \frac{7\pi}{4} 
.\end{align*}

\exercise{5}

Seja $A = A_x \times A_y\times A_z \subseteq\R^{3}$ um n-bloco tal que $\overline{B}\subseteq int A$. Por Fubini, podemos escrever o volume de $B$ como
\begin{align*}
    vol.B &= \int_A \chi_B \\
    &= \int_{A_x} \overline{\int_{A_y}} \overline{\int_{A_z}} \chi_B\left( x,y,z \right) dz dy dx 
.\end{align*}
Agora, vemos que \[
0\le z\le 2x + 2y -1 \implies x+y \ge \frac{1}{2} \implies y \ge \frac{1}{2} - x
.\] Assim, também pela definição de $B$, temos \[
\chi_B\left( x,y,z \right) = \chi_{\left[ 0,1 \right] }\left( x \right) \cdot \chi_{\left[ \frac{1}{2}-x, 1 \right] }\left( y \right) \cdot \chi_{\left[ 0, 2x+2y-1 \right] }\left( z \right) 
.\] Portanto, podemos reescrever 
\begin{align*}
    \int_{A_x} \overline{\int_{A_y}} \overline{\int_{A_z}} \chi_B\left( x,y,z \right) dz dy dx &= \int_{A_x}\chi_{\left[ 0,1 \right] }\left( x \right) \overline{\int_{A_y}} \chi_{\left[ \frac{1}{2}-x , 1\right] }\left( y \right)  \overline{\int_{A_z}} \chi_{\left[ 0, 2x+2y-1 \right] }\left( z \right)  dz dy dx \\
    &= \int_0^{1} \int_{\frac{1}{2}-x}^{1} \int_0^{2x+2y-1} dz dy dx \\
    &= \int_0^{1} \int_{\frac{1}{2}-x}^{1} \left( 2x + 2y -1 \right) dy dx  \\
    &= \int_0^{1} \left( \left( 2x-1 \right) \left( \frac{1}{2}+x \right) + \left( -x^2 + x + \frac{3}{4} \right) \right) dx \\
    &= \int_0^{1} \left( \left( 2x^2-\frac{1}{2} \right) + \left( -x^2 + x + \frac{3}{4} \right) \right) dx \\
    &= \int_0^{1} \left(  x^2 + x + \frac{1}{4} \right) dx \\
    &= \left( \frac{x^3}{3} + \frac{x^2}{2} + \frac{x}{4} \right)\Big|_0^{1}  \\
    &= \frac{13}{12}
.\end{align*}

\end{document}
