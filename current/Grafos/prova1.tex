\documentclass[a4paper]{report}
% Some basic packages
\usepackage[utf8]{inputenc}
\usepackage[T1]{fontenc}
\usepackage{textcomp}
\usepackage[english]{babel}
\usepackage{url}
\usepackage{graphicx}
\usepackage{float}
\usepackage{booktabs}
\usepackage{enumitem}

\pdfminorversion=7

% Don't indent paragraphs, leave some space between them
\usepackage{parskip}

% Hide page number when page is empty
\usepackage{emptypage}
\usepackage{subcaption}
\usepackage{multicol}
\usepackage{xcolor}

% Other font I sometimes use.
% \usepackage{cmbright}

% Math stuff
\usepackage{amsmath, amsfonts, mathtools, amsthm, amssymb}
% Fancy script capitals
\usepackage{mathrsfs}
\usepackage{cancel}
% Bold math
\usepackage{bm}
% Some shortcuts
\newcommand\N{\ensuremath{\mathbb{N}}}
\newcommand\R{\ensuremath{\mathbb{R}}}
\newcommand\Z{\ensuremath{\mathbb{Z}}}
\renewcommand\O{\ensuremath{\emptyset}}
\newcommand\Q{\ensuremath{\mathbb{Q}}}
\newcommand\C{\ensuremath{\mathbb{C}}}
\renewcommand\L{\ensuremath{\mathcal{L}}}

% Package for Petri Net drawing
\usepackage[version=0.96]{pgf}
\usepackage{tikz}
\usetikzlibrary{arrows,shapes,automata,petri}
\usepackage{tikzit}
\input{petri_nets_style.tikzstyles}

% Easily typeset systems of equations (French package)
\usepackage{systeme}

% Put x \to \infty below \lim
\let\svlim\lim\def\lim{\svlim\limits}

%Make implies and impliedby shorter
\let\implies\Rightarrow
\let\impliedby\Leftarrow
\let\iff\Leftrightarrow
\let\epsilon\varepsilon

% Add \contra symbol to denote contradiction
\usepackage{stmaryrd} % for \lightning
\newcommand\contra{\scalebox{1.5}{$\lightning$}}

% \let\phi\varphi

% Command for short corrections
% Usage: 1+1=\correct{3}{2}

\definecolor{correct}{HTML}{009900}
\newcommand\correct[2]{\ensuremath{\:}{\color{red}{#1}}\ensuremath{\to }{\color{correct}{#2}}\ensuremath{\:}}
\newcommand\green[1]{{\color{correct}{#1}}}

% horizontal rule
\newcommand\hr{
    \noindent\rule[0.5ex]{\linewidth}{0.5pt}
}

% hide parts
\newcommand\hide[1]{}

% si unitx
\usepackage{siunitx}
\sisetup{locale = FR}

% Environments
\makeatother
% For box around Definition, Theorem, \ldots
\usepackage{mdframed}
\mdfsetup{skipabove=1em,skipbelow=0em}
\theoremstyle{definition}
\newmdtheoremenv[nobreak=true]{definitie}{Definitie}
\newmdtheoremenv[nobreak=true]{eigenschap}{Eigenschap}
\newmdtheoremenv[nobreak=true]{gevolg}{Gevolg}
\newmdtheoremenv[nobreak=true]{lemma}{Lemma}
\newmdtheoremenv[nobreak=true]{propositie}{Propositie}
\newmdtheoremenv[nobreak=true]{stelling}{Stelling}
\newmdtheoremenv[nobreak=true]{wet}{Wet}
\newmdtheoremenv[nobreak=true]{postulaat}{Postulaat}
\newmdtheoremenv{conclusie}{Conclusie}
\newmdtheoremenv{toemaatje}{Toemaatje}
\newmdtheoremenv{vermoeden}{Vermoeden}
\newtheorem*{herhaling}{Herhaling}
\newtheorem*{intermezzo}{Intermezzo}
\newtheorem*{notatie}{Notatie}
\newtheorem*{observatie}{Observatie}
\newtheorem*{exe}{Exercise}
\newtheorem*{opmerking}{Opmerking}
\newtheorem*{praktisch}{Praktisch}
\newtheorem*{probleem}{Probleem}
\newtheorem*{terminologie}{Terminologie}
\newtheorem*{toepassing}{Toepassing}
\newtheorem*{uovt}{UOVT}
\newtheorem*{vb}{Voorbeeld}
\newtheorem*{vraag}{Vraag}

\newmdtheoremenv[nobreak=true]{definition}{Definition}
\newtheorem*{eg}{Example}
\newtheorem*{notation}{Notation}
\newtheorem*{previouslyseen}{As previously seen}
\newtheorem*{remark}{Remark}
\newtheorem*{note}{Note}
\newtheorem*{problem}{Problem}
\newtheorem*{observe}{Observe}
\newtheorem*{property}{Property}
\newtheorem*{intuition}{Intuition}
\newmdtheoremenv[nobreak=true]{prop}{Proposition}
\newmdtheoremenv[nobreak=true]{theorem}{Theorem}
\newmdtheoremenv[nobreak=true]{corollary}{Corollary}

% End example and intermezzo environments with a small diamond (just like proof
% environments end with a small square)
\usepackage{etoolbox}
\AtEndEnvironment{vb}{\null\hfill$\diamond$}%
\AtEndEnvironment{intermezzo}{\null\hfill$\diamond$}%
% \AtEndEnvironment{opmerking}{\null\hfill$\diamond$}%

% Fix some spacing
% http://tex.stackexchange.com/questions/22119/how-can-i-change-the-spacing-before-theorems-with-amsthm
\makeatletter
\def\thm@space@setup{%
  \thm@preskip=\parskip \thm@postskip=0pt
}


% Exercise 
% Usage:
% \exercise{5}
% \subexercise{1}
% \subexercise{2}
% \subexercise{3}
% gives
% Exercise 5
%   Exercise 5.1
%   Exercise 5.2
%   Exercise 5.3
\newcommand{\exercise}[1]{%
    \def\@exercise{#1}%
    \subsection*{Exercise #1}
}

\newcommand{\subexercise}[1]{%
    \subsubsection*{Exercise \@exercise.#1}
}


% \lecture starts a new lecture (les in dutch)
%
% Usage:
% \lecture{1}{di 12 feb 2019 16:00}{Inleiding}
%
% This adds a section heading with the number / title of the lecture and a
% margin paragraph with the date.

% I use \dateparts here to hide the year (2019). This way, I can easily parse
% the date of each lecture unambiguously while still having a human-friendly
% short format printed to the pdf.

\usepackage{xifthen}
\def\testdateparts#1{\dateparts#1\relax}
\def\dateparts#1 #2 #3 #4 #5\relax{
    \marginpar{\small\textsf{\mbox{#1 #2 #3 #5}}}
}

\def\@lecture{}%
\newcommand{\lecture}[3]{
    \ifthenelse{\isempty{#3}}{%
        \def\@lecture{Lecture #1}%
    }{%
        \def\@lecture{Lecture #1: #3}%
    }%
    \subsection*{\@lecture}
    \marginpar{\small\textsf{\mbox{#2}}}
}



% These are the fancy headers
\usepackage{fancyhdr}
\pagestyle{fancy}

% LE: left even
% RO: right odd
% CE, CO: center even, center odd
% My name for when I print my lecture notes to use for an open book exam.
% \fancyhead[LE,RO]{Gilles Castel}

\fancyhead[RO,LE]{\@lecture} % Right odd,  Left even
\fancyhead[RE,LO]{}          % Right even, Left odd

\fancyfoot[RO,LE]{\thepage}  % Right odd,  Left even
\fancyfoot[RE,LO]{}          % Right even, Left odd
\fancyfoot[C]{\leftmark}     % Center

\makeatother




% Todonotes and inline notes in fancy boxes
\usepackage{todonotes}
\usepackage{tcolorbox}

% Make boxes breakable
\tcbuselibrary{breakable}

% Verbetering is correction in Dutch
% Usage: 
% \begin{verbetering}
%     Lorem ipsum dolor sit amet, consetetur sadipscing elitr, sed diam nonumy eirmod
%     tempor invidunt ut labore et dolore magna aliquyam erat, sed diam voluptua. At
%     vero eos et accusam et justo duo dolores et ea rebum. Stet clita kasd gubergren,
%     no sea takimata sanctus est Lorem ipsum dolor sit amet.
% \end{verbetering}
\newenvironment{verbetering}{\begin{tcolorbox}[
    arc=0mm,
    colback=white,
    colframe=green!60!black,
    title=Opmerking,
    fonttitle=\sffamily,
    breakable
]}{\end{tcolorbox}}

% Noot is note in Dutch. Same as 'verbetering' but color of box is different
\newenvironment{noot}[1]{\begin{tcolorbox}[
    arc=0mm,
    colback=white,
    colframe=white!60!black,
    title=#1,
    fonttitle=\sffamily,
    breakable
]}{\end{tcolorbox}}




% Figure support as explained in my blog post.
\usepackage{import}
\usepackage{xifthen}
\usepackage{pdfpages}
\usepackage{transparent}
\newcommand{\incfig}[1]{%
    \def\svgwidth{\columnwidth}
    \import{./figures/}{#1.pdf_tex}
}

% Fix some stuff
% %http://tex.stackexchange.com/questions/76273/multiple-pdfs-with-page-group-included-in-a-single-page-warning
\pdfsuppresswarningpagegroup=1


% My name
\author{Bruno M. Pacheco}

 
\begin{document}
 
\title{Prova 1}
\author{Bruno M. Pacheco (16100865)\\
Grafos}
 
\maketitle
 
\exercise{1}

Podemos utilizar uma modificação do algoritmo de busca em largura (Algoritmo 3 das anotações), que já utiliza essa abordagem de varrer o grafo pelos vizinhos do vértice de origem. Não precisamos mais registrar a distância nem o predecessor. Entretanto, precisamos controlar os níveis de cada vértice em relação ao original, uma vez que denotam o instante de tempo em que foram contaminados. Para isso, ao invés de varrer as distâncias ao final, podemos controlar os recém infectados ($Q_{i+1}$). O algoritmo de busca em largura modificado pode ser observado abaixo.

\begin{algorithm}[H]
    \KwIn{um grafo $G=\left( V, E \right) $, vértice $x\in V$, inteiro $k$}
    $C_v \gets  $ false $\, \forall v\in V$ \\
    $C_x\gets $ true \\
    // Mantemos controle da propagação por unidade de tempo \\
    $Q_0\gets $ Fila()\\
    $Q_0$.enqueue($x$) \\

    // Conterá as unidades de tempo em que houveram mais do que $k$ infecções \\
    $K \gets $ conjuntoVazio()

    $i \gets 0$

    \While{$Q_{i}$.empty()$=$ false}{
	$i\gets i + 1$\\
	$Q_{i} \gets $ Fila() \\
	\ForEach{$u\in Q_{i-1}$}{
	    \ForEach{$v\in N\left( u \right) $}{
		\If{$C_v = $ false}{
		    $C_v \gets $ true \\
		    $Q_{i}$.enqueue($v$)
		}
	    }
	} 
	\If{$\left| Q_{i} \right|  \ge k$}{
	    $K$.adiciona$(i)$
	}
    }

    \Return{$K$}
\end{algorithm}

Apesar de adicionar um laço, não impactamos a quantidade de iterações em cima dos vértices, uma vez que esse laço se faz presente somente para separar as unidades de tempo. Portanto, mantemos a mesma complexidade computacional do algoritmo de busca em largura, ou seja, $O\left( \left| V \right| + \left| E \right|   \right) $.

\exercise{2}

Com as informações fornecidas da propriedade, podemos montar um grafo completamente conectado em que cada vértice representa um ponto de aferição ou a guarita e as arestas são valoradas pelas distâncias entre as localidades de interesse. Dessa forma, o algoritmo de Bellmann-Held-Karp como definido no Algoritmo 7 das anotações, nos retorna a distância total percorrida no caminho mínimo. Dessa forma, nos basta verificar quais dos funcionários são rápidos o suficiente. A implementação pode seguir o descrito no algoritmo abaixo.

\begin{algorithm}[H] 
    \KwIn{conjunto de funcionários $F$, função $v$, conjunto $A$, ponto $g$, função $d$, valor $k$}
    // Monta o grafo dos pontos pelos quais os guardas passam \\
    $V \gets A \cup \left\{ g \right\} $ \\
    $E \gets V\times V$ \\
    \ForEach{$\left\{ u, v \right\} \in E$}{
	$w\left( \left\{ u, v \right\}  \right) \gets \|d\left( u \right) - d\left( v \right) \|$ 
    }
    // $c$ a distância do ciclo hamiltoniano mínimo \\
    $c \gets$ Bellmann-Held-Karp$\left( G \right) $ \\

    $F_v \gets $ conjuntoVazio$()$\\
    \ForEach{$f \in F$}{
	\If{$\frac{c}{v\left( f \right)}  < k$}{
	    $F_v$.adiciona$(f)$ \\
	}
    }
    \Return{$F_v$}
\end{algorithm}

\exercise{3}

Podemos abordar esse problema utilizando o algoritmo de Dijkstra conforme Algoritmo 11 das anotações. Usaremos ele para encontrar os caminhos mínimos de $s$ até os vértices de $P$ e desses para todos os vértices de $I$. Para todo $i \in I$, como o grafo não é dirigido, podemos encontrar o caminho de todos os postos para $i$ simplesmente invertendo o caminho mínimo de $i$ para os postos, que pode ser obtido aplicando Dijkstra com $i$ como fonte. A partir disso, basta encontrar o posto de combustível tal que a soma dos caminhos mínimos de $s$ para $p$ e de $p$ para $i$ seja mínima.

Essa abordagem pode ser vista no algoritmo abaixo. Note o uso da função "caminho", que indica a construção de um caminho usando o vetor de antecessores conforme resultante do algoritmo de Dijkstra, com um dado destino. Além disso, a função "reverte" é uma simples inversão do caminho de entrada.

\begin{algorithm}
    \KwIn{um grafo $G=\left( V, A, w \right) $, vértice $s \in V$, $P\subseteq V$, $I \subseteq V$}
    $D^{(s)}, A^{(s)} \gets $ Dijkstra$\left( G, s \right) $ \\

    $D_i \gets \infty\, \forall i \in I$ \\
    $p_i \gets $ null $\forall i \in I$ \\
    \ForEach{$i \in I$}{
	$D^{(i)}, A^{(i)} \gets $ Dijkstra$\left( G, i \right) $ \\

	$V_i \gets$
	$G_i \gets \left(  \right) $
	\ForEach{$p \in P$}{
	    \If{$D^{(s)}_p + D^{(i)}_p < D_i$}{
		$D_i \gets D^{(s)}_p + D^{(i)}_p $ \\
		$p_i \gets p$
	    }
	}
	$c^{(s)}_i \gets \,$caminho$\left( A^{(s)}, p \right)$ \\
	$c^{(i)}_i \gets \,$caminho$\left( A^{(i)}, p \right)$ \\
	$c_i \gets c^{(s)}_i + $ reverte$\left(  c^{(i)}_i\right) $
    }
    \Return{c}
\end{algorithm}

\exercise{4}

Podemos modificar o algoritmo de Floyd-Warshall apresentado no Algoritmo 12 das anotações para computar a quantidade de arcos do caminho mínimo durante os relaxamentos, ou seja, cada vez que há um relaxamento nós atualizamos a quantidade de arcos no caminho. Para isso, podemos utilizar uma nova matriz de adjacências. Uma implementação possível pode ser vista no algoritmo abaixo.

\begin{algorithm}[H] 
    \KwIn{um grafo $G=\left( V, E, w \right) $}
    $D^{(0)} \gets  W\left( G  \right) $\\

    // $G_a$ o grafo com arestas de peso unitário, para gerar a matriz $A$ \\
    $w_a\left( \left( u,v \right)  \right) \gets 1\, \forall \left( u,v \right) \in E$ \\
    $G_a \gets \left( V, E, w_a \right) $
    $A^{(0)} \gets W\left( G_a \right) $ \\

    \ForEach{$k\in V$}{
	$D^{(k)} \gets D^{(k-1)}$, uma cópia  \\
	$A^{(k)} \gets A^{(k-1)}$, uma cópia  \\
	\ForEach{$u\in V$}{
	    \ForEach{$v\in V$}{
		\If{$d_{uk}^{(k-1)} + d_{kv}^{(k-1)} < d_{uv}^{(k)}$}{
		    $d_{uv}^{(k)} \gets d_{uk}^{(k-1)} + d_{kv}^{(k-1)}$ \\
		    $a_{uv}^{(k)} \gets a_{uk}^{(k-1)} + a_{kv}^{(k-1)}$ \\
		}
	    }
	}
    }

    \Return{$\left( D^{(\left| V \right| )}, A^{(\left| V \right| )} \right) $}
\end{algorithm}

Dessa forma, o algoritmo retorna não só o custo dos caminhos mínimos entre quaisquer dois vértices através da matriz $D$ como também a quantidade de arcos no caminho mínimo através da matriz $A$, uma vez que essa é "relaxada" (nem sempre diminui a quantidade de arcos) somente quando a matriz $D$ o é.

\end{document}
