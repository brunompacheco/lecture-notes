\documentclass[a4paper]{report}
% Some basic packages
\usepackage[utf8]{inputenc}
\usepackage[T1]{fontenc}
\usepackage{textcomp}
\usepackage[english]{babel}
\usepackage{url}
\usepackage{graphicx}
\usepackage{float}
\usepackage{booktabs}
\usepackage{enumitem}

\pdfminorversion=7

% Don't indent paragraphs, leave some space between them
\usepackage{parskip}

% Hide page number when page is empty
\usepackage{emptypage}
\usepackage{subcaption}
\usepackage{multicol}
\usepackage{xcolor}

% Other font I sometimes use.
% \usepackage{cmbright}

% Math stuff
\usepackage{amsmath, amsfonts, mathtools, amsthm, amssymb}
% Fancy script capitals
\usepackage{mathrsfs}
\usepackage{cancel}
% Bold math
\usepackage{bm}
% Some shortcuts
\newcommand\N{\ensuremath{\mathbb{N}}}
\newcommand\R{\ensuremath{\mathbb{R}}}
\newcommand\Z{\ensuremath{\mathbb{Z}}}
\renewcommand\O{\ensuremath{\emptyset}}
\newcommand\Q{\ensuremath{\mathbb{Q}}}
\newcommand\C{\ensuremath{\mathbb{C}}}
\renewcommand\L{\ensuremath{\mathcal{L}}}

% Package for Petri Net drawing
\usepackage[version=0.96]{pgf}
\usepackage{tikz}
\usetikzlibrary{arrows,shapes,automata,petri}
\usepackage{tikzit}
\input{petri_nets_style.tikzstyles}

% Easily typeset systems of equations (French package)
\usepackage{systeme}

% Put x \to \infty below \lim
\let\svlim\lim\def\lim{\svlim\limits}

%Make implies and impliedby shorter
\let\implies\Rightarrow
\let\impliedby\Leftarrow
\let\iff\Leftrightarrow
\let\epsilon\varepsilon

% Add \contra symbol to denote contradiction
\usepackage{stmaryrd} % for \lightning
\newcommand\contra{\scalebox{1.5}{$\lightning$}}

% \let\phi\varphi

% Command for short corrections
% Usage: 1+1=\correct{3}{2}

\definecolor{correct}{HTML}{009900}
\newcommand\correct[2]{\ensuremath{\:}{\color{red}{#1}}\ensuremath{\to }{\color{correct}{#2}}\ensuremath{\:}}
\newcommand\green[1]{{\color{correct}{#1}}}

% horizontal rule
\newcommand\hr{
    \noindent\rule[0.5ex]{\linewidth}{0.5pt}
}

% hide parts
\newcommand\hide[1]{}

% si unitx
\usepackage{siunitx}
\sisetup{locale = FR}

% Environments
\makeatother
% For box around Definition, Theorem, \ldots
\usepackage{mdframed}
\mdfsetup{skipabove=1em,skipbelow=0em}
\theoremstyle{definition}
\newmdtheoremenv[nobreak=true]{definitie}{Definitie}
\newmdtheoremenv[nobreak=true]{eigenschap}{Eigenschap}
\newmdtheoremenv[nobreak=true]{gevolg}{Gevolg}
\newmdtheoremenv[nobreak=true]{lemma}{Lemma}
\newmdtheoremenv[nobreak=true]{propositie}{Propositie}
\newmdtheoremenv[nobreak=true]{stelling}{Stelling}
\newmdtheoremenv[nobreak=true]{wet}{Wet}
\newmdtheoremenv[nobreak=true]{postulaat}{Postulaat}
\newmdtheoremenv{conclusie}{Conclusie}
\newmdtheoremenv{toemaatje}{Toemaatje}
\newmdtheoremenv{vermoeden}{Vermoeden}
\newtheorem*{herhaling}{Herhaling}
\newtheorem*{intermezzo}{Intermezzo}
\newtheorem*{notatie}{Notatie}
\newtheorem*{observatie}{Observatie}
\newtheorem*{exe}{Exercise}
\newtheorem*{opmerking}{Opmerking}
\newtheorem*{praktisch}{Praktisch}
\newtheorem*{probleem}{Probleem}
\newtheorem*{terminologie}{Terminologie}
\newtheorem*{toepassing}{Toepassing}
\newtheorem*{uovt}{UOVT}
\newtheorem*{vb}{Voorbeeld}
\newtheorem*{vraag}{Vraag}

\newmdtheoremenv[nobreak=true]{definition}{Definition}
\newtheorem*{eg}{Example}
\newtheorem*{notation}{Notation}
\newtheorem*{previouslyseen}{As previously seen}
\newtheorem*{remark}{Remark}
\newtheorem*{note}{Note}
\newtheorem*{problem}{Problem}
\newtheorem*{observe}{Observe}
\newtheorem*{property}{Property}
\newtheorem*{intuition}{Intuition}
\newmdtheoremenv[nobreak=true]{prop}{Proposition}
\newmdtheoremenv[nobreak=true]{theorem}{Theorem}
\newmdtheoremenv[nobreak=true]{corollary}{Corollary}

% End example and intermezzo environments with a small diamond (just like proof
% environments end with a small square)
\usepackage{etoolbox}
\AtEndEnvironment{vb}{\null\hfill$\diamond$}%
\AtEndEnvironment{intermezzo}{\null\hfill$\diamond$}%
% \AtEndEnvironment{opmerking}{\null\hfill$\diamond$}%

% Fix some spacing
% http://tex.stackexchange.com/questions/22119/how-can-i-change-the-spacing-before-theorems-with-amsthm
\makeatletter
\def\thm@space@setup{%
  \thm@preskip=\parskip \thm@postskip=0pt
}


% Exercise 
% Usage:
% \exercise{5}
% \subexercise{1}
% \subexercise{2}
% \subexercise{3}
% gives
% Exercise 5
%   Exercise 5.1
%   Exercise 5.2
%   Exercise 5.3
\newcommand{\exercise}[1]{%
    \def\@exercise{#1}%
    \subsection*{Exercise #1}
}

\newcommand{\subexercise}[1]{%
    \subsubsection*{Exercise \@exercise.#1}
}


% \lecture starts a new lecture (les in dutch)
%
% Usage:
% \lecture{1}{di 12 feb 2019 16:00}{Inleiding}
%
% This adds a section heading with the number / title of the lecture and a
% margin paragraph with the date.

% I use \dateparts here to hide the year (2019). This way, I can easily parse
% the date of each lecture unambiguously while still having a human-friendly
% short format printed to the pdf.

\usepackage{xifthen}
\def\testdateparts#1{\dateparts#1\relax}
\def\dateparts#1 #2 #3 #4 #5\relax{
    \marginpar{\small\textsf{\mbox{#1 #2 #3 #5}}}
}

\def\@lecture{}%
\newcommand{\lecture}[3]{
    \ifthenelse{\isempty{#3}}{%
        \def\@lecture{Lecture #1}%
    }{%
        \def\@lecture{Lecture #1: #3}%
    }%
    \subsection*{\@lecture}
    \marginpar{\small\textsf{\mbox{#2}}}
}



% These are the fancy headers
\usepackage{fancyhdr}
\pagestyle{fancy}

% LE: left even
% RO: right odd
% CE, CO: center even, center odd
% My name for when I print my lecture notes to use for an open book exam.
% \fancyhead[LE,RO]{Gilles Castel}

\fancyhead[RO,LE]{\@lecture} % Right odd,  Left even
\fancyhead[RE,LO]{}          % Right even, Left odd

\fancyfoot[RO,LE]{\thepage}  % Right odd,  Left even
\fancyfoot[RE,LO]{}          % Right even, Left odd
\fancyfoot[C]{\leftmark}     % Center

\makeatother




% Todonotes and inline notes in fancy boxes
\usepackage{todonotes}
\usepackage{tcolorbox}

% Make boxes breakable
\tcbuselibrary{breakable}

% Verbetering is correction in Dutch
% Usage: 
% \begin{verbetering}
%     Lorem ipsum dolor sit amet, consetetur sadipscing elitr, sed diam nonumy eirmod
%     tempor invidunt ut labore et dolore magna aliquyam erat, sed diam voluptua. At
%     vero eos et accusam et justo duo dolores et ea rebum. Stet clita kasd gubergren,
%     no sea takimata sanctus est Lorem ipsum dolor sit amet.
% \end{verbetering}
\newenvironment{verbetering}{\begin{tcolorbox}[
    arc=0mm,
    colback=white,
    colframe=green!60!black,
    title=Opmerking,
    fonttitle=\sffamily,
    breakable
]}{\end{tcolorbox}}

% Noot is note in Dutch. Same as 'verbetering' but color of box is different
\newenvironment{noot}[1]{\begin{tcolorbox}[
    arc=0mm,
    colback=white,
    colframe=white!60!black,
    title=#1,
    fonttitle=\sffamily,
    breakable
]}{\end{tcolorbox}}




% Figure support as explained in my blog post.
\usepackage{import}
\usepackage{xifthen}
\usepackage{pdfpages}
\usepackage{transparent}
\newcommand{\incfig}[1]{%
    \def\svgwidth{\columnwidth}
    \import{./figures/}{#1.pdf_tex}
}

% Fix some stuff
% %http://tex.stackexchange.com/questions/76273/multiple-pdfs-with-page-group-included-in-a-single-page-warning
\pdfsuppresswarningpagegroup=1


% My name
\author{Bruno M. Pacheco}

 
\begin{document}
 
\title{Prova 2}
\author{Bruno M. Pacheco (16100865)\\
Grafos}
 
\maketitle
 
\exercise{1}

Determinar se é possível acessar qualquer intersecção da cidade a partir de qualquer outra interseção é o mesmo que determinar se existe uma componente fortemente conexa que contenha todos os vértices de $G$. A partir do algoritmo de Componentes-Fortemente-Conexas (Algoritmo 15), basta que verifiquemos se existe mais de uma árvore gerada pelo algoritmo, o que é uma garantia de que não há uma componente fortemente conexa que contenha todos os vértices de $G$.

Para isso, o algoritmo abaixo agrupa os vértices (utilizando a lista $R$) através das raízes das árvores geradas pelo algoritmo de componentes fortemente conexas mencionado. Assim, caso haja mais de uma componente fortemente conexa, a lista possuirá mais de um valor único. Foi utilizada uma função genérica para essa verificação.

\begin{algorithm}
    \KwIn{um grafo $G=\left( V, A, w \right) $}
    // Utiliza-se a letra $P$ para para evitar confusão com os arcos \\
    $P\gets $Componentes-Fortemente-Conexas $\left( G \right) $ \\
    // Inicializa lista das raízes da árvore de cada elemento \\
    $R_v \gets \textbf{null} \, \forall v\in V$ \\

    // Descobre as raízes da árvore de cada vértice \\
    \ForEach{$u \in V : A_u \neq \textbf{null}$}{
	// Todos os predecessores de $u$ (incluindo esse) \\
	$P' \gets \left\{ u \right\} $ \\

	\Repeat{$A_p = \textbf{null}$}{
	    $p\gets A_u$ \\
	    $P' \text{.adiciona} \left( p \right) $ \\
	    \If{$R_p \neq \textbf{null}$}{
		$p\gets R_p$
	    }
	}
	// $r$ é a raiz da árvore \\
	$r \gets p$  \\
	$R_p \gets r\, \forall p \in P'$
    }
    \Return{$\text{contaÚnicos}\left( R \right) $ > 1}
\end{algorithm}

\exercise{2}

Para evitar a construção da ordenação topológica completa do grafo, pode-se executar uma busca em largura corrigindo as interdependências dos vértices que precedem $v$. Para isso, uma modificação do algoritmo de busca em largura visitando somente os predecessores ($N^{-}$) é proposta no algoritmo abaixo. Veja que a abordagem garante a ordenação pelos níveis de profundidade da árvore associada ao algoritmo BFS adicionando os vértices visitados à ordenação $O$ retornada. Além disso, caso hajam interdependências entre as dependências indiretas de $t$, i.e., caso algum vértice de um nível superior seja predecessor de um vértice de um nível inferior, ele é reorganizado para o início junto com os seus predecessores já visitados, de forma recursiva, por isso é necessário que o grafo não possua ciclos nas dependências de $t$.

Apesar dessa abordagem ser pouco otimizada (principalmente na reorganização da ordenação) ela permite mais flexibilidade uma vez que os vértices que sucedem $t$ podem formar ciclos, além de evitar a construção da ordenação topológica do grafo inteiro, o que seria bem custoso para grafos grandes e desconexos.

\begin{algorithm}[H]
    \KwIn{um grafo $G=\left( V, A \right) $ e um vértice $t\in V$}
    $C_v \gets  $ false $\, \forall v\in V$ \\
    $C_t\gets $ true \\
    $Q\gets $ Fila()\\
    $Q$.enqueue($t$) \\

    // Ordenação das dependências (diretas ou indiretas) de $t$ \\
    $O \gets \left(  \right) $

    \While{$Q$.empty()$=$ false}{
	$u\gets Q$.dequeue$()$ \\
	\ForEach{$v\in N^{-}\left( u \right) $}{
	    \If{$C_v = $ false}{
		$C_v \gets $ true \\
		$Q$.enqueue($v$) \\
		$O \gets \left( v \right) \cup O$ \\
	    }
	    \Else{
		// Realoca $v$ e seus predecessores de forma recursiva \\
		$O\gets $ReorganizaVértice($O, v, C$) \\
	    }
	}
    }

    \Return{$O$}
\end{algorithm}

\begin{algorithm}
    \TitleOfAlgo{ReorganizaVértice} 
    \KwIn{uma ordenação $O$, um vértice $v$, uma lista $C$}
    $O \gets \left( v \right) \cup  \left(  O \setminus (v)\right) $ \\
    \ForEach{$u\in N^{-}\left( v \right) $}{
	\If{$C_u = $false}{
	    $O \gets $ReorganizaVértice($O, u, C$)
	}
    }
    \Return{$O$}
\end{algorithm}

\exercise{3}

Vemos que o algoritmo enunciado conecta duas árvores através da aresta de menor custo. Primeiro, é claro que o resultado não possui ciclos, uma vez que $A_1$ e $A_2$ são árvores e, portanto, um ciclo no resultado deveria incluir $m$, mas como ele é a única aresta entre os vértices de $V_1$ e de $V_2$, é impossível que o resultado possua um ciclo. Assim, o resultado é uma árvore geradora para $G_3$, uma vez que cobre todos os seus vértices e não possui ciclos. Agora, nos resta verificar que é mínima. Veja que não existe aresta para ser substituída dentre aquelas de $A_1$ e $A_2$, logo, somente há espaço para que a aresta $m$ seja otimizada. Entretanto, pela verificação feita no algoritmo, sabe-se que $m$ é a aresta de menor peso dentre aquelas de $E$, logo, sua substituição não acarretará em uma árvore geradora de peso menor. Portanto, a árvore gerada pelo algoritmo é uma árvore geradora mínima de $G_3$.

\exercise{4}

Se faz necessário 2 verificações:
\begin{itemize}
    \item $A$ deve ser uma árvore geradora de $G$ (conexa e sem ciclos)
    \item $A$ tem peso mínimo
\end{itemize}

As primeiras duas verificações são feitas por uma modificação do algoritmo de busca em largura como no algoritmo \texttt{ÉÁrvore}. Caso algum vértice seja visitado mais de uma vez (salvo pelo antecessor na árvore associada) então $A$ possui um ciclo, logo, não é uma árvore. Além disso, se algum vértice de $G$ não é visitado durante a busca, $A$ não é uma árvore geradora de $G$. 

A verificação de que $A$ é uma árvore mínima é feita comparado o seu peso total com o de uma árvore garantidamente mínima, i.e., encontrada por um algoritmo como o de Kruskal (Algoritmo 21 das notas de aula).

\begin{algorithm}
    \KwIn{um grafo não dirigido e ponderado $G=\left( V,E,w \right)$ e um subconjunto $A\subseteq E$}
    // Verifica que $A$ é uma árvore \\
    $s \gets $random($V$) \\
    $G_A \gets \left( V, A \right) $ \\
    \If{\textbf{not} ÉÁrvore($G, s$)}{
	\Return{false}
    }

    // Verifica se $A$ é mínima \\
    $W_A \gets \sum_{e \in A} w(e)$ \\

    $A' \gets $Kruskal($G$) \\
    $W_{A'} \gets \sum_{e' \in A'} w(e')$ \\

    \If{$W_A = W_{A'}$}{
	\Return{true}
    }
    \Else{
	\Return{false}
    }
\end{algorithm}

\begin{algorithm}
    \TitleOfAlgo{ÉÁrvore} 
    \KwIn{um grafo $G=\left( V,E \right) $ e um vértice de origem $s \in V$}

    $C_v \gets  $ false $\, \forall v\in V$ \\
    $C_s\gets $ true \\
    $A_v \gets \textbf{null},\, \forall v\in V$ \\

    $Q\gets $ Fila()\\
    $Q$.enqueue($s$) \\

    \While{$Q$.empty()$=$ false}{
	$u\gets Q$.dequeue$()$ \\
	// Exclui o antecessor de $u$ da verificação \\
	\ForEach{$v\in N\left( u \right) \setminus \left\{ A_u \right\}  $}{
	    \If{$C_v = $ false}{
		$C_v \gets $ true \\
		$A_v \gets u$ \\
		$Q$.enqueue($v$)
	    }
	    \Else{
		\Return{false}
	    }
	}
    }
    \If{false $\in C$}{
	\Return{false}
    }
    \Else{
	\Return{true}
    }
\end{algorithm}

\end{document}
