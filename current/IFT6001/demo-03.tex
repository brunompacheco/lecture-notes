\demo{3}{18.09.2024}{}
 
\exercise{2}

Because we can write the recurrence for $n\ge 0$ as \[
t_{n} - 3t_{n-1} + 2 t_{n-2} = 2^{n-2}3 = 2^{n} \frac{3}{4}
,\] we see that it is a linear inhomogeneous recurrence.
In other words, the right-hand side is the solution to $(S-2)f(n) = 0$, which implies that $t_n$ must be the solution to \[
    (S-2)(S^2 -3S + 2)t_{n} = 0
.\]

The characteristic polynomial is, thus, $(x-2)(x-2)(x-1)$.
Because $2$ is a root with multiplicity 2, our most general solution must be of the form \[
    t_{n} = c_1 1^{n} + c_2 2^{n} + c_3 n 2^{n} = c_1 + (c_2 + c_3 n)2^{n}
.\]
Given the above, we use our initial conditions for
\begin{align}
    T(0) = 1 &\implies c_1+c_2 = 1 \\
    T(1) = 2 &\implies c_1+ 2c_2 + 2c_3 = 2 \implies c_2+2c_3 = 1 \\
    T(2) = 7 &\implies c_1+ (c_2 + 2c_3)4 = 7 \implies 3c_2 + 8c_3 = 6
.\end{align}
(1.2) and (1.3) imply that $c_3 = \frac{3}{2}$ and $c_2=-2$, which let's us use (1.1) to find $c_1=3$.
Therefore, \[
T(n) = 3 + (\frac{3}{2}n - 2)2^{n}
.\] 

\exercise{3}

Given that $n$ is a power of 2, then let $k=\log n \in \N$, and we can write \[
    T(2^{k}) = \begin{cases}
	1 &,\, k=0 \\
	2T(2^{k-1}) + k &,\,k\ge 1
    \end{cases}
.\] 
In turn, this allows us to write the equivalent linear inhomogeneous recurrence \[
t_k - 2t_{k-1} = k
,\] with $t_k = T(2^{k})$.
The characteristic polynomial is, then, $(x-2)(x-1)^2$, which implies that our general form for the solution is  \[
t_k = c_1 2^{k} + c_2 1^{k} + c_3 k 1^{k} = c_1 2^{k} + c_2 + c_3 k
.\] 
With the initial conditions, we have
\begin{align*}
    t_0 &= 1 \implies c_1+c_2=1  \\
    t_1 &= 3 \implies 2c_1+c_2+c_3 = 3 \\
    c_2 &= 8 \implies 4c_1 + c_2 = 4 \\
,\end{align*}
which lets us infer that the solution is \[
    t_k = 3 2^{k} -2 - k
,\] or, equivalently, \[
    T(n) = 3n - 2 - \log n
.\] 

\exercise{4}

\subexercise{1}

In the Akra-Bazzi form, we have $\ell=3$, $b=2$ and $g(n)=27$.
Because $g(n) = 27 \in \Theta(1) \subseteq \O(n^{0})$, we have $k=0 < \log_b \ell = \log_2 3$ and, thus, \[
    T \in \Theta \left( n^{\log_2 3} \right)
.\] 

\subexercise{2}

Now, we have $\ell=2$, $b=2$, and $g(n) = 2n$, thus, $k=1=\log_b \ell$.
Therefore, we have \[
    T(n) = \Theta(n^{\log_b \ell} \log n) = \Theta(n \log n)
.\] 

\subexercise{3}

In the Akra-Bazzi form, we have $r_1(n)=n / 2$ and $r_2(n) = \frac{n}{4}$. 

Note that $r_1$
The master theorem is not applicable here, 

\subexercise{4}

In the Akra-Bazzi form, we have $r_i(n)=n / 4, i=1,\ldots,32$, and $g \in \Theta(n^3)$.
Therefore, because $k=3 > \log_4 32 $, the master theorem gives us $T \in \Theta(n^3)$.

\subexercise{5}

Note that $g$ is not harmonious, so this is not an Akra-Bazzi recurrence.

\exercise{5}

We can rewrite the general case $n \ge 2$ as a homogeneous linear recurrence \[
    T(n) - 2T(n-1) + 5T(n-2) = 0
.\] The roots of its characteristic polynomial are 
\begin{align*}
    x &= \frac{-b + |\sqrt{b^2 - 4ac} | }{2a} \\
    &= \frac{2 + |\sqrt{-16}|}{2} \\
.\end{align*}
\textcolor{red}{\textbf{TODO: continue with the complex roots.}}

