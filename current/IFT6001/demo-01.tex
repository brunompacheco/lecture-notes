\demo{1}{05.09.2024}{}
 
\exercise{1}

\textbf{Rappel des techniques de preuve}

Direct proof: $p\implies q$.

Proof by counterposition(?): $p\implies q$ is equivalent to $\neg q \implies \neg p$.

Proof by contradiction: assume hypothesis is correct, show that it leads to a contradiction.

\exercise{2}

Note that
\begin{align*}
    3n \log n! &= 3n \sum_{i=1}^{n} \log i \\
    &\le  3n \sum_{i=1}^{n} \log n \\
    &= 3n^2 \log n \\
.\end{align*}
Therefore, $3n \log n! \le C n^2 \log n$ for $C=2$.

\exercise{3}

\paragraph{($\implies$)}
The implication is quite direct, as every function is a member of its associated $O$-set.
Precisely, let $f,g: \N \longrightarrow \N$ be functions.
Because $f(n) \in O(f(n))$, then $O(f(n)) = O(g(n)) \implies f(n) \in O(g(n))$. \emph{Mutatis mutandis}, the same is true for $g(n)$.

\paragraph{($\implies$)}
Suppose $f,g: \N \longrightarrow \N$ are such that $f(n) \in O(g(n))$ and $g(n) \in O(f(n))$.
Then, $\exists C_f > 0$ such that $|f(n)| \le C_f |g(n)|$.
Now, note that $\forall f'(n) \in O(f(n)), \exists C > 0$ such that
\begin{align*}
    |f'(n)| &\le C |f(n)| \\
    &\le C C_f |g(n)|  \\
.\end{align*}
Therefore, $O(f(n)) \subseteq O(g(n))$.
\emph{Mutatis mutandis}, the same is true for $O(g(n))$, i.e., $O(g(n)) \subseteq O(f(n))$. $\blacksquare$

\exercise{4}

We can easily show that $f(n) = \sum_{i=1}^{n} \log i \in O(n \log n)$, as
\begin{align*}
    \sum_{i=1}^{n} \log i &\le \sum_{i=1}^{n} \log n \\
    &= n \log n \\
.\end{align*}

Now, to show that $f(n) \in \Omega(n \log n)$, we have that
\begin{align*}
    \sum_{i=1}^{n} \log i &\ge \sum_{i=\left\lceil \frac{n}{2} \right\rceil }^{n} \log i \\
    &\ge \sum_{i=\left\lceil \frac{n}{2} \right\rceil }^{n} \log \left\lceil \frac{n}{2} \right\rceil  \\
    &\ge \frac{n}{2} \log \frac{n}{2} \\
    &= \frac{n}{2}\left( \log n - \log 2 \right) \\
    &\ge \frac{n}{2} \log n
    % &\ge C \frac{n}{2} \log n  \\
,\end{align*}
which, with $C = \frac{1}{2}$, shows that $f(n) \in \Omega(n \log n)$.

\exercise{5}

\subexercise{1}
\paragraph{Vrai}
For any $n\ge 2$, $n^2 + n \le n^3$.

\subexercise{2}
\paragraph{Faux}
Suppose $N\in \N$ and $C>0$ such that $\forall n \ge N \implies n^2 \ge C n^3$ (I'm omitting the modulus operation because both functions are non-negative).
However, for $n\ge 1$, this would imply that $Cn \le 1$, which is not true for $n > \frac{1}{C}$, a contradiction.

\subexercise{3}
\paragraph{Vrai}
By the limit rule, we have
\begin{align*}
    \lim_{n \to \infty} \frac{2^{n+1}}{2^{n}} &= \lim_{n \to \infty} 2 \frac{2^{n}}{2^{n}} \\
    &= 2
,\end{align*}
which ensures us that $2^{n+1} \in \Theta(2^{n})$.

\subexercise{4}
\paragraph{Faux}
Note that $(n+1)!= (n+1) n!$, therefore, $(n+1)! > n! \,\forall n\ge 1 $, which implies that \[
\lim_{n \to \infty} \frac{(n+1)!}{n!} = \infty
.\] Therefore, $n! \not\in \Omega((n+1)!)$.

\subexercise{5}
\paragraph{Faux}

Suppose $C > 0$ such that $|2^{n}| \ge C |4^{n}| \,\forall n > N $, for some $N\in \N$.
Then, we would have
\begin{align*}
    2^{n} &\ge  C 4^{n} \\
    &= C \left( 2^2 \right)^{n}  \\
    &= C (2^n)^2 \\
    \iff_{n\ge 1} 2^{n} &\le \frac{1}{C}
,\end{align*}
which, $\forall n\ge 1$, is only true for $n \le \log \frac{1}{C}$, a contradiction.

\subexercise{6}
\paragraph{Vrai}
First, note that $f(n) = \sin n + 2 \ge -1 + 2 = 1$, so  $f(n) \in \Omega(1)$.
At the same time,  $f(n) \le 1 + 2 = 3$, which implies that $f(n) \in O(1)$ for $C=3$.

\exercise{6}

This is trivial given the multiplicative rule.

For the sake of completeness, let $C > 0$ be such that $|f(n)| \ge C n$, then
\begin{align*}
    |f(n)^{2}| &= |f(n)| |f(n)| \\
    &\le C^2 n^2 
,\end{align*}
which proves that $f(n)^2 \in O(n^2)$.

\exercise{7}

Let $f(n) = 2n$.
You fall into 5.5.


