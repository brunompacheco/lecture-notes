\demo{2}{12.09.2024}{}
 
\exercise{1}

\[
    O(n \log n) \subset O(n^{1+\epsilon}) \subset O(\frac{n^2}{\log n}) \subset  O(n^{8}) = O((n^2-n+1)^{4}) \subset O((1+\epsilon)^{n})
.\] 

\paragraph{$O(n \log n) \subset O(n^{1+\epsilon})$}
By the limit theorem
\begin{align*}
    \lim_{n \to \infty} \frac{n \log n}{n^{1+\epsilon}} &= \frac{\log n}{n^{\epsilon}} \\
    &= \lim_{n \to \infty} \frac{n^{-1}}{\epsilon n^{\epsilon-1}} \\
    &= \lim_{n \to \infty} \frac{1}{\epsilon n^{\epsilon}} = 0
.\end{align*}

\paragraph{$O(n^{1+\epsilon}) \subset O(\frac{n^2}{\log n})$}
\begin{align*}
    \lim_{n \to \infty} \frac{n^2}{\log n} \frac{1}{n^{1+\epsilon}} &= \lim_{n \to \infty} \frac{n^{1-\epsilon}}{\log n} \\
    &= \lim_{n \to \infty} \frac{(1-\epsilon)n^{-\epsilon}}{1 / n} \\
    &= \lim_{n \to \infty} \frac{(1-\epsilon)n}{n^{\epsilon}} \\
    &= \lim_{n \to \infty} (1-\epsilon)n^{1-\epsilon} \to \infty \\
.\end{align*}

\paragraph{$O(\frac{n^2}{\log n}) \subset  O(n^{8})$}
\[
    \lim_{n \to \infty} \frac{n^{8}}{\frac{n^2}{\log n}} = n^{6} \log n \to \infty
.\] 

\paragraph{$O(n^{8}) = O((n^2-n+1)^{4})$}
Since both polynomials have the same degree, their $O$-set are identical.

\paragraph{$O(n^{8}) \subset O((1+\epsilon)^{n})$}
\[
    \lim_{n \to \infty} \frac{(1+\epsilon)^{n}}{n^{8}} = \lim_{n \to \infty} \frac{(\log (1+\epsilon))^{9} (1+\epsilon)^{n}}{8!} \to \infty
.\] 

\exercise{2}
\[
O(n^{\log n}) \subset O(n^{\sqrt{n}}) \subset O(2^{n}) = O(2^{n+1}) \subset O(2^{2n}) \subset O(n!) \subset O((n+1)!) \subset O(n^{n})
.\] 

\paragraph{$O(n^{\log n}) \subset O(n^{\sqrt{n}})$}
Note that\[
    \lim_{n \to \infty} \frac{n^{\log n}}{n^{\sqrt{n}}} = \lim_{n \to \infty} n^{\log n - \sqrt{n} }
.\] But we have that
\begin{align*}
    \lim_{n \to \infty} \frac{\log n}{\sqrt{n}} &= \lim_{n \to \infty} \frac{\frac{1}{x}}{\frac{1}{2\sqrt{x} }} \\
    &= \lim_{n \to \infty} \frac{2 \sqrt{x} }{x} = 0 \\
,\end{align*}
which implies that $\forall^{\infty}n, \log n - \sqrt{n} < 0 $, which, in turn, leads to $\lim_{n \to \infty} n^{\log n - \sqrt{n} }= 0$.

\paragraph{$O(n^{\sqrt{n}}) \subset O(2^{n})$}
Note that
\begin{align*}
\lim_{n \to \infty} \frac{2^{n}}{n^{\sqrt{n} }} &= \lim_{n \to \infty} \frac{2^{n}}{\left( 2^{\log_2 n} \right)^{\sqrt{n} }} \\
&= \lim_{n \to \infty} \frac{2^{n}}{2^{\sqrt{n} \log_2 n}} \\
&\ge \lim_{n \to \infty} \frac{2^{n}}{2^{n^{\frac{1}{2}} n^{\frac{1}{4}}}}  \\
&= \lim_{n \to \infty} 2^{n - n^{\frac{3}{4}}} = \infty
,\end{align*}
because $\lim_{n \to \infty} n - n^{\frac{3}{4}}=\infty$.

\paragraph{$O(2^{n}) = O(2^{n+1})$}
\[
\lim_{n \to \infty} \frac{2^{n}}{2^{n+1}} = \frac{1}{2}
.\] 

\paragraph{$O(2^{n}) \subset O(2^{2n})$}
\begin{align*}
    \lim_{n \to \infty} \frac{2^{n}}{2^{2n}} &=  \lim_{n \to \infty} 2^{n - 2n} \\
    &= \lim_{n \to \infty} 2^{-n} = 0
.\end{align*}

\paragraph{$O(2^{2n}) \subset O(n!)$}
\begin{align*}
    \lim_{n \to \infty} \frac{2^{2n}}{n!} &= \lim_{n \to \infty} \frac{1}{4}\frac{2}{4}\frac{3}{4}\frac{5}{4}\frac{6}{4}\ldots \frac{n-1}{4} \frac{n}{4} \\
    &>= \lim_{n \to \infty} \frac{1}{4}\frac{2}{4}\frac{3}{4} 1 \frac{n}{4} = \infty
.\end{align*}

\paragraph{$O(n!) \subset O((n+1)!)$}
\[
    \lim_{n \to \infty} \frac{(n+1)!}{n!} = \lim_{n \to \infty} (n+1)\frac{n!}{n!} = \infty
.\] 

\paragraph{$O((n+1)!) \subset O(n^{n})$}
\begin{align*}
    \lim_{n \to \infty} \frac{n^{n}}{(n+1)!} &= \lim_{n \to \infty} \frac{1}{n+1} \frac{n}{n} \frac{n}{n-1} \frac{n}{n-2} \ldots \frac{n}{2}\frac{n}{1} \\
    &= \lim_{n \to \infty} \frac{n}{n+1}\frac{n}{n} \frac{n}{n-1} \frac{n}{n-2} \ldots \frac{n}{2} \\
    &\ge \lim_{n \to \infty} \frac{n}{n+1} 1 \cdot 1\cdot 1\cdot \ldots\cdot \frac{n}{2}  \\
    &= \lim_{n \to \infty} \frac{n^2}{2(n+1)} = \lim_{n \to \infty} \frac{2n}{2} = \infty
.\end{align*}

\exercise{3}

Recall that the functions must be over natural numbers, but they need not to be continuous.
Let \[
f(n) = \begin{cases}
    n & \text{if }n\text{ is even} \\
    0 & \text{otherwise}
\end{cases}
\] and $g(n) = n - f(n)$.

Now, note that for every $N\in \N$ and $C \in \R^{>0}$, we can always find $n \ge N$ even such that $f(n) \ge Cg(N)$.
\emph{Mutatis mutandis}, the same applies for $g$.

\exercise{4}

\subexercise{1}

Assuming the sum operation costs $C_1$ ($\in O(1)$) and that the assignment costs $C_2$, we have that the total cost is
\begin{align*}
    T(n) &= C_2 + \sum_{i=1}^{N} \sum_{j=i}^{N} C_1 \\
    &= C_2 + \sum_{i=1}^{N} (N-i + 1)C_1 \\
    &= C_2 + C_1\left( N^2 + N + \sum_{i=1}^{N} i \right)  \\
    &= C_2 + C_1\left( N^2 + N + N \frac{N+1}{2} \right) \in \Theta(N^2) \\
.\end{align*}
Because $N$ is an integer with $n$ bits, then $N \approx 2^{n}$, which means that $T(n) \in \Theta(4^{n})$.

\subexercise{2}

Let us assume that the elementary operations performed in this function are all $\in \Theta(n)$.
Because at every iteration one bit of $a$ is removed, the loop will run, in the worst case, $n$ times.
Also, in the worst case, the \texttt{if} will enter at every iteration (case $a=2^{n}-1$), so we perform 4 elementary operations at every iteration.
Let $C(n)\in \Theta(n)$ be the cost of the elementary operations.
Then, we can write
\begin{align*}
    T(n) &= C(n) + \sum_{i=1}^{n} C(n) \\
    &= C(n) + n C(n) \\
    &=_O n^2
.\end{align*}

\subexercise{3}

Because $\text{debut}-\text{fin}$ is approximately halved at each iteration, we can be sure that the while loop, in the worst case, will run $\log \text{debut}- \text{fin} = \log n-1$ times.
Considering that all operations run in constant time, we have that $T(n) \in \Theta(\log n)$.

\subexercise{4}

We have a recurrence relation defined by \[
T(n) = \begin{cases}
    1 &,\, n<2 \\
    T(\left\lfloor \frac{n}{2} \right\rfloor) + T(\left\lceil \frac{n}{2} \right\rceil ) + g(n) &,\, n\ge 2
\end{cases}
,\] where $g(n) \in \Theta(n)$.
By the master theorem, we can conclude that $T(n) \in \Theta(n \log n)$.



