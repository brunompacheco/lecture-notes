\lecture{2}{05.09.2024}{Mathematical Prerequisites}


Let $g,f_1,\ldots,f_k: \N \longrightarrow \R$ be functions such that $f_i \in O(g) \forall i$.
Then, \[
\sum_{i=1}^{k} f_i \in O(g)
.\]\footnote{I believe this is an easy corollary of the max/sum lemma.}

An easy corollary from the above is that any polynomial with non-negative exponents is in the order of its greatest exponent.

Similarly, let $f_1,\ldots,f_k,g_1,\ldots,g_k: \N \longrightarrow \R$ be functions such that $f_i \in O(g_i) \forall i$.
Then \[
\prod_{i=1}^{k} f_i \in O\left( \prod_{i=1}^{k} g_i \right)  
.\] 

\section*{Petit $o$}

Let $f,g: \N \longrightarrow \R$ two functions.
Then \[
f\in o(g) \iff \exists N\in \N : n\ge n \implies |f(n)| \le C|g(n)|\,\forall C \in \R^{>0}
.\] 

Note that the core inequality above must hold for \textbf{all} constants $C$.

\paragraph{Transitivity}
Let $f,g,h: \N \longrightarrow \N$ be functions such that $f\in o(g)$ and $g\in O(h)$.
Then $f\in o(h)$.

To see this, note that \[
|f(n)| \le C_1|g(n)| \gets C_1 C_2 |h(n)|
\] for every $C_1>0$.

