\lecture{8}{26.09.2024}{Greedy algorithms (continuation)}

\begin{problem}
    \emph{(MatroidMax)} Given a weighted matroid $(C,I,w)$, find a set $S\in I$ that maximizes $\sum_{x\in S} w(x)$.
\end{problem}

Note that this can be framed as an optimization problem
\begin{align*}
    \max_{S} \quad & \sum_{x\in S} w(x) \\
    \textrm{s.t.} \quad & S \in I
.\end{align*}

We will show that the following algorithm is a general solution to optimization problem over matroids.

\begin{algorithm}
    \KwData{A weighted matroid $\left( C,I,w \right) $.}
    $L\gets $ sort$(C,-w)$  // sorting in decreasing order by weight \\
    $S \gets \emptyset $ \\
    \While{$L\neq \emptyset$ or we are not done}{
	$x \gets $ next$(L)$ \\
	$L\gets L \setminus \left\{ x \right\} $ \\
	\If{$S \cup \left\{ x \right\} \in I$}{
	    $S\gets S \cup \left\{ x \right\} $ 
	}
    }

    \Return{S}
\end{algorithm}

\begin{lemma}
    Let $S\le C$ be the set constructed by the algorithm above at the $k$-th iteration of the while loop, and $E\subseteq C$ be the set of elements that have been discarded at all iterations up to then.
    Then, there exists an optimal solution $T$ such that $S \subseteq T$ and $T \cap E = \emptyset $.
\end{lemma}
\begin{proof}
    We proceed by induction, noting that the case $k=0$ is trivial.

    Now, assume that the statement is true for $k$ and let us prove it for $k+1$.
    Let $S$ be the set constructed at iteration $k$, $E$ the set of discarded elements, and $T$ be some optimal solution such that $S \subseteq T$ and $T \cap E = \emptyset$.
    If $S \neq T$, the, by the definition of a matroid, there must exist $x \in T$ such that $S\cup \left\{ x \right\} \in I$, such that we are certain $L \neq \emptyset$ (because $T \cap E = \emptyset$), thus the algorithm does not stop yet.

    So let $z = \text{next}(L)$ be the element added to $S$ at iteration $k+1$.
    Note that $z \not\in S$.
    If $x \in T$, all properties are maintained.
    If $x \not\in T$, but $S\cup \left\{ x \right\} \not\in I$, then all properties are also maintained.
    Therefore, we focus on the case $x \not\in T \land S\cup \left\{ x \right\} \in I$, that is $S'=S \cup \left\{ x \right\} \not\subseteq T$.
    If $|S'| = |T|$, then $w(S) = w(T)$, given the construction of $L$ and the optimality of $T$.
    Otherwise, we can use the exchange property to construct $T' \gets S' \cup T$ such that $S'\subseteq T'$ and $|T'| = |T|$.
    Let $\hat{z} \in T \setminus T'$, and note that it is unique.
    Therefore, we have
    \begin{align*}
        w(T') &= \sum_{x \in S'} w(x) \\
	&= w(z) + \sum_{x \in T\cap T'} w(x) \\
	&= w(z) - w(\hat{z}) + \sum_{x \in T}^{} w(x) \\
    .\end{align*}
    But again, by the construction of $L$, $w(z) \ge w(\hat{z})$, as $\hat{z} \in L \setminus z$.
    Therefore, \[
    w(T') \ge  \sum_{x\in T}^{} w(x) = w(T)
    ,\] which implies that $T'$ is also optimal, which maintain the desired properties over the loop invariant, although towards a different optimal than the one in the previous iteration.
\end{proof}

Now, the following theorem becomes trivial.

\begin{theorem}
    The algorithm above solves problem \emph{MatroidMax}, i.e., it finds an optimal solution.
\end{theorem}

\begin{proof}
    The previous lemma is also true for the last iteration, but because $L = \emptyset$, then $S = T$.
\end{proof}

\begin{lemma}
    The algorithm presented for the MatroidMax problem runs in $O(n \log n + n f(n))$, where $f(n)$ is the cost of the $I$-membership function, assuming the operations over $L$ (next and emptiness) run in $O(1)$.
\end{lemma}
\begin{proof}
    Sorting can be done in $\Theta(n \log n)$, and the loop runs at most $n$ times.
    Therefore, because the trivial operations over $L$ are assumed to run in constant time, each iteration will run in $O(f(n))$.
\end{proof}

\textcolor{red}{\textbf{Study vector independency, spanning trees, Kruskal's, knapsack and Dijkstra.}}

