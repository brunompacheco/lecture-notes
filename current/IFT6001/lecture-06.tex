\lecture{6}{18.09.2024}{Mathematical Prerequisites (Continuation)}

\section*{Akra-Bazzi Recurrences}

\begin{previouslyseen}
    Recall that the exponent of an Akra-Bazzi recurrence is the value $p$ such that \[
    \sum_{i=1}^{k-1} a_i b_i^{p} = 1
    .\] 

    Furthermore, the exponent always exists and is unique.
\end{previouslyseen}

We want to show that an Akra-Bazzi recurrence respects \[
    T \in \Theta\left( n^{p}\left( 1 + \sum_{u=1}^{n-1} \frac{g(u)}{u^{p+1}} \right)  \right) 
.\] For this, we will need a couple of lemmas.

\begin{lemma}
    Let $q: \R \longrightarrow \R$ be a function such that $x \mapsto |q'(x)|$ (the absolute value of its derivative) is harmonious.
    The, for every $i$,\[
    \left| q(r_i(n)) - q(b_in) \right| \le_O |q'(n)|\cdot |r_i(n)-b_in|
    .\] 
\end{lemma}
\begin{proof}
    Let $b=\frac{1}{2}min_i b_i$, and let $n$ and $c$ such that
    \begin{align*}
	&bn \le r_i(n) \text{ (guaranteed by the properties of $r_i$)} \\
	& |q'(u)| \le c|q'(n)|,\, \forall u \in \left[ bn,n \right]  \text{ (guaranteed since $|q'|$ is harmonious)}
    .\end{align*}

    Now, for any interval containing $r_i(n)$ and $b_i n$, $q'(n)$ has a supremum $M$, and we can write \[
    \left| \frac{q(r_i(n)) - q(b_i n)}{r_i(n) - b_i n} \right| \le M
    .\] But since $M \le c|q'(n)|$, we can reformulate the above to show the asymptotic relationship desired.
\end{proof}

\begin{lemma}
    There are constants $c_3,c_4>0$ such that, $\forall^{\infty} n$, we have \[
    c_3 g(n) \le \widetilde{g_i}(n) \le c_4 g(n)
    ,\] where \[
    \widetilde{g}_i (n) := n^{p} \sum_{u=r_i(n)}^{n-1} \frac{g(u)}{u^{p+1}}
    .\] 
\end{lemma}
\begin{proof}
    See the pdf, I'm a bit tired.
\end{proof}

\begin{lemma}
    Let $\varepsilon(n) := 1 / \ln n$.
    Then, $\forall^{\infty} n$ \[
    r_i(n)^{p}\left( 1 - \varepsilon(r_i(n)) \right)  \le  (b_i n)^{p}(1 - \varepsilon(n))
    \] and \[
    r_i(n)^{p}\left( 1 - \varepsilon(r_i(n)) \right)  \ge   (b_i n)^{p}(1 + \varepsilon(n))
    .\] 
\end{lemma}
\begin{proof}
    I gave up (for now). I'll get back to this later if needed.
\end{proof}

The important thing is:
\begin{theorem}
    Let $T(n)$ be an Akra-Bazzi recurrence.
    Then \[
    T \in \Theta\left( n^{p}\left( 1 + \sum_{u=1}^{n-1} \frac{g(u)}{u^{p+1}} \right)  \right) 
    ,\] where $p$ is the exponent of $T$.
\end{theorem}

\subsection*{Master Theorem}

In the vast majority of our applications, we will use the following theorem:
\begin{theorem}
    \emph{(Master theorem)}
    Let $T : \N \longrightarrow \R$ be an Akra-Bazzi recurrence in the form \[
    T(n) = \sum_{i=0}^{\ell-1} T(r_i(n)) + g(n)
    ,\] where $\left| r_i(n) - n / b \right| \in o(n / \ln^2 n)$ for every $i$.
    Then, \[
    T \in \begin{cases}
	\Theta(n^{k}) &,\, g \in \Theta(n^{k}), k > \log_b \ell \\
	\Theta(n^{\log_b \ell} \log n) &,\, g \in \Theta(n^{\log_b l}) \\
	\Theta(n^{\log_b \ell}) &,\, g \in O(n^{k}), k < \log_b \ell \\
    \end{cases}
    .\] 
\end{theorem}
\begin{proof}
    To use the general Akra-Bazzi theorem it is necessary to determine the order of recurrence ($p$ ).
    In the particular case, we have $b_i = 1 / b, \forall i$, which implies that $p$ satisfies \[
    \sum_{i=0}^{\ell-1} (1 / b)^{p} = 1 \implies p = \log_b \ell
    .\] 

    Now, tu compute the sum $\sum_{u=1}^{n-1} \frac{g(u)}{u^{p+1}}$ of the general theorem, we use the fact that \[
    \sum_{u=1}^{n-1} u^{k} \in \begin{cases}
	\Theta(n^{k+1}) &,\, k > -1 \\
	\Theta(\log n) &,\, k = -1 \\
	\Theta(1) &,\, k < -1 \\
    \end{cases}
    ,\] which can be demonstrated by limiting the sum by the integral (?).

    Now, suppose that $g$ falls in the first case, i.e., $g \in \Theta(n^{k})$, with $k>\log_b \ell$.
    Take $N$ and $C$ such that $g(n) \le Cn^{k}$ for every $n\ge N$.
    Then, we can write
    \begin{align*}
	T&\le_O n^{\log_b \ell}\left( 1 + \sum_{u=1}^{n-1} \frac{g(u)}{u^{\log_b \ell+1}} \right) \\
	 &\le_O n^{\log_b \ell}\left( 1 + \underbrace{\sum_{u=1}^{N-1} \frac{g(u)}{u^{\log_b \ell+1}}}_{\text{Constant}} + \underbrace{C\sum_{u=N}^{n-1} \frac{u^{k}}{u^{\log_b \ell - 1}}}_{\in O(n^{k-\log_b \ell})} \right) \\
	 &\le_O n^{k}
    .\end{align*}

    The reasoning is similar for the other cases and for the lower bound.
\end{proof}

