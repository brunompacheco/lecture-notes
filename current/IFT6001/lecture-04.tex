\lecture{4}{12.09.2024}{Mathematical Prerequisites (Continuation)}

\begin{previouslyseen}
    A recurrence relationship is said linear and homogeneous if it can be written as a root-finding problem with a linear operator (with delays) over $T(n)$.
\end{previouslyseen}

Let $S: V \longrightarrow V$ be an operator over a subspace $V$ and the polynomial \[
p(x) = a_k x^{k} + \ldots + a_1 x + x_0
,\] where $a_i\in \mathbb{C}, i=0,\ldots,k$.
We can define the polynomial over operators as well, simply by replacement, as $p(S)$.

\begin{lemma}
    The set of solutions for a linear homogeneous recurrence of the form \[
	a_k T(n) + a_{k-1} T(n-1) +\ldots + a_0 T(n-k) = 0
    \] form a subspace of the function space $\N\to \mathbb{C}$ (which is a vector space) of dimension $k$.
\end{lemma}
\begin{proof}
    Let $T_1 : \N \longrightarrow \mathbb{C}$ be a solution for the recurrence and $\alpha$ a constant.
    Then $\alpha T_1$ is also a solution (easy check).

    Let $T_2: \N \longrightarrow \mathbb{C}$ be another solution.
    Then $T_1+T_2$ is also a solution (another easy check).
\end{proof}

\begin{note}
    The dimension of the function space is $k$ because an element can be uniquely determined by $k$ points (e.g., $T(0),\ldots,T(k-1)$).\footnote{I believe this can be shown through a 1-to-1 mapping between the function space and the vector space.}
\end{note}

Our goal then is to find a basis for the subspace of solutions to the target recurrence.

We now define the \emph{delay} operator 
\begin{align*}
    S: S: (\N\to \mathbb{C}) &\longrightarrow (\N\to \mathbb{C}) \\
    T(n) &\longmapsto S(T(n)) = T(n+1)
.\end{align*}
This allows us to rewrite our recurrence as \[
    (a_k S^{k} + a_{k-1}S^{k-1} + \ldots + a_0)T(n) = 0
,\] which highlights the \emph{characteristic polynomial} $p(x) = a_kx^{k}+\ldots+a_0$.

Naturally, we can factorize the polynomial and write it as \[
    p(x) = C \prod_{i=1}^{k} (x-\lambda_i) 
,\] where $\lambda_i$ are its roots and $C\in \mathbb{C}$ is a constant.
This leads us to writing the recurrence as \[
    \prod_{i=1}^{k} (S-\lambda_i) T(n) = 0
,\] which is satisfied as long as any $(S-\lambda_i)T(n) = 0$, which can be solved by writing it as $T(n) = \lambda_i T(n-1)$.

\begin{lemma}
    The solution to a recurrence of the form \[
    T(n) = \begin{cases}
	C &, \text{if }n=0 \\
	\lambda T(n-1) &
    \end{cases}
    \] is $T(n) = C\lambda^{n}$.
\end{lemma}
\begin{proof}
    The characteristic polynomial has a single root, so it is easy to see how it goes from here.
\end{proof}

For the more general case, we have that $T(n) = \lambda_i T(n-1)=\lambda_i^{2} T(n-2) = \ldots = \lambda_i^{n}T(0)$ is a solution to every case.

We should, however, look at all possible solutions, as we have the initial conditions to take into account.
More precisely, if we have $k$ distinct roots $\lambda_i$, any solution is of the form \[
T(n) = \sum_{i=1}^{k} c_i \lambda_i^{n}
.\] 
Now we can use the initial conditions to uniquely determine our solution.

If we have multiple roots, then we don't have a complete basis for our subspace.
Suppose that there is a root $\lambda$ that appears $m$ times.
Then, we must find $m$ linearly independent solutions to the problem \[
    (S-\lambda)^{m}T(n) = 0
.\] 
Recall that we already know $T(n) = T(0)\lambda^{n}$.


