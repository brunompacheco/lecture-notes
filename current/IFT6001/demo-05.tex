\demo{Devoir 1}{03.10.2024}{}
 
\exercise{1}

Obviously, the main diagonal is $\Theta$, as a function is trivially on the $\theta$-order of itself.

\paragraph{$f_1$ and $f_2$}

\begin{align*}
    \lim_{n \to \infty} \frac{n^{18}\left( \ln n \right)^{37}}{3^{n}} &\le \lim_{n \to \infty} \frac{n^{18} n^{37}}{3^{n}} \\
    &= 0
,\end{align*}
by successive l'Hôpital applications (precisely, 55 applications).
Therefore, we have that $f_1 \in o(f_2)$ and $f_2 \in  \omega(f_1)$.

\paragraph{$f_2$ and $f_3$}

We can show that $f_2$ is asymptotically greater than both cases of $f_3$.
In fact, because \[
\lim_{n \to \infty} \frac{2^{n}}{37 \ln n} = \infty
,\] we just need to show that $f_2$ dominates the exponential.
\begin{align*}
    \lim_{n \to \infty} \frac{2^{n}}{3^{n}} &= \lim_{n \to \infty} \left( \frac{2}{3} \right)^{n}  \\
    &= 0
.\end{align*}
Thus $f_3 \in \omega(f_2)$ and $f_2 \in o(f_3)$.


\paragraph{$f_1$ and $f_3$}

Because \[
\lim_{n \to \infty} \frac{f_1}{2^{n}} = 0
\] and \[
\lim_{n \to \infty} \frac{f_1}{37 \ln n} = \infty
,\] we can always find a value such that, for any constant, $f_3$ will be greater than $f_1$, and another value for which $f_3$ will be smaller than $f_1$.
Therefore, they are \emph{incomparable}.

\exercise{2}

\subexercise{a}

We show this by induction that $b\upuparrows n \ge b^{n}$.
For $n=0$, they are trivially equal.
Now, if we suppose that this is true for $n$, then we have that
\begin{align*}
    b\upuparrows (n+1) &= b^{b\upuparrows n} \\
    &\ge  b^{b^{n}} \ge b^{n+1}
,\end{align*}
because $b^{n} \in \omega(n+1)$, that is, for every $b\ge 2$, we can state that $b^{n} \ge n+1$, for $n\ge 0$.

Thus, \[
\lim_{n \to \infty} b\upuparrows n \ge \lim_{n \to \infty} b^{n} = \infty
.\] 

\subexercise{b}

We have that $\exists n_0\in \N$, such that $f(n+1) \le \frac{1}{2}f(n)$, which implies that $f(n) \le \frac{1}{2^{n-n_0}} f(n_0)\ge 0$.

Therefore, we can say that
\begin{align*}
    \lim_{n \to \infty} f(n) &=  \lim_{n \to \infty} \frac{1}{2^{n-n_0}}f(n_0) \\
    &= 0
.\end{align*}

\subexercise{c}

Let $f(n) = \frac{a^{n}}{b\upuparrows n}$.
We want to show that \[
    \lim_{n \to \infty} \frac{a^{n}}{b\upuparrows n} = 0
.\] 

Note that \[
    \frac{f(n+1)}{f(n)} = \frac{a^{n+1}}{a^{n}} \frac{b\upuparrows n}{b\upuparrows (n+1)}
.\] But we know that, $\forall C\in \N, \forall^{\infty}n, n\le C b^{n}$, which implies $b\upuparrows n\le C (b\upuparrows (n+1))$.
So, in particular, for $C=\frac{1}{2a}$, we have that \[
\forall^{\infty}n,\, \frac{f(n+1)}{f(n)} = a\frac{b\upuparrows n}{b\upuparrows (n+1)} \le \frac{a}{2a} = \frac{1}{2}
.\] 
Thus, by what has been shown in (b), we have that \[
\lim_{n \to \infty} f(n) = 0 
.\] 

\subexercise{d}

We start by showing by induction that $\ln^{*}(n) \le n-1$.
For $n=1$, this is trivially true.
Now, note that $\ln^{*} n$ is non-decreasing, that is, \[
\ln^{*}(n+1) = \ln^{*}(\ln n+1) + 1
.\] But we have that $\ln^{*}(x) \le \ln^{*}\left( \left\lceil n \right\rceil  \right) $, as given $k$ such that $x\in (e\upuparrows k-1,e\upuparrows k]$, then $\ln^{*}(x) = k$.
Therefore, \[
\ln^{*}(n+1) \le \ln^{*}(\left\lceil \ln n+1 \right\rceil) +1 \le \left\lceil \ln n+1 \right\rceil -1 +1 \le n
.\] 

Now, we can state
\begin{align*}
    \ln^{*}(n) &= \ln^{*}(\ln n) + 1 \\
    &= \ln^{*}(\ln \ln n) + 2 \\
    &= \ln^{*}(\ln \ln \ln n) + 3 \\
    &\le \ln \ln \ln n + 2
,\end{align*}
which leads us to \[
\lim_{n \to \infty} \frac{\ln^{*}(n)}{\ln \ln n} \le  \lim_{n \to \infty} \frac{\ln \ln \ln n +2}{\ln \ln n} = 0
.\] 

\exercise{3}

\subexercise{a}

By contradiction, suppose $\exists c_1,\ldots,c_k$ such that \[
\sum_{i=1}^{k} c_i f_i = 0 \,\land\, \sum_{i=1}^{k} c_i > 0
.\] 

Let $j$ be the greatest number such that $c_j\neq 0$, that is, $\forall i>j, c_i=0$.
Then, we can write
\begin{align*}
    \sum_{i=1}^{k} c_i f_i &= \sum_{i=1}^{j} c_i f_i = 0 \\
    \implies \sum_{i=1}^{j} \frac{c_i f_i}{f_j} &= 0 \\
.\end{align*}
But $i<j \implies\frac{f_i}{f_j} \to 0$, because $f_i\in o(f_j)$.
Therefore, we have that 
\begin{align*}
    \lim_{n \to \infty} \sum_{i=1}^{j} \frac{c_i f_i}{f_j} &= 0 \\
    \implies \lim_{n \to \infty} \frac{c_j f_j}{f_j} &= \lim_{n \to \infty} c_j = 0
,\end{align*}
a contradiction.

\subexercise{b}

\textbf{\textcolor{red}{TODO: use the fact that the roots are all dominated by one-another.}}

Remember that some roots have multiplicity greater than 1.


\exercise{4}

\subexercise{a}

We can write the recurrence relationship as \[
t_n - 10t_{n-1} + 9t_{n-2} = 0
.\] The characteristic polynomial, then, is \[
x^{2} - 10x + 9 = (x-1)(x-9) = 0
.\] Which implies the solution is of the form \[
T(n) = c_1 + c_2 9^{n}
.\] 

For the initial cases, we have
\begin{align*}
    T(0) &= 1 = c_1 + c_2 \\
    T(1) &= 6 = c_1 + c_2 9 \\
.\end{align*}
These lead us to \[
T(n) = \frac{3}{8} + \frac{5}{8} 9^{n}
.\] 

\subexercise{b}

Our recurrence relation here is \[
t_n - 2 t_n = (n +5) 2^{n}
.\] 
We know that if the RHS has the form $p(n) c^{n}$, where $p(n)$ is a polynomial, we can write it in the form $(x-c)^{\text{deg} p + 1}$ on the LHS, which implies that our characteristic polynomial is \[
    (x-2)^{2} (x-2) = 0
.\] which has a single root at $2$ with multiplicity 3, resulting in solutions of the form \[
T(n) = \left( c_1 + c_2n + c_3 n^{2} \right)2^{n}
.\] 

Our initial condition gives us
\begin{align*}
    T(0) &= 1 = c_1 \\
    T(1) &= 2+2+10 = 2c_1 + 2c_2+2c_3 \\
    T(2) &= 28 + 8 + 20 = 56 = 4c_1 + 8c_2 + 16c_3 \\
.\end{align*}
The solution to the above linear system gives us \[
T(n) = \left( 1 + \frac{11}{2}n + \frac{1}{2}n^2 \right) 2^{n}
.\] 

\demo{5}{03.10.2024}{}

\exercise{2}

\textbf{\textcolor{red}{TODO}}

