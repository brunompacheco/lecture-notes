\documentclass[a4paper]{report}
% Some basic packages
\usepackage[utf8]{inputenc}
\usepackage[T1]{fontenc}
\usepackage{textcomp}
\usepackage[english]{babel}
\usepackage{url}
\usepackage{graphicx}
\usepackage{float}
\usepackage{booktabs}
\usepackage{enumitem}

\pdfminorversion=7

% Don't indent paragraphs, leave some space between them
\usepackage{parskip}

% Hide page number when page is empty
\usepackage{emptypage}
\usepackage{subcaption}
\usepackage{multicol}
\usepackage{xcolor}

% Other font I sometimes use.
% \usepackage{cmbright}

% Math stuff
\usepackage{amsmath, amsfonts, mathtools, amsthm, amssymb}
% Fancy script capitals
\usepackage{mathrsfs}
\usepackage{cancel}
% Bold math
\usepackage{bm}
% Some shortcuts
\newcommand\N{\ensuremath{\mathbb{N}}}
\newcommand\R{\ensuremath{\mathbb{R}}}
\newcommand\Z{\ensuremath{\mathbb{Z}}}
\renewcommand\O{\ensuremath{\emptyset}}
\newcommand\Q{\ensuremath{\mathbb{Q}}}
\newcommand\C{\ensuremath{\mathbb{C}}}
\renewcommand\L{\ensuremath{\mathcal{L}}}

% Package for Petri Net drawing
\usepackage[version=0.96]{pgf}
\usepackage{tikz}
\usetikzlibrary{arrows,shapes,automata,petri}
\usepackage{tikzit}
\input{petri_nets_style.tikzstyles}

% Easily typeset systems of equations (French package)
\usepackage{systeme}

% Put x \to \infty below \lim
\let\svlim\lim\def\lim{\svlim\limits}

%Make implies and impliedby shorter
\let\implies\Rightarrow
\let\impliedby\Leftarrow
\let\iff\Leftrightarrow
\let\epsilon\varepsilon

% Add \contra symbol to denote contradiction
\usepackage{stmaryrd} % for \lightning
\newcommand\contra{\scalebox{1.5}{$\lightning$}}

% \let\phi\varphi

% Command for short corrections
% Usage: 1+1=\correct{3}{2}

\definecolor{correct}{HTML}{009900}
\newcommand\correct[2]{\ensuremath{\:}{\color{red}{#1}}\ensuremath{\to }{\color{correct}{#2}}\ensuremath{\:}}
\newcommand\green[1]{{\color{correct}{#1}}}

% horizontal rule
\newcommand\hr{
    \noindent\rule[0.5ex]{\linewidth}{0.5pt}
}

% hide parts
\newcommand\hide[1]{}

% si unitx
\usepackage{siunitx}
\sisetup{locale = FR}

% Environments
\makeatother
% For box around Definition, Theorem, \ldots
\usepackage{mdframed}
\mdfsetup{skipabove=1em,skipbelow=0em}
\theoremstyle{definition}
\newmdtheoremenv[nobreak=true]{definitie}{Definitie}
\newmdtheoremenv[nobreak=true]{eigenschap}{Eigenschap}
\newmdtheoremenv[nobreak=true]{gevolg}{Gevolg}
\newmdtheoremenv[nobreak=true]{lemma}{Lemma}
\newmdtheoremenv[nobreak=true]{propositie}{Propositie}
\newmdtheoremenv[nobreak=true]{stelling}{Stelling}
\newmdtheoremenv[nobreak=true]{wet}{Wet}
\newmdtheoremenv[nobreak=true]{postulaat}{Postulaat}
\newmdtheoremenv{conclusie}{Conclusie}
\newmdtheoremenv{toemaatje}{Toemaatje}
\newmdtheoremenv{vermoeden}{Vermoeden}
\newtheorem*{herhaling}{Herhaling}
\newtheorem*{intermezzo}{Intermezzo}
\newtheorem*{notatie}{Notatie}
\newtheorem*{observatie}{Observatie}
\newtheorem*{exe}{Exercise}
\newtheorem*{opmerking}{Opmerking}
\newtheorem*{praktisch}{Praktisch}
\newtheorem*{probleem}{Probleem}
\newtheorem*{terminologie}{Terminologie}
\newtheorem*{toepassing}{Toepassing}
\newtheorem*{uovt}{UOVT}
\newtheorem*{vb}{Voorbeeld}
\newtheorem*{vraag}{Vraag}

\newmdtheoremenv[nobreak=true]{definition}{Definition}
\newtheorem*{eg}{Example}
\newtheorem*{notation}{Notation}
\newtheorem*{previouslyseen}{As previously seen}
\newtheorem*{remark}{Remark}
\newtheorem*{note}{Note}
\newtheorem*{problem}{Problem}
\newtheorem*{observe}{Observe}
\newtheorem*{property}{Property}
\newtheorem*{intuition}{Intuition}
\newmdtheoremenv[nobreak=true]{prop}{Proposition}
\newmdtheoremenv[nobreak=true]{theorem}{Theorem}
\newmdtheoremenv[nobreak=true]{corollary}{Corollary}

% End example and intermezzo environments with a small diamond (just like proof
% environments end with a small square)
\usepackage{etoolbox}
\AtEndEnvironment{vb}{\null\hfill$\diamond$}%
\AtEndEnvironment{intermezzo}{\null\hfill$\diamond$}%
% \AtEndEnvironment{opmerking}{\null\hfill$\diamond$}%

% Fix some spacing
% http://tex.stackexchange.com/questions/22119/how-can-i-change-the-spacing-before-theorems-with-amsthm
\makeatletter
\def\thm@space@setup{%
  \thm@preskip=\parskip \thm@postskip=0pt
}


% Exercise 
% Usage:
% \exercise{5}
% \subexercise{1}
% \subexercise{2}
% \subexercise{3}
% gives
% Exercise 5
%   Exercise 5.1
%   Exercise 5.2
%   Exercise 5.3
\newcommand{\exercise}[1]{%
    \def\@exercise{#1}%
    \subsection*{Exercise #1}
}

\newcommand{\subexercise}[1]{%
    \subsubsection*{Exercise \@exercise.#1}
}


% \lecture starts a new lecture (les in dutch)
%
% Usage:
% \lecture{1}{di 12 feb 2019 16:00}{Inleiding}
%
% This adds a section heading with the number / title of the lecture and a
% margin paragraph with the date.

% I use \dateparts here to hide the year (2019). This way, I can easily parse
% the date of each lecture unambiguously while still having a human-friendly
% short format printed to the pdf.

\usepackage{xifthen}
\def\testdateparts#1{\dateparts#1\relax}
\def\dateparts#1 #2 #3 #4 #5\relax{
    \marginpar{\small\textsf{\mbox{#1 #2 #3 #5}}}
}

\def\@lecture{}%
\newcommand{\lecture}[3]{
    \ifthenelse{\isempty{#3}}{%
        \def\@lecture{Lecture #1}%
    }{%
        \def\@lecture{Lecture #1: #3}%
    }%
    \subsection*{\@lecture}
    \marginpar{\small\textsf{\mbox{#2}}}
}



% These are the fancy headers
\usepackage{fancyhdr}
\pagestyle{fancy}

% LE: left even
% RO: right odd
% CE, CO: center even, center odd
% My name for when I print my lecture notes to use for an open book exam.
% \fancyhead[LE,RO]{Gilles Castel}

\fancyhead[RO,LE]{\@lecture} % Right odd,  Left even
\fancyhead[RE,LO]{}          % Right even, Left odd

\fancyfoot[RO,LE]{\thepage}  % Right odd,  Left even
\fancyfoot[RE,LO]{}          % Right even, Left odd
\fancyfoot[C]{\leftmark}     % Center

\makeatother




% Todonotes and inline notes in fancy boxes
\usepackage{todonotes}
\usepackage{tcolorbox}

% Make boxes breakable
\tcbuselibrary{breakable}

% Verbetering is correction in Dutch
% Usage: 
% \begin{verbetering}
%     Lorem ipsum dolor sit amet, consetetur sadipscing elitr, sed diam nonumy eirmod
%     tempor invidunt ut labore et dolore magna aliquyam erat, sed diam voluptua. At
%     vero eos et accusam et justo duo dolores et ea rebum. Stet clita kasd gubergren,
%     no sea takimata sanctus est Lorem ipsum dolor sit amet.
% \end{verbetering}
\newenvironment{verbetering}{\begin{tcolorbox}[
    arc=0mm,
    colback=white,
    colframe=green!60!black,
    title=Opmerking,
    fonttitle=\sffamily,
    breakable
]}{\end{tcolorbox}}

% Noot is note in Dutch. Same as 'verbetering' but color of box is different
\newenvironment{noot}[1]{\begin{tcolorbox}[
    arc=0mm,
    colback=white,
    colframe=white!60!black,
    title=#1,
    fonttitle=\sffamily,
    breakable
]}{\end{tcolorbox}}




% Figure support as explained in my blog post.
\usepackage{import}
\usepackage{xifthen}
\usepackage{pdfpages}
\usepackage{transparent}
\newcommand{\incfig}[1]{%
    \def\svgwidth{\columnwidth}
    \import{./figures/}{#1.pdf_tex}
}

% Fix some stuff
% %http://tex.stackexchange.com/questions/76273/multiple-pdfs-with-page-group-included-in-a-single-page-warning
\pdfsuppresswarningpagegroup=1


% My name
\author{Bruno M. Pacheco}

 

\begin{document}
 
\title{Resumo de Qualificação de Doutorado}
\author{Bruno M. Pacheco\\
DAS410036 - Metodologia Científica}
 
\maketitle

\section{Introdução}

A qualificação relatada neste documento foi defendida por João Bernardo Aranha Ribeiro no dia 4/10/2022.
O trabalho apresentado sob o título de "\emph{Economic predictive control and optimization in offshore oil and gas production platforms}" é realizado sob orientação do prof. Julio Elias Normey-Rico, no PosAutomação, com coorientação do prof. José Dolores Vergara Dietrich, da UTFPR.
Compuseram a banca os professores Eduardo Camponogara (presidente), Rodolfo César Costa Flesch e Argimiro Resene Secchi.

\section{Resumo}

O trabalho apresentado de novas estratégias de controle para plataformas \emph{off-shore} de poços de petróleo com \emph{gas-lift} foi motivado pela crescente demanda energética global e, em particular, brasileira.
Além disso, argumentou-se que uma estratégia de controle preditivo que atacasse os problemas econômicos, atuando a nível de planta, possa aumentar também a eficiência e reduzir os impactos ambientais.
Nesse sentido, o controlador proposto é uma extensão de um controlador preditivo não-linear para que leva em consideração também fatores econômicos para além da estabilidade, aliando os objetivos estratégicos ao bom funcionamento da planta.
A proposta de trabalho deriva de resultados anteriores já publicados pelo autor de um controlador $H_{\infty}$ e de experimentos recentes utilizando essa nova abordagem de controle preditivo.

A apresentação foi bastante didática na motivação e introdução do trabalho, utilizando bem os recursos visuais sem sobrecarregá-los de texto.
Na verdade, quando pecou foi por falta de informação, utilizando imagens e ilustrações em excesso.
Durante toda a apresentação, que aproveitou com exatidão o tempo disponível, João falou bem e conseguiu comunicar bem suas ideias.
Ele poderia, entretanto, ter deixado mais clara a estrutura das subseções de sua apresentação, algo relevante em apresentações longas para manter os presentes situados no panorama geral.

As críticas da banca forma mais incisivas quanto a abrangência dos objetivos estipulados por João e a ausência de algumas abordagens para solucionar o problema em foco.
Nesses pontos, as críticas foram precisas mas muito bem-vindas pelo candidato, uma vez que se deram de forma bastante construtiva em relação aos desafios futuros no desenvolvimento da tese.
Ademais, os membros da banca apontaram algumas falhas do documento como erros no uso de superlativos e contrações e na imprecisão inadequada a um documento científico.

\section{Conclusões}

João conseguiu transmitir o valor de sua pesquisa mesmo para aqueles que não o acompanharam nem tiveram acesso ao documento escrito.
Seu trabalho parece promissor, com possibilidade de contribuições significativas na área, como apontado pelo seu orientador.
As críticas foram contundentes e, em sua maioria, serão de grande auxílio para a construção de uma sólida tese de doutorado.

% O doutorando iniciou sua apresentação com a motivação a seu trabalho, situando sua pesquisa no contexto da demanda energética global e brasileira.
% Ele apresentou, por consequência, uma demanda nacional por um aumento absoluto na produção de petróleo e seus derivados.
% Dessa forma, ele deixou claro a necessidade por maior eficiência na produção de petr
% 
% Motivação, situando seu trabalho no contexto da demanda energética global e brasileira. Demanda absoluta por petróleo deve crescer. Desafios relativos ao meio ambiente, aumento da produção, segurança, eficiência. => Controle Preditivo para atacar os problemas economicos
% 
% Introdução ao MPC no contexto do controle de processos, o processo de interesse (plataforma offshore) e os seus objetivos com o controle usando o MPC.
% 
% Plataformas offshore, gas-lifting, hierarquia dos sistemas de controle, onde se encaixaria o controlador proposto.
% Controle preditivo econômico, como uma variação ou extensão ao MPC não-linear, que leva em consideração objetivos econômicos para além da estabilidade.
% 
% Apresentou suas contribuições.
% Controle robusto $H_{\infty}$ que culminou em um artigo apresentado no Simpósio brasileiro (SBAI).
% EMPC para unir o controle avançado com o RTO, levando em conta fatores econômicos além da estabilidade.
% 
% Propostas de pesquisa derivadas da contribuição do EMPC, sugerindo com uma um avanço natural aos resultados da contribuição e com a segunda tentar atacar as dificuldades previstas na primeira.
% Planejamento para as propostas.
% 
% Críticas ao documento enviado à banca.
% Argimiro, bom conhecimento do processo e suas dificuldades, com boa revisão bibliográfica
% Correção da linguagem (superlativo e contrações), necessidade de precisão, linguagem científica.
% Objetivos muito genéricos.
% Utilizar referências mais abrangentes para evitar argumentos pouco sólidos.
% Críticas relacionadas a abordagens da literatura que não foram mencionadas.
% 
% Críticas às propostas pareceram bastante úteis ao doutorando pois lidaram com os desafios que ele ainda pretende enfrentar, numa perspectiva mais construtiva (já que ele já tinha propostas relativamente sólidas).
% 
% Definir melhor as contribuições. 

\end{document}

