\documentclass[a4paper]{report}
\input{./preamble.tex}
 
\begin{document}
 
\title{Lista 7}
\author{Bruno M. Pacheco (16100865)\\
DAS410079 - Modelagem para Otimização}
 
\maketitle
 
\exercise{1}

Seja $c_{ij}$ o custo de fornecimento do gerador $i$ para a cidade $j$, onde $j$ mapeia inteiros nas cidades segundo a associação $1\to $São José, $2\to $Florianópolis e $3\to $Palhoça.
Estende-se essa definição para abarcar a possibilidade de comprar energia da COPEL através de $i=4$, considerando a restrição de não fornecimento a Palhoça através de custo infinito.
Dessa forma, \[
C = \left\{ c_{ij} \right\} = \begin{bmatrix} 
    600 & 700 & 400 \\
    320 & 300 & 350 \\
    500 & 480 & 450 \\
    1000 & 1000 & \infty
\end{bmatrix} 
.\] 

Ademais, definimos como $s_i$ a capacidade de fornecimento do gerador $i$, de forma que \[
\left\{ s_i \right\} = \begin{bmatrix} 22,5 \\ 36 \\ 27 \\ \infty \end{bmatrix} 
.\] Note que já é desconsiderada da capacidade a perda de transmissão. Também, $d_j$ a demanda da cidade $j$, i.e., \[
\left\{ d_j \right\}  = \begin{bmatrix} 30 \\ 35 \\ 25 \end{bmatrix} 
.\] 

Assim, sendo $x_{ij}$ o fornecimento do gerador $i$ para a cidade $j$, conforme as definições anteriores, chega-se à formulação
\begin{align*}
    \min_{\left\{ x_{ij} \right\}} \quad & \sum_{i=1}^{4} \sum_{j=1}^{3} c_{ij}x_{ij} \\
    \textrm{s.t.} \quad & \sum_{j=1}^{3} x_{ij} \le  s_i,\, i=1,\ldots,3 \\
      & \sum_{i=1}^{4} x_{ij} = d_j,\, j=1,\ldots,3 \\
      & x_{ij} \ge 0,\, \forall i,j
.\end{align*}

A solução então é
\begin{align*}
    x_{1,3} = 22,5 \\
    x_{2,1} = 1 \\
    x_{2,2} = 35 \\
    x_{3,1} = 24,5 \\
    x_{3,3} = 2,5 \\
    x_{4,1} = 4,5
,\end{align*}
sendo o resto nulo.

Após o aumento de 20\%, que resulta na alteração de $\left\{ d_{ij} \right\} $, temos
\begin{align*}
    x_{1,3} = 22,5 \\
    x_{2,2} = 36 \\
    x_{3,1} = 13,5 \\
    x_{3,2} = 6 \\
    x_{3,3} = 7,5 \\
    x_{4,1} = 22,5
.\end{align*}


\exercise{2}

De forma similar ao anterior, temos o custo $c_{ij}$ do trabalhador $i$ executar a tarefa $j$, de forma que \[
\left\{ c_{ij} \right\} = \begin{bmatrix} 
    50 & 50 & \infty & 20 \\
    70 & 40 & 20 & 30 \\
    90 & 30 & 50 & \infty \\
    70 & 20 & 60 & 70
\end{bmatrix} 
.\]

Assim, modelamos através de das variáveis binárias $x_{ij} \in \left\{ 0,1 \right\} $ a execução da tarefa $j$ pelo trabalhador $i$, implicando no modelo
\begin{align*}
    \min_{\left\{ x_{ij} \right\} } \quad & \sum_{i=1}^{4} \sum_{j=1}^{4} c_{ij}x_{ij} \\
    \textrm{s.t.} \quad & \sum_{j=1}^{4} x_{ij} = 1,\, i=1,\ldots,4\\
      & \sum_{i=1}^{4} x_{ij} = 1,\, j=1,\ldots,4 \\
      & x_{ij} \in \left\{ 0,1 \right\} 
.\end{align*}

A solução ótima encontrada é \[
\left\{ x_{ij} \right\} = \begin{bmatrix} 
    1 & 0 & 0 & 0 \\
    0 & 0 & 1 & 0 \\
    0 & 0 & 0 & 1 \\
    0 & 1 & 0 & 0
\end{bmatrix} 
.\] 

\exercise{3}

Queremos definir $x_{ij}$ o fluxo de bicicletas do nó $i$ para o nó $j$. Então, o custo de transporte pelos arcos $c_{ij}$ do vértice $i$ para o vértice $j$, de forma que \[
\left\{ c_{ij} \right\} = \begin{bmatrix} 
    \infty & \infty & \infty & 1 & 0,3 & \infty & \infty & \infty \\
    \infty & \infty & \infty & 0,8 & 4,3 & \infty & \infty  & \infty \\
    \infty & \infty & \infty & 2 & 4,6 & \infty & \infty & \infty   \\
    \infty & \infty & \infty & \infty & 0,5 & 0,2 & 4,5 & 6   \\
    \infty & \infty & \infty & \infty & \infty & 3 & 2,1 & 1,9 \\
    \infty & \infty & \infty & \infty & \infty & \infty & \infty & \infty \\
    \infty & \infty & \infty & \infty & \infty & \infty & \infty & \infty \\
    \infty & \infty & \infty & \infty & \infty & \infty & \infty & \infty \\
\end{bmatrix} 
.\] 

Ademais, definimos como $b_i$ como o saldo (diferença entre saídas e entradas) almejado de cada nó. Assim, levando em consideração os fornecimentos e as demandas, \[
\left\{ b_i \right\} = \begin{bmatrix} 
    -900 \\
    -1400 \\
    -1000 \\
    240 \\
    0 \\
    1100 \\
    1000 \\
    1200
\end{bmatrix} 
.\]

Dessa forma, podemos modelar o problema como
\begin{align*}
    \min_{\left\{ x_{ij} \right\} } \quad & \sum_{i=1}^{8} \sum_{j=1}^{8} c_{ij}x_{ij} \\
    \textrm{s.t.} \quad & \sum_{i=1}^{8} x_{ik} - \sum_{j=1}^{8} x_{kj} = b_k,\, k=1,2,3,5 \\
      & \sum_{i=1}^{8} x_{ik} - \sum_{j=1}^{8} x_{kj} \le  b_k,\, k=4,6,7,8 \\
      & x_{ij} \ge 0,\, i,j \in \left\{ 1,\ldots,8 \right\} 
.\end{align*}

A solução ótima encontrada foi \[
    \left\{ x_{ij} \right\} = \begin{bmatrix} 
    0 & 0 & 0 & 0 & 900 & 0 & 0 & 0 \\
    0 & 0 & 0 & 1400 & 0 & 0 & 0  & 0 \\
    0 & 0 & 0 & 1000 & 0 & 0 & 0 & 0   \\
    0 & 0 & 0 & 0 & 1060 & 1100 & 0 & 0   \\
    0 & 0 & 0 & 0 & 0 & 0 & 760 & 1200 \\
    0 & 0 & 0 & 0 & 0 & 0 & 0 & 0 \\
    0 & 0 & 0 & 0 & 0 & 0 & 0 & 0 \\
    0 & 0 & 0 & 0 & 0 & 0 & 0 & 0 \\
\end{bmatrix} 
\] 

\end{document}
