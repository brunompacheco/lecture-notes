\documentclass[a4paper]{report}
\input{./preamble.tex}
 
\begin{document}
 
\title{Lista 2}
\author{Bruno M. Pacheco (16100865)\\
DAS410079 - Modelagem para Otimização}
 
\maketitle
 
As implementações seguem no arquivo anexo.

\exercise{1}

A implementação proposta na lista anterior,
\begin{align*}
    \max_{x_1,x_2} \quad & 4x_1 + \frac{5}{2}x_2 \\
    \textrm{s.t.} \quad & \frac{3}{4}x_1 + \frac{1}{2}x_2 \le 8 \\
      & \frac{3}{4}x_1 + \frac{3}{4}x_2 \le 7 \\
      & x_1,x_2 \in \Z^{+}
,\end{align*}
foi implementada utilizando as ferramentas do Gurobipy.

\exercise{2}

Foi adicionada uma restrição ao problema de otimização proposto na lista anterior, que apesar de descrita não foi formulada, referente à soma dos componentes de $\bm{q}$. Então, o problema agora formula-se
\begin{align*}
    \min_{\bm{q}} \quad & f\left( \bm{q} \right)  \\
    \textrm{s.t.} \quad & A\bm{q}\le \bm{u} \\
      & A\bm{q}\ge \bm{l} \\
      & \bm{q} \in \left[ 0,1 \right] ^{2} \\
      & q_A + q_B = 1
.\end{align*}

\exercise{3}

Por limitações do Gurobipy, a formulação foi modificada para fazer uso de uma variável auxiliar, 
\begin{align*}
    \min_{\bm{p}, \bm{z}} \quad & \bm{z}^{T}\bm{z}\cdot n_i  \\
    \textrm{s.t.} \quad & \bm{z}_i = \bm{p} - \bm{q_i} , \forall i=1,\ldots,4\\
			& p \in \R^{2}
.\end{align*}
Além disso, não mais otimizamos a norma do vetor, que retornaria a distância entre os pontos de interesse, mas sim seu produto interno. Isso pois assim mantemos somente operações lineares e \[
z_0\text{ t.q. } z_0^{T}z_0 \le z^{T}z, \forall z \implies \|z_0\| \le \|z\|, \forall z
,\] ou seja, otimizando o produto interno otimizamos também a norma.

\end{document}
