\documentclass[a4paper]{report}
\input{./preamble.tex}
 
\begin{document}
 
\title{Lista 6}
\author{Bruno M. Pacheco (16100865)\\
DAS410079 - Modelagem para Otimização}
 
\maketitle
 
\exercise{1}

Sejam $\bm{x} = \begin{bmatrix} c_1 \\ c_2 \end{bmatrix}  \in \Z^2$ a quantidade de colares produzidos. Seja \[
\bm{r} = \begin{bmatrix} 120 \\ 70 \end{bmatrix} 
\] os recursos disponíveis de prata e horas de mão-de-obra. Assim, temos as restrições quanto a fabricação dos colares como \[
\begin{bmatrix}
    3 & 2 \\
    1 & 2
\end{bmatrix}\bm{x} = A \bm{x} \le \bm{r}
.\] Além disso, há também as restrições relativas as encomendas feitas previamente, i.e., \[
\bm{x} \ge \begin{bmatrix} 20 \\ 25 \end{bmatrix} 
.\] 

Quanto ao ouro, temos uma não linearidade do tipo retificador, i.e., o custo associado à quantidade de ouro utilizada $o$ é \[
c_{ouro}\left( o \right) = \begin{cases}
    0, & o \le 100 \\
    100 o, & o > 100
\end{cases}
.\] Para resolver esse problema, adicionamos uma variável de folga $w$ e restrições
\begin{align*}
    o + w &\ge 100 \\
    100 \ge w &\ge 0
.\end{align*}
Dessa forma, se calculamos o custo como \[
c_{ouro}\left( o, w \right) = \left( o+w -100\right) 100
,\] fazemos com que a solução ótima seja $c_{ouro}\left( o,100-o \right) = 0\, \forall o \in \left[ 0,100 \right] $ e $c_{ouro}\left( o,0 \right) = \left( o-100 \right) 100\, \forall o>100$, como desejado.

Finalmente, sabemos que \[
	\begin{bmatrix} 2 & 3 \end{bmatrix} \bm{x} = o
\]  podemos determinar o retorno total
\begin{align*}
    c\left( \bm{x}, w \right) &= \begin{bmatrix} 400 & 500 \end{bmatrix} \bm{x} - c_{ouro}\left(  \begin{bmatrix} 2 & 3 \end{bmatrix} \bm{x}, w  \right)  \\
    &= \begin{bmatrix} 400 & 500 \end{bmatrix} \bm{x} -\begin{bmatrix} 200 & 300 \end{bmatrix} \bm{x} - 100 \left( w-100 \right)  \\
    &= \begin{bmatrix} 200 & 200 \end{bmatrix} \bm{x} - 100\left( w-100 \right) 
.\end{align*}

Agora, formulemos o problema com todas as restrições como
\begin{align*}
    \max_{\bm{x}, w} \quad & \begin{bmatrix} 200 & 200 \end{bmatrix} \bm{x} -100\left( w-100 \right)  \\
    \textrm{s.t.} \quad A\bm{x} &\le \bm{r} \\
    \bm{x} &\ge \begin{bmatrix} 20 \\ 25 \end{bmatrix} \\
			\begin{bmatrix} 2 & 3 \end{bmatrix} \bm{x} + w &\ge 100 \\
			w &\in \left[ 0, 100 \right] \\
			\bm{x} &\in \Z^2
.\end{align*}

Dadas as restrições propostas das encomendas já feitas, o resultado ótimo é exatamente a quantidade mínima, gerando um lucro de \$9000. Isso é fácil de ver uma vez que para fabricá-las já será necessária toda a quantidade de horas disponível.

\exercise{2}

\subexercise{a)}

Se esse fosse o caso, então o custo total seria ditado por
\begin{align*}
    c\left( \bm{x}, w \right) &= \begin{bmatrix} 400 & 500 \end{bmatrix} \bm{x} -\begin{bmatrix} 380 & 570 \end{bmatrix} \bm{x} - 190 \left( w-100 \right)  \\
    &= \begin{bmatrix} 20 & -70 \end{bmatrix} \bm{x} - 190\left( w-100 \right) 
.\end{align*}
Vemos que ainda seria lucrativo fabricar colares 1 além do estoque, direcionando, assim, a solução ótima.

\subexercise{b)}

Nesse caso, fabricam-se 4 colares 1 a mais, sendo somente o mínimo de colares 2 fabricados. O lucro aumenta para \$9400.

\subexercise{c)}

\exercise{3}

Sejam \[
    A = \begin{bmatrix}
	2 & 0 & 1 \\
	1 & 1 & -1
    \end{bmatrix} 
,\] \[
\bm{b} = \begin{bmatrix} 
    4 \\
    -1
\end{bmatrix} 
\] e \[
\bm{c} = \begin{bmatrix} 
    8 \\
    5 \\
    4
\end{bmatrix} 
.\] Então, podemos escrever o problema na forma matricial como
\begin{align*}
    \max_{\bm{x}} \quad & \bm{c}^{T}\bm{x} \\
    \textrm{s.t.} \quad & A\bm{x} \ge \bm{b} \\
      & x_1,x_2,x_3 \ge 0
.\end{align*}
Assim, encontramos seu dual como
\begin{align*}
    \min_{\bm{y}} \quad & \bm{b}^{T} \bm{y} \\
    \textrm{s.t.} \quad & A^{T}\bm{y} \le \bm{c} \\
      & y_1,y_2,y_3 \ge 0
.\end{align*}

\end{document}
