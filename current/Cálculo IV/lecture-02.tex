\lecture{2}{03.02.2021}{Partições e Introdução à Integral}

Definindo, então, o exposto no final da última aula.

\begin{definition}
    Dizemos que uma partição $Q=Q_1\times \ldots\times Q_m$ de $A$ \emph{refina} uma partição $P=P_1\times \ldots\times P_n$ se $\exists Q_j \text{tq} P_i\subset Q_j \forall P_i$. Dizemos $P\subset Q$.
\end{definition}

Dados $Q, P$ como na definição, podemos também encontrar que os sub-blocos $B'$ de $Q$ estão contidos nos blocos $B$ de $P$. Ainda mais, podemos ver que \[
    vol\left( B \right) = \sum_{B' \in Z_Q\left( B \right)  } vol\left( B' \right)
,\] onde $Z_Q\left( B \right) $ define o conjunto dos sub-blocos de $B$ de acordo com a partição $Q$ (que necessariamente refina a partição que gerou $B$).

\begin{prop}
    Se $Q,P$ são partições de $A$ e $Q$ refina $P$, então \[
    s\left( f, Q \right) \ge s\left( f,P \right) 
    \] e \[
    S\left( f,Q \right) \le S\left( f,P \right) 
    ,\] ou, utilizando a desigualdade já estabelecida, \[
     s\left( f,P \right) \le s\left( f, Q \right) \le S\left( f,Q \right) \le S\left( f,P \right) 
    .\] 
\end{prop}
\begin{demo}
    Temos que \[
        S\left( f, P \right) = \sum_{B\in \mathcal{B}\left( P \right) } M_B vol\left( B \right) 
    \] e 
    \begin{align*}
	S\left( f, Q \right) &=\sum_{B'\in \mathcal{Q}} M_{B'} vol\left( B' \right) \\
			     &= \sum_{B\in \mathcal{B}\left( P \right) } \left( 
				 \sum_{B'\in Z\left( B \right) } M_{B'} vol\left( B' \right) 
			     \right) \\
			     &\le \sum_{B\in \mathcal{B}\left( P \right) } \left( 
				 \sum_{B'\in Z\left( B \right) } vol\left( B' \right) 
			     \right) \\
			     &= \sum_{B\in \mathcal{B}\left( P \right) } M_B vol\left( B \right)  = S\left( f, P \right) 
    .\end{align*}
\end{demo}

\begin{corollary}
    $\forall P,Q$ partições de $A$, \[
    s\left( f, P \right) \le S\left( f,Q \right) 
    .\] 
\end{corollary}
\begin{proof}
    Tome
    \begin{align*}
	& R = \left( P_1\cup Q_1 \right) \times \ldots\times \left( P_n\cup Q_n \right) = P + Q \\
	&\implies P,Q \subset R
    ,\end{align*}
    logo $R$ refina as duas, ou seja, \[
    s\left( f,P \right) \le s\left( f,R \right) \le S\left( f,R \right) \le S\left( f,Q \right) 
    .\] 
\end{proof}

Portanto, podemos concluir que \[
s\left( f \right) = \left\{ s\left( f,P \right) : P \text{ partição de }A \right\} 
\] é limitada superiormente.

\begin{definition}
    A \emph{integral inferior} (resp. \emph{superior}) de $f$ em $A$ \[
	\underline{\int}_A f := sup\text{ }s\left( f \right) 
,\] (resp. \[
	\overline{\int}_A f := inf\text{ }S\left( f \right) 
    \]).
\end{definition}

\begin{remark}
    \[
    \underline{\int}_A f \le  \overline{\int}_A f
    .\] 
\end{remark}

\begin{eg}
    Seja \[
    f\left( x \right) = \begin{cases}
	1 &, x=\left( x_1,\ldots,x_n \right) : x_i\in \Q \forall i \\
	0 &, \text{do contrário}
    \end{cases}
    .\]
    Seja $P$ partição de $A:\forall B $ sub-bloco, então
    \begin{align*}
	& M_B = 1 \text{ e }m_B=0 \\
	&\implies S\left( f, P \right) = \sum_{B\in \mathcal{B}\left( P \right) } 1 . vol\left( B \right)= vol \left( A \right) \\
	&\implies \overline{\int}_A f = 1
    ,\end{align*}
    enquanto, de forma análoga, \[
    \underline{\int}_A f = 0
    .\] 
\end{eg}

\begin{definition}
    $f:A\to \R^{n}$ limitada é \emph{integrável} se \[
	\underline{\int}_A f = \overline{\int}_A f = \int_A f
    ,\] que é chamada \emph{integral de $f$ em $A$}.
\end{definition}

\begin{prop}
    $f:A\to \R$ limitada é integrável $\iff \forall \epsilon>0 \exists P $partição de $A$ tal que  \[
    S\left( f, P_\epsilon \right) - s\left( f,P_\epsilon \right) <\epsilon \tag{*}
    .\] 
\end{prop}
\begin{demo}
    ($\implies$) Suponha $f$ integrável. Então $\exists P_1$ de $A$ tal que \[
    s\left( f, P_1 \right) < \int_A f + \frac{\epsilon}{2}
    .\] Da mesma forma, $\exists P_2$ de $A$ tal que \[
    \int_A f - \frac{\epsilon}{2} < s\left( f, P_2 \right) 
    .\] Agora, podemos escolher $P_\epsilon$ que refine ambas $P_1$ e $P_2$ (sua existência é garantida), de forma que
    \begin{align*}
	& s\left( f, P_2 \right) \le s\left( f,P_\epsilon \right) \le S\left( f,P_\epsilon \right) \le S\left( f,P_1 \right) \\
	& \implies \int_A f - \frac{\epsilon}{2} \le s\left( f,P_\epsilon \right) \le S\left( f,P_\epsilon \right) \le \int_A f + \frac{\epsilon}{2} \\
	& \implies S\left( f, P_\epsilon \right) - s\left( f, P_\epsilon \right) \le \epsilon
    .\end{align*}

    ($\impliedby$) Se vale (*) $\forall \epsilon$ e alguma $P_\epsilon$, então suponha  $f$ não integrável. Então poderíamos tomar \[
	\epsilon = \frac{1}{2}\left( \overline{\int}_A f - \underline{\int}_A f \right) > 0
    .\] Escolhendo $P_\epsilon$ satisfazendo (*), pela definição das integrais como sup/inf das somas inferiores/superiores,
    \begin{align*}
	& s\left( f, P_\epsilon \right) \le \underline{\int}_A f \le \overline{\int}_A f\le S\left( f, P_\epsilon \right) \\
	&\implies \overline{\int}_A f - \underline{\int}_A f \le S\left( f, P_\epsilon \right) - \underline{\int}_A f \le S\left( f, P_\epsilon \right) -s\left( f, P_\epsilon \right)
    ,\end{align*}
     portanto \[
	S\left( f, P_\epsilon \right) - s\left( f, P_\epsilon \right) \ge 2\epsilon
    ,\] uma contradição com (*).
\end{demo}
\begin{corollary}
    Se $f:A\to \R$ é contínua, então $f$ é integrável.
\end{corollary}
\begin{proof}
    Seja $f:A\to \R$ contínua, então $f$ é limitada por Weierstrass. Adote em $\R^{n}$ a norma \[
    \|.\|_{max} = \|.\|
    \] do máximo.
    Como $A$ é um compacto, $f$ é uniformemente contínua.
    Seja fixado $\epsilon > 0$. Escolha $\delta > 0$ tal que $\forall x,y \in A$ \[
	\|x-y\|<\delta \implies \|f\left( x \right) -f\left( y \right) \|<\frac{\epsilon}{vol\left( A \right) }
    .\] 

    \begin{remark}
        Em um bloco degenerado, qualquer função é integrável com \[
        \int_A f = 0
        .\] 
    \end{remark}

    Escolha uma partição $P=\prod_{i=1}^{n} P_i  $ de modo que \[
    \|P_i\| < \delta
    ,\] ou seja, com que a norma de cada subintervalo seja menor que $\delta$. Tome $P = P_1\times \ldots\times P_n$. Seja $B \in \mathcal{B}\left( P \right) $. Como $f|_B$ é ainda contínua, por Weierstrass $\exists x_B,y_B\in B$ tal que
    \begin{align*}
	f\left( y_B \right) &= M_B \\
	f\left( x_B \right) &= m_B
    .\end{align*}
    Por outro lado, \[
    \|y_B - x_B\| = max\left\{ |\left( y_B \right)_j - \left( x_B \right)_j | : j = 1,\ldots,n \right\} 
    ,\] mas como $P$ foi escolhida de forma que os intervalos que constituem as partições sejam menores do que $\delta$, então \[
    |\left( y_B \right)_j - \left( x_B \right)_j | < \delta \forall j=1,\ldots,n
    \] logo \[
    \|y_B - x_B\| < \delta
    .\] Assim, vemos que \[
    M_B - m_B = \|f\left( y_B \right) -f\left( x_B \right) \|< \frac{\epsilon}{vol\left( A \right) }
    ,\] para todo $B\in \mathcal{B}\left( P \right) $. Portanto, \[
    S\left( f, P \right) - s\left( f, P \right) = \sum_{B\in \mathcal{B}\left( P \right) } \left( M_B - m_B \right) vol\left( B \right) <\frac{\epsilon}{vol\left( A \right) } \sum_{B\in \mathcal{B}\left( P \right) } vol\left( B \right) = \epsilon
    ,\] o que nos garante, pela proposição anterior, que $f$ é integrável.
\end{proof}

De forma direta, podemos ver que as funções constantes são integráveis e \[
\int_A f_c = c . vol\left( A \right) 
.\] 

\begin{theorem}
    (Propriedades aritméticas da integração) Seja $A$ um bloco, $f, g: A\to \R$ integráveis e $c\in \R$. Então, vale o seguinte
    \begin{enumerate}
	\item $f+g:A\to \R$, $f+g(x) = f(x) + g(x)$ é integrável e \[
	\int_A \left( f+g \right) = \int_A f + \int_A g
	;\] 
	\item $c.f: A \to R$, é integrável e  \[
	\int_A \left( c.f \right) = c. \int_A f
	;\] 
	\item Se $f(x) \le g(x) \forall x \in A$, então \[
	\int_A f \le \int_A g
	.\] 
    \end{enumerate}
\end{theorem}
\begin{proof}
    (1) Seja $P$ uma partição de $A$. Seja $B\in \mathcal{B}\left( P \right) $. $\forall  x\in B$
    \begin{align*}
	& m_B^{f} + m_B^{g} \le  f+g \left( x \right) =f(x) + g(x) \le M_B^{f} + M_B^{g}\\
	&\implies m_B^{f} + m_B^{g} \le m_B^{(f+g)} \\
	&\implies M_B^{(f+g)}\le M_B^{f} + M_B^{g}
    .\end{align*}
    Veja que com a soma desses componentes com o volume dos blocos já nos leva às somas superiores e inferiores. Ou seja, temos que
    \begin{align*}
	s\left( f+g, P \right) &= \sum_{B\in \mathcal{B}\left( P \right)} m_B^{(f+g)} vol\left( B \right) \\ 
			       &\ge \sum_{B\in \mathcal{B}\left( P \right)} \left( m_B^{f}+m_B^{g} \right) vol\left( B \right) = s\left( f, P \right) + s\left( g, P \right) 
    .\end{align*}
    De forma análoga, \[
    S\left( f+g,P \right) \le S\left( f,P \right) +S\left( g,P \right) 
    .\]
    Agora podemos trabalhar as integrais de $f+g$
    \begin{align*}
	\underline{\int}_A \left( f+g \right)  &= sup \left\{ s\left( f+g,P \right)  \right\} \\
			       &\ge sup \left\{ s\left( f,P \right)  \right\} + sup \left\{ s\left( g,P \right)  \right\} = \underline{\int}_A f + \underline{\int}_A g
    \end{align*}
    e \[
	\overline{\int}_A \left( f+g \right) \le  \overline{\int}_A f + \overline{\int}_A g
    .\] Assim, pela integrabilidade de $f$ e $g$ e pela desigualdade já estabelecida entre as integrais inferiores e superiores, \[
    \int_A f + \int_A g \le \underline{\int}_A \left( f+g \right) \le \overline{\int}_A \left( f+g \right) \le \int_A f + \int_A g
    ,\] ou seja, não só $f+g$ é integrável como sua integral é \[
    \int_A \left( f+g \right) = \int_A f + \int_A g
    .\] 

    (2) Para o caso $c<0$, temos que, dada uma partição $P$ de $A$,
    \begin{align*}
	m_B^{c.f} &= inf\left\{ c.f(x): x\in B \right\} \\
		  &= c.sup\left\{ f(x): x\in B \right\} \\
		  &= c.M_B^{f} \\
	M_B^{c.f} &= c.m_B^{f}
    ,\end{align*}
    ou seja,
    \begin{align*}
	S\left( c.f, P \right) &= c.s\left( f,P \right) \\
	s\left( c.f, P \right) &= c.S\left( f,P \right) \\
    .\end{align*}
    Assim, podemos ver que \[
	\overline{\int}_A c.f = c.\underline{\int}_A f
    \] e \[
	\underline{\int}_A c.f = c.\overline{\int}_A f
    ,\] logo, $c.f$ é integrável e \[
    \int_A c.f = c.\int_A f
    .\] 
    Para $c\ge 0$ isso se dá de forma análoga, partindo de 
    \begin{align*}
	m_B^{c.f} &= inf\left\{ c.f(x): x\in B \right\} \\
		  &= c.inf\left\{ f(x): x\in B \right\} \\
		  &= c.m_B^{f} \\
	M_B^{c.f} &= c.M_B^{f}
    ,\end{align*}
    vemos que \[
	\overline{\int}_A c.f = c.\overline{\int}_A f
    \] e \[
	\underline{\int}_A c.f = c.\underline{\int}_A f
    ,\] logo, $c.f$ é integrável novamente e \[
    \int_A c.f = c.\int_A f
    .\] 

    (3) $f(x) \le g(x), \forall x\in A$. $\forall B\in \mathcal{B}\left( P \right) $, onde $P$ é uma partição de $A$,
    \begin{align*}
	m_B^{f}&\le m_B^{g} \\
	M_B^{f} &\le M_B^{g}
    ,\end{align*}
    que nos permite estabelecer a mesma relação para as somas superiores e inferiores, que desencadeia em  \[
    \int_A f\le \int_A g
    .\] 
\end{proof}


