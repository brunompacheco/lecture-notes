\documentclass[a4paper]{report}
\input{./preamble.tex}
 
\begin{document}
 
\title{Exercícios - Semana 1}
\author{Bruno M. Pacheco (16100865)\\
DAS5341 - Inteligência Artificial Aplicada a Controle e Automação}
 
\maketitle
 
\exercise{3}

\subexercise{a}

Um estado precisa representar a quantidade de missionários e canibais em uma das margens (uma vez que a outra margem terá um valor complementar) e em qual margem se encontra o barco.

Podemos representar os estados através do conjunto $E = \N\times \N\times \{0,1\} $, de forma que um estado $e\in E$ seja da forma $e = \left( m, c, b \right) $, onde $m$ indica a quantidade de missionários na margem inicial do rio, $c$ a quantidade de canibais nessa mesma margem, e $b$ caso o barco esteja na margem original (1) ou não.

\subexercise{b}

O estado inicial pode ser representado por \[
e_i = \left( 3,3,1 \right) 
,\] enquanto o final pode ser \[
e_f = \left( 0,0,0 \right) 
.\] 

\subexercise{c}

Podemos descrever as operações através de funções $f_1,f_2,f_3,f_4,f_5,g_1,g_2,g_3,g_4,g_5: E \longrightarrow E$ de forma que 
\begin{align*}
    f_1: E &\longrightarrow E \\
    \left( m,c,1 \right)  &\longmapsto f_1(\left( m,c,1 \right) ) = \left( m-1,c,0 \right) 
,\end{align*}
\begin{align*}
    f_2: E &\longrightarrow E \\
    \left( m,c,1 \right)  &\longmapsto f_1(\left( m,c,1 \right) ) = \left( m-2,c,0 \right) 
,\end{align*}
\begin{align*}
    f_3: E &\longrightarrow E \\
    \left( m,c,1 \right)  &\longmapsto f_1(\left( m,c,1 \right) ) = \left( m,c-1,0 \right) 
,\end{align*}
\begin{align*}
    f_4: E &\longrightarrow E \\
    \left( m,c,1 \right)  &\longmapsto f_1(\left( m,c,1 \right) ) = \left( m,c-2,0 \right) 
\end{align*}
e
\begin{align*}
    f_5: E &\longrightarrow E \\
    \left( m,c,1 \right)  &\longmapsto f_1(\left( m,c,1 \right) ) = \left( m-1,c-1,0 \right) 
\end{align*}
sejam as operações de travessia no sentido da margem desejada, ou seja, possuem como pré-requisito que $b=1$ e que tenham suficientes passageiros para a mesma na margem de origem. De forma equivalente, as operações $g_i=f_i^{-1}, i=1,\ldots,5$ fazem o mesmo mas no sentido oposto, com pré-requisitos, portanto, exatamente opostos.

\exercise{6}

\subexercise{a}

Precisamos representar a posição de cada caixa em cada pilha. Para isso, podemos utilizar uma linguagem $L\subseteq \Sigma^*$ sobre o alfabeto $\Sigma=\{a,b,c,d,e,f,g\} $. Para garantir que as palavras dessa linguagem representam algum empilhamento das caixa, definimos \[
L = \left\{ w\in \Sigma^* : \|w\| = \|Alph\left( w \right) \|\right\} 
,\] ou seja, a quantidade de letras no alfabeto de cada palavra deve ser idêntica ao tamanho da palavra, evitando palavras com letras repetidas.

Assim, podemos definir os estados como \[
E = \left\{ \left( p_1,p_2,p_3 \right) \in L\times L\times L : \|p_1\|+\|p_2\|+\|p_3\| = \|\Sigma\| \right\} 
.\] Note que a condição sobre os estados junto com a restrição sobre as palavras resulta na ausência de uma mesma caixa em múltiplas pilhas ($p_1,p_2,p_3$).

\subexercise{b}

\[
e_i = \left( abc, de, fg \right) 
\] e \[
e_f = \left( \epsilon, abcgf, ed \right) 
.\] 

\subexercise{c}

A movimentação das caixas pode ser modelada pelas funções $f_{i,j}: E \longrightarrow E$ que movimentam uma caixa da pilha $i$ para a pilha $j$, de forma que, por exemplo,
\begin{align*}
    f_{1,2}: E &\longrightarrow E \\
    \left( p_1,p_2,p_3 \right)  &\longmapsto f_{1,2}(\left( p_1,p_2,p_3 \right) ) = \left( p_1 \setminus last\left( p_1 \right), p_2\cdot last\left( p_1 \right)  , p_3 \right) 
,\end{align*}
onde $last(\cdot )$ indica a última letra de uma palavra e as demais operações indicam remoção e concatenação.


\end{document}
