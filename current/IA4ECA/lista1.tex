\documentclass[a4paper]{report}
% Some basic packages
\usepackage[utf8]{inputenc}
\usepackage[T1]{fontenc}
\usepackage{textcomp}
\usepackage[english]{babel}
\usepackage{url}
\usepackage{graphicx}
\usepackage{float}
\usepackage{booktabs}
\usepackage{enumitem}

\pdfminorversion=7

% Don't indent paragraphs, leave some space between them
\usepackage{parskip}

% Hide page number when page is empty
\usepackage{emptypage}
\usepackage{subcaption}
\usepackage{multicol}
\usepackage{xcolor}

% Other font I sometimes use.
% \usepackage{cmbright}

% Math stuff
\usepackage{amsmath, amsfonts, mathtools, amsthm, amssymb}
% Fancy script capitals
\usepackage{mathrsfs}
\usepackage{cancel}
% Bold math
\usepackage{bm}
% Some shortcuts
\newcommand\N{\ensuremath{\mathbb{N}}}
\newcommand\R{\ensuremath{\mathbb{R}}}
\newcommand\Z{\ensuremath{\mathbb{Z}}}
\renewcommand\O{\ensuremath{\emptyset}}
\newcommand\Q{\ensuremath{\mathbb{Q}}}
\newcommand\C{\ensuremath{\mathbb{C}}}
\renewcommand\L{\ensuremath{\mathcal{L}}}

% Package for Petri Net drawing
\usepackage[version=0.96]{pgf}
\usepackage{tikz}
\usetikzlibrary{arrows,shapes,automata,petri}
\usepackage{tikzit}
\input{petri_nets_style.tikzstyles}

% Easily typeset systems of equations (French package)
\usepackage{systeme}

% Put x \to \infty below \lim
\let\svlim\lim\def\lim{\svlim\limits}

%Make implies and impliedby shorter
\let\implies\Rightarrow
\let\impliedby\Leftarrow
\let\iff\Leftrightarrow
\let\epsilon\varepsilon

% Add \contra symbol to denote contradiction
\usepackage{stmaryrd} % for \lightning
\newcommand\contra{\scalebox{1.5}{$\lightning$}}

% \let\phi\varphi

% Command for short corrections
% Usage: 1+1=\correct{3}{2}

\definecolor{correct}{HTML}{009900}
\newcommand\correct[2]{\ensuremath{\:}{\color{red}{#1}}\ensuremath{\to }{\color{correct}{#2}}\ensuremath{\:}}
\newcommand\green[1]{{\color{correct}{#1}}}

% horizontal rule
\newcommand\hr{
    \noindent\rule[0.5ex]{\linewidth}{0.5pt}
}

% hide parts
\newcommand\hide[1]{}

% si unitx
\usepackage{siunitx}
\sisetup{locale = FR}

% Environments
\makeatother
% For box around Definition, Theorem, \ldots
\usepackage{mdframed}
\mdfsetup{skipabove=1em,skipbelow=0em}
\theoremstyle{definition}
\newmdtheoremenv[nobreak=true]{definitie}{Definitie}
\newmdtheoremenv[nobreak=true]{eigenschap}{Eigenschap}
\newmdtheoremenv[nobreak=true]{gevolg}{Gevolg}
\newmdtheoremenv[nobreak=true]{lemma}{Lemma}
\newmdtheoremenv[nobreak=true]{propositie}{Propositie}
\newmdtheoremenv[nobreak=true]{stelling}{Stelling}
\newmdtheoremenv[nobreak=true]{wet}{Wet}
\newmdtheoremenv[nobreak=true]{postulaat}{Postulaat}
\newmdtheoremenv{conclusie}{Conclusie}
\newmdtheoremenv{toemaatje}{Toemaatje}
\newmdtheoremenv{vermoeden}{Vermoeden}
\newtheorem*{herhaling}{Herhaling}
\newtheorem*{intermezzo}{Intermezzo}
\newtheorem*{notatie}{Notatie}
\newtheorem*{observatie}{Observatie}
\newtheorem*{exe}{Exercise}
\newtheorem*{opmerking}{Opmerking}
\newtheorem*{praktisch}{Praktisch}
\newtheorem*{probleem}{Probleem}
\newtheorem*{terminologie}{Terminologie}
\newtheorem*{toepassing}{Toepassing}
\newtheorem*{uovt}{UOVT}
\newtheorem*{vb}{Voorbeeld}
\newtheorem*{vraag}{Vraag}

\newmdtheoremenv[nobreak=true]{definition}{Definition}
\newtheorem*{eg}{Example}
\newtheorem*{notation}{Notation}
\newtheorem*{previouslyseen}{As previously seen}
\newtheorem*{remark}{Remark}
\newtheorem*{note}{Note}
\newtheorem*{problem}{Problem}
\newtheorem*{observe}{Observe}
\newtheorem*{property}{Property}
\newtheorem*{intuition}{Intuition}
\newmdtheoremenv[nobreak=true]{prop}{Proposition}
\newmdtheoremenv[nobreak=true]{theorem}{Theorem}
\newmdtheoremenv[nobreak=true]{corollary}{Corollary}

% End example and intermezzo environments with a small diamond (just like proof
% environments end with a small square)
\usepackage{etoolbox}
\AtEndEnvironment{vb}{\null\hfill$\diamond$}%
\AtEndEnvironment{intermezzo}{\null\hfill$\diamond$}%
% \AtEndEnvironment{opmerking}{\null\hfill$\diamond$}%

% Fix some spacing
% http://tex.stackexchange.com/questions/22119/how-can-i-change-the-spacing-before-theorems-with-amsthm
\makeatletter
\def\thm@space@setup{%
  \thm@preskip=\parskip \thm@postskip=0pt
}


% Exercise 
% Usage:
% \exercise{5}
% \subexercise{1}
% \subexercise{2}
% \subexercise{3}
% gives
% Exercise 5
%   Exercise 5.1
%   Exercise 5.2
%   Exercise 5.3
\newcommand{\exercise}[1]{%
    \def\@exercise{#1}%
    \subsection*{Exercise #1}
}

\newcommand{\subexercise}[1]{%
    \subsubsection*{Exercise \@exercise.#1}
}


% \lecture starts a new lecture (les in dutch)
%
% Usage:
% \lecture{1}{di 12 feb 2019 16:00}{Inleiding}
%
% This adds a section heading with the number / title of the lecture and a
% margin paragraph with the date.

% I use \dateparts here to hide the year (2019). This way, I can easily parse
% the date of each lecture unambiguously while still having a human-friendly
% short format printed to the pdf.

\usepackage{xifthen}
\def\testdateparts#1{\dateparts#1\relax}
\def\dateparts#1 #2 #3 #4 #5\relax{
    \marginpar{\small\textsf{\mbox{#1 #2 #3 #5}}}
}

\def\@lecture{}%
\newcommand{\lecture}[3]{
    \ifthenelse{\isempty{#3}}{%
        \def\@lecture{Lecture #1}%
    }{%
        \def\@lecture{Lecture #1: #3}%
    }%
    \subsection*{\@lecture}
    \marginpar{\small\textsf{\mbox{#2}}}
}



% These are the fancy headers
\usepackage{fancyhdr}
\pagestyle{fancy}

% LE: left even
% RO: right odd
% CE, CO: center even, center odd
% My name for when I print my lecture notes to use for an open book exam.
% \fancyhead[LE,RO]{Gilles Castel}

\fancyhead[RO,LE]{\@lecture} % Right odd,  Left even
\fancyhead[RE,LO]{}          % Right even, Left odd

\fancyfoot[RO,LE]{\thepage}  % Right odd,  Left even
\fancyfoot[RE,LO]{}          % Right even, Left odd
\fancyfoot[C]{\leftmark}     % Center

\makeatother




% Todonotes and inline notes in fancy boxes
\usepackage{todonotes}
\usepackage{tcolorbox}

% Make boxes breakable
\tcbuselibrary{breakable}

% Verbetering is correction in Dutch
% Usage: 
% \begin{verbetering}
%     Lorem ipsum dolor sit amet, consetetur sadipscing elitr, sed diam nonumy eirmod
%     tempor invidunt ut labore et dolore magna aliquyam erat, sed diam voluptua. At
%     vero eos et accusam et justo duo dolores et ea rebum. Stet clita kasd gubergren,
%     no sea takimata sanctus est Lorem ipsum dolor sit amet.
% \end{verbetering}
\newenvironment{verbetering}{\begin{tcolorbox}[
    arc=0mm,
    colback=white,
    colframe=green!60!black,
    title=Opmerking,
    fonttitle=\sffamily,
    breakable
]}{\end{tcolorbox}}

% Noot is note in Dutch. Same as 'verbetering' but color of box is different
\newenvironment{noot}[1]{\begin{tcolorbox}[
    arc=0mm,
    colback=white,
    colframe=white!60!black,
    title=#1,
    fonttitle=\sffamily,
    breakable
]}{\end{tcolorbox}}




% Figure support as explained in my blog post.
\usepackage{import}
\usepackage{xifthen}
\usepackage{pdfpages}
\usepackage{transparent}
\newcommand{\incfig}[1]{%
    \def\svgwidth{\columnwidth}
    \import{./figures/}{#1.pdf_tex}
}

% Fix some stuff
% %http://tex.stackexchange.com/questions/76273/multiple-pdfs-with-page-group-included-in-a-single-page-warning
\pdfsuppresswarningpagegroup=1


% My name
\author{Bruno M. Pacheco}

 
\begin{document}
 
\title{Exercícios - Semana 1}
\author{Bruno M. Pacheco (16100865)\\
DAS5341 - Inteligência Artificial Aplicada a Controle e Automação}
 
\maketitle
 
\exercise{3}

\subexercise{a}

Um estado precisa representar a quantidade de missionários e canibais em uma das margens (uma vez que a outra margem terá um valor complementar) e em qual margem se encontra o barco.

Podemos representar os estados através do conjunto $E = \N\times \N\times \{0,1\} $, de forma que um estado $e\in E$ seja da forma $e = \left( m, c, b \right) $, onde $m$ indica a quantidade de missionários na margem inicial do rio, $c$ a quantidade de canibais nessa mesma margem, e $b$ caso o barco esteja na margem original (1) ou não.

\subexercise{b}

O estado inicial pode ser representado por \[
e_i = \left( 3,3,1 \right) 
,\] enquanto o final pode ser \[
e_f = \left( 0,0,0 \right) 
.\] 

\subexercise{c}

Podemos descrever as operações através de funções $f_1,f_2,f_3,f_4,f_5,g_1,g_2,g_3,g_4,g_5: E \longrightarrow E$ de forma que 
\begin{align*}
    f_1: E &\longrightarrow E \\
    \left( m,c,1 \right)  &\longmapsto f_1(\left( m,c,1 \right) ) = \left( m-1,c,0 \right) 
,\end{align*}
\begin{align*}
    f_2: E &\longrightarrow E \\
    \left( m,c,1 \right)  &\longmapsto f_1(\left( m,c,1 \right) ) = \left( m-2,c,0 \right) 
,\end{align*}
\begin{align*}
    f_3: E &\longrightarrow E \\
    \left( m,c,1 \right)  &\longmapsto f_1(\left( m,c,1 \right) ) = \left( m,c-1,0 \right) 
,\end{align*}
\begin{align*}
    f_4: E &\longrightarrow E \\
    \left( m,c,1 \right)  &\longmapsto f_1(\left( m,c,1 \right) ) = \left( m,c-2,0 \right) 
\end{align*}
e
\begin{align*}
    f_5: E &\longrightarrow E \\
    \left( m,c,1 \right)  &\longmapsto f_1(\left( m,c,1 \right) ) = \left( m-1,c-1,0 \right) 
\end{align*}
sejam as operações de travessia no sentido da margem desejada, ou seja, possuem como pré-requisito que $b=1$ e que tenham suficientes passageiros para a mesma na margem de origem. De forma equivalente, as operações $g_i=f_i^{-1}, i=1,\ldots,5$ fazem o mesmo mas no sentido oposto, com pré-requisitos, portanto, exatamente opostos.

\exercise{5}

\subexercise{a}

Podemos modelar os estados como uma matriz $n\times n$ normal (pela definição do enunciado) e um valor binário que indica se a matriz é um quadrado mágico ou não.

Dessa forma, podemos escrever o conjunto de estados \[
E = \left\{ \left( M, m \right) \in  \Z_{n\times n} \times \{0,1\} : M\text{ é normal} \right\} 
.\] 

\subexercise{b}

O estado inicial $e_i=\left( M_i, b_i \right) $ pode ser composto por matriz normal $M_i$ e valor $b_i$, enquanto o final $e_f=\left( M_f,b_f \right) $ é qualquer estado com $b_f=1$.

\subexercise{c}

As operações possíveis são trocas de valores entre os campos da matriz, mantendo-a normal. Nota-se também que o valor de $b$ deve ser corrigido a cada operação para condizer com a matriz resultante.

\exercise{6}

\subexercise{a}

Precisamos representar a posição de cada caixa em cada pilha. Para isso, podemos utilizar uma linguagem $L\subseteq \Sigma^*$ sobre o alfabeto $\Sigma=\{a,b,c,d,e,f,g\} $. Para garantir que as palavras dessa linguagem representam algum empilhamento das caixa, definimos \[
L = \left\{ w\in \Sigma^* : \|w\| = \|Alph\left( w \right) \|\right\} 
,\] ou seja, a quantidade de letras no alfabeto de cada palavra deve ser idêntica ao tamanho da palavra, evitando palavras com letras repetidas.

Assim, podemos definir os estados como \[
E = \left\{ \left( p_1,p_2,p_3 \right) \in L\times L\times L : \|p_1\|+\|p_2\|+\|p_3\| = \|\Sigma\| \right\} 
.\] Note que a condição sobre os estados junto com a restrição sobre as palavras resulta na ausência de uma mesma caixa em múltiplas pilhas ($p_1,p_2,p_3$).

\subexercise{b}

\[
e_i = \left( abc, de, fg \right) 
\] e \[
e_f = \left( \epsilon, abcgf, ed \right) 
.\] 

\subexercise{c}

A movimentação das caixas pode ser modelada pelas funções $f_{i,j}: E \longrightarrow E$ que movimentam uma caixa da pilha $i$ para a pilha $j$, de forma que, por exemplo,
\begin{align*}
    f_{1,2}: E &\longrightarrow E \\
    \left( p_1,p_2,p_3 \right)  &\longmapsto f_{1,2}(\left( p_1,p_2,p_3 \right) ) = \left( p_1 \setminus last\left( p_1 \right), p_2\cdot last\left( p_1 \right)  , p_3 \right) 
,\end{align*}
onde $last(\cdot )$ indica a última letra de uma palavra e as demais operações indicam remoção e concatenação.


\end{document}
