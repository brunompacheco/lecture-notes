\documentclass[a4paper]{report}
\input{./preamble.tex}
 
\begin{document}
 
\title{Integer Programming for Timetabling in Higher Education Institutions\\Project Proposal}
\author{Bruno M. Pacheco (16100865)\\
Pedro Kretzschmar (18100679) \\
DAS410049 - Programação Inteira}
 
\maketitle

\section*{Characterization}

Timetabling is the allocation of resources to objects being placed in space-time aiming to satisfy as nearly as possible a set of desirable objectives [1]. At the first moment, it may look simple to solve the timetabling problem, but the complexity of the problem is growing alongside the Higher Education Institutions (HEIs) [2].

HEIs timetabling has many formulations, but the two most "standard" ones are the Post Enrollment Course Timetabling (PE-CTT) and the Curriculum-Based Course Timetabling (CB-CTT) [3]. These two different approaches have their differences in the main constraints of the problem.
The PE-CTT intends to simulate the situation where the students are given a choice of lectures that they wish to attend and the timetable is to be constructed with the goal of satisfying all the enrollments [4].
The CB-CTT approach consists of the scheduling of the lectures for several university courses within a given number of rooms and periods. The conflicts between courses are set according to the curricula published by the University [5]. So, the source of conflicts in PE-CTT is the actual student enrolment whereas in CB-CTT the courses in conflict are those that belong to the same predefined group of courses, or curricula.
However, this is only one of the differences. When in PE-CTT, the lectures can be seen as a self-standing event, penalizing late, consecutive, and isolated ones, in CB-CTT it is important to focus on trying to evenly spread the lectures of a course over the weekdays, preserve the same room, and focus on developing a good experience to each curriculum [3].

Following the definition of Feiring, in \emph{Linear Programming: An Introduction} that mathematical programming is the allocation of limited resources to activities to maximize intereset and minimize cost, it is natural to think of timetabling as an application for it. Limited resources in timetabling can be seen as rooms, time-periods, teachers' availability, equipment, and lecture requirements. The interest is the satisfaction of students and teachers with the allocation, at the same time that the cost is related to the number of rooms used, the gap time between lectures, the number of time that a class changes room, etc. As one can clearly notice, most of objects of interest are of integer nature, thus the need for integer programming.
 
\section*{Methodology}

We plan to open the lecture with an introduction to the problem, its main features and particularities, and what makes it relevant nowadays. Afterward, the formulations of the timetabling problem for IP will be shown for its different variants. During this moment, we will present the challenges that arise alongside the complexity of the formulations. Lastly, we will bring a brief overview of the state-of-the-art and a comparison with other approaches, besides references that may be useful for further research.

\section*{Expected Results}

With the lecture, we plan to provide a good understanding of the problem and the challenges that come with it, besides the formulations that can be used and the implications of this choice. Also, by giving an overview of the state-of-the-art on the subject we aim to provide a starting point on further research on this subject.

\section*{References}

1. Scheduling, timetabling and rostering - A special relationship? Wren, A. 1995. \\
2. Practices in timetablingin higher education institutions: a systematic review. Vrielink. 2019. \\
3. Feature-based tuning of simulated annealingappliedto the curriculum-based course timetabling problem - greatintroduction. Bellio. 2014. \\
4. Post enrolment based course timetabling: a description of the problem model used for track two of the second international timetabling competition. Lewis, R. 2007. \\
5. The second international timetabling competition (ITC-2007): Curriculum-based course timetabling (track 3). Di Gaspero, L. 2007. \\
6. A survey of approaches for university course timetabling problem. Babaei, H., Karimpour, J., \& Hadidi, A. 2015. \\
7. Practice and Theory of Automated Timetabling (PATAT) and the Multidisciplinary International Scheduling Conference: Theory and Application (MISTA). \\
\end{document}

