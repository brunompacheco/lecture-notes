\documentclass[a4paper]{report}
\input{./preamble.tex}
 
\begin{document}
 
\title{Integer Programming for Timetabling in Higher Education Institutions\\Project Proposal}
\author{Bruno M. Pacheco (16100865)\\
Pedro Kretzschmar (???) \\
DAS410049 - Programação Inteira}
 
\maketitle

\section*{Characterization}

\begin{itemize}
    \item What is timetabling? (definition) (Vrielink et al., Practices in timetabling in higher education institutions: a systematic review)
	\begin{itemize}
	    \item Wren 1995: (Scheduling, timetabling and rostering—a special relationship?) "The allocation of given resources to specific objects being placed in space time, in such way as to satisfy as nearly as possible a set of desirable objectives"
	    \item NP-hard or NP-complete problem (Even 1975, On the complexity of time table and multi-commodity flow problems)
	    \item HEI timetabling has many formulations, the two most "standard" being (via Bellio et al., Feature-based tuning of simulated annealing appliedto the curriculum-based course timetabling problem - great introduction)
		\begin{itemize}
		    \item Post-Enrollment Course Timetabling(PE-CTT) (see Lewis et al., Post enrolment based course timetabling: a description of the problem model used for track two of the second international timetabling competition.)
		    \item Curriculum-Based Course Timetabling(CB-CTT) (see Di Gaspero et al., The second international timetabling competition (ITC-2007): Curriculum-based course timetabling (track 3).)
		    \item University Course Timetabling (UCTT) Babaei et al.
		\end{itemize}
	\end{itemize}
    \item Integer programming \& timetabling - so what?
	\begin{itemize}
	    \item Thus, it is natural to think of it as an IP problem.
	    \item (Perhaps not necessary) Some solutions through heuristics (Hidalgo-Herrero, 2013, Comparing problem solving strategies for NP-hard optimization problems)
	    \item Babaei et al., A survey of approaches for university course timetabling problem - great overview of the problem formulation for IP at the end of section 2.2.1
	    \item Burke et al, Decomposition, reformulation, and diving in university course timetabling (decomposes the problem and uses IP)
	\end{itemize}
\end{itemize}
 
\section*{Methodology}

\begin{itemize}
    \item What we aim to reach with our lecture?
	\begin{itemize}
	    \item Present the problem
	    \item Formulation for IP
	    \item Which challenges arise?
	    \item What is the limit? (What are the boundaries of computability? what are the parameters that make it impossible/very hard to compute?)
	    \item Variations
	    \item Overview of the state-of-the-art
	    \item comparison with other approaches
	\end{itemize}
    \item How?
	\begin{itemize}
	    \item State-of-the-art papers
	    \item Competition summary
	    \item Books on the subject?
	\end{itemize}
\end{itemize}

\section*{Expected Results}

\begin{itemize}
    \item Good understanding of the problem and its challenges
    \item Good understanding of the problem formulation and its nuances
    \item Broad understanding of the state-of-the-art solutions
\end{itemize}

\section*{References}

Practice and Theory of Automated Timetabling (PATAT) and the Multidisciplinary International Scheduling Conference: Theory and Application (MISTA).

\end{document}
