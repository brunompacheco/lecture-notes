\documentclass[a4paper]{report}
\input{./preamble.tex}
 
\begin{document}
 
\title{Integer Programming for Timetabling in Higher Education Institutions\\Project Proposal}
\author{Bruno M. Pacheco (16100865)\\
Pedro Kretzschmar (18100679) \\
DAS410049 - Programação Inteira}
 
\maketitle

\section*{Characterization}

Timetabling is the allocation of given resources to specific objects being placed in space-time, in such a way as to satisfy as nearly as possible a set of desirable objectives [1]. At the first moment, it may look simple to solve the timetabling problem, but the complexity of the problem is growing alongside the Higher Education Institutions (HEIs) [2].

HEIs timetabling has many formulations, but the two most "standard" ones are the Post Enrollment Course Timetabling (PE-CTT) and the Curriculum-Based Course Timetabling (CB-CTT) [3]. These two different approaches have their differences in the main constraints of the problem. The PE-CTT intends to simulate the situation where the students are given a choice of lectures that they wish to attend and the timetable is to be constructed after the students have selected which lectures they wish to attend [4]. The CB-CTT approach consists of the scheduling of the lectures for several university courses within a given number of rooms and periods. The conflicts between courses are set according to the curricula published by the University [5]. So, the source of conflicts in PE-CTT is the actual student enrolment whereas in CB-CTT the courses in conflict are those that belong to the same predefined group of courses, or curricula. However, this is only one of the differences. When in PE-CTT, the lectures can be seen as a self-standing event, penalizing late, consecutive, and isolated ones. In CB-CTT it's important to focus on trying to evenly spread the lectures of a course on the weekdays, preserve the same room, and focus on developing a good experience to each curriculum [3].

Feiring, in Linear Programming: An Introduction, had defined Linear Programming as a subset of mathematical programming where this aim has been followed by efficient assigning of limited resources to the specified activities to maximize the interest and minimize the cost. Limited resources in timetabling can be seen as rooms, time-periods, teachers' disponibility, rooms, equipment, and lecture requirements. Maximize interest is to allocate all lectures in a way that satisfies (subjective, we have to bring something more quantitative here, I think -PDK) the greatest amount of students and teachers, at the same time that lower the number of rooms used, the gap time between lectures, the number of time that a class changes room, and others.

\begin{itemize}
    \item What is timetabling? (definition) (Vrielink et al., Practices in timetabling in higher education institutions: a systematic review)
	\begin{itemize}
	    \item Wren 1995: (Scheduling, timetabling and rostering—a special relationship?) "The allocation of given resources to specific objects being placed in space time, in such way as to satisfy as nearly as possible a set of desirable objectives"
	    \item NP-hard or NP-complete problem (Even 1975, On the complexity of time table and multi-commodity flow problems)
	    \item HEI timetabling has many formulations, the two most "standard" being (via Bellio et al., Feature-based tuning of simulated annealing appliedto the curriculum-based course timetabling problem - great introduction)
		\begin{itemize}
		    \item Post-Enrollment Course Timetabling(PE-CTT) (see Lewis et al., Post enrolment based course timetabling: a description of the problem model used for track two of the second international timetabling competition.)
		    \item Curriculum-Based Course Timetabling(CB-CTT) (see Di Gaspero et al., The second international timetabling competition (ITC-2007): Curriculum-based course timetabling (track 3).)
		    \item University Course Timetabling (UCTT) Babaei et al.
		\end{itemize}
	\end{itemize}
    \item Integer programming \& timetabling - so what?
	\begin{itemize}
	    \item Thus, it is natural to think of it as an IP problem.
	    \item (Perhaps not necessary) Some solutions through heuristics (Hidalgo-Herrero, 2013, Comparing problem solving strategies for NP-hard optimization problems)
	    \item Babaei et al., A survey of approaches for university course timetabling problem - great overview of the problem formulation for IP at the end of section 2.2.1
	    \item Burke et al, Decomposition, reformulation, and diving in university course timetabling (decomposes the problem and uses IP)
	\end{itemize}
\end{itemize}
 
\section*{Methodology}

During the lecture, we will present the problem, showing the main features and particularities that make it important nowadays. After that, the formulation for IP of the timetabling problem will be demonstrated. During this moment, we'll elucidate how the addition of new variables and constraints can increase the difficulty of the problem reaches the limitations of the model. Alongside that, we'll bring the most popular variations of the subject, a brief overview of the state-of-the-art, and a comparison with other approaches.

\begin{itemize}
    \item During the lecture, we will:
	\begin{itemize}
	    \item Present the problem, showing .
	    \item Formulation for IP
	    \item Which challenges arise?
	    \item What is the limit? (What are the boundaries of computability? what are the parameters that make it impossible/very hard to compute?)
	    \item Variations
	    \item Overview of the state-of-the-art
	    \item comparison with other approaches
	\end{itemize}
    \item How?
	\begin{itemize}
	    \item State-of-the-art papers
	    \item Competition summary
	    \item Books on the subject?
	\end{itemize}
\end{itemize}

\section*{Expected Results}

\begin{itemize}
    \item Good understanding of the problem and its challenges
    \item Good understanding of the problem formulation and its nuances
    \item Broad understanding of the state-of-the-art solutions
\end{itemize}

\section*{References}

Practice and Theory of Automated Timetabling (PATAT) and the Multidisciplinary International Scheduling Conference: Theory and Application (MISTA). \\
1. Scheduling, timetabling and rostering - A special relationship? Wren, A. 1995. \\
2. Practices in timetablingin higher education institutions: a systematic review. Vrielink. 2019. \\
3. Feature-based tuning of simulated annealingappliedto the curriculum-based course timetabling problem - greatintroduction. Bellio. 2014. \\
4. Post enrolment based course timetabling: a description of the problem model used for track two of the second international timetabling competition. Lewis, R. 2007. \\
5. The second international timetabling competition (ITC-2007): Curriculum-based course timetabling (track 3). Di Gaspero, L. 2007. \\
6. A survey of approaches for university course timetabling problem. Babaei, H., Karimpour, J., & Hadidi, A. 2015. \\

\end{document}

