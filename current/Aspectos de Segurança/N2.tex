\documentclass[a4paper]{report}
\input{./preamble.tex}
 
\begin{document}
 
\title{N2}
\author{Bruno M. Pacheco\\
DAS 5401 - Aspectos de Segurança}
 
\maketitle
 
\exercise{1}

\subexercise{I}

\begin{figure}[H]
    \centering
    \includegraphics[width=0.8\textwidth]{N2_Q1.png}
    \caption{Curva de confiabilidade.}
    \label{fig:}
\end{figure}

\subexercise{II}

\[
    10000 \left( 10 R(1440)\right)  = 10000 \left( 1- e^{-10^{-5}1440}\right)  \cong 143 \text{ unidades}
\] 

\subexercise{III}

\[
    \frac{R(5760)}{R(4320)} = \frac{0,9440}{0,9577} \cong 98,57 \text{ \% de chance}
\] 

\exercise{2}

Sabemos que a probabilidade de falha de cada um dos motores pode ser modelada pelas funções
\begin{align*}
    F_1(t) &= 1 - e^{-10^{-3}t} \\
    F_2(t) &= 1 - e^{-10^{-4}t} \\
.\end{align*}

Assim, podemos calcular a confiabilidade do equipamento como \[
    F_s(t) = F_1(t)F_2(t) \implies R_s(t) = 1-F_s(t)
.\] 

Suponde que $\lambda_1$ e $\lambda_2$ são valores de falhas por hora, a probabilidade do equipamento funcionar por 6 meses é \[
    R_s(4320) \cong 65,39 \text{ \%}
.\] 

\exercise{3}

Podemos determinar a confiabilidade do equipamento como \[
    R(t) = e^{-10^{-5}t}
.\] Assim, \[
MTTF = \int_0^{\infty}R(t)dt = 10^5 \text{ horas}
.\] 

\[
MTBF = MTTF + MTTR = 101500 \text{ horas}
.\]  

Considerando \emph{up time} como MTBF e \emph{down time} como MTTR, temos a disponibilidade \[
A = \frac{MTBF}{MTBF + MTTR} = 98,54 \text{ \%}
.\] 

O número de unidades que irão falhar em 6 meses de uso é \[
    100 \left( 1 - R(6\cdot 30\cdot 24) \right) \cong 4 \text{ unidades.} 
\] 

\exercise{4}

Podemos modelar a probabilidade de falha do sistema como \[
    F(t) = (1 - e^{-\lambda t})^2
\] e a confiabilidade como \[
R(t) = 1- F(t)
.\] Assim, \[
R(600) \cong 99,66 \text{ \%}
,\] com \[
MTBF = \frac{1}{\lambda}\left( 1 + \frac{1}{2} \right) = 15000 \text{ horas}
.\] 



\end{document}
