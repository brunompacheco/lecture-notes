\documentclass[a4paper]{report}
\input{./preamble.tex}
 
\begin{document}
 
\title{Laboratório 2}
\author{Bruno M. Pacheco (16100865)\\
Pedro Y. F. Ceripes (18100681) \\
EEL 7550 - Eletrônica Aplicada}
 
\maketitle

\section*{Objetivo}
 
Os experimentos executados visam desenvolver familiaridade e experiência com o uso de amplificadores operacionais em montagem de circuitos comparadores e amplificadores.
 
\section*{Simulações}

\subsection*{Circuito 1}

Conforme requisitado, foi montado o circuito representado pela figura \ref{fig:figures-lab2-1-circuito-png}.

\begin{figure}[H]
    \centering
    \includegraphics[width=0.9\textwidth]{figures/lab2-1-circuito.png}
    \caption{Primeiro circuito do roteiro.}
    \label{fig:figures-lab2-1-circuito-png}
\end{figure}

\subsubsection*{a)}

O resultado da simulação pode ser visualizado na figura \ref{fig:figures-lab2-1-a-png}.

\begin{figure}[H]
    \centering
    \includegraphics[width=0.9\textwidth]{figures/lab2-1-a.png}
    \caption{Resultado da simulação do item a.}
    \label{fig:figures-lab2-1-a-png}
\end{figure}

\subsubsection*{b)}

O resultado da simulação após a alteração da tensão $V_x$, pode ser vista na figura \ref{fig:figures-lab2-1-b-png}.

\begin{figure}[H]
    \centering
    \includegraphics[width=0.9\textwidth]{figures/lab2-1-b.png}
    \caption{Resultado da simulação do item b.}
    \label{fig:figures-lab2-1-b-png}
\end{figure}

\subsubsection*{c)}

O circuito é denominado de comparador porque a saída ($CH2$) apresenta apenas dois valores possíveis: $+V_{cc}$ e 0. Nesse sentido, a saída segue as seguintes regras:
\begin{itemize}
    \item $CH2=0$, caso a tensão de entrada ($CH1$) seja menor que o valor de referência ($V_x$);
    \item $CH2=+V_{cc}$, caso a tensão de entrada ($CH1$) seja maior que o valor de referência ($V_x$);
    \item $CH2=0$, caso a tensão de entrada ($CH1$) seja maior que a tensão de alimentação ($+V_{cc}$).
\end{itemize}
\subsection*{Circuito 2}

Conforme instruído, o circuito foi montado como na figura \ref{fig:figures-lab2-2-circuito-png}.

\begin{figure}[H]
    \centering
    \includegraphics[width=0.8\textwidth]{figures/lab2-2-circuito.png}
    \caption{Segundo circuito do roteiro.}
    \label{fig:figures-lab2-2-circuito-png}
\end{figure}

\subsubsection*{a)}

O resultado da simulação pode ser verificado na figura \ref{fig:figures-lab2-2-a-png}. Conforme visível através do cursor, a tensão de saída respeita, aproximadamente, a relação teórica \[
V_{out}= \left( 1+\frac{3k3}{1k} \right)V_{in} = 8,6 V_p
\] 

\begin{figure}[H]
    \centering
    \includegraphics[width=0.8\textwidth]{figures/lab2-2-a.png}
    \caption{Formas de onda da tensão de entrada \emph{V(ch1)} e tensão de saída \emph{V(ch2)}.}
    \label{fig:figures-lab2-2-a-png}
\end{figure}

\subsubsection*{b)}

Para o valor de ganho desejado, precisamos que \[
\frac{R_2}{R_1} = 2 \implies R_2 = 2k\Omega
.\] Podemos ver no resultado, através da figura \ref{fig:figures-lab2-2-b-png}, que a relação teórica é uma boa aproximação.

\begin{figure}[H]
    \centering
    \includegraphics[width=0.8\textwidth]{figures/lab2-2-b.png}
    \caption{Formas de onda do segundo circuito após ajuste de $R_2$.}
    \label{fig:figures-lab2-2-b-png}
\end{figure}

\subsubsection*{c)}

Para determinar o valor real de saturação da saída, o resistor $R_2$ foi incrementado para que se tornasse visível uma saturação na forma de onda monitorada, conforme visível na figura \ref{fig:figures-lab2-2-c-png}. Pelo valor encontrado, podemos calcular o valor da resistência pela relação dos valores de pico \[
    V_{out} = 10,90 = \left( 1 +\frac{R_2}{1k}\right) 2 \implies R_2 = 4k45 \Omega
.\] 

\begin{figure}[H]
    \centering
    \includegraphics[width=0.8\textwidth]{figures/lab2-2-c.png}
    \caption{Saturação da saída do circuito.}
    \label{fig:figures-lab2-2-c-png}
\end{figure}

\subsubsection*{d)}

Com um sinal de entrada de $2 V_{pp}$ dificilmente o amplificador entraria em uma zona de não linearidade do ganho, então manteria a relação teórica entrada-saída verificada na parte a). Com isso, os resultados de b) se mantém, uma vez que o ganho deriva dessa relação. Finalmente, a resposta de c) provavelmente mudaria uma vez que a tensão de saída, para um mesmo ganho, seria muito menor, ou seja, temos uma margem maior de ganho sem que haja distorção do sinal de saída.

\section*{Conclusão}

Os resultados do circuito comparador apresentou as características desse arranjo, principalmente a alternação entre os dois níveis possíveis de saída a partir dos valores de referência, entrada e alimentação.

A simulação do arranjo como amplificador de sinal mostrou uma proximidade do componente simulado com o componente teórico quanto à relação entrada-saída, ainda que a faixa de saturação do sinal de saída seja muito menor do que no componente teórico.

\end{document}
