\documentclass[a4paper]{report}
\input{./preamble.tex}
 
\begin{document}
 
\title{Laboratório 9}
\author{Bruno M. Pacheco (16100865)\\
Pedro Y. F. Ceripes (18100681) \\
EEL 7550 - Eletrônica Aplicada}

\maketitle
\section*{Objetivo}

As experiências relatadas neste documento visam demonstrar as regiões de operação do transistor do tipo MOSFET. Além disso, as experiencias visaram polarizar o transistor para atingir a região de interesse.

\section*{Simulações}

\subsection*{Circuito 1}
\subsection*{a}

A porta G é aquela conectada ao resistor $R_G$, o dreno está conectado ao resistor $R_D$ e a fonte ao resistor $R_S$. As tensões são estabelecidas entre esses terminais, ou seja, $V_{GS}$ na malha da fonte $V_{GG}$ e do resistor $R_S$ enquanto $V_{DS}$ naquela do resistor $R_S$ e da fonte $V_{DD}$.

\subsection*{b}
\subsection*{c}
\subsection*{d}

\begin{table}[H]
    \centering
    \begin{tabular}{c | c | c | c | c | c}
	$V_{GG}$ [V]  & $I_G$ & $V_{GS}$ [V]  & $I_D$ [mA]  & $V_{DS}$ [V]  & Região \\
    \hline
    0 & 0 & 0 & 0 & 12 & Corte \\
    4 & 0 & 1,73 & 1,90 & 4,61 & Saturação \\
    8 & 0 & 4,31 & 3,07 & 0,01 & Triodo
    \end{tabular}
\end{table}

\section*{Circuito 2}

O circuito foi simulado conforme na figura abaixo.

\begin{figure}[H]
    \centering
    \includegraphics[width=0.6\textwidth]{figures/lab9_2.png}
\end{figure}

Mediu-se
\begin{align*}
    I_D &= 0,00100947 \\
    V_G &= 5 \\
    V_{GS} &= 1,66874
.\end{align*}

Dessa forma, estimamos \[
k_n \frac{W}{L} = \frac{2I_D}{\left( V_{GS}-V_{th} \right)^2 } \approx 0,214
.\] 


\section{Conclusão}

No primeiro circuito, foi possível detectar os 3 tipos de região de operação com os valores de tensão requisitados. Já no segundo circuito, foi estimado os parâmetros a partir dos valores simulados. 

\end{document}
