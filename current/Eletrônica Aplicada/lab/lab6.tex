\documentclass[a4paper]{report}
\input{./preamble.tex}
 
\begin{document}
 
\title{Laboratório 6}
\author{Bruno M. Pacheco (16100865)\\
Pedro Y. F. Ceripes (18100681) \\
EEL 7550 - Eletrônica Aplicada}

\maketitle
\section*{Objetivo}

Nesta experiência visamos adquirir familiaridade com a simulação de circuitos eletrônicos com transistores, utilizando o software LTspice para as análises.
 
\section*{Simulações}

\subsection*{Circuito 1}
\subsection*{a}

No circuito apresentado, a base do transistor NPN encontra-se conectada ao resistor $R_B$, o coletor ao resistor $R_C$ e o emissor ao terra.

\subsection*{b}

O circuito foi montado conforme a figura abaixo.

\begin{figure}[H]
    \centering
    \includegraphics[width=0.8\textwidth]{figures/lab6-1.png}
\end{figure}

Pela simulação, vemos que a junção BE do transistor está polarizada diretamente e a junção CB está polarizada de forma direta, i.e., a corrente amplificada pelo transistor não pode ser fornecida à carga pela fonte. Portanto, o transistor está saturado.

\subsection*{c}

Agora, vemos que, diferentemente da situação anterior, a junção CB está polarizada de forma reversa, além de uma queda de tensão significativa sobre o transistor (CE), ou seja, o transistor opera amplificando a corrente da base. Portanto, modo de operação ativa.

\subsection*{d}

Como a junção BE não está polarizada, i.e., não há corrente na base do transistor, ele opera em corte.

\subsection*{e}

Pela construção (teoricamente) simétrica do transistor, neste caso, ele opera em modo reverso. É exatamente a mesma configuração vista anteriormente no modo ativo mas invertendo-se o coletor com o emissor, ou seja, a junção BC está polarizada diretamente enquanto a junção BE está polarizada de forma reversa.

\begin{table}[H]
    \centering
    \caption*{Tabela 2}
    \begin{tabular}{c | c | c | c | c | c}
     & $I_B$ [$\mu$A] & $V_{BE}$ [V] & $I_C$ [mA] & $V_{CE}$ [V] & Região \\
     \hline
	b) & 287,7 & 0,69 & 3,3 & 0,04 & Saturação \\
	c) & 6,4 & 0,65 & 1,8 & 2,24 & Ativa \\
	d) & 0 & 0 & 0 & 5 & Corte \\
	e) & 289,7 & -1,3 & 2,3 & -1,97 & Corte \\
    \end{tabular}
\end{table}

\subsection*{f}

Pela Tabela 2, temos \[
\beta \approx \frac{1,8 \text{ mA}}{6,4\,\mu\text{A}} \approx 281
.\]

\subsection*{g}

Para a operação reversa \[
\beta_R \approx \frac{2,3 \text{ mA}}{289,7\,\mu\text{A}} \approx 8
.\] 

\section{Conclusão}

Os resultados dos circuitos com transistores apresentaram as características desse arranjo, considerando os 3 tipos de operação: em corte, operação ativa e saturado. Nesse sentido, todos os resultados representam os conceitos teóricos do componente eletrônico.
\end{document}
