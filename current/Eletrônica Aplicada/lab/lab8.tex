\documentclass[a4paper]{report}
\input{./preamble.tex}
 
\begin{document}
 
\title{Laboratório 8}
\author{Bruno M. Pacheco (16100865)\\
Pedro Y. F. Ceripes (18100681) \\
EEL 7550 - Eletrônica Aplicada}

\maketitle
\section*{Objetivo}

TODO

\section*{Simulações}

\subsection*{Circuito 1}
\subsection*{a}

Considerando valores de $V_{BE} = 0,66\,V$ e $\beta = 150$,

\begin{table}[H]
    \centering
    \begin{tabular}{c c}
    $V_{TH} = 4,8\,V$ & $I_C \approx 1,83\,mA$ \\
    $R_{TH} = 7k2\,\Omega$ & $I_E \approx 1,84\, mA $ \\
    $I_B = 12,20\,\mu A$ & $V_{CE} \approx 4,68\,V$
    \end{tabular}
\end{table}

\subsection*{b}

O circuito foi montado conforme a figura abaixo. Nota-se que os capacitores foram emitidos uma vez que se comportam como circuito aberto em regime permanente devido à natureza contínua dos sinais.

\begin{figure}[H]
    \centering
    \includegraphics[width=0.4\textwidth]{figures/lab8_1_b.png}
\end{figure}

Assim, os valores encontrados foram
\begin{table}[H]
    \centering
    \begin{tabular}{c | c | c | c | c | c}
	 & $I_B$ & $I_C$ & $I_E$ & $V_{BE}$ & $V_{CE}$ \\
	 \hline
	Teórico & $12,20\,\mu A$ & $1,83\,mA$ & $1,84\,mA$ & $0,66\,V$ & $4,68\,V$ \\
	Prático & $6,22\,\mu A$ & $1,86\,mA$ & $1,86\,mA$ & $0,65\,V$ & $4,55\,V$
    \end{tabular}
\end{table}

\subsection*{c}

\subsection*{d}

\begin{table}[H]
    \centering
    \begin{tabular}{c | c | c | c}
     & $r_\pi$ & $g_m$ & $A_v$ \\
     \hline
	Teórico & $2k13\,\Omega$ & $0,07$ & $-63,34$ \\
	Prático & $2k01\,\Omega$ & $0,07$ & $-55,80$
    \end{tabular}
\end{table}

\subsection*{e}

Notamos que a partir de $150\,mV$ o sinal de saída já apresenta deformações similares a um achatamento do semiciclo positivo, portanto, esse foi o valor utilizado. Nessa condição, $V_C$ possui máximo em $9,28\,V$ e mínimo em $7,71\,V$.

Entretanto, somente a partir de uma amplitude de $460\,mV$ o sinal apresenta ceifamento claro, visível no semiciclo negativo. Nesse cenário, $V_C$ oscila entre $9,96\,V$ e $4,47\,V$.

\section*{Circuito 2}

TODO

\section{Conclusão}

TODO

\end{document}
