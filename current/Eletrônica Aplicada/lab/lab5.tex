\documentclass[a4paper]{report}
\input{./preamble.tex}
 
\begin{document}
 
\title{Laboratório 4}
\author{Bruno M. Pacheco (16100865)\\
Pedro Y. F. Ceripes (18100681) \\
EEL 7550 - Eletrônica Aplicada}
 
\maketitle
\section*{Objetivo}
 
TODO
 
\section*{Simulações}

\subsection*{Circuito 1}
\subsection*{a}

Seguindo as instruções do roteiro, foi montado o circuito representado pela figura \ref{fig:figures-lab5-1-a-png} abaixo no software LTspice XII.

\begin{figure}[H]
    \centering
    \includegraphics[width=0.8\textwidth]{figures/lab5-1-a.PNG}
    \caption{Primeiro circuito configurado no LTSpice.}
    \label{fig:figures-lab5-1-a-png}
\end{figure}

\subsubsection*{b}

Considerando uma queda de tensão no LED de 2V, podemos calcular R como:

\begin{equation*}
    R = \frac{V_{fonte}-V_{D0}}{I_D} = \frac{5-2}{0.02} = 150 \Omega
\end{equation*}

\subsubsection*{c}

Definimos $R_x = 150 \Omega$, como esse é um valor comercial, podemos dizer que $R_1 = 150 \Omega$. Após isso simulamos o circuito e alcançamos os resultados abaixo.

\begin{table}[H]
    \centering
    \begin{tabular}{c | c}
	Medida & Resultado \\
	\hline
	$R_{x}$ & $150 \Omega$ \\
	$V_{D}$ & 1,94 V \\
	$I_{D}$ & 20,42 mA
\end{tabular}
\end{table}

\subsubsection*{d}
Foi dobrada a resistência utilizada na simulação e refeita a simulação. Dessa forma, foi obtido os resultados abaixo.

\begin{table}[H]
    \centering
    \begin{tabular}{c | c}
	Medida & Resultado \\
	\hline
	$R_{x}$ & $300 \Omega$ \\
	$V_{D}$ & 1,85 V \\
	$I_{D}$ & 10,49 mA
\end{tabular}
\end{table}

Analisando os resultado é possível perceber que ocorre uma queda pequena na tensão diodo e outra grande, cerca de 50\%, na tensão. Sendo assim, o aumento da resistência faz com que a potência do LED diminua, diminuindo também o brilho gerado por ele.

\subsection*{Circuito 2}
TODO

\subsection*{Circuito 3}
\subsubsection*{a}
Conforme instruído no roteiro, foi montado o circuito no software LTspice como é mostrado na figura \ref{fig:figures-lab5-3-a-png}.

\begin{figure}[H]
    \centering
    \includegraphics[width=0.8\textwidth]{figures/lab5-3-a.PNG}
    \caption{Terceiro circuito configurado no LTSpice.}
    \label{fig:figures-lab5-3-a-png}
\end{figure}

\subsubsection*{b}

Após isso, o circuito acima foi simulado e a figura \ref{fig:figures-lab5-3-b-png} mostra o sinal de entrada e de saída nas cores azul e verde, respectivamente.

\begin{figure}[H]
    \centering
    \includegraphics[width=0.8\textwidth]{figures/lab5-3-b.PNG}
    \caption{Simulação do terceiro circuito.}
    \label{fig:figures-lab5-3-b-png}
\end{figure}

\subsubsection*{c}

Utilizando as ferramentas do LTspice e a simulação apresentada no item anterior, foi possível preencher a tabela abaixo.

\begin{table}[H]
    \centering
    \begin{tabular}{c | c}
	Medida & Resultado \\
	\hline
	$V_{i,pp}$ & 20V \\
	$V_{o,pp}$ & 10,84 V \\
	$V_{o,min}$ & -5,42 V \\
	$V_{0,max}$ & 5,42
\end{tabular}
\end{table}

\section{Conclusão}

TODO

\end{document}