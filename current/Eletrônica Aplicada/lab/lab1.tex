\documentclass[a4paper]{report}
\input{./preamble.tex}
 
\begin{document}
 
\title{Laboratório 1}
\author{Bruno M. Pacheco (16100865)\\
NOME AQUI (MATRICULA) \\
EEL 7550 - Eletrônica Aplicada}
 
\maketitle

\section*{Objetivo}
 
O objetivo dos experimentos descritos neste relatório é desenvolver familiaridade com o software LTSPICE, utilizado para simular os circuitos propostos, através de circuitos resistivos simples alimentados a corrente contínua.
 
\section*{Simulações}

\subsection*{Circuito 1}

Conforme requisitado, o circuito foi montado no LTSPICE como visível na figura \ref{fig:ltspice-1}.

\begin{figure}[h]
    \centering
    \includegraphics[width=0.8\textwidth]{figures/lab1-ltspice1.png}
    \caption{Primeiro circuito do roteiro montado no LTSPICE.}
    \label{fig:ltspice-1}
\end{figure}

Os resultados dos cálculos e da simulação podem ser observados na tabela \ref{tab:circ-1}.

\begin{table}[h]
    \centering
    \caption{Tabela de comparação dos resultados teóricos e simulados.}
    \label{tab:circ-1}
    \begin{tabular}{c | c | c | c | c}
	$V_f$ & $I_{R1}$ (teoria) & $I_{R1}$ (simulação) & $I_{R2}$ (teoria) & $I_{R2}$ (simulação)  \\
	\hline
	5V & 0,5 mA & 0,5 mA & 0,185 mA & 0,185185 mA \\
	12V & 1,2 mA & 1,2 mA & 0,444 mA & 0,444444 mA \\
	24V & 2,4 mA & 2,4 mA & 0,889 mA & 0,888889 mA
    \end{tabular}
\end{table}

\subsection*{Circuito 2}

O circuito foi simulado também conforme na figura \ref{fig:figures-lab1-ltspice2-png}.

\begin{figure}[H]
    \centering
    \includegraphics[width=0.8\textwidth]{figures/lab1-ltspice2.png}
    \caption{Segundo circuito do roteiro.}
    \label{fig:figures-lab1-ltspice2-png}
\end{figure}

Os resultados obtidos podem ser observados na tabela \ref{tab:circ-2}.

\begin{table}[H]
    \centering
    \caption{Resultados da simulação do segundo circuito.}
    \label{tab:circ-2}
    \begin{tabular}{c | c | c}
     & Valor teórico & Valor simulado \\
     \hline
	$V_f$ & 12 V & 12 V \\
	$V_1$ & 1,24 V & 1,243 C \\
	$V_2$ & 10,76 V & 10,757 V \\
	$V_3$ & 4,30 V & 4,303 V \\
	$V_4$ & 6,45 V & 6,454 V \\
	$I_f$ & 0,829 mA & 0,829 mA \\
	$I_1$ & 0,829 mA & 0,829 mA \\
	$I_2$ & 0,398 mA & 0,398 mA \\
	$I_3$ & 0,430 mA & 0,430 mA \\
	$I_4$ & 0,430 mA & 0,430 mA \\
	$P_f$ & 10 mW & 9,944 mW \\
	$P_1$ & 1,028 mW & 1,030 mW \\
	$P_2$ & 4,282 mW & 4,286 mW \\
	$P_3$ & 1,849 mW & 1,851 mW \\
	$P_4$ & 2,773 mW & 2,777 mW \\
    \end{tabular}
\end{table}

\section*{Conclusão}

Conforme esperado, o uso de componentes ideais resultou em valores de simulação bastante próximos dos teóricos. Acredita-se que as pequenas diferenças presentes são oriundas de erros de arredondamento e/ou erros numéricos.

\end{document}
