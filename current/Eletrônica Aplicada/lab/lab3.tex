\documentclass[a4paper]{report}
\input{./preamble.tex}
 
\begin{document}
 
\title{Laboratório 3}
\author{Bruno M. Pacheco (16100865)\\
Pedro Y. F. Ceripes (18100681) \\
EEL 7550 - Eletrônica Aplicada}
 
\maketitle

\section*{Objetivo}
 
TODO
 
\section*{Simulações}

\subsection*{Circuito 1}

O circuito foi montado conforme visível na figura \ref{fig:figures-lab3-1-circuit-png}.

\begin{figure}[H]
    \centering
    \includegraphics[width=0.8\textwidth]{figures/lab3-1-circuit.png}
    \caption{Primeiro circuito como montado no LTSpice.}
    \label{fig:figures-lab3-1-circuit-png}
\end{figure}

\subsubsection*{a)}

figura \ref{fig:figures-lab3-1-a-measurements-png}.

\begin{figure}[H]
    \centering
    \includegraphics[width=0.8\textwidth]{figures/lab3-1-a-measurements.png}
    \caption{Medidas tomadas com $R_0=0 \Omega$.}
    \label{fig:figures-lab3-1-a-measurements-png}
\end{figure}

\subsubsection*{b)}


\begin{figure}[H]
    \centering
    \includegraphics[width=0.8\textwidth]{figures/lab3-1-b-measurements.png}
    \caption{Medidas tomadas com $R_0=9k1 \Omega$.}
    \label{fig:figures-lab3-1-a-measurements-png}
\end{figure}

\subsubsection*{c)}

\begin{figure}[H]
    \centering
    \includegraphics[width=0.8\textwidth]{figures/lab3-1-c-measurements.png}
    \caption{Medidas tomadas com $R_0=0 \Omega$ e $R_1 = 1 k\Omega$.}
    \label{fig:figures-lab3-1-a-measurements-png}
\end{figure}

\subsubsection*{d)}

Para determinar a resposta do circuito quanto à frequência de entrada, utilizamos o comando de simulação "AC Analysis", que faz uma varredura ao longo de uma faixa de frequência, levantando o ganho e a fase da saída em relação à entrada. Dessa forma, obtemos a resposta conforme na figura \ref{fig:figures-lab3-1-a-measurements-png}.

\begin{figure}[H]
    \centering
    \includegraphics[width=0.8\textwidth]{figures/lab3-1-d.png}
    \caption{Diagrama de Bode do circuito.}
    \label{fig:figures-lab3-1-a-measurements-png}
\end{figure}

Vemos que o circuito tem característica de filtrar altas frequências, portanto, se comporta como um filtro passa-baixa.

\subsection*{Circuito 3}

Destaca-se o uso do gerador de pulso em modo contínuo (sem definir quantidade de pulsos). Também, para garantir uma proximidade do sinal gerado a uma onda quadrada, foi necessário especificar o tempo de subida e descida de forma real, uma vez que deixando como 0 causava uma resposta mais lenta (onda trapezoidal). O circuito montado assim como os resultados podem ser vistos na figura \ref{fig:figures-lab3-3-png}.

\begin{figure}[H]
    \centering
    \includegraphics[width=0.8\textwidth]{figures/lab3-3.png}
    \caption{Circuito e resultado da simulação com as medidas necessárias para o cálculo do \emph{slew rate}.}
    \label{fig:figures-lab3-3-png}
\end{figure}

Dessa forma, chegamos em um valor de \emph{slew rate} de aproximadamente $0,32 \frac{V}{\mu s}$, próximo do valor mínimo indicado para a versão LM741A do componente.

\section*{Conclusão}

TODO

\end{document}
