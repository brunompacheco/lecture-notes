\documentclass[a4paper]{report}
\input{./preamble.tex}
 
\begin{document}

\title{Relatório 8}
\author{Bruno M. Pacheco\\
DAS 5142 - Sistemas Dinâmicos}
 
\maketitle

\exercise{E1}

\subexercise{a)}

\begin{figure}[H]
    \centering
    \includegraphics[width=0.8\textwidth]{figures/lab8_1a_state_1.png}
    \caption{Espaço de estados para $k_1 = k_2 = 1$.}
    \label{fig:figures-lab8_1a_state-png}
\end{figure}

\begin{figure}[H]
    \centering
    \includegraphics[width=0.8\textwidth]{figures/lab8_1a_time_1.png}
    \caption{Comportamento da saída e do sinal de controle para $k_1 = k_2 = 1$.}
    \label{fig:figures-lab8_1a_state-png}
\end{figure}

\begin{figure}[H]
    \centering
    \includegraphics[width=0.8\textwidth]{figures/lab8_1a_state_2.png}
    \caption{Espaço de estados para $k_1 = k_2 = 1$.}
    \label{fig:figures-lab8_1a_state-png}
\end{figure}

\begin{figure}[H]
    \centering
    \includegraphics[width=0.8\textwidth]{figures/lab8_1a_time_2.png}
    \caption{Comportamento da saída e do sinal de controle para $k_1 = 1$ e $k_2 = 2$.}
    \label{fig:figures-lab8_1a_state-png}
\end{figure}

Notamos que como o aumento de $k_2$ afeta a inclinação da reta, ele controla a velocidade de convergência para o ponto de equilíbrio.


\subexercise{b}

Temos \[
    \nabla \sigma = \begin{bmatrix} -k_1 & -k_2 \end{bmatrix} 
\] e 
\begin{align*}
    f^{+} = \begin{bmatrix} x_2 \\ 1 \end{bmatrix} \\
    f^{-} = \begin{bmatrix} x_2 \\ -1 \end{bmatrix} \\
,\end{align*}
portanto, a região de deslizamento se dá em
\begin{align*}
    L_f^{+} \sigma = -k_1x_2 -k_2 \implies L_f^{+}\sigma <0 \forall x_2> \frac{-k_2}{k_1}\\
    L_f^{-} \sigma = -k_1x_2 +k_2 \implies L_f^{-} > 0 \forall  x_2 < \frac{k_2}{k_1} \\
    \implies \left( L_f^{+} \sigma \right) \left( L_f^{-} \sigma \right) \le 0 \forall x_2 \in \left\{ -\frac{k_2}{k_1}, \frac{k_2}{k_1} \right\} 
.\end{align*}

\subexercise{c}

\begin{figure}[H]
    \centering
    \includegraphics[width=0.8\textwidth]{figures/lab8_1c_state_1.png}
    \caption{Espaço de estados para $k_1 = k_2 = 1$.}
    \label{fig:figures-lab8_1c_state-png}
\end{figure}

\begin{figure}[H]
    \centering
    \includegraphics[width=0.8\textwidth]{figures/lab8_1c_time_1.png}
    \caption{Comportamento da saída e do sinal de controle para $k_1 = k_2 = 1$.}
    \label{fig:figures-lab8_1c_state-png}
\end{figure}

\begin{figure}[H]
    \centering
    \includegraphics[width=0.8\textwidth]{figures/lab8_1c_state_2.png}
    \caption{Espaço de estados para $k_1 = 1$ e $k_2 = 2$.}
    \label{fig:figures-lab8_1c_state-png}
\end{figure}

\begin{figure}[H]
    \centering
    \includegraphics[width=0.8\textwidth]{figures/lab8_1c_time_2.png}
    \caption{Comportamento da saída e do sinal de controle para $k_1 = 1$ e $k_2 = 2$.}
    \label{fig:figures-lab8_1c_state-png}
\end{figure}

Com o relê com histerese (de $\pm 0,2$), vemos uma diminuição da frequência de atuação do sinal de controle mas uma oscilação também na saída. Ambas essas características são influenciadas pelo ganho $k_2$ como visível nas figuras.

\subexercise{d}

\begin{figure}[H]
    \centering
    \includegraphics[width=0.8\textwidth]{figures/lab8_1d_state.png}
    \caption{Trajetória dos estados do sistema com função regularizadora no lugar da histerese.}
    \label{fig:figures-lab8_1d_state-png}
\end{figure}

\begin{figure}[H]
    \centering
    \includegraphics[width=0.8\textwidth]{figures/lab8_1d_time.png}
    \caption{Resposta no tempo do sistema com função regularizadora no lugar da histerese.}
    \label{fig:figures-lab8_1d_state-png}
\end{figure}

\end{document}
