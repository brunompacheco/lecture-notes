\documentclass[a4paper]{report}
\input{./preamble.tex}
 
\begin{document}
 
\title{Relatório 2}
\author{Bruno M. Pacheco\\
DAS 5142 - Sistemas Dinâmicos}
 
\maketitle
 
\exercise{}

\subsubsection*{a)}

Para o primeiro sistema, podemos encontrar os pontos de equilíbrio igualando $\bm{\dot{x}}$ a zero, ou seja
\begin{align*}
    \dot{x}_1 = 0 \implies \overline{x}_1 = \pm\sqrt{\mu} \\
    \dot{x}_2 = 0 \implies \overline{x}_2 = 0 \\
\end{align*}
, assim, os pontos do equilíbrio do sistema dependem do parâmetro $\mu$ e têm forma $(\pm\sqrt{\mu} ,0)$. Como não consideramos soluções imaginárias, o sistema só possui equilíbrio para $\mu\ge 0$.

Em relação à estabilidade no ponto de equilíbrio, temos que a matriz jacobiana do sistema é \[
    \begin{bmatrix} -2x_1 & 0 \\ 0 & -1  \end{bmatrix} 
\] e possui autovalores $-1$ e $\pm 2\sqrt{\mu}$. Assim, concluímos que o ramo superior dos pontos de equilíbrio, aquele da forma $\sqrt{\mu}$ é estável, enquanto o ramo inferior, de forma $-\sqrt{\mu}$ é instável e um ponto de sela.

Para $\mu = 0$, temos um autovalor no semiplano esquerdo e um autovalor nulo, o que nos indica que esta é uma bifurcação sela-nó.

\subsubsection*{b)}

Podemos observar o comportamento dos ramos do ponto de equilíbrio na figura \ref{fig:figures-lab2_1_pitchfork-pdf}.

\begin{figure}[H]
    \centering
    \includegraphics[width=0.6\textwidth]{figures/lab2_1_pitchfork.png}
    \caption{Diagrama de variação dos equilíbrios do sistema 1 em função do parâmetro $\mu$. Em azul, o ramo estável e em vermelho o ramo instável.}

    \label{fig:figures-lab2_1_pitchfork-pdf}
\end{figure}

\subsubsection*{c)}

Utilizando o aplicativo \emph{pplane8}, simulou-se o sistema para 3 combinações do parâmetro $\mu$. Os resultados podem ser observados na figura \ref{fig:pplane-1}.

\begin{figure}[H]
    \centering
    \begin{subfigure}{0.29\textwidth}
	\includegraphics[width=\textwidth]{figures/lab2_1_pplane_inst.png}
	\caption{$\mu=-0,5$}
    \end{subfigure}
    \begin{subfigure}{0.29\textwidth}
	\includegraphics[width=\textwidth]{figures/lab2_1_pplane_0.png}
	\caption{$\mu=0$}
    \end{subfigure}
    \begin{subfigure}{0.29\textwidth}
	\includegraphics[width=\textwidth]{figures/lab2_1_pplane_st.png}
	\caption{$\mu=0,5$}
    \end{subfigure}
    \caption{Espaço de estados do modelo do primeiro sistema para diferentes parâmetros. Pontos de equilíbrio são representados por pontos vermelhos e as trajetórias dos estados são os traços azuis.}
    \label{fig:pplane-1}
\end{figure}

\exercise{}

\subsubsection*{a)}

Para o segundo sistema, identificou-se duas possíveis formas para o ponto de equilíbrio:
\begin{align*}
    \bm{\overline{x}}_1 &= \begin{bmatrix} 0 \\ 0 \end{bmatrix} \\
    \bm{\overline{x}}_2 &= \begin{bmatrix} \mu \\ 0 \end{bmatrix} 
.\end{align*}

Para determinar a estabilidade, analisamos a jacobiana do sistema, ou seja, a matriz \[
    \begin{bmatrix} \mu-2x_1 & 0 \\ 0 & -1 \end{bmatrix} 
\]. Temos que, para os pontos $\bm{\overline{x}}_1$, a matriz possui autovalores \[
\bm{\lambda} = \begin{bmatrix}   -1 \\ \mu \end{bmatrix}
\] sendo, portanto, estável somente para $\mu<0$. Agora para os pontos $\bm{\overline{x}}_2$, os autovalores são \[
\bm{\lambda} = \begin{bmatrix}   -1 \\ -\mu \end{bmatrix}
\], o que nos indica que esses pontos são estáveis somente para $\mu>0$.

Para o caso de $\mu=0$, têm-se um autovalor nulo, portanto não podemos concluir nada.

\subsubsection*{b)}

Pode-se observar o comportamento dos ramos do ponto de equilíbrio na figura \ref{fig:figures-lab2_2_pitchfork-png}.

\begin{figure}[H]
    \centering
    \includegraphics[width=0.6\textwidth]{figures/lab2_2_pitchfork.png}
    \caption{Diagrama de variação dos equilíbrios do sistema 2 em função do parâmetro $\mu$. Em azul, o ramo estável e em vermelho o ramo instável.}
    \label{fig:figures-lab2_2_pitchfork-png}
\end{figure}

\subsubsection*{c)}

As simulações através do aplicativo \emph{pplane8} podem ser observadas na figura \ref{fig:pplane-2}. Verifica-se o encontrado de forma analítica.

\begin{figure}[H]
    \centering
    \begin{subfigure}{0.29\textwidth}
	\includegraphics[width=\textwidth]{figures/lab2_2_pplane_neg.png}
	\caption{$\mu=-0,5$}
    \end{subfigure}
    \begin{subfigure}{0.29\textwidth}
	\includegraphics[width=\textwidth]{figures/lab2_2_pplane_0.png}
	\caption{$\mu=0$}
    \end{subfigure}
    \begin{subfigure}{0.29\textwidth}
	\includegraphics[width=\textwidth]{figures/lab2_2_pplane_pos.png}
	\caption{$\mu=0,5$}
    \end{subfigure}
    \caption{Espaço de estados do modelo do segundo sistema para diferentes parâmetros. Pontos de equilíbrio são representados por pontos vermelhos e as trajetórias dos estados são os traços azuis.}
    \label{fig:pplane-2}
\end{figure}

\exercise{}

\subsubsection*{a)}

Para o terceiro sistema, os pontos de equilíbrio têm três possíveis formas:
\begin{align*}
    \bm{\overline{x}}_1 &= \begin{bmatrix} \sqrt{\mu} \\ 0 \end{bmatrix} \\
    \bm{\overline{x}}_2 &= \begin{bmatrix} -\sqrt{\mu} \\ 0 \end{bmatrix} \\
    \bm{\overline{x}}_3 &= \begin{bmatrix} 0 \\ 0 \end{bmatrix} \\
.\end{align*}
Nota-se que os dois primeiros só são válidos para $\mu>0$, uma vez que pontos de equilíbrio complexos não são válidos.

A estabilidade desses pode ser determinada através da matriz jacobiana \[
    \begin{bmatrix} \mu-3x_1^2 & 0 \\ 0 & -1 \end{bmatrix} 
\]. Os respectivos autovalores são
\begin{align*}
    \bm{\lambda}_1 &= \begin{bmatrix}   -1 \\ -2\mu \end{bmatrix} \\
    \bm{\lambda}_2 &= \begin{bmatrix}   -1 \\ -2\mu \end{bmatrix} \\
    \bm{\lambda}_3 &= \begin{bmatrix}   -1 \\ \mu \end{bmatrix} \\
.\end{align*}
Concluímos, então, que os ramos referentes a $\bm{\overline{x}}_1$ e $\bm{\overline{x}}_2$ são estáveis para $\mu>0$, enquanto o ramo associado a $\bm{\overline{x}}_3$ é estável para $\mu<0$, o que resulta em uma bifurcação do tipo tridente supercrítica.

\subsubsection*{b)}

O diagrama de variação dos equilíbrio pode ser observado na figura \ref{fig:figures-lab2_3_pitchfork-png}.

\begin{figure}[H]
    \centering
    \includegraphics[width=0.6\textwidth]{figures/lab2_3_pitchfork.png}
    \caption{Diagrama de variação dos equilíbrios do sistema 3 em função do parâmetro $\mu$. Em azul, o ramo estável e em vermelho o ramo instável.}
    \label{fig:figures-lab2_3_pitchfork-png}
\end{figure}

\subsubsection*{c)}

A simulação do sistema no aplicativo \emph{pplane8} pode ser observado na figura \ref{fig:pplane-3}.

\begin{figure}[H]
    \centering
    \begin{subfigure}{0.29\textwidth}
	\includegraphics[width=\textwidth]{figures/lab2_3_pplane_neg.png}
	\caption{$\mu=-0,5$}
    \end{subfigure}
    \begin{subfigure}{0.29\textwidth}
	\includegraphics[width=\textwidth]{figures/lab2_3_pplane_0.png}
	\caption{$\mu=0$}
    \end{subfigure}
    \begin{subfigure}{0.29\textwidth}
	\includegraphics[width=\textwidth]{figures/lab2_3_pplane_pos.png}
	\caption{$\mu=0,5$}
    \end{subfigure}
    \caption{Espaço de estados do modelo do terceiro sistema para diferentes parâmetros. Pontos de equilíbrio são representados por pontos vermelhos e as trajetórias dos estados são os traços azuis.}
    \label{fig:pplane-3}
\end{figure}

\exercise{}

\subsubsection*{a)}

Para o sistema 4, os pontos de equilíbrio têm três possíveis formas:
\begin{align*}
    \bm{\overline{x}}_1 &= \begin{bmatrix} \sqrt{-\mu} \\ 0 \end{bmatrix} \\
    \bm{\overline{x}}_2 &= \begin{bmatrix} -\sqrt{-\mu} \\ 0 \end{bmatrix} \\
    \bm{\overline{x}}_3 &= \begin{bmatrix} 0 \\ 0 \end{bmatrix} \\
.\end{align*}
Nota-se que os dois primeiros só são válidos para $\mu<0$, uma vez que pontos de equilíbrio complexos não são válidos.

A estabilidade desses pode ser determinada através da matriz jacobiana \[
    \begin{bmatrix} \mu+3x_1^2 & 0 \\ 0 & -1 \end{bmatrix} 
\]. Os respectivos autovalores são
\begin{align*}
    \bm{\lambda}_1 &= \begin{bmatrix}   -1 \\ -2\mu \end{bmatrix} \\
    \bm{\lambda}_2 &= \begin{bmatrix}   -1 \\ -2\mu \end{bmatrix} \\
    \bm{\lambda}_3 &= \begin{bmatrix}   -1 \\ \mu \end{bmatrix} \\
.\end{align*}
Concluímos, então, que os ramos referentes a $\bm{\overline{x}}_1$ e $\bm{\overline{x}}_2$ são instáveis enquanto o ramo associado a $\bm{\overline{x}}_3$ é estável para $\mu<0$, o que resulta em uma bifurcação subcrítica.

\subsubsection*{b)}

O diagrama de variação dos equilíbrio pode ser observado na figura \ref{fig:figures-lab2_4_pitchfork-png}.

\begin{figure}[H]
    \centering
    \includegraphics[width=0.6\textwidth]{figures/lab2_4_pitchfork.png}
    \caption{Diagrama de variação dos equilíbrios do sistema 4 em função do parâmetro $\mu$. Em azul, o ramo estável e em vermelho o ramo instável.}
    \label{fig:figures-lab2_4_pitchfork-png}
\end{figure}

\subsubsection*{c)}

A simulação do sistema no aplicativo \emph{pplane8} pode ser observado na figura \ref{fig:pplane-4}.

\begin{figure}[H]
    \centering
    \begin{subfigure}{0.29\textwidth}
	\includegraphics[width=\textwidth]{figures/lab2_4_pplane_neg.png}
	\caption{$\mu=-0,5$}
    \end{subfigure}
    \begin{subfigure}{0.29\textwidth}
	\includegraphics[width=\textwidth]{figures/lab2_4_pplane_0.png}
	\caption{$\mu=0$}
    \end{subfigure}
    \begin{subfigure}{0.29\textwidth}
	\includegraphics[width=\textwidth]{figures/lab2_4_pplane_pos.png}
	\caption{$\mu=0,5$}
    \end{subfigure}
    \caption{Espaço de estados do modelo do quarto sistema para diferentes parâmetros. Pontos de equilíbrio são representados por pontos vermelhos e as trajetórias dos estados são os traços azuis.}
    \label{fig:pplane-4}
\end{figure}

\end{document}
