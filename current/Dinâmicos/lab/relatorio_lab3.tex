\documentclass[a4paper]{report}
\input{./preamble.tex}
 
\begin{document}
 
\title{Relatório 3}
\author{Bruno M. Pacheco\\
DAS 5142 - Sistemas Dinâmicos}
 
\maketitle
 
\exercise{E1}

\subsubsection*{a)}

O lugar das raízes do sistema proposto, ignorando a saturação, pode ser observado na figura \ref{fig:figures-lab3_1_rlocus-pdf}.

\begin{figure}[H]
    \centering
    \includegraphics[width=0.6\textwidth]{figures/lab3_1_rlocus.png}
    \caption{Lugar das raízes do sistema, ignorando a saturação.}
    \label{fig:figures-lab3_1_rlocus-pdf}
\end{figure}

Nota-se que, para $k>0$, o sistema é estável. Além disso, para o ganho proposto, tem-se um par de polos complexos conjugados.

\subsubsection*{b)}

Utilizou-se a construção visível na figura \ref{fig:figures-lab3_1_simulink_model-png} no \emph{Simulink} para simular o comportamento do sistema com saturação.

\begin{figure}[H]
    \centering
    \includegraphics[width=0.6\textwidth]{figures/lab3_1_simulink_model.png}
    \caption{Modelo do sistema proposto no software \emph{Simulink}.}
    \label{fig:figures-lab3_1_simulink_model-png}
\end{figure}

A simulação foi executada aplicando um degrau na referência no instante inicial e observando os sinais de controle (após saturação) e de saída. Os resultados das simulações para os diferentes valores de referência podem ser observados na figura \ref{fig:figures-lab3_1_saturation-png}.

\begin{figure}[H]
    \centering
    \includegraphics[width=0.8\textwidth]{figures/lab3_1_saturation.png}
    \caption{Resposta do sistema a diferentes sinais tipo degrau na referência.}
    \label{fig:figures-lab3_1_saturation-png}
\end{figure}

\subsubsection*{c)}

Pelo modelo de variáveis de estado fornecido, sabemos que os dois estados são 
\begin{align*}
    x_1(t) = \int x_2(t) \\
    x_2(t) = \int u(t)
\end{align*}
, ou seja, podemos construir o sistema da forma visível na figura \ref{fig:figures-lab3_1_state_space_model-png}, utilizando os pontos destacados para observar a trajetória dos estados do sistema.

\begin{figure}[H]
    \centering
    \includegraphics[width=0.6\textwidth]{figures/lab3_1_state_space_model.png}
    \caption{Modelo do sistema por variáveis de estado.}
    \label{fig:figures-lab3_1_state_space_model-png}
\end{figure}

Assim, o comportamento do sistema pode ser observada na figura \ref{fig:figures-lab3_1_state_space_response-png}.

\begin{figure}[H]
    \centering
    \includegraphics[width=0.8\textwidth]{figures/lab3_1_state_space_response.png}
    \caption{Trajetória dos estados para diferentes condições iniciais, utilizando uma referência degrau $y_r = 6$.}
    \label{fig:figures-lab3_1_state_space_response-png}
\end{figure}

\subsubsection*{d)}

Pode-se observar nas trajetórias dos estados o que já se esperava pelo lugar das raízes: o sistema se torna estável com o ganho estabelecido. Quanto a convergência, observa-se que, conforme a referência se distancia da origem, o sinal de controle se torna mais oscilatório, atingindo mais e mais a região de saturação, enquanto a saída caracteriza-se por um \emph{settling time} maior e mais oscilação. A resposta foi conforme o esperado, uma vez que a saturação impede o sistema de receber o sinal de controle necessário para se deslocar rapidamente para o ponto de referência, ao mesmo tempo que também impede uma resposta que possa estabilizar o sistema uma vez que uma grande "velocidade" de deslocamento dos estados é atingida. Uma forma de modelar essa intuição da saturação é através de um ganho fracionário inversamente proporcional à amplitude do sinal de referência, ou seja, quanto maior o sinal de referência é em relação à condição inicial, maior o impacto da saturação na redução do ganho.

\exercise{E2}

\subsubsection*{a)}

Pela trajetória das raízes visível na figura \ref{fig:figures-lab3_2_rlocus-pdf}, vê-se que para ganho aproximadamente menores que $0,5$, o sistema apresenta polos com parte real positiva, ou seja, é instável.

\begin{figure}[H]
    \centering
    \includegraphics[width=0.6\textwidth]{figures/lab3_2_rlocus.png}
    \caption{Lugar das raízes do sistema.}
    \label{fig:figures-lab3_2_rlocus-pdf}
\end{figure}

\subsubsection*{b)}

A resposta do sistema e o comportamento do sinal de controle pode ser observado na figura \ref{fig:figures-lab3_2_saturation-png}.

\begin{figure}[H]
    \centering
    \includegraphics[width=0.8\textwidth]{figures/lab3_2_saturation.png}
    \caption{Resposta do sistema a diferentes sinais tipo degrau na referência.}
    \label{fig:figures-lab3_2_saturation-png}
\end{figure}

\subsubsection*{c)}

Para modelar o sistema proposto por variáveis de estado, selecionou-se as seguintes variáveis:
\begin{align*}
    x_1 = \int x_2 \\
    x_2 = \int x_3 \\
    x_3 = \int u \\
    y = x_1 + 2x_2 + x_3 \\
.\end{align*}
Nota-se que, em regime permanente, $y(t)\to x_1(t)$. Assim, a realização desse sistema pode ser observada na figura \ref{fig:figures-lab3_2_simulink_model-png}.

\begin{figure}[H]
    \centering
    \includegraphics[width=0.6\textwidth]{figures/lab3_2_simulink_model.png}
    \caption{Modelo do sistema por variáveis de estado no software \emph{Simulink}.}
    \label{fig:figures-lab3_2_simulink_model-png}
\end{figure}

Com o modelo realizado, o sistema foi simulado com condições iniciais nulas e os valores de referência propostos. As trajetórias podem ser observadas na figura \ref{fig:figures-lab3_2_state_space_response-png}.

\begin{figure}[H]
    \centering
    \includegraphics[width=0.8\textwidth]{figures/lab3_2_state_space_response.png}
    \caption{Trajetória dos estados para diferentes referências, com condição inicial nula.}
    \label{fig:figures-lab3_2_state_space_response-png}
\end{figure}

\subsubsection*{d)}

Sabe-se que a saturação pode ser aproximada por um ganho fracionário em função da amplitude da diferença entre a condição inicial e a referência, ou seja, para grandes amplitudes na referência, tem-se um ganho do controlador cada vez menor. Pode-se demonstrar isso através da resposta do sistema para uma condição inicial diferente, mais próxima da referência, na figura \ref{fig:figures-lab3_2_ci_3-png}.

\begin{figure}[H]
    \centering
    \includegraphics[width=0.8\textwidth]{figures/lab3_2_ci_3.png}
    \caption{Simulação do sistema com $y_r=3.5$ e $x_{1_0}=3$.}
    \label{fig:figures-lab3_2_ci_3-png}
\end{figure}

Pelo observado através do lugar das raízes, tem-se, na prática, uma aproximação dos polos do sistema da zona instável através da redução do ganho causada pelo aumento da amplitude da referência em conjunto com a saturação.

\exercise{E3}

\subsubsection*{a)}

Observando a trajetória dos polos do sistema, visível na figura \ref{fig:figures-lab3_3_rlocus-png}, é possível afirmar que um ganho aproximadamente superior à 0,2 leva o sistema à instabilidade, com dois polos complexos conjugados no semiplano direito.

\begin{figure}[H]
    \centering
    \includegraphics[width=0.8\textwidth]{figures/lab3_3_rlocus.png}
    \caption{Lugar das raízes do sistema.}
    \label{fig:figures-lab3_3_rlocus-png}
\end{figure}

\subsubsection*{b)}

Através do software \emph{Simulink}, simulou-se a resposta do sistema para diferentes referências. O resultado pode ser observado na figura \ref{fig:figures-lab3_3_saturation-png}.

\begin{figure}[H]
    \centering
    \includegraphics[width=0.8\textwidth]{figures/lab3_3_saturation.png}
    \caption{Resposta do sistema a diferentes sinais tipo degrau na referência.}
    \label{fig:figures-lab3_3_saturation-png}
\end{figure}

\subsubsection*{c)}

Para converter o sistema para um modelo em variáveis de estado, primeiro determinou-se a expressão da saída no domínio do tempo, \[
    \dddot{y}(t) + 0,2\ddot{y}(t) + \dot{y}(t) = u(t)
\] e escolheu-se os estados como \[
\begin{cases}
    \bm{x}(t) = \begin{bmatrix} y(t) \\ \dot{y}(t) \\ \ddot{y}(t) \end{bmatrix} 
\end{cases}
\], construindo o sistema, então, na forma canônica controlável da seguinte forma: \[
\begin{cases}
    \dot{x}_1(t) = x_2(t) \\
    \dot{x}_2(t) = x_3(t) \\
    \dot{x}_3(t) = -0,2x_3(t) - x_2(t) +u(t)
\end{cases}
\].

Esse sistema foi realizado conforme visível na figura \ref{fig:figures-lab3_3_state_space_model-png}.

\begin{figure}[H]
    \centering
    \includegraphics[width=0.6\textwidth]{figures/lab3_3_state_space_model.png}
    \caption{Modelo do sistema por variáveis de estado.}
    \label{fig:figures-lab3_3_state_space_model-png}
\end{figure}

A resposta do sistema simulado pode ser observada na figura \ref{fig:figures-lab3_3_state_space_response-png}. Notamos que a resposta oscilatória do sistema é presente não só na saída ($x_1$), como também nas outras variáveis de estado. Também é possível observar a existência de um ciclo limite estável em torno do ponto de equilíbrio desejado.

\begin{figure}[H]
    \centering
    \includegraphics[width=0.8\textwidth]{figures/lab3_3_state_space_response.png}
    \caption{Trajetória dos estados para $y_r=4$, com condição inicial nula.}
    \label{fig:figures-lab3_3_state_space_response-png}
\end{figure}

\subsubsection*{d)}

Projetou-se, então, um filtro $N(s)$ de tal forma que possua dois zeros próximos aos polos conjugados originais do sistema, atraindo os polos para o SPE, e dois polos rápidos para não alterar a dinâmica do sistema. Projetou-se o filtro da seguinte forma: \[
    N(s) = \frac{200(s^{2}+0.2s+0.5}{(s+10)^2}
\].

Confirmou-se a eficácia do filtro através da simulação. A resposta do sistema pode ser observada na figura \ref{}.

\begin{figure}[H]
    \centering
    \includegraphics[width=0.8\textwidth]{figures/lab3_3_filter.png}
    \caption{Resposta do sistema com filtro.}
    \label{fig:figures-lab3_3_filter-png}
\end{figure}

\end{document}
