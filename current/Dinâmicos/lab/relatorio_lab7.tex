\documentclass[a4paper]{report}
\input{./preamble.tex}
 
\begin{document}

\title{Relatório 7}
\author{Bruno M. Pacheco\\
DAS 5142 - Sistemas Dinâmicos}
 
\maketitle

\exercise{E1}

\subexercise{a)}

Podemos modelar os sistema proposto através da seguinte equação:
\begin{align*}
    & \dot{h}(t) = \frac{500}{\pi\left( Dh - h^2 \right) }u(t) - \frac{A_2\sqrt{2gh} }{\pi\left( Dh - h^2 \right) } \\
    & 0\le h(t)\le 100
.\end{align*}

Linearizando esse sistema em torno de um ponto de operação $\overline{h}$, temos \[
h(t) = \overline{h} + \Delta h(t)
,\] ou seja, reduzimos o sistema à sua variação linear $\Delta h(t)$ em torno do ponto de equilíbrio. Encontramos $\overline{u}$ sabendo que, para um dado ponto de operação $\overline{h}$, é verdade que \[
\frac{d h(t)}{dt}\Bigr|_{(\overline{h},\overline{u})} = 0
.\] Para a dinâmica do modelo, vemos que
\begin{align*}
    \dot{h}(t) &= \Delta \dot{h}(t) = \frac{\delta \Delta \dot{h}}{\delta h}\Bigr|_{(\overline{h},\overline{u})}\Delta h(t) + \frac{\delta \Delta \dot{h}}{\delta u}\Bigr|_{(\overline{h},\overline{u})}\Delta u(t) \\
    a &= \frac{\delta \Delta \dot{h}}{\delta h}\Bigr|_{(\overline{h},\overline{u})} = \frac{-A_2g}{\pi\left( D\overline{h}- \overline{h}^2 \right)\sqrt{2g\overline{h}}}\\
    b &= \frac{\delta \Delta \dot{h}}{\delta u}\Bigr|_{(\overline{h},\overline{u})} = \frac{500}{\pi\left( D \overline{h}- \overline{h}^2 \right) } \\
.\end{align*}

Assim, vemos que o sistema resultante pode ser modelado no domínio da frequência complexa por \[
    \frac{\Delta H(s)}{U(s)} = \frac{b}{s-a} = \frac{k}{\tau s + 1}
,\] onde $k=\frac{-b}{a}$ e $\tau = \frac{-1}{a}$.

Essa linearização foi implementada no \emph{Simulink} conforme pode ser visto na figura \ref{fig:figures-lab7_simulink_MA-png}.

\begin{figure}[H]
    \centering
    \includegraphics[width=0.8\textwidth]{figures/lab7_simulink_MA.png}
    \caption{Implementação do sistema linearizado e não linear no \emph{Simulink}.}
    \label{fig:figures-lab7_simulink_MA-png}
\end{figure}

[ADICIONAR IMPLEMENTAÇÃO DO CONTROLADOR] (destacar tuning do ganho de anti-windup)

Para o projeto do controlador nos pontos de operação, utilizou-se a ferramenta \emph{rltool} do Matlab. Assim, chegou-se aos valores do controlador conforme a tabela \ref{tab:ganhos-PI}.

\begin{table}[H]
    \centering
    \caption{Ganhos dos controladores PI projetados para os pontos de operação $\overline{h}$.}
    \label{tab:ganhos-PI}
    \begin{tabular}{c | c | c}
	$\overline{h}$ & $K_c$ & $K_i$ \\
	\hline 
	10 & 2,2129 & 0,2154 \\
	50 & 6,7742 & 0,1770 \\
	90 & 2,0389 & 0,1887
    \end{tabular}
\end{table}

O controlador com esses ganhos foi implementado no \emph{simulink}. A implementação do sistema em malha fechada pode ser observada na figura \ref{fig:figures-lab7_simulink_MF-png}. Nota-se que ambos os sistemas original e linearizado foram implementados.

\begin{figure}[H]
    \centering
    \includegraphics[width=0.8\textwidth]{figures/lab7_simulink_MF.png}
    \caption{Implementação do sistema em malha fechada com o controlador PI.}
    \label{fig:figures-lab7_simulink_MF-png}
\end{figure}

Simulou-se a resposta de ambos os sistemas implementados à degraus entre os pontos de operação. O resultado pode ser observado na figura \ref{fig:figures-lab4_1_resposta_simulink}.

\begin{figure}[H]
    \centering
    \begin{subfigure}{0.32\textwidth}
	\includegraphics[width=\textwidth]{figures/lab7_PI_10.png}
	\caption{$\overline{h} = 10$.}
    \end{subfigure}
    \begin{subfigure}{0.32\textwidth}
	\includegraphics[width=\textwidth]{figures/lab7_PI_50.png}
	\caption{$\overline{h} = 50$.}
    \end{subfigure}
    \begin{subfigure}{0.32\textwidth}
	\includegraphics[width=\textwidth]{figures/lab7_PI_90.png}
	\caption{$\overline{h} = 90$.}
    \end{subfigure}
    \caption{Resposta dos sistemas em malha fechada à troca de ponto de operação.}
    \label{fig:figures-lab4_1_resposta_simulink}
\end{figure}

\subexercise{b)}

A partir do equacionamento não linear do sistema, desejamos encontrar um FLC. Assim, temos
\begin{align*}
    & \dot{h}(t) = \frac{500u(t) - A_2\sqrt{2gh} }{\pi\left( Dh - h^2 \right) } = v(t)\\
,\end{align*}
onde $v(t)$ é a saída do controlador linear desejado. Assim, podemos determinar a nova entrada para a planta como \[
    u(t) = \frac{v(t)\pi\left( Dh - h^2 \right) + A_2\sqrt{2gh} }{500}
,\] o que resulta em um novo sistema \[
\dot{h}(t) = v(t) + \frac{A_2\sqrt{2gh(t)} - A_2\sqrt{2gh(t)}  }{\pi\left( Dh(t) h^2(t) \right) } = v(t)
,\] considerando que a identificação de $A_2$ é perfeita.

Para a implementação do controle, podemos entender $u(t)$ como \[
    u(t) = f(h) + g(h)v(t)
,\] onde
\begin{align*}
    f(h) &= \frac{A_2\sqrt{2gh} }{500} \\
    g(h) &= \frac{\pi\left( Dh - h^2 \right) }{500}
.\end{align*}
Além disso, para garantirmos que o acionamento se mantém nos níveis aceitos pelo sistema propaga-se a saturação para $v(t)$, de forma que \[
    \frac{-f(h)}{g(h)} < v(t) < \frac{100-f(h)}{g(h)}
.\] 

A partir disso, implementou-se o controle linearizante conforme visível na figura \ref{fig:figures-lab7_FLC-png}. Destaca-se o uso de valores aproximados para os limites de saturação para evitar erros numéricos.

\begin{figure}[H]
    \centering
    \includegraphics[width=0.8\textwidth]{figures/lab7_FLC.png}
    \caption{Controle linearizante implementado no Simulink.}
    \label{fig:figures-lab7_FLC-png}
\end{figure}

Para o controle, implementou-se a lei de controle \[
v(t) = \dot{h}_r + K_0e(t)
,\] que foi implementada junto à planta conforme visível na figura \ref{fig:figures-lab7_FLC_simulink-png}.

\begin{figure}[H]
    \centering
    \includegraphics[width=0.8\textwidth]{figures/lab7_FLC_simulink.png}
    \caption{Implementação do sistema em malha fechada com controlador.}
    \label{fig:figures-lab7_FLC_simulink-png}
\end{figure}

Assim, a resposta do sistema pode ser observada na figura \ref{fig:figures-lab7_b_result-png}. Um ajuste fino do ganho $K_0$ foi realizado mas, pelo baixo impacto, optou-se por mantê-lo unitário. O aprimoramento da resposta é claro quando comparado à figura \ref{fig:figures-lab4_1_resposta_simulink}, com um tempo de acomodação significativamente maior e sem nenhum sobressinal.

\begin{figure}[H]
    \centering
    \includegraphics[width=0.8\textwidth]{figures/lab7_b_result.png}
    \caption{Resposta do sistema à variação na referência.}
    \label{fig:figures-lab7_b_result-png}
\end{figure}

\subexercise{c)}

Para analisar a robustez do sistema, fixamos o parâmetro $A_2$ do ponto de vista do controlador linearizante e variamos o valor real desse parâmetro na planta. A resposta do sistema nesse cenário pode ser observada na figura \ref{fig:figures-lab7_c_resposta-png}.

\begin{figure}[H]
    \centering
    \includegraphics[width=0.8\textwidth]{figures/lab7_c_resposta.png}
    \caption{Resposta do sistema com variação do parâmetro $A_2$ da planta.}
    \label{fig:figures-lab7_c_resposta-png}
\end{figure}

\end{document}
