\documentclass[a4paper]{report}
\input{./preamble.tex}
 
\begin{document}
 
\title{Teste 0}
\author{Bruno M. Pacheco (16100865)\\
EMC 5235 - Metrologia Industrial}
 
\maketitle
 
Como o mensurando é a vazão de água para consumo da casa, primeiro nos certificamos de que, com todos os registros da casa fechados, não há vazamento, pois afetaria diretamente a comparação do indicado pelo sistema de medição com o hidrômetro. Assumindo que não há vazamento, i.e., o valor registrado pelo hidrômetro não se altera com os registros internos fechados, definimos um volume de referência para comparar com o hidrômetro. Sabendo que um hidrômetro mede as grandezas com resolução de decilitros, nos restringimos a um volume que possua resolução maior ou igual para maximizar a precisão da comparação. Podemos imaginar o uso de um recipiente de um liquidificador para tal.

Agora, com o volume de referência definido, construímos nosso sistema de medição através do uso de uma saída única de água diretamente no volume de referência, tomando os cuidados para garantir que toda a água permanece no recipiente. Dessa forma, visamos garantir que toda a água que passa pelo hidrômetro termina no recipiente. Podemos, então, comparar a quantidade de água indicada pelas marcações do recipiente (indicação direta do nosso sistema de medição) com a variação do hidrômetro antes e depois de enchê-lo. Para garantir maior confiabilidade do nosso resultado, realizamos esse processo uma quantidade estatisticamente significativa de vezes, sempre enchendo o recipiente até a mesma marcação. Assim, caso o resultado base indicado pelo hidrômetro esteja acima do resultado base do apontado pelo nosso sistema de medição e, ainda mais, se as faixas de incertezas tiverem pequena sobreposição, temos um forte indicativo de que há um problema com o hidrômetro.

\end{document}
