\documentclass[a4paper]{report}
\input{./preamble.tex}
\newenvironment{alltt}{\ttfamily}{\par}
 
\begin{document}
 
\title{Devoir 1}
\author{Bruno M. Pacheco (20308350) \\
IFT6512 - Programmation Stochastique}
 
\maketitle
 
\exercise{1}

The EVPI can be defined as the difference between the value of the solution to the recourse problem (RP) and the value of the wait-and-see (WS) solution.
Both values are defined as 
\begin{align*}
    WS &= \mathbb{E}_\xi \min_{x\in X} c^Tx + Q(x,\xi) \\
    RP &= \min_{x\in X} c^Tx + \mathbb{E}_\xi Q(x,\xi)
.\end{align*}
Intuitively, the EVPI is positive because the WS approach computes the optimal solution for every realization of the random variable, while the RP solution is merely feasible.
In other words, for every realization of $\xi$, the WS approach results in a solution at least as good as the RP solution.

Formally, assume that our random variable has a finite support of size $N>0$.
Furthermore, for any realization $\xi_i$, let \[
\overline{x}(\xi_i) = \arg\min_{x\in X} c^Tx  + Q(x,\xi_i)
\] denote the WS solution and \[
x^{\star} = \arg\min_{x\in X} c^Tx + \mathbb{E}_\xi Q(x,\xi)
\] be the RP solution.
Then,
\begin{align*}
    WS &= \sum_{i=1}^{N} P(\xi = \xi_i) \min_{x\in X} c^Tx + Q(x, \xi_i) \\
       &= \sum_{i=1}^{N} P(\xi = \xi_i)\left( c^{T}\overline{x}(\xi_i) + Q(\overline{x}(\xi_i),\xi_i \right) 
,\end{align*}
while
\begin{align*}
    RP &= \min_{x\in X} c^Tx + \sum_{i=1}^{N} P(\xi=\xi_i) Q(x,\xi_i) \\
    &= c^{T}x^{\star} + \sum_{i=1}^{N} P(\xi=\xi_i) Q(x^{\star},\xi_i) \\
    &= \sum_{i=1}^{N} P(\xi=\xi_i)\left( c^{T}x^{\star} + Q(x^{\star},\xi_i) \right)
.\end{align*}
Thus,
\begin{align*}
    EVPI &= RP - WS \\
	 &= \sum_{i=1}^{N} P(\xi=\xi_i)\left( c^{T}x^{\star} + Q(x^{\star},\xi_i) - c^{T}\overline{x}(\xi_i) - Q(\overline{x}(\xi_i),\xi_i) \right) \ge 0
\end{align*}
because the definition of $\overline{x}(\xi_i)$ implies that \[
c^{T}x + Q(x,\xi_i) \ge c^{T}\overline{x}(\xi_i) + Q(\overline{x}(\xi_i),\xi_i)
\] for any $x\in X$.
Note that minor modifications in the above reasoning yield the desired result for the case of continuous random variables.

Now, to show that the VSS is always non-negative, we follow a similar approach.
The VSS is the difference between the value of the solution to the RP and the value of the solution to the expected-value problem (EV), defined as \[
EV = \min_{x\in X} c^{T}x + Q(x,\overline{\xi} )
,\] where $\overline{\xi}$ is the expected value of our random variable.
Let $\widetilde{x}$ be the solution of EV.
Then, we can write 
\begin{align*}
    VSS &= \left( \mathbb{E}_\xi c^{T}\widetilde{x} + Q(\widetilde{x},\xi) \right) - RP \\
    &= \mathbb{E}_\xi \left( c^{T}\widetilde{x} + Q(\widetilde{x},\xi) \right) - \mathbb{E}_\xi \left( c^{T}x^{\star} + Q(x^{\star},\xi) \right)  \\
    &= c^{T}\widetilde{x} + \mathcal{Q}(\widetilde{x}) - c^{T}x^{\star} - \mathcal{Q}(x^{\star})
,\end{align*}
where $\mathcal{Q}(\cdot )$ is as defined in class.
Similar to the previous result, the definition of $x^{\star}$ implies that $VSS \ge 0$.

\exercise{2}

In the proposed problem, $\overline{\xi}=1.5$, which means that  \[
    EV = \min_{\substack{x+y \ge 1 \\ x,y \ge 0}} 2x + \frac{3}{2}y = 2
\] at $x=1,y=0$.

Now, the WS approach yields
\begin{align*}
    WS &= \frac{3}{4} \left( \min_{\substack{x+y \ge 1 \\ x,y \ge 0}} 2x + y \right) + \frac{1}{4} \left( \min_{\substack{x+y \ge 1 \\ x,y \ge 0}} 2x + 3y \right) \\
    &= \frac{3}{4}2 + \frac{1}{4}3 = \frac{9}{4} > EV
.\end{align*}

\exercise{3}

I have solved this question by following the original problem statement.

See \texttt{devoir.ipynb}.

\exercise{4}

\subexercise{a}

This problem can be seen as a variation of the newsvendor problem.
The agent has to decide how much bread to make (decision variables), taking into consideration the randomness of the demand (random variable).
More precisely, we can formulate the problem as
\begin{align*}
    \min_{\bm{x}} \quad & \bm{c}^{T}\bm{x} + \mathbb{E}_{\bm{\delta}\sim \mathcal{N}(\bm{\mu},\Sigma)} Q(\bm{x},\bm{\delta}) \\
    \textrm{s.t.} \quad & \bm{x} \ge \bm{0}
,\end{align*}
where
\begin{align*}
    \mathcal{Q}(\bm{x}, \bm{\delta}) = \min_{\bm{y}, \bm{w}} \quad & -\bm{q}^{T}\bm{y} - \bm{r}^{T}\bm{w} \\
    \textrm{s.t.} \quad & \bm{y} \le \bm{\delta} \\
      & \bm{y} + \bm{w} \le \bm{x} \\
      & \bm{y}, \bm{w} \ge \bm{0}
.\end{align*}

The first stage decision variables are $\bm{x}= (x_b,x_e)$, which represent the amount of white and whole bread produced, respectively.
The second stage variables $\bm{y}=(y_b,y_e)$ represent the amount of bread sold during the day, and $\bm{w}=(w_b,w_e)$ the amount of bread sold in the evening.
The random variables $\bm{\delta}=(\delta_b,\delta_e)$ are the demands for each type of bread.

The parameters of the problem are adjusted as required, with \[
\bm{c} = \begin{bmatrix} 1.5 \\ 1.8 \end{bmatrix} ;\; \bm{q} = \begin{bmatrix} 3 \\ 4 \end{bmatrix} ;\; \bm{r} = \begin{bmatrix} 1 \\ 1.2 \end{bmatrix} 
\] being, resp., the production costs, the price of bread during the day, and the price of bread at the evening.
Finally, the demand is drawn from a normal distribution with mean and variances \[
    \bm{\mu} = \begin{bmatrix} 50 \\ 30 \end{bmatrix} ;\; \Sigma = \begin{bmatrix} 5^2 & 0 \\ 0 & 2^2 \end{bmatrix} 
.\] 

\subexercise{b}

In the problem above, we have $K_1 = \R^{2}_+$.
However, $K_2(\bm{\delta})$ is only non-empty for non-negative realizations of the random variable, in other words, given that the random variables are drawn from a normal distribution, they can assume negative values, for which the second-stage problem becomes infeasible.

On the other hand, negative demand is nonsensical, thus we assume that \emph{the support of the random variable} is actually $\Delta = \R^{2}+$, so that it \emph{approximately} follows a normal distribution over its support.
Now, we can state that $K_2(\bm{\delta}) = \R^{2}+$, which yields $K_2 = K_2^{P} = \R^{2}+$.

It is easy to see that the problem has complete recourse.
In fact, we could reformulate the problem into the standard two-stage form (by adding slack variables and transforming the constraints into equalities) and show that the $W$ matrix has full rank.

\subexercise{c,d,e,f}

I have solved these problems following the original statement.

See \texttt{devoir.ipynb}.


\end{document}
