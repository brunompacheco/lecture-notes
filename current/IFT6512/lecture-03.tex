\lecture{3}{10.09.2024}{Random Numbers Generation}

We started with a quick recap of the Farmer's problem implementation in Julia+JuMP along with a new formulation of the entire problem (three cases).

\begin{note}
    The EVPI is the Expected Value \emph{OF} the Perfect Information.
    It shows how much the optimal expected revenue would increase if we could perfectly predict which case is going to materialize.
    Another way of phrasing it is by the difference of the expected revenue under perfect information and the expected revenue of the stochastic problem.

    Similarly, the VSS is the value \emph{OF} the stochastic solution, which measures how much the revenue increases when taking the uncertainties into consideration during the optimization, i.e., in contrast to optimizing for the expected scenario.
\end{note}


\section*{Random Numbers Generation (RNG)}

\subsection*{$\mathcal{U}[0,1]$-distributed random numbers}

A random generator over $\mathcal{U}[0,1]$ is a major component of good RNG libraries.
This is mostly due to being able to drawn from other distributions by applying transformations to the unitary uniform distribution.

However, in computers, we are only able to generate integer values (binary), which is why we need some sort of discrete generation process?

\begin{definition}
    A \emph{transition function} $f: \mathcal{S} \longrightarrow \mathcal{S}$ is a functions over the \emph{state space} $\mathcal{S}$.
    It is assumed that $\mathcal{S}$ has finite cardinality.
\end{definition}

\begin{notation}
    The \emph{initial state} will be denoted $s_0\in \mathcal{S}$, and we will write \[
    s_n = f(s_{n-1}
    .\] 
\end{notation}

We assume that $f$ is periodic for all $n \ge \tau$ (often $\tau=0$) with period $\rho$: \[
s_{n+\rho} = s_n,\,n\ge \tau
.\] 

Given a transition function, we want to map the states into the output space $\mathcal{U}=(0,1)$.
We will use an output function $g: \mathcal{S} \longrightarrow \mathcal{U}$.

Now, how to choose $f$ and $g$?
Goals:
\begin{itemize}
    \item Large $\rho$ 
    \item Good uniformity
    \item "random" behavior
\end{itemize}

\subsection*{Linear Congruential Generator (LCG)}

The state space is $\mathcal{S}=\N_{\ge 0}$ and the transition function is \[
s_n = f(s_{n-1}) = (a s_{n-1} + c ) \mod m
.\] 

