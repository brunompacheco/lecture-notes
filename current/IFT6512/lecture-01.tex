\lecture{1}{03.09.2024}{Introduction to Stochastic Programming}

\paragraph{Main goal:} introduce stochastic programming, which aims to perform mathematical optimization while taking into account \emph{uncertainty}.

\section*{Introduction}

Are deterministic problems realistic?
\begin{itemize}
    \item Measurement errors
    \item Uncertainties on the future
    \item Unavailable data
\end{itemize}

Mathematical programming (optimization) = decision problem

Stochastic programming = decision under uncertainty, where the uncertainty is represented through \emph{random parameters}.

For a problem
\begin{align*}
    \min_{x\in X} \quad & g_0(x,\xi) \\
    \textrm{s.t.} \quad & g_i(x,\xi) \le 0, i=1,\ldots,m 
,\end{align*}
where $\xi$ is a random variable, what does it mean to optimize?

Assuming that the random vector is finite, we define a \emph{scenario} as a realization of the optimization.

\section*{Farmer's problem}

Let there be a problem over variables $x,y$.
If variables $y$ can be uniquely determined by variables $x$ in the optimal solution, then we could formulate the problem as a bilevel program, splitting over the variables, which lets us determine the expected cost/objective over the scenarios.
The bilevel formulation shows us that a solution to $x$ has to be feasible for all scenarios.

Question :what is the definition of EVPI (expected value of perfect information)? I believe it is the difference between the expected value solution, and the expected cost/objective that results from using the optimal solutions for each scenario.

The difference between the stochastic solution and the solution to the average scenario is called VSS (value of the stochastic solution).

