\documentclass[a4paper]{report}
\input{./preamble.tex}
\newenvironment{alltt}{\ttfamily}{\par}
 
\begin{document}
 
\title{Homework Assignment}
\author{Bruno M. Pacheco (202100361) \\
DAS410062 - Intro to Optimization}
 
\maketitle
 
\exercise{1}

Given $x_1,\ldots,x_7 \in \{0,1\} $, the variables that indicate whether the stock will be acquired (in any amount), the restrictions will follow.

\subexercise{a}

\[
\sum_{n=1}^{7} x_{n} < 7
\]

\subexercise{b}

\[
\sum_{n=1}^{7} x_{n} > 0
\] 

\subexercise{c}
\[
x_1+x_3 \le 1
\]
\subexercise{d}
\[
x_4\le x_2
\] 
I understood this restriction as "stock 4 can be selected only if stock 2 \emph{is} selected".

\subexercise{e}
\[
x_1=x_5
\]
\subexercise{f}
\[
\left( x_1+x_2+x_3 \right) + \left( \frac{x_2+x_4+x_5+x_6}{2} \right) \ge 1
\] 

\exercise{2}

\subexercise{$\|x\|_0 \le k$}

First, we create variables $x_{i}^{+}, x_{i}^{-} \ge 0, i=1,\ldots,n$ constrained to \[
x_{i}^{+}-x_{i}^{-} = x_{i},\,i=1,\ldots,n
.\] It is clear that $x_{i}^{-}+x_{i}^{+} = |x_{i}|$.

Now, let us assume that $\epsilon$ is the smallest value possible: the resolution of our floating point representation. Then, with the constraints
\begin{align*}
    y_{i} &\le \left( x_{i}^{-}+x_{i}^{+} \right)  \frac{1}{\epsilon} \\
    y_{i} &\le 1
\end{align*}
for $i=1,\ldots,n$ the variables $y_{i}$ will be 1 just in case $x_{i}$ is non-zero. Therefore, \[
\sum_{i=1}^{n} y_{i} \le k \iff \|x\|_0 \le k
.\] 

\subexercise{$\|x\|_1 \le k$}

Using our variables defined in the previous, the constraint with the Manhattan norm can be modeled by \[
\sum_{i=1}^{n} \left( x_{i}^{-} + x_{i}^{+} \right) \le k
.\] 

\subexercise{$\|x\|_{\infty} \le k$}

This norm can be modeled through a simple optimization formulation. Again, reusing our variables,
\begin{align*}
    \max_{z} \quad & z \\
    \textrm{s.t.} \quad & z \le x_{i}^{-}+x_{i}^{+},\,i=1,\ldots,n \\
      & z \ge 0
.\end{align*}
This way, we will have \[
    z = \max \{|x_1|,\ldots,|x_{n}|\} = \|x\|_\infty
.\]

In an optimization problem with a cost function already, one can add such constraint by adding $z$ to the cost function (or $-z$ in case it is a minimization problem) and then adding $z\le k$ to the constraints.

\exercise{3}

\subexercise{(a)}

Let $x_i \in \left\{ 1,\ldots,n \right\} $ be the order in which task $i$ is performed. For it to be a feasible solution, no more than one task must be processed at each time, that is, \[
x_i \neq x_j ,\, i\neq j,\, i,j \in V \tag{$*$}
.\] Also, the precedence defined by $G=\left( V,E \right) $ must be respected, thus \[
x_i < x_j\,\forall \left( i,j \right) \in E \tag{$**$}
.\]

Still, one must compute the start time \[
s_i = \sum_{j \in V : x_j < x_i } p_j
,\] which is a nonlinear constraint. For us to handle this constraint, we add a support variable $y_{ij}$ which represents that task $i$ precedes task $j$. This way, with a big enough $M$ constant value, one can write
\begin{align*}
    x_i - x_j &\ge 1 - My_{ij} \\
    x_i - x_j &\le -1 + M \left( 1 -  y_{ij}\right)
.\end{align*}
Notice how these constraints also imply ($*$). Also, ($**$) can now be written \[
y_{ij} = 1\,\forall \left( i,j \right) \in E 
.\] 

Finally, we can write the problem
\begin{align*}
    \min_{x_1,\ldots,x_n} \quad & \sum_{i=1}^{n} w_i s_i \\
    \textrm{s.t.} \quad & x_i - x_j \ge 1 - My_{ij},\, i,j \in V ,\, i\neq j \\
      & x_i - x_j \le  1 - M \left( 1 - y_{ij}\right) ,\, i,j \in V ,\, i\neq j \\
      & y_{ij} = 1,\, \forall \left( i,j \right) \in E \\
      & s_i = \sum_{j \in V} p_j y_{ji} \\
      & y_{ij} \in \left\{ 0,1 \right\},\, i,j \in V \\
      & x_i \in \left\{ 1,\ldots,n \right\}
.\end{align*}

The AMPL formulation can be seen in files \texttt{ampl-6.\{mod,dat,run\}}. The problem was solved using CPLEX, in the NEOS server. The output was:

\begin{alltt}
Presolve eliminates 36 constraints and 6 variables. \\
Adjusted problem: \\
162 variables: \\
\-\hspace{2cm}138 binary variables \\
\-\hspace{2cm}12 integer variables \\
\-\hspace{2cm}12 linear variables \\
270 constraints, all linear; 918 nonzeros \\
\-\hspace{2cm}12 equality constraints \\
\-\hspace{2cm}258 inequality constraints \\
1 linear objective; 12 nonzeros. \\

CPLEX 20.1.0.0: threads=4 \\
CPLEX 20.1.0.0: optimal integer solution; objective 1888 \\
288 MIP simplex iterations \\
0 branch-and-bound nodes \\
:      \_varname  \_var    := \\
1     'x[1]'        7 \\
2     'x[2]'       10 \\
3     'x[3]'        8 \\
4     'x[4]'       12 \\
5     'x[5]'        1 \\
6     'x[6]'        5 \\
7     'x[7]'       11 \\
8     'x[8]'        4 \\
9     'x[9]'        2 \\
10    'x[10]'       6 \\
11    'x[11]'       3 \\
12    'x[12]'       9 \\
\end{alltt}

\exercise{4}

Given that the non-linearities of the formulation reside in the trigonometric functions, the piece-wise linearization is chosen to be on $P_{i,j}$ as a whole. For this, 9 points of the domain are chosen for the approximation, from $-\pi$ to $\pi$ in steps of $\frac{\pi}{4}$, that is, the set of points are such that
\begin{align*}
    \theta^{(k)} &= -\pi + k\left(  \frac{\pi}{4} \right),\,k=0,\ldots,8  \\
    P^{(k)}_{i,j} &= g_{i,j} -\left( g_{i,j}\cos\theta^{(k)} + b_{i,j}\sin\theta^{(k)} \right), i \in N , j\in N_i
.\end{align*}

This way, using the \emph{Convex Combination} model, one gets the following appendix to the original formulation (replacing equation $\left( 5c \right) $)
\begin{align*}
    & \theta_i - \theta_j = \sum_{k=1}^{8} \lambda_k \theta^{(k)},\, P_{i,j} = \sum_{k=0}^{8} \lambda_k P^{(k)}_{i,j} \\
    & 1 = \sum_{k=0}^{8} \lambda_k,\, 1 = \sum_{k=1}^{8} z_k \\
    & \lambda_k \le z_k + z_{k+1},\, k=1,\ldots,7 \\
    & \lambda_{0} \le z_1,\, \lambda_8 \le z_8 \\
    & \lambda_k \ge 0,\, k=0,\ldots,8 \\ 
    & z_k \in \{0,1\},\, k=1,\ldots,8
.\end{align*}

If one has access to a solver which handles \emph{Special Ordered Sets of Variables Type 2} (SOS2), then the formulation may become simply
\begin{align*}
    & \theta_i - \theta_j = \sum_{k=0}^{8} \lambda_k \theta^{(k)},\, P_{i,j} = \sum_{k=0}^{8} \lambda_k P^{(k)}_{i,j} \\
    & 1 = \sum_{k=0}^{8} \lambda_k \\
    & \lambda_k \ge 0,\, k=0,\ldots,8 \\
    & \{\lambda_k \}_{k=0}^{8} \text{ is SOS2}
.\end{align*}

Finally, so we can minimize the absolute value of the powers, we add
\begin{align*}
    & P_i^{(+)} - P_i^{(-)} = P_i, \\
    & P_i^{(+)}, P_i^{(-)} \ge 0,\, i = 1,\ldots,3
,\end{align*}
so our cost function can be \[
\sum_{i\in N} P_i^{(+)} + P_i^{(-)}
.\]

The AMPL implementation as well as the results can be seen in the .dat and .mod files.

\end{document}
