\documentclass[a4paper]{report}
\input{./preamble.tex}
 
\begin{document}
 
\title{Approximation Algorithms\\Project Proposal}
\author{Bruno M. Pacheco (16100865)\\
DAS410047 - Introdução a Algoritmos}
 
\maketitle

\section*{Goals}

Assuming $P\neq NP$, approximation algorithms are feasible solutions for most instances of relevant NP problems \cite{10.5555/1614191}. Thus, their study become of great relevance. Besides that, it has a great convergence with many of the subjects studied in the optimization path of our department. Therefore, with this lecture I plan to introduce the ideas and definitions behind approximation algorithms, bring some examples and design techniques (briefly) \cite{Williamson2011} and present a glimpse of its relation to linear programming to solve optimization problems \cite{Johnson1974,10.5555/241938.241949}.

\section*{Planned Content}

\begin{enumerate}
    \item Introduction
    \begin{enumerate}
	\item NP-complete algorithms
	\item Motivation for approximation algorithms
    \end{enumerate}
    \item Definition
    \begin{enumerate}
        \item Performance ratios
	\item Approximation schemes
	\item Running time for approximation schemes/algorithms
    \end{enumerate}
    \item Examples
    \begin{enumerate}
	\item Traveling-salesman
	\item Subset-sum
    \end{enumerate}
    \item Design techniques
    \begin{enumerate}
        \item Greedy algorithm
	\item Dynamic programming
    \end{enumerate}
    \item Linear Programming
    \begin{enumerate}
        \item Relaxations
	\item Approximation algorithmic solutions
	\item Examples
    \end{enumerate}
\end{enumerate}


\bibliographystyle{plain}
\bibliography{project_proposal}

\end{document}

