\documentclass[a4paper]{report}
\input{./preamble.tex}
 
\begin{document}
 
\title{Fast Fourier Transform\\Project Proposal}
\author{Bruno M. Pacheco (16100865)\\
DAS410047 - Introdução a Algoritmos}
 
\maketitle

\section*{Goals}

Gilbert Strang, in his introduction to wavelets, called the FFT "the most important numerical algorithm of our lifetime" \cite{strang}. With a considerable improvement in complexity ($O\left( n^2 \right) $ to $O\left( n \log n\right) $), the algorithm published by Cooley and Tukey in 1965 \cite{Cooley1965} is still in the roots of many of the numerical softwares used daily by engineers and mathematicians. Thus, with this lecture I aim to introduce and detail the algorithm \cite{Lathi}, including a complexity analysis and different implementations \cite{ThomasH.Cormen2009}. Also, I plan to provide some applications so my colleagues can grasp how impactful is the FFT.

I also included a short introduction to Fourier analysis as it is not taught anymore in the Signals and Linear Systems course. This way, I hope to level everyone with the necessary basics to follow the FFT explanation.

\section*{Planned Content}

\begin{enumerate}
    \item Fourier Transform
    \begin{enumerate}
	\item Fourier analysis
	\item Discrete Fourier Transform (DFT)
    \end{enumerate}
    \item Fast Fourier Transform - FFT
    \begin{enumerate}
	\item Linearity of the DFT
	\item The Cooley-Tukey algorithm
	\item The Gauss algorithm? \cite{Heideman1985}
	\item Complexity
	\item Efficient implementations
    \end{enumerate}
    \item Applications
    \begin{enumerate}
	\item Convolutions
	\item Polynomials multiplication
	\item Multiplication of large integers
	\item Spectral method
    \end{enumerate}
    \item Conclusion
\end{enumerate}


\bibliographystyle{plain}
\bibliography{project_proposal}

\end{document}

