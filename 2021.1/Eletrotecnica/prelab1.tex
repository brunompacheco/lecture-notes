\documentclass[a4paper]{report}
\input{./preamble.tex}
 
\begin{document}
 
\title{Laboratório 1}
\author{Bruno M. Pacheco (16100865)\\
EEL5354 - Eletrotécnica para Automação}
 
\maketitle
 
\section*{Preparação}

\exercise{1}

O ensaio de curto-circuito é utilizado para determinar a reatância síncrona do gerador, enquanto o ensaio de circuito aberto é utilizado para determinar a relação entre a corrente de campo e a tensão gerada enquanto o gerador roda na velocidade rotacional nominal. Dessa forma, basta determinar a resistência do enrolamento para que o modelo do gerador seja levantado.

\exercise{2}

A frequência angular mecânica do rotor estabelece a frequência da tensão elétrica gerada pelo número de pares de polos no rotor.

\exercise{3}

Para medir a tensão gerada $E_A$ deve ser utilizado um voltímetro na faixa de 1000 V e um amperímetro com limite de 100 A para a corrente do estator $I_A$.

\exercise{4}

A velocidade de rotação do rotor determina a velocidade com que os polos do rotor interagem com os enrolamentos do estator, portanto, determinando a frequência com que a força eletromotriz oscila.

\exercise{5}

A resistência pode ser estimada aplicando uma tensão contínua nos terminais do estator e medindo a corrente que circula. A partir disso, pode-se determinar a impedância em um ensaio de curto-circuito, i.e., sabendo a tensão que é gerada na operação nominal, conecta-se somente o amperímetro como carga, o que possibilita a determinação da impedância uma vez que \[
I_A = \frac{E_A}{\sqrt{R_A^2 + X_S^2} }
.\] 

\exercise{6}

Para um gerador operando em 60 Hz, podemos estimar que a tensão gerada será determinada por \[
v_{60} = -N \frac{d \Phi}{d t} = -N \frac{d}{d t}\left( K \sin\left( 2 \pi 60 t \right)  \right) = -N K 2\pi 60 \sin\left( 2\pi 60 t \right) 
,\] onde $K$ é uma constante, enquanto se operarmos em 50 Hz \[
v_{50} = -N K 2\pi 50 \sin\left( 2\pi 50 t \right) 
.\] Dessa forma, a tensão eficaz em cada um dos casos será
\begin{align*}
    v_{60,RMS} &= 60\frac{ N K 2\pi }{\sqrt{2} }  \\
    v_{50,RMS} &= 50\frac{ N K 2\pi }{\sqrt{2} }
.\end{align*}

Podemos, então, estimar que a potência resultante, para uma mesma carga, será
\begin{align*}
    P_{60} &= \frac{v_{60,RMS}^2}{R} = 3600\frac{\left( \frac{ N K 2\pi }{\sqrt{2} } \right) ^2}{R} \\
    P_{50} &= 2500\frac{\left( \frac{ N K 2\pi }{\sqrt{2} } \right) ^2}{R} \\
    \implies P_{50} &= \frac{25}{36} P_{60} 
.\end{align*}
Ou seja, a potência será reduzida em cerca de 31\%.

\exercise{7}

Espera-se que um gerador de 60 Hz, para atingir a mesma potência e tensão nominal, precise de mais espiras no estator, portanto, aumentando de tamanho em relação ao de 400 Hz.

\exercise{8}

\begin{figure}[H]
    \centering
    \includegraphics[width=0.8\textwidth]{figures/prelab1-8.png}
\end{figure}

\exercise{9}

\begin{figure}[H]
    \centering
    \includegraphics[width=0.8\textwidth]{figures/prelab1-9.png}
\end{figure}

\exercise{10}

\begin{figure}[H]
    \centering
    \includegraphics[width=0.8\textwidth]{figures/prelab1-10.png}
\end{figure}

\exercise{11}

57,735 V

\exercise{12}

Assumindo que a potência consumida se mantenha dentro do valor nominal, um valor muito alto da corrente de armadura gerará um superaquecimento da armadura por efeito Joule, o que seria de grande risco para o operador e para a manutenção do equipamento.

\end{document}
