\section{Planta}

As dinâmicas do líquido no interior do tanque foram simuladas no LabVIEW utilizando as ferramentas para equações diferenciais, em particular, para a equação do calor bidimensional para domínios retangulares. Como o tanque é cilíndrico, assumiu-se que simulando uma seção do tanque seria o suficiente para aproximar o comportamento da temperatura no interior do líquido frente aos pontos de medição almejados.

As condições de contorno foram utilizadas para simular as perdas e o aquecimento. Troca de calor nula nas paredes do tanque simulam o seu isolamento, enquanto uma injeção de calor constante (dentro de um passo de cálculo) nas paredes próximas ao aquecedor simulam a potência do aquecedor aquecendo o líquido. Para simular as perdas ao ambiente, a condução térmica entre o líquido e o ambiente externo (na distância entre a superfície e a abertura do tanque) foi calculada e definida como a perda de calor na superfície a cada passo de cálculo.

As variações volumétricas foram implementadas através de modificações no domínio de cálculo. Podemos entender o domínio de cálculo em uma situação computacional como uma matriz que representa as temperaturas nos pontos do líquido. Dessa forma, remoção do líquido perto da base foi emulada com a remoção das linhas da matriz mais próximas da base, enquanto adições de líquido foram feitas através de adições de linhas na matriz na região dos pontos da superfície do líquido.

Dessa forma, a cada passo de cálculo é necessário:
\begin{enumerate}
    \item Inicializar as equações e o domínio com a matriz de temperaturas resultante do passo anterior;
    \item Calcular as condições de contorno a partir dos valores do aquecedor e da temperatura ambiente;
    \item Resolver a equação do calor bidimensional para 1 passo de cálculo;
    \item Renderizar os resultados;
    \item Adicionar e remover linhas da matriz de acordo com a vazão de entrada e saída.
\end{enumerate}

