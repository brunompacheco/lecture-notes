\section{Desafios}

Ainda que devido à modalidade remota o projeto não envolva um protótipo real, vários desafios foram encontrados principalmente na tentativa de fazer simulações fidedignas. O primeiro desafio encontrado, em ordem cronológica das implementações, foi quanto à estabilidade da simulação do líquido no tanque. As variações do domínio das equações decorrentes das vazões de entrada e saída facilmente ocasionavam instabilidades nos resultados ao longo do tempo. Muitas vezes, isso decorria de erros de arredondamento que se acumulavam ao longo do tempo, fazendo com que o nível da água esperado e o tamanho da matriz que armazena os pontos no interior do líquido divergissem. A solução encontrada foi limitar a vazão máxima (a 0,5\,m$^3$/s, um valor bastante alto) e realizar somente incrementos proporcionais à resolução da matriz do domínio, passando ao passo seguinte as variações de volume restantes.

Além disso, a definição do modelo da planta antes da escolha dos componentes (devido à ordem das entregas) limitou as escolhas a soluções não necessariamente ideais. Um exemplo é o aquecedor elétrico, que esperava-se utilizar um sistema de aquecimento por contato com a base do tanque, mas soluções satisfatórias (grande porte, fornecimento de potência suficiente, disponibilidade de dados) dessa forma não foram encontradas, resultando na escolha de um sistema de aquecimento não muito adequado à simulação.

Já no âmbito da integração dos componentes da simulação, a pequena vazão das válvulas acabou limitando o seu potencial como fontes de ruído para a malha de controle. A falta de experiência no dimensionamento desse tipo de componente hidráulico acabou limitando a escolha a válvulas de baixo coeficiente de vazão, de forma que a pequena diferença de pressão estipulada não fosse suficiente para gerar um impacto significativo na condução térmica.

Finalmente, a ausência de suporte de muitas das ferramentas para o sistema operacional Linux tornou necessária a simulação dos circuitos de condicionamento utilizando um software especifico para isso (LTSpice) que não consegue se comunicar com o software desenvolvido em LabVIEW para monitoramento e controle. Isso gerou um esforço adicional para modelar o desempenho dos circuitos no LabVIEW, de forma que a malha de controle pudesse ser completada.

\section{Próximos Passos}

É evidente que o presente trabalho possui muita margem para melhoria. Ainda assim, o foco desta seção será em potenciais aprimoramentos no que tange o escopo da disciplina. Dessa forma, por exemplo, modificações na simulação do líquido no interior do tanque ou no processo em questão não serão abordadas. Além disso, as melhorias propostas são sempre tendo em vista a experiência adquirida com o presente trabalho, e não necessariamente o desempenho do processo imaginado no qual se enquadra o sistema simulado.

Dito isso, uma escolha de componentes mais adequados a aplicação pode ser feita. Válvulas que permitam maior vazão ou o uso de uma bomba para aumentar a diferença de pressão na válvula de saída permitira um controle mais dinâmico do nível. Um aquecedor com controle do calor gerado permitiria a implementação de um controlador PI, por exemplo, dando muito mais precisão à temperatura do líquido de interesse.

Melhorias nos modelos dos transdutores e atuadores seriam de grande benefício para que os resultados se tornem mais confiáveis. Exemplos seriam um modelo do aquecedor elétrico que considere seu tempo de resposta e um modelo do transdutor de nível que considere os efeitos da temperatura do líquido. Além disso, seria interessante simular ruídos de forma a tornar válida a implementação de filtros nos sinais mais críticos, analisando o impacto dos ruídos no desempenho do controlador. Finalmente, seria também interessante simular, através de um componente aleatório, os erros dos transdutores de acordo com as margens fornecidas pelos fornecedores.

