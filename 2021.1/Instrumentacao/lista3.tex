\documentclass[a4paper]{report}
% Some basic packages
\usepackage[utf8]{inputenc}
\usepackage[T1]{fontenc}
\usepackage{textcomp}
\usepackage[english]{babel}
\usepackage{url}
\usepackage{graphicx}
\usepackage{float}
\usepackage{booktabs}
\usepackage{enumitem}

\pdfminorversion=7

% Don't indent paragraphs, leave some space between them
\usepackage{parskip}

% Hide page number when page is empty
\usepackage{emptypage}
\usepackage{subcaption}
\usepackage{multicol}
\usepackage{xcolor}

% Other font I sometimes use.
% \usepackage{cmbright}

% Math stuff
\usepackage{amsmath, amsfonts, mathtools, amsthm, amssymb}
% Fancy script capitals
\usepackage{mathrsfs}
\usepackage{cancel}
% Bold math
\usepackage{bm}
% Some shortcuts
\newcommand\N{\ensuremath{\mathbb{N}}}
\newcommand\R{\ensuremath{\mathbb{R}}}
\newcommand\Z{\ensuremath{\mathbb{Z}}}
\renewcommand\O{\ensuremath{\emptyset}}
\newcommand\Q{\ensuremath{\mathbb{Q}}}
\newcommand\C{\ensuremath{\mathbb{C}}}
\renewcommand\L{\ensuremath{\mathcal{L}}}

% Package for Petri Net drawing
\usepackage[version=0.96]{pgf}
\usepackage{tikz}
\usetikzlibrary{arrows,shapes,automata,petri}
\usepackage{tikzit}
\input{petri_nets_style.tikzstyles}

% Easily typeset systems of equations (French package)
\usepackage{systeme}

% Put x \to \infty below \lim
\let\svlim\lim\def\lim{\svlim\limits}

%Make implies and impliedby shorter
\let\implies\Rightarrow
\let\impliedby\Leftarrow
\let\iff\Leftrightarrow
\let\epsilon\varepsilon

% Add \contra symbol to denote contradiction
\usepackage{stmaryrd} % for \lightning
\newcommand\contra{\scalebox{1.5}{$\lightning$}}

% \let\phi\varphi

% Command for short corrections
% Usage: 1+1=\correct{3}{2}

\definecolor{correct}{HTML}{009900}
\newcommand\correct[2]{\ensuremath{\:}{\color{red}{#1}}\ensuremath{\to }{\color{correct}{#2}}\ensuremath{\:}}
\newcommand\green[1]{{\color{correct}{#1}}}

% horizontal rule
\newcommand\hr{
    \noindent\rule[0.5ex]{\linewidth}{0.5pt}
}

% hide parts
\newcommand\hide[1]{}

% si unitx
\usepackage{siunitx}
\sisetup{locale = FR}

% Environments
\makeatother
% For box around Definition, Theorem, \ldots
\usepackage{mdframed}
\mdfsetup{skipabove=1em,skipbelow=0em}
\theoremstyle{definition}
\newmdtheoremenv[nobreak=true]{definitie}{Definitie}
\newmdtheoremenv[nobreak=true]{eigenschap}{Eigenschap}
\newmdtheoremenv[nobreak=true]{gevolg}{Gevolg}
\newmdtheoremenv[nobreak=true]{lemma}{Lemma}
\newmdtheoremenv[nobreak=true]{propositie}{Propositie}
\newmdtheoremenv[nobreak=true]{stelling}{Stelling}
\newmdtheoremenv[nobreak=true]{wet}{Wet}
\newmdtheoremenv[nobreak=true]{postulaat}{Postulaat}
\newmdtheoremenv{conclusie}{Conclusie}
\newmdtheoremenv{toemaatje}{Toemaatje}
\newmdtheoremenv{vermoeden}{Vermoeden}
\newtheorem*{herhaling}{Herhaling}
\newtheorem*{intermezzo}{Intermezzo}
\newtheorem*{notatie}{Notatie}
\newtheorem*{observatie}{Observatie}
\newtheorem*{exe}{Exercise}
\newtheorem*{opmerking}{Opmerking}
\newtheorem*{praktisch}{Praktisch}
\newtheorem*{probleem}{Probleem}
\newtheorem*{terminologie}{Terminologie}
\newtheorem*{toepassing}{Toepassing}
\newtheorem*{uovt}{UOVT}
\newtheorem*{vb}{Voorbeeld}
\newtheorem*{vraag}{Vraag}

\newmdtheoremenv[nobreak=true]{definition}{Definition}
\newtheorem*{eg}{Example}
\newtheorem*{notation}{Notation}
\newtheorem*{previouslyseen}{As previously seen}
\newtheorem*{remark}{Remark}
\newtheorem*{note}{Note}
\newtheorem*{problem}{Problem}
\newtheorem*{observe}{Observe}
\newtheorem*{property}{Property}
\newtheorem*{intuition}{Intuition}
\newmdtheoremenv[nobreak=true]{prop}{Proposition}
\newmdtheoremenv[nobreak=true]{theorem}{Theorem}
\newmdtheoremenv[nobreak=true]{corollary}{Corollary}

% End example and intermezzo environments with a small diamond (just like proof
% environments end with a small square)
\usepackage{etoolbox}
\AtEndEnvironment{vb}{\null\hfill$\diamond$}%
\AtEndEnvironment{intermezzo}{\null\hfill$\diamond$}%
% \AtEndEnvironment{opmerking}{\null\hfill$\diamond$}%

% Fix some spacing
% http://tex.stackexchange.com/questions/22119/how-can-i-change-the-spacing-before-theorems-with-amsthm
\makeatletter
\def\thm@space@setup{%
  \thm@preskip=\parskip \thm@postskip=0pt
}


% Exercise 
% Usage:
% \exercise{5}
% \subexercise{1}
% \subexercise{2}
% \subexercise{3}
% gives
% Exercise 5
%   Exercise 5.1
%   Exercise 5.2
%   Exercise 5.3
\newcommand{\exercise}[1]{%
    \def\@exercise{#1}%
    \subsection*{Exercise #1}
}

\newcommand{\subexercise}[1]{%
    \subsubsection*{Exercise \@exercise.#1}
}


% \lecture starts a new lecture (les in dutch)
%
% Usage:
% \lecture{1}{di 12 feb 2019 16:00}{Inleiding}
%
% This adds a section heading with the number / title of the lecture and a
% margin paragraph with the date.

% I use \dateparts here to hide the year (2019). This way, I can easily parse
% the date of each lecture unambiguously while still having a human-friendly
% short format printed to the pdf.

\usepackage{xifthen}
\def\testdateparts#1{\dateparts#1\relax}
\def\dateparts#1 #2 #3 #4 #5\relax{
    \marginpar{\small\textsf{\mbox{#1 #2 #3 #5}}}
}

\def\@lecture{}%
\newcommand{\lecture}[3]{
    \ifthenelse{\isempty{#3}}{%
        \def\@lecture{Lecture #1}%
    }{%
        \def\@lecture{Lecture #1: #3}%
    }%
    \subsection*{\@lecture}
    \marginpar{\small\textsf{\mbox{#2}}}
}



% These are the fancy headers
\usepackage{fancyhdr}
\pagestyle{fancy}

% LE: left even
% RO: right odd
% CE, CO: center even, center odd
% My name for when I print my lecture notes to use for an open book exam.
% \fancyhead[LE,RO]{Gilles Castel}

\fancyhead[RO,LE]{\@lecture} % Right odd,  Left even
\fancyhead[RE,LO]{}          % Right even, Left odd

\fancyfoot[RO,LE]{\thepage}  % Right odd,  Left even
\fancyfoot[RE,LO]{}          % Right even, Left odd
\fancyfoot[C]{\leftmark}     % Center

\makeatother




% Todonotes and inline notes in fancy boxes
\usepackage{todonotes}
\usepackage{tcolorbox}

% Make boxes breakable
\tcbuselibrary{breakable}

% Verbetering is correction in Dutch
% Usage: 
% \begin{verbetering}
%     Lorem ipsum dolor sit amet, consetetur sadipscing elitr, sed diam nonumy eirmod
%     tempor invidunt ut labore et dolore magna aliquyam erat, sed diam voluptua. At
%     vero eos et accusam et justo duo dolores et ea rebum. Stet clita kasd gubergren,
%     no sea takimata sanctus est Lorem ipsum dolor sit amet.
% \end{verbetering}
\newenvironment{verbetering}{\begin{tcolorbox}[
    arc=0mm,
    colback=white,
    colframe=green!60!black,
    title=Opmerking,
    fonttitle=\sffamily,
    breakable
]}{\end{tcolorbox}}

% Noot is note in Dutch. Same as 'verbetering' but color of box is different
\newenvironment{noot}[1]{\begin{tcolorbox}[
    arc=0mm,
    colback=white,
    colframe=white!60!black,
    title=#1,
    fonttitle=\sffamily,
    breakable
]}{\end{tcolorbox}}




% Figure support as explained in my blog post.
\usepackage{import}
\usepackage{xifthen}
\usepackage{pdfpages}
\usepackage{transparent}
\newcommand{\incfig}[1]{%
    \def\svgwidth{\columnwidth}
    \import{./figures/}{#1.pdf_tex}
}

% Fix some stuff
% %http://tex.stackexchange.com/questions/76273/multiple-pdfs-with-page-group-included-in-a-single-page-warning
\pdfsuppresswarningpagegroup=1


% My name
\author{Bruno M. Pacheco}

 
\begin{document}
 
\title{Lista 3}
\author{Bruno M. Pacheco (16100865)\\
DAS5151 - Instrumentação em Controle}
 
\maketitle
 
\exercise{1}

\subexercise{a}

A melhor instalação seria com os extensômetros na mesma orientação, i.e., paralelos, mas em lados opostos da viga, de forma que, em um circuito de meia ponte, não só a sensibilidade seja dobrada uma vez que a deformação em ambas as superfícies da peça são utilizadas, mas também os efeitos da temperatura também são compensados. Ainda mais, pode-se utilizar ambos em série no circuito de meia ponte, de forma a ter uma variação linear da resposta.

\subexercise{b}

Sabendo que \[
    V_{AB} = \frac{1}{2}\frac{\Delta R}{R} V_E = \frac{K \epsilon}{2} V_E
,\] podemos determinar \[
\epsilon = \frac{2 V_{AB}}{V_E K} = 19,048\,\frac{\mu m}{m}
.\] 

\subexercise{c}

Com mais dois extensômetros seria ideal arranjá-los em pares nas faces opostas, conectando-os em um circuito de ponte completa. Isso resultaria em uma aumento considerável na sensibilidade da medição, que não seria exatamente o dobro uma vez que os novos extensômetros possuem ganhos diferentes.

\exercise{2}

Instalando ambos termorresistores imersos no líquido, podemos conectá-los nas diagonais de um circuito de ponte cuja tensão diferencial é conectada ao amplificador de instrumentação ideal. Dessa forma, temos que a tensão lida $V_{in} = 100 V_{AB}$, enquanto, assumindo $\Delta R \ll R$, chegamos em
\begin{align*}
    V_{AB} &= \frac{\left( R+\Delta R \right)^2 - R^2}{\left( 2R + \Delta R \right)^2 } V_E \\
    &= \frac{2R\Delta R + \Delta R^2}{4R^2 + 4R\Delta R + \Delta R^2} V_E \\
    &\approx \frac{1}{2}\frac{\Delta R}{R + \Delta R}V_E \\
    &\approx \frac{1}{2}\frac{\Delta R}{R} V_E
.\end{align*}
A partir disso, podemos calcular a sensibilidade do circuito observando o valor lido a partir da variação de $1\,K$ na temperatura (uma vez que o comportamento do circuito é linear), ou seja, $\Delta R = 0,38\,\Omega \implies V_{AB} = 0,19\,mV \implies V_{in} = 19\,mV$.

\exercise{3}

Para a primeira alternativa, tem-se
\begin{align*}
    V_{OUT} &= \frac{\left( 100+\Delta R \right) 49,9k - 4990k}{50k\left( 50k + \Delta R \right) } V_{IN} \\
    &= \frac{49,9}{50}\frac{\Delta R}{50k + \Delta R} V_{IN} \\
    &\approx \frac{\Delta R}{50k}V_{IN}
.\end{align*}

Da mesma forma, para a alternativa b,
\begin{align*}
    V_{OUT} &= \frac{25k\left( 25k + \Delta R \right) -25k.25k}{50k\left( 50k + \Delta R \right) } V_{IN} \\
    &= \frac{1}{2}\frac{\Delta R}{50k + \Delta R} V_{IN} \\
    &\approx \frac{1}{2}\frac{\Delta R}{50k}V_{IN}
.\end{align*}
Assim, é fácil ver que a alternativa a terá aproximadamente o dobro da sensibilidade da alternativa b.

\exercise{4}

\subexercise{a}

Assumindo que a alimentação da célula de carga é tal que a tensão de excitação $V_{ex} = 5\,V$, tem-se uma tensão de saída de \[
\frac{8}{10}0,002 V_{ex} = 8\,mV
.\] 

\subexercise{b}

Agora, sabendo já que sob $10\,kg$ tem-se $V_{AB} = 10\,mV$, pode-se encontrar \[
\epsilon = \frac{V_{AB}}{K V_{ex}} = 1000\,\mu m / m
.\]

\exercise{5}

Uma abordagem possível é a já estudada nos exercícios anteriores: extensometria em circuitos de ponte. Entretanto, é necessário atentar que os extensômetros devem ficar a $45^\circ$ do eixo para capturar as linhas de tração e compressão. Apesar de todas as vantagens já sabidas, essa abordagem apresenta grandes problemas quanto a alimentação no caso de eixos girantes, quando pensa-se em mensurar torque em rotações contínuas (e.g., motor elétrico).

Para evitar esse problema, é possível utilizar uma abordagem direta e medir o ângulo entre duas partes do eixo através de referências (e.g., roda dentada). Dessa forma, medem-se os deslocamentos angulares entre duas secções do eixo girante e compara-se a variação relativa aos valores iniciais (eixo em repouso). Apesar de ser bastante interessante por ter pouca interferência na operação o eixo, essa abordagem se limita a rotações mais baixas (cerca de 10000 rpm) e não costuma apresentar grande resolução.

\end{document}
